\section{Parser Construction}
\label{sect:method:alternatives}
Writing a parser from scratch was ruled out early for being too time consuming.
Instead it was decided to use tools for compiler and parser construction to
generate a parser from the XQuery 1.0 and XPath 2.0 grammar  
specifications\cite{w3c01} developed by the W3C.

Some important requirements were taken into account when evaluating these
alternatives. Specifically, the generator must:
\begin{itemize}
  \item generate Java source code for the parser
  \item be able to generate an LL parser, because the grammar  
  specification given by W3C is LL(k) \footnote{This can be argued, as later 
  seen in section \ref{sect:discussion:lookahead}}
  \item be licensed liberally, either GPL or BSD, or an equivalent license
  approved by OSI
\end{itemize}

The alternative parser generators evaluated are presented in the following
sections.

\subsection{JFlex/CUP}
JFlex and CUP is a versatile combination consisting of JFlex which is a lexer
generator, and CUP which is a parser generator. These tools can be interfaced to
generate a complete parser with a separate lexical analyzer.

JFlex and CUP produces only LALR parsers, and since the W3C has specified an
LL grammar for XQuery 1.0 and XPath 2.0, the combination of JFlex and CUP was
rejected from this project. Additionally, some basic occurence operators such as
*,+ and ? is not supported by CUP.

In the light of this decision, it is important to note that a large amount of
work can be avoided by rejecting a LALR parser generator solution for this
project. Consider the examples in figure \ref{code:parsers:lalr} and
\ref{code:parsers:ll} and compare it to the reference grammar specification in
figure \ref{code:parsers:w3c}.

\begin{figure}[h!]
\begin{Verbatim}
Expr : ExprSingle
     | ExprSingle "," Expr
     ;
\end{Verbatim}
\caption[LALR grammar example]{LALR grammar example from the Pathfinder project\cite{pathfinderHome} 
(slightly edited for brevity)}
\label{code:parsers:lalr}
\end{figure}

\begin{figure}[h!]
\begin{Verbatim}
expr : exprSingle (COMMASi exprSingle)* ;
\end{Verbatim}
\caption[LL grammar example]{LL grammar example, actual code used in this project}
\label{code:parsers:ll}
\end{figure}

\begin{figure}[h!]
\begin{Verbatim}
[32] Expr ::= ExprSingle ("," ExprSingle)*
\end{Verbatim}
\caption{ W3C reference grammar example }
\label{code:parsers:w3c}
\end{figure}

Obviously, the LALR-grammar has suffered an rewrite into a recursive
rule to compensate for the lack of support for occurence indicators. The LL
example stands out as clear and readable, and is similar to the example from the
W3C specification.

In the case of JFlex/CUP, building abstract syntax trees is a process that must
be done manually by adding action code snippets to the grammar. This is
unfortunate for several reasons - it dilutes the grammar and decreases
readability and clarity. Again, we use the same example from Pathfinder in
figure \ref{code:parsers:lalr2}, however with tree building logic included.
Contrast this to the tree rewrite rules and operators described later in section
\ref{sect:antlr:ast}.

\begin{figure}[h!]
\begin{Verbatim}
Expr : ExprSingle { $$ = wire2 (p_exprseq, @$,
                    $1,
                    leaf (p_empty_seq, @1)); }
     | ExprSingle "," Expr { $$ = wire2 (p_exprseq, @$, $1, $3); }
\end{Verbatim}
\caption[LALR grammar example]{Verbatim LALR grammar example from the Pathfinder
project\cite{pathfinderHome}}
\label{code:parsers:lalr2}
\end{figure}

The CUP parser generator as well as the code it produces is licensed under a
custom GPL\footnote{http://www.gnu.org/copyleft/gpl.html, also see section \ref{sect:method:licensing}
}-compatible license. The JFlex lexer generator itself is licensed under the GPL
license, however the code produced inherits the license of the lexer specification.

\subsection{JavaCC}
JavaCC could have been a viable alternative as it produces LL(k) parsers,
however compared to ANTLR its grammar specification syntax deviated more
from the W3C EBNF syntax, meaning that the grammar would have had to be
even more rewritten. However, JavaCC supports the same occurence indicators
used in the XQuery grammar specification. Figure \ref{code:parsers:javacc} shows
a grammar fragment translated to JavaCC syntax -- compare to the original grammar
in figure \ref{code:parsers:w3c}.

\begin{figure}[h!]
\begin{Verbatim}
void expr()
{}
{
    exprSingle() (<COMMA> exprSingle())* ;
}
\end{Verbatim}
\caption{ JavaCC reference grammar example }
\label{code:parsers:javacc}
\end{figure}

For AST construction, JavaCC requires the use of a preprocessor tool known as  
JJTree. This tool will parse the grammar and insert tree building actions where
appropriate. 

JavaCC is licensed under a BSD-style license (see section
\ref{sect:method:licensing}).


\subsection{ANTLR}
ANTLR is a renowned tool for parser generation, and can generate LL(*) parsers
(explained in the next chapter). The ANTLR parser generator was chosen for this
project based on the basic requirements outlined previously in this section.  

ANTLR meets all of these requirements, and also adds some commodities such as a
grammar syntax which is syntactically close to the syntax used by the W3C in
their grammar specification\cite{w3c01}. Additionally ANTLR supports some
interesting features such as syntactic and semantic predicates, as well as
automatic AST (abstract syntax tree) generation. These features are described in
detail later in chapter\ref{sect:antlr}, and the usage of these features has been described
in chapter \ref{chapter:implementation}.

ANTLR is licensed under a BSD-style license (see section
\ref{sect:method:licensing}).

\section{Debugging}
\label{sect:method:debugging}
There are several approaches to debugging. ANTLRWorks \cite{antlrwrks00} is a
simple tool for writing, testing and debugging ANTLR grammars. The debugging
interface is useful in that it draws a realtime step-by-step parse tree as the
input is being parsed, as well as displaying a list of parser events.
ANTLRWorks also helps eliminate grammar nondeterminisms by drawing the syntax
diagram associated with a grammar and highlighting nondeterministic paths. A
big drawback, though, is as the grammar grows bigger, the application tends to
become unstable.    

When working properly, ANTLRWorks is a great tool for debuging the parser, but
no so much for the lexer. It will show the tokens returned to the parser, but
no information indicating the reason for returning just that
token. A more appropriate tool for such a task could be debugging facilities in
a regular integrated development environment. % Fjernet ref til eclipse her
