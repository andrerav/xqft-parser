% Discussion / ScopingAndSymbol
\section{Scoping, Symbol Tables, and Typechecking}
Scoping and symbol tables are important for safely enabling features like
forward references\footnote{A rundown can be found
at http://en.wikipedia.org/wiki/Forward\_declaration},
symbol management, and semantic error detection. This section will discuss the
current implementation of scoping and symbol tables.

\subsection{Scoping and Symbol Tables}
\label{sect:discussion:scoping_and_symtab}
Currently the scoping and symbol table code is inlined directly in the grammar, and executed during parse time, as described in section
\ref{sect:impl:scoping_and_symtab}. It could be benefitial to move this
functionality into being performed at a higher level of a system, and combine it with type checking
functionality. This possibility was previously mentioned in section
\ref{sect:discussion:ast:extend}.

This would decouple some amount of code from the grammar, improving readability
slightly, and creating a more self-contained parser with no forced semantic
checks.

The current implementation is adequate, but maintainability and extendability
suffers from the constraints of being locked into a monolithic grammar file.

\subsection{Type Checking}
Currently the parser will not perform type checking on the parsed queries. This
is an essential feature and will be necessary to implement for the parser to be
applicable in any realistic setting. A type checking system with proper type
inference and synthesis\cite{compiler_tech} could be a
complex feature to implement in a language such as XQuery, and might require
considerable effort -- especially in quality assurance. 

However, the resulting AST output from the parser (presented in section
\ref{sect:results:parser_output_ast}) should be a solid base to improve upon by
adding proper type inference and type checking. This would require a framework
for dataflow analysis. This is a substantial amount of work, but it would
benefit the project in several ways, for example the possibility of adding
optimizations.