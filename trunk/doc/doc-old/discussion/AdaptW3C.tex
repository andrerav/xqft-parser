\section{Adapting the W3C Grammar}
\label{sect:discussion:adaptW3C}

In the beginning of this project, very inexperienced parser developers as we
were, had the na\"{i}ve idea that since the XQuery grammar was given and
expressed in EBNF, our task was to run this through a parser generator,
possibly with some syntax changes, and the job would be done. After a while of
trying to adapt the given grammar to something ANTLR would accept, we came to
the conclusion that W3C probably intended the specification to be read by
humans, and not by computers. 

An alternative to insert and adapt would be to write a grammar from scratch
while keeping the semantics specified by W3C. We belive that by doing so there
is a risk of over-simplifying, causing latent semantics such as operator
precedence to be distorted, or in worst case lost alltogether. Another negative
aspect with this approach is that it does not properly utilize the work already
done, making it time consuming compared to just adapting.

The terminal productions had to be completely rewritten. By choosing to write
these rules from scratch in the first place, we would in all likelyhood have
discovered the problem with the ambiguous terminals earlier, but at the cost of
not having a parser up and running (albeit a very reduced one) until the
grammar was completly finished. This would have prohibited us from working with
e.g. the scoping system in parallell with the grammar.

Considering the non-terminal productions, their rewrites were not as extensive.
Most of the work done on the parser grammar were in the form of left factoring
and augmenting with syntactic predicates to reduce required lookahead, or in the form of
acommodating for the extra-grammatical constrains. We can not see that any
particular benefit would be gained from writing the non-terminal productions
from scratch, compared to adapting the W3C specification.
