\newpage
\section{BAREBAJZ}
\label{sect:translation:mXr}



\subsection{Path Expressions}
\label{sect:translation:mXr:smpPathExpr}
XQuery implements XPath 2.0 path expressions as described in section
\ref{sect:theory:xquery:PathExpressions}. The \textsf{scope} operator of
MQL(section \ref{sect:method:marsOperators}) makes it possible to
filter results based on the scope. The operator does however only support paths
equivalent to path expressions with only \texttt{child} axis steps. Methods to
translate expressions with other types of axis steps will be discussed in
section \ref{sect:discussion:notAxis}.

Figure \ref{fig:translation:pathAST} shows the abstract syntax tree for the
generic child axis and name test only path expression \texttt{/a/\ldots/y/z}. It
should be noted that for the sake of simplicitly the AST node \textbf{AST\_STEPEXPR} is
omitted from the representation. The corresponding translation into MQL of such
expressions can be seen in rule \ref{eq:translation:pathExpr}.


\begin{figure}[h]
\centering
\tikzstyle{astNode}=[circle, draw=blue!70,fill=blue!20,solid,thick, minimum
size=26pt]
\begin{tikzpicture}[grow via three points={one child at (0,-1.5) and two
children at (-1.5,-1.0) and (1.5,-1.0)}]

\node at (0,0) [ellipse,
draw=blue!70,fill=blue!20]{AST\_PATHEXPR\_SGL} 
child{node [astNode] {/}
	child{node [astNode] {/}
		child{node [astNode] {/} 
			child{node [astNode] {a} edge from parent [draw, solid, thin]}
			child{node [astNode] {\ldots} edge from parent [draw, solid, thin]}
			edge from parent [draw,dotted, thick] 
			}
		child{node [astNode] {y}}
		}
	child{node [astNode] {z}}
	};
\end{tikzpicture}
\label{fig:translation:pathAST}
\caption[AST of path expression]{A simplified version of the abstract syntax
tree of a simple path expression}
\end{figure}


\begin{equation}
\centering
\texttt{/a/\ldots/y/z}
\Longrightarrow
\begin{array}{l}
	\mbox{\textsf{scope(/a/\ldots/y;}} \\
	\quad \mbox{\textsf{index(valloc; lookup(\$z))}}
\end{array}
\label{eq:translation:pathExpr}
\end{equation}

Rule \ref{eq:translation:singleStep} shows how a single child axis step can be
translated.

\begin{equation}
\centering
\alpha \textsf{/} \beta
\Longrightarrow
\begin{array}{l}
	\mbox{\textsf{project([r.fields,}} \vartheta \mbox{\textsf{];}} \\ \quad
 	\mbox{\textsf{select(isInScope(r.scope, l.scope);}} \\ \quad \quad
 	\mbox{\textsf{join([documentId], [documentId]; }} 
 	\alpha;\, \beta \mbox{\textsf{)))}}; \\
\end{array}
\label{eq:translation:singleStep}
\end{equation}

Where $\vartheta = \alpha.\vartheta \cup \beta.\vartheta$. If $\alpha.\vartheta \cap \beta.\vartheta \neq
\emptyset$, the algebra will have to be augmented with a \textsf{select} operator removing tuples where the common
variables iteration number is not equal. The synchronisation of iteration numbers may be done either before or
after or in combination with the existing \texttt{select} operator in the rule. This is a general rule, and is
applied whenever two expressions are joined or crossed together.

\subsection{Predicates}
\label{sect:translation:mXr:predicates}
As described in section \ref{sect:theory:xqueryPredicates}, predicates are used to restrict the items returned
from the expression it is assigned to. The predicate expression returns either true/false or an integer value. Rule
\ref{eq:translation:predicate} shows how predicates, which neither implisitly nor explicitly makes use of the
context item, are translated into MQL. 

\begin{equation}
\centering
\begin{array}{c}
	\frac{\displaystyle \textbf{put(ctx,}\alpha\textbf{)}\vdash\textbf{ctx}\notin\beta.\vartheta}
	{\displaystyle \alpha\texttt{[}\beta\texttt{]}}

	\Longrightarrow 
	\\
	\mbox{}
	\\
	\begin{array}{l}
		\mbox{\textsf{project([fields = l.fields,}} \vartheta \mbox{\textsf{];}} 
		\\ \quad \mbox{\textsf{select(ifThenElse(isNumber(r.value),}} \\ \quad 
		\qquad\qquad \mbox{\textsf{scopeIndex(scope,r.value),xqBool(r.value));}} \\ \quad\quad
		\mbox{\textsf{cross(}} 
		\alpha \textsf{;}
		\beta \mbox{\textsf{)))}}
	\end{array}
\end{array}
\label{eq:translation:predicate}
\end{equation}


\textsf{isNumber(r.value), scopeIndex(scope, r.value)} and \textsf{xqBool(r.value))} are explained in section
\ref{sect:method:marsAddedOperators}. The use of \textsf{scopeIndex(scope,pred)} may however not be valid in cases
where there are multiple predicates, or when the predicate is assigned to a path expression which utilises some
axes like e.g. \texttt{decendant-or-self}. This will be discussed in section \ref{sect:discussion:predicates}.
With the same limitations rule \ref{eq:translation:iterationPredicate} shows how a predicate which \emph{does}
reffer to the context node (implicitly or explicitly) is translated.

\begin{equation}
\centering
\begin{array}{c}
	\frac{\displaystyle \textbf{put(ctx,}\alpha\textbf{)}\vdash\textbf{ctx}\in\beta.\vartheta}
	{\displaystyle \alpha\texttt{[}\beta\texttt{]}}

	\Longrightarrow 
	\\
	\mbox{}
	\\
	\begin{array}{l}
		\mbox{\textsf{project([fields = l.fields,}} \vartheta \mbox{\textsf{];}} 
		\\ \quad \mbox{\textsf{select(ifThenElse(isNumber(r.value),}} \\ \quad 
		\qquad\qquad \mbox{\textsf{scopeIndex(scope,r.value),xqBool(r.value));}} \\ \quad\quad
		\mbox{\textsf{cross( join?}} \\ \quad\quad\quad
		\mbox{\textsf{project(counter())}}\alpha \textsf{;} \\ \quad\quad\quad
		\beta \mbox{\textsf{)))}}
	\end{array}
\end{array}
\label{eq:translation:iterationPredicate}
\end{equation}

Where $\vartheta = \alpha \cup (\beta -$\textbf{ctx}$)$. Because of the predicate exists in its own scope, any
refferal to the context node in $\alpha$ will not be distracted by adding a new context node in the symbol table
for the predicate. As with the single step expression (rule \ref{eq:translation:singleStep}), any common iterator
variable references will have to be synchronised.

\textbf{\underline{\Large TODO:}} Dette fuker ikke helt\ldots hva skjer om predikatet er et sett? YEYEYE samme
problem som med operatorer og ifz etc\ldots Det vi m\aa~finne ut er hvordan sekvenser skal representeres\ldots.
trenger vi order? JEEEEVLER\ldots\ldots\ldots\ldots\ldots\ldots\ldots  BAJZ

