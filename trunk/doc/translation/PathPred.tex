\section{Path Expressions and Predicates}
\label{sect:trans:TD:pathNpred}

XQuery implement XPath 2.0 path expressions and predicates as described in section
\ref{sect:theory:xquery:PathExpressions} and \ref{sect:theory:xquery:Predicates}. In this section we will present
a method for translating some of these expressions into MQL relational algebra.

For the translations in this section to be correct, we have a few assumptions. A scope defined as a parameter to
the \textsf{scope}-operator starting with a slash (\textsf{/}) is an absolute scope. That means if the parameter
is e.g. \textsf{/a/b}, the operator will remove any tuples in the input relation where the $scope$ attribute does
not define the tuple to stem from a \texttt{b} scope within the root-scope \texttt{a}. Without a slash first in
the parameter, the scope is relative. E.g. \texttt{a/b} as a parameter will lead to removal of all tuples where
the $scope$ attribute does not define the tuple to stem from a \texttt{b} scope within an \texttt{a} scope.

We also assume that the tuples returned from a scope lookup in the \textsf{valocc}-index will have information of
the scope it self in the $scope$ attribute, and the contents of the scope in an $value$ attribute. E.g. a lookup of
\textsf{\$c} (the \textsf{\$}-sign indicates it is a scope) in this index may return a tuple with
\textsf{a[1].b[2].c[1]} as its $scope$ value.

The tuples returned from a lookup in the value occurence index will have to be ordered according to document
order. This order will have to hold even after the result is run through a \textsf{scope}

The concept of the context item is important in this section, and is by the W3C defined as follows\cite{w3c00}:
\begin{quote}
[Definition: The \textbf{context item} is the item currently being processed. An item is either an atomic value or
a node.][\ldots] The context item is returned by an expression consisting of a single dot (\texttt{.}). When an
expression $e_1$\texttt{/}$e_2$ or $e_1$\texttt{[}$e_2$\texttt{]} is evaluated, each item in the sequence obtained
by evaluating $e_1$ becomes the context item in the inner focus for an evaluation of $e_2$.
\end{quote}

\subsection{Path Expressions}
\label{sect:trans:TD:pathExprs}
A path expression consists of a series of one or more steps, separated by ``/'' or ``//'', and optionally beginning
with ``/'' or ``//''.  Each step is either a axis step or a filter expression, and a axis step consist of a axis
and a node test. This can be seen from the exerpt of the W3C XQuery specification in figure \ref{fig:trans:TD:pathEBNF}.
A node test can be either a kind test or a name test, we will focus on the latter. 

\begin{figure}[h]
\begin{Verbatim}
[68] PathExpr        ::= ("/" RelativePathExpr?)
                       | ("//" RelativePathExpr)
                       | RelativePathExpr
[69] RelativePathExpr::= StepExpr (("/" | "//") StepExpr)*
[70] StepExpr        ::= FilterExpr | AxisStep
[71] AxisStep        ::= (ReverseStep | ForwardStep) PredicateList
[72] ForwardStep     ::= (ForwardAxis NodeTest) | AbbrevForwardStep
[75] ReverseStep     ::= (ReverseAxis NodeTest) | AbbrevReverseStep
\end{Verbatim}
\label{fig:trans:TD:pathEBNF}
\caption{Path expressions as specified by W3C\cite{w3c00}}
\end{figure}
The semantics of such expressions reading from left to right, the result of one step expression is used as input
for the next. Within one step expression, the result from the preceding step will first be used as input for the
axis expression. The result of this will be filtered by a name or kind test, before this again is filtered by
possible predicates.

Step expressions can be abbreviated. If the axis name is omitted from an axis step, the default axis is
\texttt{child}. \texttt{decendant-or-self} can be replaced by using ``\texttt{//}'' istead of ``\texttt{/}''
between the steps. \texttt{@} is an abbreviation of \texttt{attribute::}. We will only present translation of
unabbreviated syntax.

To accomodate for the context item, the result of each step of a path expression will be stored on the symbol
table as \textbf{dot} -- each time replacing the the last entry. The context item, and references to it will be
treated as iterator dependencies. The attribute corresponding to a dependency on \textbf{dot} is $dotNumb$. No
expression not within the path expression can be dependant on \textbf{dot}, which is why \textbf{dot} will not be
part of the dependencies $\vartheta$ returned. A general path expression is translated in the following manner:

\begin{equation}
\frac{}{[\mbox{\texttt{/}}]e_1\mbox{\texttt{/}}\ldots\mbox{\texttt{/}}e_n} \longmapsto
\begin{array}{l}
\mbox{\textbf{put(dot, t(r(}}e_1\mbox{\textbf{), \{dot\}))}} \\
\qquad \qquad \vdots \\
\mbox{\textbf{put(dot, t(r(}}e_n\mbox{\textbf{), \{dot\}))}} \\
\mbox{\textsf{numberate(index, [dotNumb, index], value, }}\vartheta\mbox{\textsf{;}} \\ \quad
\mbox{\textbf{get(dot)}\textsf{)}}
\end{array}
\label{rule:trans:TD:pathExpr}
\end{equation} 

Where $\vartheta = (e_n.\vartheta-$\textbf{dot}$)$. Axis step expressions are all dependent on the context item and
will take no effect of the tainting. Tainting is used in this context to ensure filter expressions to be evaluated
correctly. The $\vartheta$ piggybacking the tree in the symbol table will contain \textbf{dot}, as with
the iterator variables.

A general step expression, axis + name test, can be translated like this:
\begin{equation}
\begin{array}{c}
\frac{}{\displaystyle \texttt{AXIS::}QName_1} \\ 
\longmapsto \begin{array}{c}\mbox{\tiny } \\ \mbox{\tiny } \end{array} \\
\begin{array}{l}
\mbox{\textsf{project(dotNumb, docId, index, value, pos, scope, }}\vartheta\mbox{\textsf{;}} \\ \quad
\mbox{\textsf{numberate(dotNumb, [dotNumb, subIdx], [}}(\vartheta-\mbox{\textbf{dot}})\mbox{\textsf{];}} \\
\quad\quad 
\mbox{\textsf{numberate(index, [index], [}}\vartheta\mbox{\textsf{];}} \\ \quad\quad\quad
\mbox{\textsf{select(isFUNC(scope, lsc);}} \\ \quad \quad\quad\quad
\mbox{\textsf{hhjoin([docId],[docId],}} \\ \quad\quad\quad\quad\quad\quad\quad
\mbox{\textsf{[dotNumb,lsc=l.scope,subIdx=r.index,r.value,}}\vartheta\mbox{\textsf{];}}\\
\quad\quad\quad\quad\quad \mbox{\textbf{get(dot)}\textsf{;}} \\ \quad \quad\quad \quad\quad
\mbox{\textsf{numberate(index, [], [];}}\\ \quad\quad\quad\quad\quad\quad
\mbox{\textsf{index(valocc;}} \\ \quad\quad\quad\quad\quad\quad\quad
\mbox{\textsf{lookup(\$}}QName_1\mbox{\textsf{))))))))}}
\end{array}
\end{array}
\label{rule:trans:TD:pathStep}
\end{equation}

Where $\vartheta$ is the $\vartheta$ returned from \textbf{get(dot)}, $QName_1$ is any XML-qualified name.
\textsf{r.value} is short for \textsf{value = right.value, index = right.index, \ldots etc}, and \textsf{docId} is
short for \textsf{documentId}. \texttt{AXIS} will map to an MQL funciton \textsf{isFUNK} as described in table
\ref{tab:trans:TD:axisMap}. To ensure correct ordering, the order of the tuples returned from the \textsf{lookup}
operator must be the same as the order of the tuples recieved by the \textsf{numberate} operator. 

\begin{table}[h]
\centering
\begin{tabular}{c|c}
\texttt{AXIS} & \textsf{isFUNC} \\ \hline
\texttt{child} & \textsf{isChild} \\
\texttt{descendant} & \textsf{isDescendant} \\
\texttt{attribute} & \textsf{isChild} $^{*}$ \\
\texttt{self} & \textsf{isSelf} \\
\texttt{descendant-or-self} & \textsf{isDescendantOrSelf} \\
\texttt{following} & \textsf{isFollowing} \\
\texttt{following-sibling} & \textsf{isFollowingSibling} \\
\texttt{parent} & \textsf{isParent} \\
\texttt{ancestor} & \textsf{isAncestor} \\
\texttt{ancestor-or-self} & \textsf{isAncestorOrSelf} \\
\texttt{preceding} & \textsf{isPreceding} \\
\texttt{preceding-sibling} & \textsf{isPrecedingSibling} 
\end{tabular}
\caption{Mapping between XQuery axes and MQL functions. \label{tab:trans:TD:axisMap}}
\end{table}

$^{*}$For the \texttt{attribute} axis the parameter to \textsf{lookup} will have a \textsf{\$@}-prefix
instead of the \textsf{\$}-prefix described in the rule.

The relation after the join will contain to copies of the $index$ attribute stemming from the lookup, $index$ and
$subIdx$. This is because the \textsf{numberate} operator will remove the attributes specified in the sort list.
The next to last \textsf{numberate} operator will number the tuples according to the document order, but among
others partition on its dependency of the context item. This numbering is necessary to solve e.g. numeric predicates. The
last \textsf{numberate} operator will update the $dotNumb$ field as it is possible that the axis step do not
match the items from the last step in a 1:1 ratio.

\begin{myExample}
Consider the exerpt of a XML-document of figure \ref{fig:trans:TD:pathExXml}. The subscript numbers are used to
differentiate the different elements with same names, and are not a part of the names. Further, let a non-iterator
variable \texttt{\$a} be bound to a sequence of the \texttt{A} elements of the figure, but \emph{not} in document order.
An illustration of the relational representation of \texttt{\$a} as it is stored as the context item in the symbol
table is shown in figure \ref{fig:trans:TD:pathEx:varA}. The $val$-attribute indicates which XML-element is
represented and the $scope$-attribute indicates the scope of this element. Figure \ref{fig:trans:TD:pathQu} shows
an exerpt of a query referring to the variable \texttt{\$a}. \begin{figure}[h!]
\centering
\texttt{\ldots\$a/child::B/\ldots}
\caption{Example XQuery path expression \label{fig:trans:TD:pathQu}}
\end{figure} 

\begin{figure}[h]
\centering
\begin{equation*}
\begin{array}{l}
\qquad \qquad \vdots \\
\mbox{\texttt{<A}}_{1}\mbox{\texttt{>}} \\ \quad
\mbox{\texttt{<B}}_{1}\mbox{\texttt{/>}}\mbox{\texttt{<B}}_{2}\mbox{\texttt{/>}} \\
\mbox{\texttt{</A}}_{1}\mbox{\texttt{>}} \\
\mbox{\texttt{<A}}_{2}\mbox{\texttt{>}} \\ 
\mbox{\texttt{</A}}_{2}\mbox{\texttt{>}} \\
\mbox{\texttt{<A}}_{3}\mbox{\texttt{>}} \\ \quad
\mbox{\texttt{<B}}_{3}\mbox{\texttt{/>}}\mbox{\texttt{<B}}_{4}\mbox{\texttt{/>}}\mbox{\texttt{<B}}_{5}\mbox{\texttt{/>}}
\\ \mbox{\texttt{</A}}_{3}\mbox{\texttt{>}} \\
\qquad \qquad \vdots
\end{array}
\end{equation*}
\caption[Exerpt of example XML-document.]{Exerpt of example XML-document. The subscript numbers indicate the
instance of the elements, and are not part of the QName \label{fig:trans:TD:pathExXml}}
\end{figure}

First, a lookup of \texttt{\$B} (where \texttt{\$} indicates to find a element, not a word) is done in the value
occurence index. The result of this is numerated by a \textsf{numberate}-operator, as illustrated in figure
\ref{fig:trans:TD:pathEx:luB}. The $index$-attribute ($idx$ in the figure) now holds the document-order of the
\texttt{B} elements. As there may be more \texttt{B} elements in the document, the $index$ attribute may not start
at the value 1 for the tuples relevant to the query.

\begin{figure}[h]
\centering
\subfigure[\textbf{r(}\texttt{\$a}\textbf{)}]{
\begin{tabular}{|c|c|c|}  \hline
$dNb$ & $val$ & $scope$ \\ \hline
1 & \texttt{A}$_{2}$ & \textsf{..{A}[2]} \\ \hline
2 & \texttt{A}$_{3}$ & \textsf{..{A}[3]} \\ \hline 
3 & \texttt{A}$_{1}$ & \textsf{..{A}[1]} \\ \hline
\end{tabular}
\label{fig:trans:TD:pathEx:varA}
}
\qquad
\subfigure[]{
\begin{tabular}{|c|c|c|}  \hline
$idx$ & $val$ & $scope$ \\ \hline
$\ldots$ & \texttt{\ldots} & \textsf{\ldots} \\ \hline
5 & \texttt{B}$_{1}$ & \textsf{..{A}[1].B[1]} \\ \hline
6 & \texttt{B}$_{2}$ & \textsf{..{A}[1].B[2]} \\ \hline
7 & \texttt{B}$_{3}$ & \textsf{..{A}[3].B[1]} \\ \hline
8 & \texttt{B}$_{4}$ & \textsf{..{A}[3].B[2]} \\ \hline
9 & \texttt{B}$_{5}$ & \textsf{..{A}[3].B[3]} \\ \hline
$\ldots$ & \texttt{\ldots} & \textsf{\ldots} \\ \hline  
\end{tabular}
\label{fig:trans:TD:pathEx:luB}
}
\caption[Evaluating the expression of figure \ref{fig:trans:TD:pathQu}]{Illustration of results evaluating the
expression in figure \ref{fig:trans:TD:pathQu}. (a) The variable \texttt{\$a}. (b) Experpt of a lookup on the term
\textsf{\$B}, and the following numbering. Some attribute names are shortened. $val$ attribute indicates which
XML-element is represented in the tuple.
\label{fig:trans:TD:pathEx}}
\end{figure}

The result of the lookup will be joined with the context item relation on their $documentId$-attribute. As we
assume only one XML-document this attribute is omitted from our example. A \textsf{select} operator is applied to
the result of the join to prune the relation. Only the tuples where the $scope$ attribute stemming from the lookup
defines a scope which is the child scope (as defined by the MQL function \textsf{isChild() in section
\ref{sect:method:marsAddedOperators}}) of the scope defined by the $scope$-attribute stemming from the
\texttt{\$a} relation (called $lsc$). After the selection the result will be as illustrated in figure
\ref{fig:trans:TD:pathEx:joinSel}. The copy of the $index$ column is not shown.

\begin{figure}[h]
\subfigure[]{
\begin{tabular}{|c|c|c|c|c|} \hline
$dNb$ & $idx$ & $val$ & $scope$ & $lsc$ \\ \hline
2 & 9 & \texttt{B}$_{5}$ & \textsf{..{A}[3].B[3]} & \textsf{..{A}[3]} \\ \hline
2 & 8 & \texttt{B}$_{4}$ & \textsf{..{A}[3].B[2]}& \textsf{..{A}[3]} \\ \hline
2 & 7 & \texttt{B}$_{3}$ & \textsf{..{A}[3].B[1]}& \textsf{..{A}[3]} \\ \hline 
3 & 5 & \texttt{B}$_{1}$ & \textsf{..{A}[1].B[1]} & \textsf{..{A}[1]} \\ \hline
3 & 6 & \texttt{B}$_{2}$ & \textsf{..{A}[1].B[2]} & \textsf{..{A}[1]} \\ \hline
\end{tabular}
\label{fig:trans:TD:pathEx:joinSel}
}
\quad
\subfigure[]{
\begin{tabular}{|c|c|c|c|} \hline
$dNb$ & $idx$ & $val$ & $scope$ \\ \hline
1 & 1 & \texttt{B}$_{3}$ & \textsf{..{A}[3].B[1]}\\ \hline 
2 & 2 & \texttt{B}$_{4}$ & \textsf{..{A}[3].B[2]}\\ \hline
3 & 3 & \texttt{B}$_{5}$ & \textsf{..{A}[3].B[3]}\\ \hline
4 & 1 & \texttt{B}$_{1}$ & \textsf{..{A}[1].B[1]}\\ \hline
5 & 2 & \texttt{B}$_{2}$ & \textsf{..{A}[1].B[2]}\\ \hline
\end{tabular}
\label{fig:trans:TD:pathEx:done}
}
\caption[Further evaluation of the path expression]{Further evaluation of the path expression in figure
\ref{fig:trans:TD:pathQu}. (a) The result of the selection of the joining of \textbf{r(}\texttt{\$a}\textbf{)} and
the numerated result of the lookup of \textsf{\$B}. (b) Renumbering an projection on the relation of (a). Some
attribute names are shortened.}
\end{figure}

Finally, numbering is employed two times, followed by a projection removing the last attributes stemming from the
context item relation. The result of the expression is illustrated in figure \ref{fig:trans:TD:pathEx:done}.

\end{myExample}

\subsection{Predicates}
\label{sect:trans:TD:predicates}
A predicate consists of an expression, called a predicate expression which evaluates to a boolean value. A
predicate serves to filter a sequence, retaining the items where the expression evaluates to $true$ and discarding
all other. In the case of multiple adjacent predicates, the predicates are applied from left to right, and the
result of applying each predicate serves as the input sequence for the following predicate. If the value of the
predicate expression is a singleton atomic value of a numeric type, the predicate truth value is $true$ if the
value of the predicate expression is equal to the position of the context item within the input sequence.

When entering a predicate a new scope is pushed to the symbol table, and the relational algebra tree corresponding
to the expression assigned the predicate is stored as \textbf{dot}. The predicate expression is then evaluated,
and then joined with the relational representation of the expression it is assigned to on their common
dependencies. If the context item is referred to within the predicate, $dotNumb$ is one of the attributes the
relations are joined on. The context item may be reffered to explicitly with the dot-operator (.), or implicitly
through a relative path expression. When evaluating path expressions it is important to note that predicates have
higher precedence than the / (step expression separator). As the predicate may remove items, renumbering is
needed. 

\begin{equation}
\frac{}{e_1\mbox{\texttt{[}}e_2\mbox{\texttt{]}}}\longmapsto
\begin{array}{l}
\mbox{\textsf{project(index, value = l.value, }}\vartheta\mbox{\textsf{;}} \\ \quad
\mbox{\textsf{numberate(index, [l.index], [}}\vartheta\mbox{\textsf{];}} \\ \quad\quad
\mbox{\textsf{select(ifthenelse(isNumber(pred),}} \\\quad\quad\quad\quad\quad
\mbox{\textsf{eq(index,pred), xqBoolean(pred));}} \\\quad\quad\quad
\mbox{\textsf{hhjoin([}}(e_1.\vartheta\cap e_2.\vartheta)\mbox{\textsf{], [}}(e_2.\vartheta\cap e_1.\vartheta)\mbox{\textsf{],}} 
\\ \quad\quad\quad \quad \quad \quad\mbox{\textsf{[l.value, pred=r.value, }}\vartheta\mbox{\textsf{];}}
\\\quad\quad\quad\quad
 \mbox{\textbf{r(}}e_1\mbox{\textbf{)}\textsf{;}} \\\quad\quad\quad\quad
\mbox{\textbf{r(}}e_2\mbox{\textbf{)}\textsf{))))}} \\
\end{array}
\label{rule:trans:TD:predicate}
\end{equation}

Where $\vartheta=e_1.\vartheta \cup (e_2.\vartheta-$\textbf{dot}$)$. As the expression may be a part of a path
expression, a possible dependency to \textbf{dot} in $e_1.\vartheta$ will not be removed. The fields of the left
relation in the join will follow \textsf{l.value} as described in section \ref{sect:trans:TD:basics}. The final
\textsf{project} operator will restore the names of these attributes.

As described earlier, before translation of the expression, a new scope will be pushed to the symbol table,
followed by \textbf{put(dot, r(}$e_1$\textbf{)}. If $e_1$ is not allready dependent on the context item,
\textbf{r(}$e_1$\textbf{)} will have to be run through a \textsf{project} operator with \textsf{dotNumb=index,
index=1, value} as parameters both in rule \ref{rule:trans:TD:predicate} and before being stored in the symbol
table. The sorting attribute of the \textsf{numberate} operator will also have to be \textsf{dotNumb} not
\textsf{l.index}, and $\vartheta=(e_1.\vartheta \cup e_2.\vartheta)-$\textbf{dot}.

\begin{myExample}
Consider the XQuery query of figure \ref{fig:trans:TD:predQu}. In this query the predicate expression is dependent
on the context item.
\begin{figure}[h]
\centering
\verb!(2, 3, 4, 5)[. mod 2 = 0]!
\caption{Example query with a predicate. \label{fig:trans:TD:predQu}}
\end{figure}

The sequence assigned a predicate is not dependent on the context item and will have its $index$ attribute renamed
to $dotNumb$ ($dNb$ in the figure) as shown in figure \ref{fig:trans:TD:pred:seq}. A reference to the context item
is like a reference to a variable, and the predicate expression evaluates to the relation illustrated in figure
\ref{fig:trans:TD:pred:predEx}. The two relations are joined together on their common dependencies, the context
item ($dotNumb$). Further, as the predicate expression does not evaluate to a numeric type, only the tuples where
$pred$ is $true$ are selected. This is shown in figure \ref{fig:trans:TD:pred:prune}.

\begin{figure}[h]
\subfigure[]{
\begin{tabular}{|c|c|c|} \hline
$dNb$ & $idx$ & $val$ \\ \hline
1 & 1 & 2 \\ \hline
2 & 1 & 3 \\ \hline
3 & 1 & 4 \\ \hline
4 & 1 & 5 \\ \hline
\end{tabular}
\label{fig:trans:TD:pred:seq}
}
\quad
\subfigure[]{
\begin{tabular}{|c|c|c|} \hline
$dNb$ & $idx$ & $val$ \\ \hline
1 & 1 & $true$ \\ \hline
2 & 1 & $false$ \\ \hline
3 & 1 & $true$ \\ \hline
4 & 1 & $false$ \\ \hline
\end{tabular}
\label{fig:trans:TD:pred:predEx}
}
\quad
\subfigure[]{
\begin{tabular}{|c|c|c|c|} \hline
$dNb$ & $idx$ & $val$ & $pred$ \\ \hline
1 & 1 & 2 & $true$ \\ \hline
3 & 1 & 4 & $true$ \\ \hline
\end{tabular}
\label{fig:trans:TD:pred:prune}
}
\caption[Evaluating the query in figure \ref{fig:trans:TD:predQu}]{Evaluating the query in figure
\ref{fig:trans:TD:predQu}. (a) The sequence assigned the predicate. (b) The predicate expression. (c) The
relations (a) and (b) joined together and pruned with a selection. Attribute names are shortened.}
\end{figure}

To finish the evaluation of the query, after the selection, the relation will have to be renumbered and the $pred$
attribute will have to be removed with a \textsf{project} operator.

\end{myExample}