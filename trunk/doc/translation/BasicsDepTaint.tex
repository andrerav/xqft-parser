\section{Basics}
\label{sect:trans:TD:basics}
The method assumes left-to-right traversal of the assymetric syntax tree. In most cases the traversal is
postorder, meaning a subtree can be evaluated independently from its ancestors -- the exception being the logical
context set by boolean operators. The relational algebra will thus be generated bottom-up. In addition to the
evaluated subtree, a node must be able to inform its parent node about its variable dependencies ($\vartheta$),
which we will discuss later.

One XQuery sequence is represented as one relation and one XQuery item is represented as one tuple. This is sound,
as all XQuery items are sequences, and all sequences are one-dimensional (section
\ref{sect:theory:xquery:basics}). As we mentioned in section \ref{sect:trans:MarkXRemove}, the MarkXRemove method
did actually not consider the ordering of items in sequences at all. In Tainting Dependencies, however, we have
introduced an attribute $index$ holding the intra sequence number of the item. Consider the XQuery sequence
\texttt{('a','b',}$\ldots$\texttt{,'z')}. With this attribute, the relational representation will be as follwing:

\begin{center}
\begin{tabular}{|c|c|} \hline
$index$ & $value$ \\\hline
1		& \texttt{'a'} \\\hline
2		& \texttt{'b'} \\\hline
$\ldots$& $\ldots$ \\\hline
$n$		& \texttt{'z'} \\\hline
\end{tabular}
\end{center}


As can be seen, the item value is stored in the $value$ attribute. For the course of this chapter we will, for the
sake of simplicity, treat $value$ as a polymorphic type attribute. This simplification has minimal consequences
for the method and the way XQuery expressions are translated. XQuery types and relational representation of such
will is handled in section \ref{sect:disc:typeSystem}.

Also for simplicity, the $index$, $documentId$, $pos$ and $scope$ attributes have sometimes been left out of the
fields specified in \textsf{project} operators. If the \textsf{project} operator is applied to the result of a
join or cartesian product, these fields will follow the $value$ attribute if nothing else is specified. That is, if
$r.value$ is projected, then so is $r.documentId$, etc\ldots if applicable.

Tainting Dependencies utilises a symbol table for storing of variables declared. The table has two functions:
\begin{itemize}
  \item \textbf{put(}$\chi$\textbf{, }\textbf{r(}$e$\textbf{))} -- will store the
  algebraic version of the expression bound to the variable \texttt{\$}$\chi$ with $\chi$ as the key.  
  \item \textbf{get(}$\chi$\textbf{)} -- will do a lookup in the table based on the name of the variable
  \texttt{\$}$\chi$ and return the algebraic version of the expression linked to it.
\end{itemize}
The symbol table handles scoping according to XQuery semantics (section \ref{sect:theory:xquery:Flwor}), meaning
the translator will always be able to find the right declared variable based on which node in the AST the
translator is visiting.

\subsection{Iterator Dependency Inheritance}
\label{sect:trans:TD:dependency}

The concept of iterator dependency form the basis of the Tainting Dependency method. Such dependency is
defined as follows:

\noindent
\begin{myDefinition}
An XQuery expression $e$ is \textbf{dependent} on an iterator $I_{\chi}$ if $e$
occurs within the iterator body of $I_{\chi}$ and if the evaluation of $e$ depends on the iteration number of $I_{\chi}$.
\label{def:iterVarDep}
\end{myDefinition}

An variable reference to an iterator variable \texttt{\$}$\chi$ is by this definition dependent on the iterator
$I_{\chi}$. Intuitively, an expression which subexpression is dependent on an iterator $I_{\chi}$ is also
dependent on this iterator -- we say the dependency is inherited. Consider the example subexpression of figure
\ref{fig:trans:td:varDep}, where \texttt{\$x} and \texttt{\$y} both are
iterator variables. Here, the expression $e_{1}$ is dependent on the two
iterators $I_{\mbox{\texttt{x}}}$ and $I_{\mbox{\texttt{y}}}$, while expression $e_{2}$ is only dependent on $I_{\mbox{\texttt{x}}}$.

\begin{figure}[h]
\centering
\tikzstyle{astNode}=[circle, draw=blue!70,fill=blue!20,solid,thick, minimum
size=26pt]
\begin{tikzpicture}[grow via three points={one child at (0,-1.5) and two
children at (-1.5,-1.0) and (1.5,-1.0)}]
\draw[loosely dotted, thick] (0,0) -- (0,-1);
\node at (0,-1) [astNode, label=above left:$e_{1}$ ] {\texttt{and}} 
child{node [astNode, label=above left:$e_{2}$] {\texttt{+}}
	child{node [astNode] {\texttt{\$x}}}
	child{node [astNode] {\texttt{3}}}
 }
child{node [astNode] {\texttt{\$y}}}
 ;
\end{tikzpicture}
\label{fig:trans:td:varDep}
\caption[Iterator variable dependency]{Iterator variable dependency}
\end{figure}

The iterator dependencies of an expression $e$ are part of the set $e.\vartheta$. As mentioned earlier,
an AST node must be able to inform its parent about the node's dependencies as well as the algebra generated. For
an expression $e$ this can be done by letting $e.\vartheta$ piggyback the \textbf{r(}$e$\textbf{)} returned. The
variable dependencies for an expression $e$ with the subexpressions $e_{1},\ldots,e_{n}$ can be described as
following:
\begin{equation}
e.\vartheta = e_{1}.\vartheta\cup\ldots\cup e_{n}.\vartheta
\label{eq:trans:TD:depInheritance}
\end{equation}

The dependency on the iterator $I_{\chi}$ manifest itself relationally by the
attribute $\chi$$numb$. The value of $\chi$$numb$ is the iteration number of $I_{\chi}$, that is, for a tuple ($\chi$$numb$, $value$) the value $value$
will appear in the $\chi$$numb$th iteration of $I_{\chi}$.

When an iterator variable \texttt{\$}$\chi$ is declared it is assigned a $\chi$$numb$ by renaming the $index$
field of the corresponding iterator sequence $\chi$$numb$. Which leads us to the inference rule for translating the
(optional) \texttt{for} clause of a FLWOR expression:
\begin{equation}
\frac{}{\mbox{\texttt{for \$}}\chi \mbox{\texttt{ in }} e \mbox{\texttt{\ldots}}}\longmapsto
\begin{array}{l}
\mbox{\textbf{put(}}\chi\mbox{\textbf{, }} \\ \quad
\mbox{\textsf{project(}}\chi\mbox{\textsf{numb = index, index=1, value;}} \\ \quad \quad
\mbox{\textbf{r(}}e\mbox{\textbf{)}\textsf{)}\textbf{)}}
\end{array}
\label{rule:trans:TD:forbind}
\end{equation}
Where the dependencies piggybacking the \textsf{project} operator can be
expressed as: $\vartheta = e.\vartheta \cup \chi$.

For a \texttt{for} clause with multiple variable bindings the rule must be applied once per binding as if there
is one FLWOR expression per binding, and the $n$th binding is a FLWOR expression in the $(n-1)$th bindings
return clause. This is in accordance with the XQuery semantics, and is one of
the rewrite rules into XQuery Core (see section \ref{sect:theory:xquery:XQcore}).

From definition \ref{def:iterVarDep} it can be seen that $\chi$ is not part of
the set of dependencies the iterator $I_{\chi}$ returns its parent. This is in
fact the only case where a variable is removed from a dependency set. Because of
this, we must be careful not to incidentally remove a $\chi$$numb$ attribute
from a relation by means of a \textsf{project} operator.

When we in this chapter write $\vartheta$ enclosed in MQL syntax it is to be
interpreted as a comma separated list of all the attributes linked to the
dependencies in the set $\vartheta$. As an example, the dependency set
$\vartheta = \left\{\mbox{\texttt{x}},\mbox{\texttt{y}},\mbox{\texttt{z}}\right\}$, is read as \textsf{xnumb, ynumb, znumb} in an MQL environment.

XQuery variable reference expressions, be it iterator, \texttt{let} or \texttt{declare} variables, are translated
to relational algebra quite simply by fetching the tree linked to the variable name in the symbol table:
\begin{equation}
\frac{}{\mbox{\texttt{\$}}\chi}\longmapsto
\mbox{\textbf{get(}}\chi\mbox{\textbf{)}}
\label{rule:trans:TD:varRef}
\end{equation}


\subsection{Iterator Dependency Tainting}
\label{sect:trans:TD:tainting}

The iterator body of an iterator with a iterator sequence with length $n$ will
have to be executed $n$ times. This can be done by e.g. evaluating the cartesian
product between the body or the sequence, as with the MarkXRemove method. To
avoid any denormalised intermediate results, an ideal solution would be to
always calculate such products after all other evaluations of the query is
done. Consider the following simple example of the query $e$:

\begin{center}
\begin{tabular}{l}
\texttt{for \$a in (1, 2) return} \\ \qquad
\texttt{for \$b in (3, 4) return} \\ \qquad \qquad
\texttt{5 + 6}
\end{tabular}
\end{center}

For this query the result can be calculated like this:
\noindent
\begin{center}
\textbf{r(}$e$\textbf{)}$=$\textbf{r(}\texttt{(1, 2)}\textbf{)}$\times$\textbf{r(}\texttt{(3,
4)}\textbf{)}$\times$\textbf{r(}\texttt{5 + 6}\textbf{)}.
\end{center}
\noindent

But such a simple solution is not adequate if there is a reference to an iterator variable somewhere within the
iterator body. This was managed by MarkXRemove by implementing inheritance of iterator dependencies, similar to
the concept discussed in section \ref{sect:trans:TD:dependency}, and replacing the cartesian product operator with
something like a natural join operator (section \ref{sect:trans:mxr:basics}).

MarkXRemove has shortcomings when it comes to evaluating expressions where a
sequence constructed with at least one iterator dependent expression is a subexpression. Tainting Dependencies mend for this by requiring that
all items constituting the result of an iterator dependent expression are
iterator dependent. To meet this requirement, dependency tainting is
introduced.

\noindent
\begin{myDefinition}
Iterator dependency \textbf{Tainting} is to impose a representation of one expression for each iteration of the
iterators another expression is dependent on.
\end{myDefinition}

During sequence construction, expressions explicitly taint all other expressions part of the construction with
their dependencies. Consider this subexpression:
\begin{center}
\begin{tabular}{l}
\quad \;\, $\vdots$  \\
\texttt{(}$e_{1}$\texttt{, }$e_{2}$\texttt{)}\\
\quad \;\, $\vdots$  
\end{tabular}
\end{center}
Where $e_{1}.\vartheta = \left\{\chi_{1}\right\}$ and $e_{2}.\vartheta = \emptyset$. Here $e_{2}$ will be tainted
by $e_{1}$'s dependency on $\chi_{1}$, but as $e_{2}$ have no dependencies, $e_{1}$ will not be tainted. The
tainting process is carried out by calculating the cartesian product of $e_{2}$ and the $\chi_{1}$$numb$ column of
\texttt{\$}$\chi_{1}$ stored in the symbol table.

More generally, for a sequence constructing expression $e$, \texttt{(}$e_{1}$\texttt{, \ldots, }$e_{n}$\texttt{)},
tainting of an subexpression $e_{i}$ can be expressed like this: 
\begin{center}
\begin{equation}
\begin{array}{l}
e.\vartheta = e_{1}.\vartheta \cup \ldots \cup e_{n}.\vartheta = \left\{\chi_{1},\ldots\chi_{m}\right\} \\
i \in \left\{1,\ldots,n\right\} \\
\mbox{\textbf{t(r(}}e_{i}\mbox{\textbf{),}}e.\vartheta\mbox{\textbf{)}} = 
\mbox{\textbf{r(}}e_{i}\mbox{\textbf{)}} \times {\displaystyle \prod_{\chi_{j} \in (e.\vartheta -
e_{i}.\vartheta)}} \mbox{\textsf{project(}}\chi_{j}numb\mbox{\textsf{;
}\textbf{get(}}\chi_{j}\mbox{\textbf{)}\textsf{)}}
\end{array}
\label{eq:trans:TD:taint}
\end{equation}
\end{center}

\subsection{Unique Iterations}
\label{sect:trans:TD:implic}
Consider an XQuery expression consisting of nested iterators $I_{\chi_1},\ldots,I_{\chi_n}$, where $I_{\chi_j}$
($1<j<n$) occurs within the iterator body of $I_{\chi_{j-1}}$. As per XQuery semantics, the iterator body of a
iterator $I_{\chi_j}$ is evaluated once for each of the items in the iterator sequence of $I_{\chi_j}$. And because
of the nesting, $I_{\chi_j}$ will have to be evaluated once per item in the iterator sequence of $I_{\chi_{j-1}}$.
The consequence of this is that the number of unique iterations the body of $I_{\chi_j}$ is actually evaluated can
be expressed like this: 
\begin{equation*}
\mbox{\textit{unique iterations evaluated for body of}}I_{\chi_j}=\displaystyle \prod_{i=1}^{j}card(I_{\chi_i})
\end{equation*}  
Where $card(I_{\chi})$ is a function returning the cardinality of the iterator sequence of $I_{\chi}$.

Of these nested iterators, a subexpression $e$ is dependent on the subset
$\left\{I_{\chi_k},I_{\chi_l}\right\}$. Because of dependency tainting and inheritance, the relational
depiction of $e$ will have a representation in all possible iteration
combinations of $I_{\chi_k}$ and $I_{\chi_l}$. A tuple in $e$, ($index, \chi_k{numb},\chi_l{numb}, value$), represents one of these unique
iterations. When $I_{\chi_k}$ is in its $\chi_k{numb}$th iteration and $I_{\chi_l}$ is in its $\chi_l{numb}$th
iteration, the item in position $index$ of the sequence returned from $e$ will be $value$. Let $I_{\chi_m}$ also
be one of the nested iterators, but one which $e$ is not dependent on. $e$ will evaluate to the same result
regardless of which iteration $I_{\chi_m}$ is in given the iteration number of
$I_{\chi_k}$ and $I_{\chi_l}$ is the same.

When an subexpression such as $e$ is used in further evaluation, it is important to seperate these iterations from
each other. This is done by grouping the relation on all unique combinations of its iterator dependency attributes.
Grouping can be done either by the \textsf{group} operator or by specifying the attributes to group by in the
partition list of the \textsf{numberate} operator.

Often the evaluation of an expression use the values of each of its
subexpressions. E.g. an addition expression is evaluated by adding the value of its first subexpression with the value of the second. To be able to calculate
such expressions, the values of the subexpressions will have to be in the same relation. This can be achieved by
evaluating the cartesian product of the subexpressions. Assumed that the subexpressions are independent of
iterators or are not dependent on the same iterators this is sufficient. But if they are depentent on one or more
iterators in common, the result of the cartesian product will have to be synchronised on the common iterators
iterations. This allows evaluation in each unique iteration, and is solved by turning the cartesian product into
an equi-join.

Generally, for such an expression $e$, with the subexpressions $e_1$ and $e_2$ this can be written
like this:
\begin{equation*}
\mbox{\textbf{r(}}e\mbox{\textbf{)}}=
\begin{array}{l}
\mbox{\texttt{\ldots}} \\
\mbox{\textsf{hhjoin([}}(e_1.\vartheta\cap e_2.\vartheta)\mbox{\textsf{], [}}(e_2.\vartheta\cap e_1.\vartheta)
\mbox{\textsf{]\ldots}} \\ \quad
\mbox{\textbf{r(}}e_1\mbox{\textbf{)}} \\ \quad
\mbox{\textbf{r(}}e_2\mbox{\textbf{)}\textsf{)}}
\end{array}
\end{equation*}

The dependencies $e.\vartheta$ is described by equation \ref{eq:trans:TD:depInheritance}. If $e_{1}.\vartheta \neq
e_{2}.\vartheta$ each subexpression will be implicitly tainted by the other's unique dependencies.

How the effective boolean function \textbf{B(r(}$e$\textbf{))} works will be discussed in section
\ref{sect:disc:effBool}. In this chapter is sufficient to consider it as a grouping operator, grouping on the
attributes specified by $e.\vartheta$ (i.e. all known unique iterations). For each group it will produce a field
$pred$ representing the effective boolean value of $e$ in that unique iteration. If $e$ holds a singleton numeric
value in one group $pred$ will hold this value, in all other cases it will hold a boolean value. The result
relation of the function will in addition to the $pred$-attribute contain all the attributes implied by
$e.\vartheta$. The main reason this function is introduced at all is that it ensures that a incoming relation
will have \emph{exactly one} tuple per unique iteration.

\subsection{Literals}
\label{sect:trans:TD:literal}

The XQuery Full Text specification\cite{w3c01} defines a number of literals as seen in figure
\ref{fig:trans:TD:litEBNF}. A \texttt{StringLiteral} is a text string enclosed in apostrophes or quotation marks,
and the numeric literals are similar to numeric types from other programming languages. 

\begin{figure}[h]
\begin{Verbatim}
[85] Literal        ::= NumericLiteral|StringLiteral
[86] NumericLiteral ::= IntegerLiteral|DecimalLiteral|DoubleLiteral
\end{Verbatim}
\caption[Literals in XQuery]{Definition of literals in XQuery Full Text}
\label{fig:trans:TD:litEBNF}
\end{figure}

To be able to include such expressions in evaluation of relational algebra, they need a relational representation.
As we in this chapter assume $value$ is a polymorphic type attribute, with the help of the \textsf{make} operator
translation of literals is done in the following way:

\begin{equation}
\frac{e \in \left\{Literals\right\}}{e}\longmapsto
\mbox{\textsf{make(name:=["index","value"], [1, }}e\mbox{\textsf{])}}
\label{rule:trans:TD:literal}
\end{equation}

This is a general way to translate literals, but there exists quite a few simplifications. Most importantly when
constructing sequences entirely composed of literals, as we will discuss in section
\ref{sect:trans:TD:simplifications}.

