
\section{Inference Rule Language Explanation}
\label{sect:trans:TD:langExpl}
\underline{\textbf{\LARGE TODO:}} Skal jeg bruke greske bokstaver i stedet for tekstlig marsgreie n\aa r jeg
skriver regler?

During this chapter we will present some inference rules. In this section we will explain the various
typographical representations.

\begin{table}[h]
\centering
\begin{tabular}{l|l}

  $\longmapsto$  			& Translates into \\
  $\vartheta$ 				& A set of iterator dependencies \\
  \textsf{sans serif} 		& MQL expressions \\
  \texttt{monospaced} 		& XQuery expressions \\
  $e,\ldots,e_{n}$			& Generic expressions \\
  $\chi,\ldots,\chi_{n}$	& Generic variable names \\
  $I_{\chi}$				& The iterator expression which declares $\chi$ \\
  \textbf{bold} 			& Operations done during the generation of the algebra \\
  \textbf{r(}$e$\textbf{)} 	& Returns the relational algebraic representation of $e$   \\
  \textbf{t(}\textbf{r(}$e$\textbf{)}\textbf{,}$\vartheta'$\textbf{)} & Returns \textbf{r(}$e$\textbf{)} tainted
  with the dependencies $\vartheta'$ \\
  $\Delta$ 					&  The default context \\ 
  $\Lambda$ 				& The logical/boolean context \\
  
\end{tabular}
\caption{Explanation of inference rule symbols}
\label{tab:trans:td:langExpl}
\end{table}

Inference rules are generally in this format:
\begin{equation*}
\frac{\mbox{\textbf{cxt}}\vdash cond}{e}\longmapsto \mbox{\textbf{r(}}e\mbox{\textbf{)}}
\end{equation*}

This should be read as: in the context of \textbf{cxt}, if condition $cond$ is satisfied, the XQuery expression
$e$ will be translated into \textbf{r(}$e$\textbf{)}.

The translator is in the logical context, $\Lambda$, if the AST node it is currently visiting is a successor of a
boolean operator or within the condition part of an \texttt{if..then..else} expression. In all other cases the
translator is in the default context, $\Delta$. If no context is mentioned in the inference rules the default
context is assumed.

Often MQL operator trees are depicted like this:
\begin{center}
\begin{tabular}{l}
\textsf{operator1(\ldots,l.field, r.field\ldots; } \\ \quad
\textsf{operator2(\ldots);} \\ \quad
\textsf{operator3(\ldots))}
\end{tabular}
\end{center}

This is to be interpreted as \textsf{operator2} is the left child of \textsf{operator1} and \textsf{operator3} is
the right child. Mars does not allow attribute names to contain punctuation marks or allow two attributes with the
same name within one relation. An operator combining two relations will therefore have renaming functionality. A
typical projection list of such an operator combining two relations which both contain the attribute $field$ would
look something like: \textsf{\ldots rfield=right.field, lfield=left.field\ldots}. To make the inference rules
easier to read, this step has been dropped. The rules assume that the equal named fields will automaticly be given
a prefix \textsf{l.} (left) or \textsf{r.} (right) depending on which child the attribute stems from.