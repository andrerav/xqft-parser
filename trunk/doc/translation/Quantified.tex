\section{Quantified Expressions}
\label{sect:disc:not:quantified}

Figure \ref{fig:disc:not:quantified} shows the EBNF specification of the XQuery quantified expression. These
expressions support existential and universal quantification, and will always result in a single $true$ or $false$
value.

\begin{figure}[h]
\begin{Verbatim}
[42] QuantifiedExpr ::= ("some" | "every") "\$" VarName "in" ExprSingle 
                        ("," "\$" VarName "in" ExprSingle)* "satisfies" ExprSingle
\end{Verbatim}
\caption{W3C specification of quantified expressions \cite{w3c00} \label{fig:disc:not:quantified}}
\end{figure}

If the quantifier is \texttt{some}, the expression only returns $true$ if at least one evaluation of the
\texttt{satisfies}-expression yields $true$. For the \texttt{every} quantifier, the expression will only return
$true$ if every evaluation of the \texttt{satisfies}-expression yields $true$.

It can be suitable to treat such expressions as iterators with the \texttt{satisfies}-expression as the iterator
body. The result of the iterator is then checked if every or some of its items have the effective boolean value
$true$. This can be done with a \textsf{group} operator similarily to the general comparison operator expressions.
If we assume $true$ has a bigger value than $false$, a \textsf{max} aggregator will reveal if there exists a tuple
with the value $true$ and a \textsf{min} operator will reveal if there exist a tuple with the value $false$. All
variables bound in quantified expressions will be treated by the iterator binding translation of rule
\ref{rule:trans:TD:forbind}. $\beta$ will refer to the set of all variables bound in one such expression

\begin{equation}
\frac{}{\mbox{\texttt{QUANT \$\ldots  satisfies }}e_1} \longmapsto
\begin{array}{l}
\mbox{\textsf{project(index = 1, value, }}\vartheta\mbox{\textsf{;}} \\ \quad
\mbox{\textsf{group((}}\vartheta\mbox{\textsf{), AGG(value);}} \\ \quad\quad
\mbox{\textsf{project(xqBoolean(value), }}\vartheta\mbox{\textsf{;}} \\ \quad\quad\quad
\mbox{\textbf{B(r(}}e_1\mbox{\textbf{))}\textsf{)))}}
\end{array}
\label{rule:trans:TD:quantified}
\end{equation}

Where $\vartheta=e_1.\vartheta-\beta$, and a quantifier specification \texttt{QUANT} maps to an aggregator
function as seen in Table \ref{tab:disc:quant}. The \textsf{xqBoolean()} funcion will have to be run on the
$value$ fields of the \texttt{satisfies} expression, as there is no requirement that it is a boolean expression.

\begin{table}[h]
\centering
\begin{tabular}{c|c} 
\texttt{QUANT} & \textsf{AGG} \\ \hline
\texttt{some} & \textsf{max} \\
\texttt{every} & \textsf{min}
\end{tabular}
\caption{Mapping XQuery quantifiers to MQL aggregators \label{tab:disc:quant}}
\end{table}

\begin{myExample}
Consider the XQuery query of Figure \ref{fig:trans:TD:quantQu}. In this query the iterator body of the FLWOR is a
quantified expression.
\begin{figure}[h]
\begin{equation*}
\begin{array}{l}
\mbox{\texttt{for \$a in ("a","b") return}} \\ \quad
\mbox{\texttt{every \$e in (\$a, "b") satisfies \$e eq "b"}}
\end{array}
\end{equation*}
\caption{Example query with quantified expression \label{fig:trans:TD:quantQu}}
\end{figure}

The sequence sequentially bound to the quantifier variable is treated as explained in section
\ref{sect:trans:TD:seqBuild}: where \texttt{\$a} taints \texttt{"b"} and the relations are spliced together. As
quantifier variables are to be handled as iterator variables, the $index$ attribute of this relation will be
renamed $enumb$, as illustrated in Figure \ref{fig:trans:TD:quant:seq}. This relation (or rather the algebra
evaluating to it) is stored in the symbol table, and is fetched during the evaluation of the
\texttt{satisfies}-expression. The result of this comparison expression is shown in figure
\ref{fig:trans:TD:quant:satEx}. Here, the double line illustrates grouping.

\begin{figure}[h]
\centering
\subfigure[]{
\begin{tabular}{|c|c|c|c|} \hline
$idx$ & $anb$ & $enb$ & $val$ \\ \hline
1 & 1 & 1 & \texttt{"a"} \\ \hline
1 & 1 & 2 & \texttt{"b"} \\ \hline
1 & 2 & 1 & \texttt{"b"} \\ \hline
1 & 2 & 2 & \texttt{"b"} \\ \hline
\end{tabular}
\label{fig:trans:TD:quant:seq}
}
\qquad
\subfigure[]{
\begin{tabular}{|c|c|c|c|} \hline
$idx$ & $anb$ & $enb$ & $val$ \\ \hline
1 & 1 & 1 & $false$ \\ \hline
1 & 1 & 2 & $true$ \\ \hline\hline
1 & 2 & 1 & $true$ \\ \hline
1 & 2 & 2 & $true$ \\ \hline
\end{tabular}
\label{fig:trans:TD:quant:satEx}
}
\qquad
\subfigure[]{
\begin{tabular}{|c|c|c|} \hline
$idx$ & $anb$ & $val$ \\ \hline
1 & 1 & $false$ \\ \hline
1 & 2 & $true$ \\ \hline
\end{tabular}
\label{fig:trans:TD:quant:quant}
}

\caption[Evaluating the quantified expression]{Evaluating the quantified expression of figure
\ref{fig:trans:TD:quantQu}. (a) The quantifier variable fetched from the symbol table. (b) The result of the
\texttt{satisfies}-expression. The double line illustrates the groups. (c) The result of the quantified
expression. Attribute names are shortened. \label{fig:trans:TD:quantInter} }
\end{figure}

Except for the quantifier variable, this expression is dependent on $I_{\mbox{\texttt{a}}}$, which the result of
the \texttt{satisfies}-expression is grouped on. Each group will be run through the \texttt{min} aggregator,
because the quantifier is \texttt{every}. As $false$ has a lower value than $true$, the aggregator will reveal if
the group contains a $false$. The result of the grouping is shown in Figure \ref{fig:trans:TD:quant:quant}. This
result will have to be renumbered to finalise the evaluation of the query.
\end{myExample}