\section{Simplifications}
\label{sect:trans:TD:simplifications}
\begin{itemize}
  \item sekvenser:
	  \begin{itemize}
	    \item suprIndx kan doppes n\aa r bare singelton
	    \item singeltons trenger ikke index
      \end{itemize} 
  \item Litterals: 
		\begin{itemize}
          \item implies that any sequence of
				explicitly stated items will first have to be made as one-tuple relations and
				then spliced together. As the \textsf{make} operator supports multiple items, a
					better solution would be to collect all items in one MQL operator. Further, if
					the item is part of a singleton sequence, there is no need for a representation
					in form of an relation, as the item could be made part of the parameters of an
					operator (e.g. $\alpha$ \texttt{> 1} $\Longrightarrow$
					\textsf{select(gt(value, 1) $\alpha$)}).   
        \end{itemize}
  \item FLWOR:
  	\begin{itemize}
        \item  kan bli -> project , ved singelton return
      \end{itemize}   
  \item Arithmetic /comp / logical
	\begin{itemize}
      \item multiple +- kan bli dyttet sammen\ldots oddetall - = -, partal = drit i det.
      \item singleton inn i and/or -> slippe select
    \end{itemize}
  \item             
\end{itemize}