\section{Simple Binary Expressions}
\underline{\textbf{\LARGE TODO:}} innledning

\subsection{Arithmetic Expressions}
\label{sect:trans:TD:atrith}
W3C defines the arithmetic expressions as shown in figure \ref{fig:trans:TD:arithEBNF}\cite{w3c00}. Notice how the
specified grammar handles operator precedence. \texttt{UnaryExpr} is a decendant production of \texttt{UnionExpr}.

\begin{figure}[h]
\begin{Verbatim}
[50] AdditiveExpr       ::= MultiplicativeExpr ( ("+" | "-") 
                                MultiplicativeExpr )*
[51] MultiplicativeExpr ::= UnionExpr 
                                ( ("*" | "div" | "idiv" | "mod")
                                UnionExpr )*
[58] UnaryExpr          ::= ("-" | "+")* ValueExpr
\end{Verbatim}
\caption{The arithmetic expressions of XQuery}
\label{fig:trans:TD:arithEBNF}
\end{figure}

The translation of such expressions will have to be separated in binary and unary operators. For the binary
operators the two values to be operated on will have to be in the same relation. To ensure the values to be
operated on are from the same unique iteration (if there is any iterations at all), the relations corresponding to
the two expressions will have to be joined on their common iterator dependencies, as described in section
\ref{sect:trans:TD:implic}. Both the unary and the binary XQuery arithmetic operators accept only singleton
sequences.

\begin{equation}
\frac{}{e_1 \mbox{\texttt{ OP }} e_2}\longmapsto
\begin{array}{l}
\mbox{\textsf{project(index=1,value=FUNC(l.value,r.value),}}\vartheta\mbox{\textsf{;}} \\ \quad
\mbox{\textsf{hhjoin([}}(e_1.\vartheta\cap e_2.\vartheta)\mbox{\textsf{], [}}(e_2.\vartheta\cap e_1.\vartheta)
\mbox{\textsf{],}} 
\\ \quad \quad \quad \quad\mbox{\textsf{[r.value, l.value, }}\vartheta\mbox{\textsf{];}} \\ \quad \quad
\mbox{\textbf{r(}}e_1\mbox{\textbf{)}} \\ \quad \quad
\mbox{\textbf{r(}}e_2\mbox{\textbf{)}\textsf{))}}
\end{array}
\label{rule:trans:TD:arithmetic}
\end{equation}

Where $\vartheta = e_1.\vartheta \cup e_2.\vartheta$, \texttt{OP} will map to a MQL function replacing
\textsf{FUNC} as shown in table \ref{tab:trans:TD:binOpMap}. The projecting functionality of the
\textsf{hhjoin} operator will in this case remove the $index$ attributes and any possible $documentId$, $scope$
and $pos$ attributes.

\begin{table}[h]
\centering
\begin{tabular}{c|c}
\texttt{OP} & \textsf{FUNC} \\ \hline
\texttt{+} 	& \textsf{sum} \\
\texttt{-} 	& \textsf{subtract} \\
\texttt{*} 	& \textsf{prod} \\
\texttt{div} 	& \textsf{div} \\
\texttt{idiv} 	& \textsf{idiv} \\
\texttt{mod} 	& \textsf{mod} \\
\end{tabular}
\label{tab:trans:TD:binOpMap}
\caption{Mapping XQuery arithmetic operators to MQL functions.}
\end{table}

Considering the unary operators, the \texttt{+} operator will never have any effect, and can therefore be dropped.
The unary \texttt{-} operator will change the sign of the value it is assigned to. This is equal to multiplying
the value with $-1$:
\begin{equation}
\frac{}{\mbox{\texttt{-}}e_1}\longmapsto 
\begin{array}{l}
\mbox{\textsf{project(value = prod(-1, value);}} \\ \quad
\mbox{\textbf{r(}}e_1\mbox{\textbf{)}\textsf{)}}
\end{array}
\label{rule:trans:TD:unaryMin}
\end{equation}

\begin{myExample}
Consider the XQuery query of figure \ref{fig:trans:TD:arithQuery}. Here $e_1$ is an arithmetic expression.

\begin{figure}[h]
\centering
\begin{equation*}
\begin{array}{l}
\mbox{\texttt{for \$a in (1,2) return}} \\ \quad
\mbox{\texttt{for \$b in (3,4) return}} \\ \quad \quad
\underbrace{\mbox{\texttt{\$a + \$b}}}_{e_1}
\end{array}
\end{equation*}
\caption{Example query containting a arithmetic expression.}
\label{fig:trans:TD:arithQuery}
\end{figure}

Both \texttt{\$a} and \texttt{\$b} is translated to simple two-tuple relations by rules
\ref{rule:trans:TD:forbind} and \ref{rule:trans:TD:varRef}. As the two expressions do not have any iterator
dependencies the \texttt{hhjoin} operator of rule \ref{rule:trans:TD:arithmetic} will be treated as a cartesian
product (ref. \ref{sect:trans:TD:implic}). The cross product of the two relations are shown in figure
\ref{fig:trans:TD:aritJoin}. Lastly, the \texttt{project} operator is employed to calculate the sum for each
iteration, the result of which is shown in figure \ref{fig:trans:TD:aritEnd}.

\begin{figure}[h]
\centering
\subfigure[]{
\begin{tabular}{|c|c|c|c|}\hline
$anb$ & $bnb$ & $l.val$ & $r.val$ \\ \hline
1 & 1 & 1 & 3 \\ \hline
2 & 1 & 2 & 3 \\ \hline
1 & 2 & 1 & 4 \\ \hline
2 & 2 & 2 & 4 \\ \hline
\end{tabular}
\label{fig:trans:TD:aritJoin}
}
\qquad
\subfigure[\textbf{r(}$e_1$\textbf{)}]{
\begin{tabular}{|c|c|c|c|}\hline
$idx$ & $anb$ & $bnb$ & $val$ \\ \hline
1 & 1 & 1 & 4 \\ \hline
1 & 2 & 1 & 5 \\ \hline
1 & 1 & 2 & 5 \\ \hline
1 & 2 & 2 & 6 \\ \hline
\end{tabular}
\label{fig:trans:TD:aritEnd}
}
\caption[Results of evaluating expression $e_1$ of figure \ref{fig:trans:TD:arithQuery}.]{Results of evaluating
expression $e_1$ of figure \ref{fig:trans:TD:arithQuery}. (a) shows the result of the cross product. (b) is the
fully evaluated $e_1$. Attribute names are shortened. \label{fig:trans:TD:arithRes}}
\end{figure}

\end{myExample}

\subsection{Logical Expressions}
\label{sect:trans:TD:logical}
An XQuery logical expression is either an \texttt{and} expression or an \texttt{or} expression. If a logical
expression does not raise an error, its value is always one of the boolean values $true$ or $false$.

A logical expression is translated in a matter very similar to the arithmetic expressions. The XQuery logical
operators does however operate on the effective boolean value (see section \ref{sect:theory:xquery:basics}) rather
than the direct value. Rule \ref{rule:trans:TD:arithmetic} is valid for logical expressions, and the mapping
between the XQuery operators and the MQL functions are shown in table JEJEJEJE. As the operators require effective
boolean values, the $value$ fields will have to be run through the \textsf{xqBoolean()} function. E.g. the
following will be a part of the parameters of the \textsf{project} operator of a translated \texttt{or} expression:
\textsf{\ldots value = or(xqBoolean(l.value),xqBoolean(r.value))\ldots}.

\begin{table}[h]
\centering
\begin{tabular}{c|c}
\texttt{OP} & \textsf{FUNC} \\ \hline
\texttt{or} & \textsf{or} \\
\texttt{and} & \textsf{and} \\
\end{tabular}
\label{tab:trans:TD:logMap}
\caption{Mapping XQuery boolean operators to MQL functions}
\end{table}

For logical expressions, $e_1$\texttt{ logOp }$e_2$, $e_1$ and $e_2$ and their possible subexpressions will have
to be translated in the logical context $\Lambda$.

\subsection{Comparative Expressions}
\label{sect:trans:TD:compArit}
Comparison expressions allow two values to be compared. XQuery provides three kinds of comparison expressions,
called value comparisons, general comparisons, and node comparisons. The comparison operators as specified by W3C
are shown in figure \ref{fig:trans:TD:compEBNF}. The Tainting Dependency method does at this time not support node
comparison, as will be discussed in section \ref{sect:disc:notSupported}.

\begin{figure}[h]
\begin{Verbatim}
[61] ValueComp   ::= "eq" | "ne" | "lt" | "le" | "gt" | "ge"
[60] GeneralComp ::= "=" | "!=" | "<" | "<=" | ">" | ">="
[62] NodeComp    ::= "is" | "<<" | ">>"
\end{Verbatim}
\caption{The comparison operators of XQuery}
\label{fig:trans:TD:compEBNF}
\end{figure}

Value comparisons are used for comparing two single values. With the same premises and restrictions the rule for
translating arithmetic expressions (rule \ref{rule:trans:TD:arithmetic}) can be used to translate such comparison
expressions. The mapping between the XQuery value comparators and MQL functions can be seen in table
\ref{tab:trans:TD:valueComp}.

\begin{table}[h]
\centering
\begin{tabular}{c|c}
\texttt{OP} & \textsf{FUNC} \\ \hline
\texttt{eq} & \textsf{eq} \\
\texttt{ne} & \textsf{neq} \\
\texttt{lt} & \textsf{lt} \\
\texttt{le} & \textsf{leq} \\
\texttt{gt} & \textsf{gt} \\
\texttt{ge} & \textsf{geq} \\
\end{tabular}
\label{tab:trans:TD:valueComp}
\caption{Mapping XQuery value comparators to MQL functions}
\end{table}

General comparisons are existentially quantified comparisons that may be applied to sequences of any length. If
employing a general comparator on any pair of items consisting of one from each of the sequences yields $true$,
the comparison expression yields true. As an example, all the comparison expressions of figure
\ref{fig:trans:TD:genCompEx} evaluates to $true$.

\begin{figure}[h]
\centering
\begin{tabular}{l}
\texttt{(1, 2) = (2, 3)} \\
\texttt{(1, 2) != (2, 3)} \\
\texttt{(1, 200) < (10, 20)} \\
\texttt{(1, 200) > (10, 20)} \\
\end{tabular}
\label{fig:trans:TD:genCompEx}
\caption[Example general comparisons]{Example general comparisons: all expressions evaluate to $true$}
\end{figure}

The big difference between general and value comparisons is that the first must accomodate for sequences. This is
solved by grouping expressions' iterator dependencies, meaning that each group will contain the sequence of one
unique iteration. By defining $true$ as having a larger value than $false$, the \textsf{max()} aggregator will
identify the groups with \emph{at least} one $true$ value.

As with the arithmetic expressions, the relational representation of the two operands is joined on their common
dependencies to ensure that both values are from the same iteration.

\begin{equation}
\frac{}{e_1\mbox{\texttt{ OP }}e_2}\longmapsto
\begin{array}{l}
\mbox{\textsf{project(index = 1, value=max, }}\vartheta \textsf{;} \\ \quad
\mbox{\textsf{group((}}\vartheta\mbox{\textsf{), max(value);}} \\ \quad \quad
\mbox{\textsf{project(value=FUNC(l.value,r.value),}}\vartheta\mbox{\textsf{;}} \\ \quad \quad \quad
\mbox{\textsf{hhjoin([}}(e_1.\vartheta\cap e_2.\vartheta)\mbox{\textsf{], [}}(e_2.\vartheta\cap e_1.\vartheta)
\mbox{\textsf{],}} 
\\ \quad \quad \quad \quad \quad \quad\mbox{\textsf{[l.value, r.value, }}\vartheta\mbox{\textsf{];}} \\ \quad \quad
\quad \quad \quad \mbox{\textbf{r(}}e_1\mbox{\textbf{)}} \\ \quad \quad \quad \quad \quad
\mbox{\textbf{r(}}e_2\mbox{\textbf{)}\textsf{))))}}
\end{array}
\label{rule:trans:TD:comparative}
\end{equation}

Where $\vartheta = e_1.\vartheta \cup e_2.\vartheta$. \texttt{OP} wil map to a MQL function replacing
\textsf{FUNC} as shown in table \ref{tab:trans:TD:genCompMap}. 

The result of a general comparison is always a singleton sequence, thus it is safe to project in an $index$
attribute with the value $1$. The $index$ and possible $documentId$, $scope$ and $pos$ attributes are left out of
the project list of the \textsf{hhjoin} operator.

\begin{table}[h]
\centering
\begin{tabular}{c|c}
\texttt{OP} & \textsf{FUNC} \\ \hline
\texttt{=} & \textsf{eq} \\
\texttt{!=} & \textsf{neq} \\
\texttt{<} & \textsf{lt} \\
\texttt{<=} & \textsf{leq} \\
\texttt{>} & \textsf{gt} \\
\texttt{>=} & \textsf{geq} \\
\end{tabular}
\label{tab:trans:TD:genCompMap}
\caption{Mapping XQuery general comparators to MQL functions.}
\end{table}

\begin{myExample}
Figure \ref{fig:trans:TD:genCompQu} shows a simple XQuery query with a general comparison expression, $e_1$.
\begin{figure}[h]
\centering
\begin{equation*}
\begin{array}{l}
\mbox{\texttt{for \$a in (10, 20) return}} \\ \quad
\underbrace{\mbox{\texttt{\$a > (5, 15)}}}_{e_1}
\end{array}
\end{equation*}
\caption{Example query with a general comparison expression \label{fig:trans:TD:genCompQu}}
\end{figure}
 
Because the operands of the \texttt{>} operator have no common iterator dependencies, the cartesian product of the
two relations are calculated, as seen in figure \ref{fig:trans:TD:gen:join}. After the inner \textsf{project}
operator is employed, the result will be as in figure \ref{fig:trans:TD:gen:projGr}. The double line illustrates
the grouping on $anumb$ ($anb$ in the figure). The maximum value of $value$ for each group is calculated and the
attributes are renamed, which gives the relation of figure \ref{fig:trans:TD:gen:end}.

\begin{figure}[h]
\centering
\subfigure[]{
\begin{tabular}{|c|c|c|} \hline
$anb$ & $l.val$ & $r.val$ \\ \hline
1 & 10 & 5 \\ \hline
1 & 10 & 15 \\ \hline
2 & 20 & 5 \\ \hline
2 & 20 & 15 \\ \hline
\end{tabular}
\label{fig:trans:TD:gen:join}
}
\qquad
\subfigure[]{
\begin{tabular}{|c|c|} \hline
$anb$ & $val$ \\ \hline
1 & $true$ \\ \hline
1 & $false$ \\ \hline \hline
2 & $true$ \\ \hline
2 & $true$ \\ \hline
\end{tabular}
\label{fig:trans:TD:gen:projGr}
}
\qquad
\subfigure[\textbf{r(}$e_1$\textbf{)}]{
\begin{tabular}{|c|c|c|} \hline
$idx$ & $anb$ & $val$ \\ \hline
1 & 1 & $true$ \\ \hline
1 & 2 & $true$ \\ \hline
\end{tabular}
\label{fig:trans:TD:gen:end}
}
\caption[Results of evaluating $e_1$ in figure \ref{fig:trans:TD:genCompQu}]{Results of evaluating expression $e_1$
in figure \ref{fig:trans:TD:genCompQu}. (a) The relations of the operands joined. (b) Each combination of the
sequences evaluated. Double line illustrates grouping. (c) The final result. Attribute names are shortened.
\label{fig:trans:TD:genCompRes}}
\end{figure}
 
\end{myExample}
