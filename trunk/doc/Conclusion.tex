\chapter{Conclusion}
\label{chapter:conclusion}

In this project we have developed a front-end parser for XQuery with support
for full-text extensions. The parser produces a usable AST that can be easily
modified, and is suitable for further usage.  

\begin{itemize}
  \item A fairly complete XQuery full-text parser front end capable of
  generating abstract syntax trees and with a test coverage of
  99.3\%\footnote{As noted and emphasized in section
  \ref{sect:discussion:coverageResults},
  these results are not immediately comparable with officially published results
  from the W3C}
  \item Research on the current state of XQuery implementations (with and
  without full-text extensions) as well as XQuery as a language, and research and
  evaluation of parser generators suitable for a project based on our requirements
  \item Documentation of features of interest as well as limitations in the
  ANTLR parser generator. Addditionally, research and evaluation of
  strategies for the problems encountered
  \item A suitable XQuery full-text grammar specification for ANTLR based on the
  W3C recommendation
  \item Investigation and documentation as well as resolve for several unexpected
  problems and ambiguities in the grammar
  punktene over?
  \item Grammar rewrite rules and operators to generate abstract
  syntax trees for XQuery full-text queries, as well as evaluation of their
  usefulness
  \item A prototype for scoping, symbol tables, and lookups in an XQuery
  full-text parser
  \item Improved default ANTLR error handling by giving the responsibility
  of error handling to the calling program
  \item A test harness for executing detailed coverage tests based on the
  official XQuery test suite
\end{itemize}


\underline{\textbf{\LARGE //TODO:}}
s\aa~konklusjon


hva ble gjort for \aa ~f\aa~ opp dekningskrav hovedgrunner, litt fremtid, litt
av alt egentlig? 

Vi maa si noe om hva vi synes selv ogsaa, om vi er fornoyde osv. Vi maa ha en
definitiv konklusjon, f.eks ``Vi har utviklet en parser av grei kvalitet som vi
er fornoyd med'' 
-AR

\underline{\textbf{\LARGE //ODOT:}}