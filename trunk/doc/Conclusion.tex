\chapter{Conclusion}
\label{chapter:conclusion}

In the course of this project we have investigated the current state of XQuery implementations (with and without full-text extensions) as well as XQuery as a language. In addition we have examinated and evaluated multiple parser generators suitable for a project based on our requirements, and finally chose ANTLR. Features of interest as well as limitations of this parser generator were also documented.
The reason for choosing ANTLR was firstly because it utilizes a grammar specification syntactically close to the EBNF W3C makes use of. In addition, the parser generator supports various types of predicates and automatic abstract syntax tree generation -- features really convenient when creating more than a non-trivial recognizer. These as well as other features of interest and limitations in ANTLR were documented in this report.

Rewriting the XQuery full-text specification to conform to ANTLR grammar syntax and semantics proved to be a non-trivial task. This is because the specification contains ambiguous terminals. An LL lexer capable of handling these must be aware of the context the incomming characters. We evaluated a number of different approaches to this. 

After implementing a prototype state driven lexer for a subset of XQuery, we discovered that only a few states would be needed to solve the non-determinism. Additionally we found that by making the parser communicate the state to the lexer the implementation would be much less complex. To make this work we had to implement functionality in the lexer to generate tokens one by one, overriding the default ANTLR generated lexer manner of operation. A great deal of the terminal ambiuities were solved by this parser controlled state driven strategy. The reminder were solved by converting some non-terminals to enclosed composite terminals.

Because of the chosen lexer strategy, a goal was to reduce required parser lookahead. A low lookahead combined with XQuery's property of being reserved keyword free created a number of non-determinsisms. These were solved by left factoring and augmenting the grammar with predicates guiding the parser.

Using grammar rewrite rules and rewrite operators, the parser produces an AST well suited for further data flow analysis and transformation. The AST can be structured at will simply by changing the rewrite rules and regenerating and recompiling the parser.

A simplistic system for scoping and symbol tables was developed and implemented in the parser by adding action code to the grammar file. This system would be more useful and extendable if it was re-implemented to operate on an AST instead of inlining it in the generated parser. This would also make it possible to integrate scoping and symbol table with a future type inference system.

We implemented a test harness for executing detailed coverage tests based on the official XQuery test suite\cite{w3c05}. These tests are meant for XQuery 1.0 and not the full-text extension. We belive, however, that they are a good indicator of the degree of compliance with the W3C specificaion. The parser failed on 84 of the 12478 test queries applicable. The causes of the failed queries are accounted for, and possible solutions are presented.

As as response to the task description, this project fulfills the requirements stated. We have developed a XQuery parser in Java under a BSD license. It accommodates for the full-text extensions specified by W3C. The parser is capable of generating abstract syntax trees and has a XQuery test suite test coverage of 99.3\%\footnote{As noted and emphasized in section\ref{sect:discussion:coverageResults}, these results are not immediately comparable with officially published results from the W3C}.
