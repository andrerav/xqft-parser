%Relasjonsalgebra
\chapter{Architectual Decisions}
\label{chapter:method}
Because we decided to create the parser with a generator, different alternatives of such tools had to be evaluated. In this chapter we will document this process. Further, we will present other decisions we made before implementating the recognizer. This constitutes the choice of licence and means testing the system. Finally we will present the problem with the ambiguous terminals and evaluate different lexer strategies that would solve this.

\section{Parser construction}
Writing a parser from scratch was ruled out early for being too time consuming.
Instead it was decided to use tools for compiler and parser construction to
generate a parser from the XQuery 1.0 and XPath 2.0 grammar  
specifications\cite{w3c01} developed by the W3C.

Some important requirements were taken into account when evaluating these
alternatives. Specifically, the generator must:
\begin{itemize}
  \item generate Java source code for the parser
  \item be able to generate an LL(1) (or LL(k)) parser, because the grammar  
  specification is given by W3C in LL(k)
  \item be licensed liberally, either GPL or BSD, or an equivalent license
  approved by OSI
\end{itemize}

\subsection{Evaluated alternatives}
\label{sect:method:alternatives}
\subsubsection{JFlex/CUP}
JFlex and CUP is a versatile combination consisting of JFlex which is a lexer
generator, and CUP which is a parser generator. These tools can be interfaced to
generate a complete parser with a separate lexical analyzer.

JFlex and CUP produces only LALR parsers, and since the W3C has specified an
LL(1) grammar for XQuery 1.0 and XPath 2.0, the combination of JFlex and CUP was
rejected from this project. Additionally, some basic occurence operators such as
*,+ and ? is not supported by CUP.

In the light of this decision, it is important to note that a large amount of
work can be avoided by rejecting a LALR parser generator solution for this
project. Consider the examples in figure \ref{code:parsers:lalr} and
\ref{code:parsers:ll} and compare it to the reference grammar specification in
figure \ref{code:parsers:w3c}.

\begin{figure}[h!]
\begin{verbatim}
LetClause : "let $" LetBindings_
LetBindings_ : VarName OptTypeDeclaration_ ":=" ExprSingle
        | VarName OptTypeDeclaration_ ":=" ExprSingle
        "," "$" LetBindings_
\end{verbatim}
\caption[LALR grammar example]{LALR grammar example from the Pathfinder project
(see section \ref{sect:soa:pathfinder})}
\label{code:parsers:lalr}
\end{figure}

\begin{figure}[h!]
\begin{verbatim}
letClause : LET varBinding (COMMA varBinding)*

varBinding : (DOLLAR varName typeDeclaration? | 
             SCORE DOLLAR varName )
             ASSIGN exprSingle
\end{verbatim}
\caption[LL grammar example]{LL grammar example, actual code used in this project}
\label{code:parsers:ll}
\end{figure}

\begin{figure}[h!]
\begin{verbatim}
LetClause ::=  (("let" "$" VarName TypeDeclaration?) 
 				| ("let" "score" "$"  VarName)) ":=" ExprSingle ("," 
 				(("$" VarName TypeDeclaration?) 
 				| FTScoreVar) ":=" ExprSingle)*
\end{verbatim}
\caption{ W3C reference grammar example }
\label{code:parsers:w3c}
\end{figure}

Obviously, the LALR-grammar has suffered an extensive rewrite into a recursive
rule to compensate for the lack of support for occurence indicators. The LL
example stands out as clear and readable, even more so than the W3C
grammar specification.

In the case of JFlex/CUP, building abstract syntax trees is a process that must
be done manually by adding action code snippets to the grammar. This is
unfortunate for several reasons - it dilutes the grammar and decreases
readability and clarity.


\subsubsection{JavaCC}
JavaCC could have been a viable alternative as it produces LL(k) parsers,
however compared to Antlr its grammar specification syntax deviated more
from the W3C EBNF syntax, meaning that the grammar would have had to be
extensively rewritten. However, JavaCC supports the same occurence indicators
used in the XQuery grammar specification. Figure \ref{code:parsers:javacc} shows
a grammar fragment translated to JavaCC syntax, compare to the original grammar
in figure \ref{code:parsers:w3c}.

\begin{figure}[h!]
\begin{verbatim}
void letClause() :
{ }
{
    <LET> varBinding() (COMMA varBinding())*
} 

void varBinding()
{ }
{
    (<DOLLAR> varName() typeDeclaration()? | 
     <SCORE> <DOLLAR> varName() )
     <ASSIGN> exprSingle()
}
\end{verbatim}
\caption{ JavaCC reference grammar example }
\label{code:parsers:javacc}
\end{figure}

For AST construction, JavaCC requires the use of preprocessor tool known as  
JJTree. This tool will parse the grammar and insert tree building actions where
appropriate. 

\subsubsection{ANTLR}
ANTLR is a renowned tool for parser generation, and can generate LL(*) parsers.
Additionally, ANTLR accepts a grammar specification syntactically very close to
the EBNF used by the W3C. This is the parser generater chosen for our project,
based on the criteria outlined in this section.

The details of this parser generator is outlined in chapter \ref{sect:antlr}.

\section{Debugging}
\label{sect:method:debugging}
There are several approaches to debugging. ANTLRWorks \cite{antlrwrks00} is a
simple tool for writing, testing and debugging Antlr grammars. The debugging
interface is useful in that it draws a realtime step-by-step parse tree as the
input is being parsed, as well as displaying a list of parser events.
ANTLRWorks also helps eliminate grammar nondeterminisms by drawing the syntax
diagram associated with a grammar and highlighting nondeterministic paths. A
big drawback, though, is as the grammar grows bigger, the application tends to
become unstable.    

When working properly, ANTLRWorks is a great tool for debuging the parser, but
no so much for the lexer. It will show the tokens returned to the parser, but
no information or decision path indicating the reason for returning just that
token. A more appropriate tool for such a task would be the standard debuging
environment in Eclipse (\cite{eclipse}).    

% Method / Testing
\section{Testing}
\label{sect:method:testing}
To improve quality insurance in this projects, it was decided to systematically
write tests for new features as well as perform coverage testing for comparison
with existing implementations. This section will outline the various test
methods used in this project. Also see section \ref{sect:results:tests}, where
the test results are presented.

\subsection{gUnit}
\label{sect:method:gUnit}
Unit testing can be a powerful tool for asserting functionality and can be a
helpful aid in debugging and prevention of regression errors.  For unit testing the
grammar specification, gUnit \cite{gunit00} was employed. This tool uses a
syntax similar to Antlr itself, however instead of defining productions, one
defines a set of inputs for some rule, as well as the expected result. Consider
this example:

\begin{Verbatim}
gunit XQFT;
@header{package no.ntnu.xqft.parse;}

piTarget: // Test piTarget rule

    // Any case permutation of 'XML' must fail
    "Xml" FAIL
    "XMl" FAIL
    "XML" FAIL
    "XmL" FAIL
\end{Verbatim}

This is a complete input file for gUnit, and will automatically discover the
classes XQFTLexer and XQFTParser in the package no.ntnu.xqft.parse. gUnit will
then proceed to invoke the lexer with ``Xml'', ``XMl'', ``XML'', and ``XmL'' as
input, and pass the lexer to an instance of XQFTParser and execute the production
piTarget. For all these inputs, it will assert that the parser emits an error
(i.e it must fail to pass the test).

In case of a test where the parser should not fail, the syntax is as follows:
\begin{Verbatim}
forClause:
	"for $a in document(\"abc.xml\")/a/b/text()" OK
\end{Verbatim}
Here gUnit will assert that the parser will not fail for the given inpu (i.e it
must not fail to pass the test).

gUnit is also capable of parsing abstract syntax trees built by the generated
parser, but this feature has not been used in this project.

\subsection{jUnit}
In addition to the gUnit tests for the grammar, jUnit was used to write unit tests for
the additional classes needed for this parser. In particular, the classes
related to scoping and symbol tables. Here is one example, asserting that the
setParent() method in the scope class works as expected:
\begin{Verbatim}
    @Test
    public void testSetParent() {
        this.scope = new Scope();
        Scope tmp = new Scope();
        this.scope.setParent(tmp);
        assertEquals(this.scope.getParent(), tmp);
    }
\end{Verbatim}

\subsection{Manual Coverage Tests}
Finally, we employed manual tests for running the complete XQuery Test Suite\cite{w3c05}
as a benchmark for XQuery compliance. Specifically, we simplified this approach
into counting any test query accepted by the parser as passed. Since the parser
does not generate any results from queries, it was impossible to run a
comparison to the given expected results.

The test suite consists of a total of 12584 different queries with a
corresponding set of expected output. These tests span a large amount of XQuery
functionality. Out of these tests, a total of 10363 tests are applicable for
this project. These tests are categorized as scenarios: \emph{standard}, 
\emph{parse-error}, and \emph{trivial}. The remaining tests are only possible to
perform in run time, and are thus not applicable for this parser at this time.
\section{Licensing}
\label{sect:method:licensing}
One requirement for the parser is to license the produced software under a
permissive open source license. Several such liberal licenses exist, however here we
shall review a short selection of popular and widely adopted licenses, and
choose one for this project. These licenses have been gathered from the Open
Source Initiative (OSI) website\cite{osi_website}, from the category
``\emph{Licenses that are popular and widely used or with strong communities}''.

\subsection*{GPLv2 -- General Public License version 2}
The GPL license is a widely adopted license known for its reciprocality and
enforcement of ``copyleft''\footnote{http://www.gnu.org/copyleft/} to ensure
preservation of software freedom. This license was rejected in favor of more
permissive licenses.

\subsection*{Apache License, Version 2.0}
The Apache 2.0 license is a GPL-compatible\footnote{This is a subject of debate.
The Free Software Foundation argues\cite{fsf_licenses} that the Apache License
2.0 is incompatible  with GPL versions older than version 3 due to patent
clauses in the Apache license} software license proposed by the Apache Software
Foundation. This license is slightly more permissive than the GPL, but was
rejected in favor of a more permissive license. 

\subsection*{New BSD}
BSD-style licenses are named after Berkeley Software Distribution, a Unix-like
operating system. The ``New BSD'' license is a permissive license which
essentially puts the software in the public domain. This is the license chosen
for this project due to its permissive and uncomplicated nature.

\section{Ambiguous Terminals}
\label{sect:ambiguousgrammar:ambigTerm}
The W3C EBNF specification of XQuery 1.0 with full-text extensions\cite{w3c01} defines among other these terminals (somewhat simplified):
\begin{Verbatim}
ElementContentChar       ::= Char - [{}<&]
QuotAttrContentChar      ::= Char - ["{}<&]
AposAttrContentChar      ::= Char - ['{}<&]
IntegerLiteral           ::= Digits
NCName                   ::= NCNameStartChar NCNameChar*
NCNameChar               ::= Letter | Digit | '.' | '-' | '_'
NCNameStartChar          ::= Letter | '_'
\end{Verbatim}
Where \verb!Char! denotes all possible legal characters in XQuery, \verb!Digits! is all possible number characters and \verb!Letter! denotes all possible letter characters. In their specification, the W3C uses a dash operator, which has the following
semantic meaning in a grammar (\cite{w3c03}, section 6):
\begin{quote}
``A - B: matches any string that matches A but does not match B."
\end{quote}
In addition, the specification contains explicit defined literals and symbols
such as \verb!for! and \verb!::!, as well as \verb!StringLiteral! which is
indeed comparable with the non-terminal \verb!DirAttributeList! \footnote{worth
noting, the eXist\cite{exist_doc}~implementation has also accounted for this
ambiguity} as shown here (simplified):
\begin{Verbatim}
DirAttributeValue        ::= ('"' (QuotAttrContentChar)* '"')
                           | ("'" (AposAttrContentChar)* "'")
StringLiteral            ::= ('"' ([^"&])* '"') 
                           | ("'" ([^'&])* "'")
\end{Verbatim}
Where \verb!QuotAttrContentChar! and \verb!AposAttrContentChar! is as defined earlier, and the hat operator (\verb!^!) is defined as follows(\cite{w3c03}, section 6):
\begin{quote}
``[\^{}abc] matches any Char with a value not among the characters given."
\end{quote}
Meaning that e.g. \verb![^"&]! equals \verb!Char - ["&]!. It is easy to see that the characters that can be matched by these productions would overlap significantly. This means that the lexer must be aware of the context of a incoming symbol to differanciate between the rules. We considered several alternatives for solving this, as discussed in the following sections.

\subsection{Scan-While-Parse scanner}
\label{sect:ambiguousgrammar:scanWhileParse}
With a scan-while-parse strategy the parser would evaluate the character stream
into tokens, while at the same time branching to the respective hierarchical
context. This means essentially that there would be no lexer (it would only
return one and one character), only a large parser. Such a parser would be
overly complex, and would possibly suffer from performance degradations.

\subsection{Fuzzy Token Lexer}
A fuzzy token lexer is typically a simple lexer which leaves the meaning of the
tokens somewhat ambiguous. Such a lexer may e.g. have a production
\verb!Letters! which bundles all consecutive letter characters into tokens. The
tokens would then be refined to their appropriate type during the context
sensitive parsing process. Problems would occur, though, for example of the
parser encounters a \verb!Letters! token but is expecting a token consisting of
both letters, numbers and other symbols. The parser would then need to do the
comlicated task of splitting and merging such tokens.    

\subsection{State Driven Lexer}
\label{sect:amiguousgrammar:stateDriven}
With this strategy the lexer would only recognize sets of words when in an
appropriate state. The acceptance of a word in one state may lead to a state
transition. In other words, the lexical recognition would be implemented as
transition tables for a push-down automata. A suggestion for such transition
tables for XQuery (not full-text) are published as W3C working
draft\cite{createTokenizer} with the following disclaimer:  

\begin{quote}
``The following tables were hand-constructed and have not, at the time of this writing, been exhaustively verified against all possible paths that may be legal in the XQuery and XPath EBNFs, which is to say, it is possible they contain bugs."
\end{quote}
We made a prototype of a state driven lexer for a subset of XQuery, leading us
to the conclusion that it would be a cumbersome task to implement one for the
full XQuery full-text grammar. 

\subsection{Island Grammars}
\label{sect:amiguousgrammar:islandGrammar}
Island grammars \cite{islandGrammar} is a strategy where ambiguous parts of a
grammar can be taken out of the main section and turned into its own
subgrammar. This will lead to a case where the system will consist of more than
one lexer. The parser would then, in the power of being aware of the context,
switch between the lexers when necessary. Communication between the lexers is
imperative, as they need to know where to start scanning in the input stream.
ANTLR does not support island grammars by default, although there are plans
to implement it in future versions
\footnote{http://www.antlr.org/wiki/display/ANTLR3/island+grammar+formalization}.
It can, though, be manually implemented. We abandoned this idea because of the
more intuitive strategy of the parser controlled state driven lexer.

\subsection{Parser Controlled State Driven Lexer}
\label{sect:amiguousgrammar:parserControlled}
When working with the prototype of the state driven lexer, we discovered that
we would not need more than maybe four or five states to differentiate between
the ambiguous terminals. A possible solution would then be to transform the
transition tables in \cite{createTokenizer} to a system with a lot fewer states
aswell as adding state transitions for the tokens added in the full-text
version of XQuery. But a more comprehensible solution would be to let the
parser be responsible for the state transitions.

Depending on which state it is in, the lexer will block or promote some
productions with the help of gated semantic predicates or disambiguating
semantic predicates, respectively. We choose this strategy mainly because its
intuitive and easier to implement compared to the other alternatives. A more
thorough review of our implementation of the parser controlled state driven
lexer is found in section \ref{sect:impl:parser_controlled_state_driven_lexer}.



\section{Summary}
We have now outlined the requirements for our parser generator, evaluated
several alternatives, and chosen one of these parser generators (ANTLR) based on
these requirements. We have also briefly presented debugging
strategies using ANTLRWorks and/or integrated development environments.

We have presented methods for performing general unit/regression testing
using jUnit as well as ANTLR-specific grammar unit testing using gUnit. We also
outlined how to use the XQuery test suite to perform compliance
benchmarking and compare our results to existing implementations.

Further, a small collection of popular open source licenses were reviewed, and
one (the new BSD license) was selected.

Finally we discussed the problem of ambiguous terminals in the grammar. A number of lexer strategies to solve this were presented and evaluated. We chose the parser controlled state driven lexer due to its intuitivity and ease of implementation.

The next chapter will present the parser generator chosen for this
project, ANTLR, in more detail.