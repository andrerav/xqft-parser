%Relasjonsalgebra
\section{Method}
\subsection{Parser construction}
Writing a parser from scratch was ruled out early for being too time consuming.
Instead we decided to use tools for compiler and parser construction to generate
a parser from one or more grammar specifications.
\subsection{Evaluated alternatives}
\subsubsection{JFlex/CUP}
JFlex and CUP is a common combination consisting of JFlex which is a lexer
generator, and CUP which is a parser generator. These tools can be interfaced to
generate a complete parser with a separate lexical analyzer.

JFlex and CUP produces only LALR parsers, and since the W3C has specified an
LL(1) grammar for XQuery 1.0 and XPath 2.0, the combination of JFlex and CUP was
rejected from this project.
\subsubsection{JavaCC}
JavaCC could have been a viable alternative as it produces LL(k) parsers,
however compared to Antlr its grammar specification syntax deviated too much
from the W3C specification, so the grammar would have had to be extensively rewritten.
\subsubsection{Antlr}
Antlr is a renowned tool for parser generation, and can generate LL(k) parsers.
Additionally, Antlr accepts a grammar specification syntactically very close to
the EBNF used by the W3C. This is the parser generater chosen for our project,
based on the criteria outlined in this section.

It is important to note, however, that Antlr has limited support for unicode.
In this project this implies that our parser will not accept unicode characters
in the range above 0x10000. This will exclude the supplementary multilingual
plane (SMP) range of unicode characters, as well as the supplementary
ideographic plane (SIP) and the supplementary special-purpose plane (SSP). These
are seldomly used, but this is an important limitation nonetheless. The Antlr
developers have indicated that support for this unicode range will be added in
future versions of Antlr.
