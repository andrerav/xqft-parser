% Discussion / Coverage test results
\section{Coverage Test Results}
\label{sect:discussion:coverageResults}
As noted in section \ref{sect:results:tests}, our coverage testing ended at 
99.3\%. We can compare this result to official results published by W3C, with one
important constraint: as described in section \ref{sect:method:testing}, the
coverage tests is limited by the fact that we can only execute the set of tests
that are applicable in parse-time (and not run-time), since our parser does not
have a backend which produces output to be returned and compared to expected
output. Neither is our parser capable of detecting semantics such as
typing errors. This means that the official results from W3C include a number
of tests that have not been run on our parser.

Further, the XQuery test suite does not contain any full-text tests. So the
full-text capabilities of our parser are left untested, leaving only the minimum
XQuery features.

As such, it is crucial to keep in mind that our test results are not directly
comparable with official results. However, they give a fair indication of the
parsers abilities. 

\begin{figure}[h!]
  \begin{center}
    \begin{tabular}{ |c | c | c | c | c | c | c | c | c | c | c | c | c | c | c | }
      \hline
      Anglo-DT        & BaseX           & Berkeley DB XML   & DataDirect XQuery \\ \hline
      14630 / 100\%   & 14532 / 99.3\%  & 14566 / 99.5\%    & 14593 / 99.7\%  \\ \hline \hline
      eXist-db        & Galax           & Qexo              & Qizx \\ \hline            
      14544 / 99.4\%  & 14555 / 99.4\%  & 14535 / 99.3\%    & 14620 / 99.9\% \\ \hline \hline
      Saxon-SA        & Sedna XML       & Stylus Studio     & xbird/open \\ \hline
      14637 / 100\%   & 14459 / 98.8\%  & 14593 / 99.7\%    & 12041 / 82.3\% \\ \hline \hline
      X-Hive/DB       & xq2xsl          & XQuantum          & \\ \hline
      14589 / 99.7\%   & 14588 / 99.7\%  & 14378 / 98.2\%    & \\ 
    \hline
    \end{tabular}
  \end{center}
  \caption[Official XQuery test suite results]{Official XQuery test suite
  results compiled from the test suite web site\cite{w3ctestresults}} 
  \label{figure:table:w3c_test_results}
\end{figure}

Figure \ref{figure:table:w3c_test_results} gives a basic overview of the current
tests results published by the W3C at the current time of writing.

One interesting byproduct of the coverage tests were the utility of using it to
find and fix previously unknown bugs quickly. As such, this test suite also
works well as a regression test, since any improvement to the parser should
never degrade our coverage result -- however, it is hard to tell if tests that
previously failed would pass along with an identical amount of previously
passed tests that would fail, leaving the coverage percentage as before even
though the actual result was changed drastically. This kind of pitfall could be
avoided by keeping detailed track of which tests fail and pass, and compare new
results to old results and inspect the difference.


\begin{itemize}
\item Testresultater, bra/d\aa rlig
\item testresultater (hva var de forskjellige feilene)
\item Generelt resultater vi skrev om i forrige kapittel
\item Resultatene maa sees relativt - vi har ikke faatt testet hele settet siden
vi ikke kan kjore run-time tester, og vi returnerer heller ikke noe som kan
sammenlignes med forventet resultat
\end{itemize}

W3C Unconformities delen forklarer egentlig testresultatet.

Coverage er ikke h\o yere mest sannsynlig p\aa~grunn av vi har reserved
keywords, og noen ganger ser parseren for langt frem... skal se om jeg f\aa~r
fiksa noe av dette til helga, sp\o rs p\aa~ hvor lang vi har kommet med
rapporten.... Covrage testene er eXtremt bra til \aa~finne bugz.