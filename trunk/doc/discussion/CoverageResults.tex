% Discussion / Coverage test results
\section{Coverage Test Results}
\label{sect:discussion:coverageResults}
As noted in section \ref{sect:results:tests}, our coverage testing ended at 
99.3\%. We can compare this result to official results published by W3C, with
some  important constraints: as described in section \ref{sect:method:testing},
the coverage tests is limited by the fact that we can only execute the set of
tests that are applicable in parse-time (and not run-time), since our parser
does not have a backend which produces output to be returned and compared to
expected output. Neither is our parser capable of detecting semantics such as
typing errors. This means that the official results from W3C include a number
of tests that have not been run on our parser.

Further, the XQuery test suite does not contain any full-text tests. So the
full-text extension capabilities of our parser are left untested, leaving only the XQuery features.

As such, it is crucial to keep in mind that \emph{our test results are not directly
comparable with official results}. However, they give a fair indication of the
parsers abilities, seen from a relative point of view.

On a side note, the results from the \emph{parse-error} scenario tests
(described in  section \ref{sect:method:testing}) would be directly comparable.
However, W3C does not annotate their result summaries with associated scenarios,
so this proved to be difficult to investigate within our given time frame.

\begin{figure}[h!]
  \begin{center}
    \begin{tabular}{ |c | c | c | c | c | c | c | c | c | c | c | c | c | c | c | }
      \hline
      Anglo-DT        & BaseX           & Berkeley DB XML   & DataDirect XQuery \\ \hline
      14630 / 100\%   & 14532 / 99.3\%  & 14566 / 99.5\%    & 14593 / 99.7\%  \\ \hline \hline
      eXist-db        & Galax           & Qexo              & Qizx \\ \hline            
      14544 / 99.4\%  & 14555 / 99.4\%  & 14535 / 99.3\%    & 14620 / 99.9\% \\ \hline \hline
      Saxon-SA        & Sedna XML       & Stylus Studio     & xbird/open \\ \hline
      14637 / 100\%   & 14459 / 98.8\%  & 14593 / 99.7\%    & 12041 / 82.3\% \\ \hline \hline
      X-Hive/DB       & xq2xsl          & XQuantum          & \\ \hline
      14589 / 99.7\%   & 14588 / 99.7\%  & 14378 / 98.2\%    & \\ 
    \hline
    \end{tabular}
  \end{center}
  \caption[Official XQuery test suite results]{Official XQuery test suite
  results compiled from the test suite web site\cite{w3ctestresults}} 
  \label{figure:table:w3c_test_results}
\end{figure}

Figure \ref{figure:table:w3c_test_results} gives a basic overview of the current
tests results published by the W3C at the current time of writing.

One interesting byproduct of the coverage tests were the utility of using it to
find and fix previously unknown bugs quickly. As such, this test suite also
works well as a regression test, since any improvement to the parser should
never degrade our coverage result -- however, it is hard to tell if tests that
previously failed would pass along with an identical amount of previously
passed tests that would fail, leaving the coverage percentage as before even
though the actual result was changed drastically. This kind of pitfall could be
avoided by keeping detailed track of which tests fail and pass, and compare new
results to old results and inspect the difference.

Of the 12478 test queries run 84 failed. These failed test cases can be broken down as follows:
\begin{itemize}
\item 46 erroneous queries caught by ANTLRs default lexer error handling.
\item 12 errors allowed by the parser
\item 9 type cast errors
\item 9 white space errors.
\item 5 XML specifiaction nonconformaties were allowed.
\item 2 untimely state transitions
\item 1 UTF-16 query
\end{itemize}

The majority of the failed test are because exceptions are caught by the lexer and not forwarded, leaving the parser unaware of the error (section \ref{sect:discussion:error_handling}). The 12 errors allowed by the parser and the 9 type cast errors are not syntactic errors, as such, as they are allowed by the W3C EBNF grammar. This test also shows that our lexer is to liberal with where white space can and cannot occur. The reason the tests which allowed illigal XML syntax and the untimely state transition failed are discussed in section \ref{sect:future:knownBugs}.
