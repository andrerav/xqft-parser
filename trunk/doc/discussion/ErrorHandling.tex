% Discussion / ErrorHandling
\section{Error Handling}
\label{sect:discussion:error_handling}
In this section we will discuss the implemented error handling in the generated
lexer and parser classes, as well as our rationale for doing so. 

\subsection{Parser Error Handling}
In section \ref{sec:impl:errorhandling} we described how error handling was
implemented in the parser to make sure that all errors were handled properly by
catching their respective exceptions and throwing them on to the top-level
program. Contrast this to the default error handling, which will simply catch
any exceptions and print error messages to \verb!stdout!.

Our solution is a common approach to error handling, by pushing the
responsibility upwards until the top-level program recieves the exception and
can handle it gracefully. This would have been more difficult with the default
ANTLR behaviour, which essentially hides all errors.

Note however, that to achieve this we had to modify the grammar file and add
methods to override the default behaviour (described in detail in section
\ref{sec:impl:errorhandling}). This cluttered the grammar somewhat, which is a
fair trade-off for the improved error handling.

\subsection{Lexer Error Handling}
\label{sect:future_work:lexer_error_handling} % TODO: fikse denne referansen?
As noted in sections \ref{sect:error_handling:syntax_errors} and
\ref{sec:impl:errorhandling}, the \verb!nextToken()! method in the lexer base class
will  ``hijack'' lexical exceptions and handle them by printing an error message
to stderr. Further it attempts to recover from the error by simply consuming the
offending character and ignoring it in the following construction of tokens.

However, it is not immediately appareant, but the \verb!nextToken()! method can
not be overridden and forced to throw the exception onwards. This is due to the method
signature itself which does not allow exceptions to be thrown. 

One possible solution to this problem could be to override the
\verb!nextToken()! method and employ the observer design
pattern\cite{elements_of_oop} to allow a  simple and decoupled way of flagging
an exception to the parser. It could also be possible to add a new base lexer
class and modify the \verb!nextToken()! method accordingly - however it is
likely that this API change would break the interface to the parser base class,
which would regress into a series of modifications to the original ANTLR
codebase. 

This behaviour is unfortunate, as some lexer errors will be illusive and
impossible to detect by a top-level program. 