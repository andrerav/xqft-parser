\section{Results}
\label{sect:disc:res}
\textbf{\LARGE TODO: {{ANDREAS}}}

\subsection{Translation output}
\label{sect:disc:res:translation_output}
In sections \ref{sect:result:theoretical_algebra} and
\ref{sect:result:implementation_algebra}, a series of XQuery queries were
translated using the novel ``Tainting Dependencies'' (TD) methodology developed
and described in chapter \ref{sect:translation}. In section
\ref{sect:result:theoretical_algebra}, where hand-computed translations were
presented, a series of simplifications were applied (these simplifications were
described in section \ref{sect:trans:TD:simplifications}). However, the
prototype developed in chapter \ref{chapter:implementation} did not implement
any of these simplifications. This was an important point to keep in mind when
later comparing this algebra to that generated by Pathfinder/MonetDB, and is
discussed more thoroughly in the next section.

One characterisation of the algebra generated by TD is that nodes with more
than one parent node are typically located far towards the bottom of the algebra
tree. 

Another characteristic of the algebra is that is seems to maintain a fairly
compact form. This is partly due to the fact that the tainting process does not
affect constant subexpressions, and thus the potential size of the algebra is
reduced significantly. Compare this to loop lifting used by
Pathfinder/MonetDB, where all expressions within a loop body are loop lifted --
as explained in sections \ref{sect:trans:ll:ConstExprs},
\ref{sect:trans:ll:mappingBack}, and \ref{sect:trans:ll:OtherExpr}.

TODO: parallellisering \^ ?

\subsection{Complexity comparison}
\label{sect:disc:res:comparison}
The complexity calculation method (see section \ref{sect:method:complexity} on
page \pageref{sect:method:complexity}) defined by \O ystein Torbj\o rnsen at
FAST was used to compare complexity in the algebra generated by the prototype
implementation to that of Pathfinder/MonetDB. This comparison was
based on three queries (dubbed \emph{Trivial}, \emph{Complex}, and
\emph{Conditional}). For each of these queries, algebra was generated on both
the prototype implementation as well as Pathfinder/MonetDB. Then the described
method of calculating complexity was applied to these trees, and the result was
used to compare the prototype and Pathfinder/MonetDB.

Though an interesting comparison, this is a sparse source of data -- it
is difficult if not impossible to draw certain conclusions. However, with the
exception of the most trivial query, 



\begin{itemize} 
  \item usikkerhetsgreier, forenklinger / er det forenklinger? LL folger ikke
  helt slik diskutert i teori tror jeg, 
  \item pathfinder har noen faa optimiseringer men ikke alle
  \item kompleksitetsestimatet
  \item kunne kjort vaar algebra paa monetdb? og sammenlignet ytelse?
  (ogsaa ta dette i future work)
  \item spesiallagd algebra for monetdb
\end{itemize}