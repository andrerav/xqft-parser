\section{XQuery sequences}
\label{sect:disc:singelton}

There is no distinction between an item, that is, a node or an atomic value, and a singleton sequence containing
that item in XQuery. An item is equivalent to a singleton sequence containing that item and vice versa. A sequence
may contain nodes, atomic values, or any mixture of nodes and atomic values. But it may be advantageous for a
translator to differentiate singleton sequences from other sequences. 

As we saw in section \ref{sect:trans:TD:simplifications}, by knowing that all subexpressions return singleton
sequences, the translation of the sequence construction expression may be simplified. If the
\texttt{return}-clause expression is a singleton sequence the translation of iterator ordered FLWOR expressions
may also be simplified. If the FLWOR only contains only one iterator and no \texttt{where}-clause the renumbering
can be replaced by a renaming of the $-numb$ field corresponding to the iterator to $index$. Understanding that
this works can be done by considering the rule for translation of iterator ordering (rule
\ref{rule:trans:TD:itOrd}). As the \texttt{return}-clause is a singleton, the $index$ fields will have the
constant value 1. $\beta$ will contain only one $-numb$ attribute, holding information of which iteration the
value in $value$ occurs. The iteration number will then become the $index$ field of the sequence created by the FLWOR.

Some expressions, such as arithmetic expressions, \texttt{order by}-expressions and value comparisons, require
their subexpressions or operands to be singleton sequences. This means that a query such as \texttt{(1, 2) + 3}
will raise a type error. By having the knowledge of the cardinality of the sequence returned to such an
expression, the translator may raise the error, and avoid a faulty query being run on the MQL processor.
Evaluating the cardinality of seqences returned from expressions is in many cases a simple task. Some expressions
will always return singletons, such as logical, comparison and aritmetic expressions, iterator variable references
and literals. The cardinality of sequences constructed of such expressions may also be calculated in the
translator. A problem arises, however, when dynamic content (not from the query
itself) is included. Consider the following query:

\begin{center}
\begin{tabular}{l}
\texttt{for \$a in doc("people.xml")//person} \\
\texttt{order by \$a/surname} \\
\texttt{return \$a}
\end{tabular}
\end{center}

As previously stated, the \texttt{order by}-expression only accepts singletons. If the document contains a
\texttt{person} node containing two \texttt{surename} node the query should fail. The translator does
however not have the ability to evaluate if a type error should be raised or not. The query stated will, without a
check for multiple \texttt{surname} node per \texttt{person}, result in a sequence where the \texttt{person}
nodes containing more than one \texttt{surname} node will occur more than one time. 

The problem lies in the fact that the query is not a erroneous MQL query, but a erroneous XQuery query. One
solution would be to implement a check in MQL, which would inform the MQL processor of any potential error. This
can be done e.g.\ by a MQL function \textsf{raiseError()}, which would abort the evaluation of the query and e.g.\
throw an exception. All items in the relational representation of sequences in
Tainting Dependencies are marked with their position within the sequence with
the $index$ field. One way to check if an expression $e$ does not return a singleton sequence can be the following:
\begin{center}
\begin{tabular}{l}
\textsf{select(ifthenelse(eq(index, 1), true, raiseError());} \\ \quad
\textbf{r(}$e$\textbf{)}\textsf{)}
\end{tabular}
\end{center}

The check may of course be omitted if the translator is sure, by the means discussed earlier, $e$ will return a
singleton.

\subsection{Effective boolean value}
\label{sect:disc:effBool}
Within certain circumstances it is necessary to find the effective boolean value
of a sequence. The effective boolean value is by W3C defined as follows\cite{w3c00}: 
\begin{quote}
The effective boolean value of a value is defined as the result of applying the \texttt{fn:boolean} function to
the value.
\end{quote}

Where the function is declared as \texttt{fn:boolean(\$arg as item()*) as xs:boolean} and the dynamic semantics are
defined as follows\cite{w3cfuncOps}:

\begin{enumerate}
  \item If \texttt{\$arg} is the empty sequence, \texttt{fn:boolean} returns \textit{false}.
  \item If \texttt{\$arg} is a sequence whose first item is a node, \texttt{fn:boolean} returns \textit{true}.
  \item If \texttt{\$arg} is a singleton value of type \texttt{xs:boolean} or a derived from \texttt{xs:boolean},
  \texttt{fn:boolean} returns \$arg.
  \item If \texttt{\$arg} is a singleton value of type \texttt{xs:string} or a type derived from \texttt{xs:string},
  \texttt{xs:anyURI} or a type derived from \texttt{xs:anyURI} or \texttt{xs:untypedAtomic}, \texttt{fn:boolean}
  returns \textit{false} if the operand value has zero length; otherwise it returns \textit{true}.
  \item If \texttt{\$arg} is a singleton value of any numeric type or a type derived from a numeric type,
  \texttt{fn:boolean} returns \textit{false} if the operand value is \textit{NaN} or is numerically equal to zero;
  otherwise it returns \textit{true}.
  \item In all other cases, \texttt{fn:boolean} raises a type error.
\end{enumerate}

In the chapter presenting Tainting Dependencies, chapter \ref{sect:translation}, we solved resolving the effective
boolean value of sequences with the help of an assumed MQL function \textsf{xqBoolean()} and an undefined abstract
function \textbf{B()}. In that chapter we said it was enough to consider it as a grouping function returning one
tuple per unique iteration (as implied by the iterator dependencies). The points 3-5 of the definition of the
effective boolean value are all handled by \textsf{xqBoolean()} without problems, as they all concern singleton
sequences. Points 2 and 6 together, however, creates a interesting situation.
They imply that when finding the effective boolean value of all possible XQuery sequences, all should cause an error, except the sequences where a
node is the first element, regardless of the other items in the sequence.

This means that a simple check as proposed for evaluating the cardinality of operands for e.g.\ arithmetic
expressions earlier in this section will not do. A possible solution for the \textbf{B()} function would be to run
a \textsf{group} operator on the relation, grouping on the unique iterations, and run a \textsf{count} and some
aggregator functions selecting the value and type of the first item of each group. Further, the result will then
have to be checked similar to the check proposed earlier.
\begin{center}
\begin{tabular}{l}
\textsf{select(or(eq(type, node), ifthenelse(eq(count,1), true, raiseError());} \\\quad
\textsf{group((}$e.\vartheta$\textsf{), count(), selProj(eq(index,1), value), selProj(eq(index,1), type);}
\\\quad\quad
\textbf{r(}$e$\textbf{)}\textsf{))}
\end{tabular}
\end{center}
Where \textsf{selProj(}$pred$, $field$\textsf{)} is a kind of selection and projection hybrid, selecting field
$field$ if predicate $pred$ is \textit{true}. Such a aggregator function may however seem strange, and there is
nothing similar implemented in the MQL processor.

An alternative can be to utilise the \textsf{groupby} operator. The semantics of this operator is the same as for
the \textsf{group} operator, except that the input relation is unchanged and returned as the output. Further, the
result of the grouping is returned as a separate, named relation (``result set'' in MQL lingo). The
\textsf{resultset} operator with the name as its parameter will fetch this named result:

\begin{center}
\begin{tabular}{l}
\textsf{select(and(eq(index,1), or(eq(type,node), ifthenelse(eq(count,1), true, raiseError())));} \\ \quad 
\textsf{hhjoin([}$e.\vartheta$\textsf{], [}$e.\vartheta$\textsf{], [l.value, r.count];} \\ \quad\quad
\textsf{groupby(countrelation, (}$e.\vartheta$\textsf{), count();} \\ \quad\quad\quad
\textbf{r(}$e$\textbf{)}\textsf{);} \\ \quad\quad
\textsf{resultset(countrelation)))}
\end{tabular}
\end{center}

Both solutions assume a type system where a field $type$ holds the type of the item represented in the tuple. The
last proposal of a implementation of the \textbf{B()} function only uses operators and functions already
implemented in the MQL processor. A disadvantage with this solution, however, is that it will probably be more
resource-demanding as it consists of both a grouping and a join.

Point \#1 in the definition of evaluation of effective boolean value is also a
bit problematic. This is because Tainting Dependencies, as it is presented in chapter \ref{sect:translation},
does not handle empty sequences explicitly. And in most cases this is sufficient, as non-empty sequences containing
empty items should be normalised. In the following query it is clear that the result must be all the
\texttt{maybe} nodes of the document which is a child of a \texttt{elem} node:

\begin{center}
\begin{tabular}{l}
\texttt{fn:doc("nodes.xml")//elem/maybe}
\end{tabular}
\end{center}

The \texttt{elem} nodes which does \emph{not} contain a \texttt{maybe} node (making \texttt{../elem/maybe}
an empty sequence) are irrelevant to the evaluation of this query. The only time the need for knowledge of a
possible empty sequence arises is when evaluating the effective boolean value. Consider the following XQuery
expression:

\begin{center}
\begin{tabular}{l}
\texttt{if(\$a/maybe) then} \\ \quad
\texttt{"exists"} \\ 
\texttt{else} \\ \quad
\texttt{"empty"}
\end{tabular}
\end{center}

Further, let \texttt{\$a} be an iterator variable consecutively bound to one and one \texttt{elem} node. If
some of the \texttt{elem} nodes does not contain a \texttt{maybe} node, some items of the resulting sequence
should be the string literal \texttt{"empty"}.

A possible solution to this would be to differantiate between the times the empty sequences are needed, and the
times they are not. This can be done by evaluating all the descendant
expressions of an expression which is to be calculated into a effective boolean value, in its own context. To evaluate some expressions in another matter
according to the context is made simple by implementing a context sensitive visitor pattern (section
\ref{sect:theory:contextVisitorPattern}). The difference in translating in the logical context as opposed to the
default context will only be that all joins on common dependencies will have to be made into full outer joins. In
addition joins such as in the axis+name test translation (rule \ref{rule:trans:TD:pathStep}) where only the left
operand may have dependencies, will have to be turned into a left-outer join. Always employing outer joins will
ensure that no $-numb$ field corresponding to one unique iteration will be removed from the relation. Using this
solution \textsf{xqBoolean()} must return $false$ if it is run with \textit{NULL} as input.