\section{Optimisations - {MADS}}
\label{sect:disc:optimisations}
\textbf{\LARGE TODO: {MADS}}
\begin{itemize}
  \item Med bare \emph{EN} referanse til context item i predikat i path expressions kan hele symtab bajabajaz
  droppes, og bare prune p\aa~scope\ldots
  \item holde styr p\aa~scopes, bruke scope operatoren mer flittig = smaller relasjoner \aa~joine.
  \item kanskje bruke den alt-fremover omskrivinga av pathexprs
  \item pushe projects sammen ex: pathgreia i resultatkap. (sikkert allerede gjort i Mars) Fjerne attributter som
  aldri blir lest, og ikke er en del av resultatet\ldots
\end{itemize}

fra 2.3.4 snakk om tilstrekkelig kunnskap til \aa~ evaluere expr:
\begin{quote}
These rules apply to all the operands of an expression considered in combination: thus if an expression has two
operands E1 and E2, it may be evaluated using any samples of the respective sequences that satisfy the above rules.

The rules cascade: if A is an operand of B and B is an operand of C, then the processor needs to evaluate only a
sufficient sample of B to determine the value of C, and needs to evaluate only a sufficient sample of A to
determine this sample of B.

The effect of these rules is that the processor is free to stop examining further items in a sequence as soon as
it can establish that further items would not affect the result except possibly by causing an error. For example,
the processor may return true as the result of the expression S1 = S2 as soon as it finds a pair of equal values
from the two sequences.

Another consequence of these rules is that where none of the items in a sequence contributes to the result of an
expression, the processor is not obliged to evaluate any part of the sequence. Again, however, the processor
cannot dispense with a required cardinality check: if an empty sequence is not permitted in the relevant context,
then the processor must ensure that the operand is not an empty sequence.
\end{quote}