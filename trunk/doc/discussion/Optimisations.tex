\section{Optimisation - {MADS}}
\label{sect:disc:optimisations}

In section \ref{sect:trans:TD:simplifications} we presented some situations where the translation from XQuery into
MQL can be specialised and simplified. In this section we will outline some ideas for further simplification and
optimisation.

As can be seen of the tree in figure \ref{fig:results:query_ifthenelse_result} on page
\pageref{fig:results:query_ifthenelse_result}, Tainting Dependecies can produce relational algebra trees with
consecutive \textsf{project} operators. This is of cource unnecessary, and such operators can be merged into one
single \textsf{project}. Attributes which is not part of the result or part of evaluation of the result can be
pruned. This may however not be anything the translator would need to consider, as the Mars optimiser allready
implements methods for detecting and simplifying such trees.

Rule \ref{rule:trans:TD:conditional} on page \pageref{rule:trans:TD:conditional} describes the translation of
XQuery \texttt{if..then..else} expressions. It was implemented considering the result would be a operator tree and
not a DAG, and the number of operations should be minimised. In this translation both the \texttt{then}-expression
and the \texttt{else}-expression will be evaluated and their results spliced together. From this relation the
tuples stemming from the right expression according to the result of the test expression is selected. If the MQL
processor labours DAGs it may be a better solution to evaluate the two expressions only for the cases where they
are to be returned. This may be done in a matter very similar to the Loop Lift solution (equation
\ref{eq:ll:ifthenelse} on page \pageref{eq:ll:ifthenelse}). In this solution different $loop$ relations are made
depending on the outcome of the test-expression. One of the relations are used to evaluate the
\texttt{then}-expression, another the \texttt{else}-expression. As Tainting Dependencies does not utilise $loop$
relations, some changes would have to be made. Evaluation of the test-expression will reveal which unique
iterations which expression is to be evaluated in. Based on this information iterations can be pruned from the two
expressions before they are evaluated.

\subsection{XQuery Semantics}

As XQuery is a purely functional language an implementation is always free to evaluate the operands of an operator
in any order\cite{w3c00}. This means that a MQL optimiser is able to rearrange operators as it sees fit, with the
possible outcome of a less expensive query execution. One such case would be in a path expression such as this:
\begin{center}
\texttt{\$i/child::elem/descendant::maybe}
\end{center}
If the optimiser has some notion of the frequency of occurence of the \texttt{maybe} and \texttt{elem} elements,
as well as the cardinality of the relational representation of \texttt{\$i}, it may not be evaluated from left to
right. If there exists only a few \texttt{maybe} elements and the \texttt{\$i} relation is relatively large, the
most effective execution plan for this path expression would most likely be from right to left.

In some cases, a XQuery implementation can determine the result of an expression without accessing all the data
that is implied by the formal expression semantics. A consequence of this is that some errors goes undetected. W3C
specify to which extent an implementation may optimise its access to data at the cost of not detecting errors
like this \cite{w3c00}:

\begin{quote}
Consider an expression Q that has an operand (sub-expression) E. In general the value of E is a sequence. At an
intermediate stage during evaluation of the sequence, some of its items will be known and others will be unknown.
If, at such an intermediate stage of evaluation, a processor is able to establish that there are only two possible
outcomes of evaluating Q, namely the value V or an error, then the processor may deliver the result V without
evaluating further items in the operand E. [\ldots]

There is an exception to this rule: If a processor evaluates an operand E (wholly or in part), then it is required
to establish that the actual value of the operand E does not violate any constraints on its cardinality. [\ldots] 
\end{quote}

This feature could be utilised in situations where path expressions is to be evaluated to effective boolean values
(as described in section \ref{sect:disc:effBool}). If the path expression is to return either nodes or an empty
sequence, it is sufficient to find \emph{one} node per unique iteration. This might prove difficult to implement
on an MQL processor however, as it is hard to know which unique iteration a element belongs to before the element
relation and a relation with representations in all unique iterations are joined together.

Another situation where not accessing all data is required is with numeric predicates. The formal description of
filter expressions\cite{xquery_semantics} says that a expression such as \texttt{\$s[1]} should be evaluated by
consecutively examining the items of the sequence \texttt{\$s}, and selecting all items where
\texttt{position()=1}. A better solution would be to only pick the first item of the sequence. But as Tainting
Dependencies stores sequences from many iterations in the same relation, and no ordering can be assumed, this can
not be done easily. The MQL processor has no way of knowing where the tuples holding the items which are the
first item of their respected sequences unless there is some kind of indexing of the relation on the $index$
fields. In such a case the processor could lookup the tuples where $index$ has the value 1, the alternative is a
sequencial scan of the relation, checking each tuple.

\subsection{Path Expressions}
Quite a lot of research has been done on optimising XPath (Path expressions in XQuery stem from XPath) since it
became a W3C Recommendation in 1999. Of the documents produced by this research we would recommend
\cite{optimize_logic}, \cite{optimize_michiels} and \cite{optimize_xsltPath}. As this is outside the scope of this
report the contents of these will not be resited here. However, \cite{optimize_forward} presents a reverse axis
removal algorithm which may be interesting in a MQL and Taiting Dependencies specific setting. The algorith
recognises path expressions containting reverse axis steps, and rewrites them into pure forward axis step
expressions with possible predicates. This may be helpful as the MQL \textsf{scope} operator only accepts steps
equivalent to the \texttt{child} axis.

By letting the translator keep track over valid consecutive \texttt{child} axis steps, the \textsf{scope} operator
may be employed to filter the results from the lookup (ref rule \ref{rule:trans:TD:pathStep}, page
\pageref{rule:trans:TD:pathStep}), most likely reducing the tuples involved in the subsequent join. Consider the
following path expression:

\begin{center}
\texttt{/a/b[g]/c//d[h]/e}
\end{center}

Here, the result of looking up \texttt{g} may be filtered by a \textsf{scope} operator with \textsf{/a/b/g} as its
parameter. Similarly, the lookup of \texttt{c} may be filtered on \textsf{/a/b/c}. Because the name test \texttt{d}
is not part of a \texttt{child} axis step (rather a \texttt{descendant-or-self} axis step), no filtering can be
done. \texttt{e}, however, may be filtered by \textsf{scope} with the parameter \textsf{d/e}. Some consideration
will still have to be done concerning how much cheaper the join will become with the filtering compared to the cost
of filtering one of its operands.

One idea would be to move all predicates out of the path expression, and apply them as post filters. This would
probably reduce the number of needed joins -- at least in some cases. In other cases it may be to costly to build
a filter fitted to the whole path expression. If e.g. \texttt{h} from the last example query was a number instead,
the path expression up until \texttt{d} would have to be evaluated, filtered with the \texttt{g} predicate, and
finally joined with the evaluation of the path expression without the last predicate on their \texttt{d} step in
the $scope$ fields.

The rule for translating predicates, rule \ref{rule:trans:TD:predicate} on page \pageref{rule:trans:TD:predicate},
is a general rule. Here, a reference to the context item from the outside of the predicate is ``copied'' inside the
predicate (containting $sprDotNumb$). This copy is then operated on, and when the predicate expression is
evaluated, the predicated expression and the predicate expression is joined on the reference to the context item.
An example of this process is seen in figure \ref{fig:results:query_ifthenelse_result_dag} on page
\pageref{fig:results:query_ifthenelse_result_dag}. Here, the copying is in form of a upwards split, and the If the
context item is only referred to once within the predicate this translation may however be simplified.

Consider the following XQuery query:
\begin{center}
\texttt{//person[name eq "Robert"]}
\end{center}

This can be solved by joining the \texttt{name} relation with the \texttt{person} relation, and keeping all the
attributes from both relations. The attributes stemming from the \texttt{name} relation would have to be marked in
some way, as these are not a part of the result of the query. This relation will have to be filtered with a
selection removing the tuples where the $scope$ stemming from the \texttt{name} is not a child scope of the
$scope$ stemming from the \texttt{person} relation. Further, only the tuples where the $value$ stemming from the
\texttt{name} relation is \texttt{"Robert"} is retained. Finally the attributes stemming from the \texttt{name}
relation are projected away.

A similar solution may be considered for other types of predicates as well, but if there is more than one
reference to the context item within the predicate things get more cumbersome. In such a case the method would
have to be sure that no reference to the context item is removed from the relation at any time before the
finalisation of the evaluation.