\section{Creating the AST}
As described in section \ref{sect:theory:antlr}, ANTLR provides a operators and
rewrite rules for augmenting the grammar specification into generating a useable
AST. In many cases, it is a simple matter of defining which tokens to use as
root, and which to omit. This is done using the two basic operators \verb!hat!
(\^{}) and \verb!exclamation mark! (!). Figure \ref{code:ast:exoperator} shows an
example where the exclusion operator is used to exclude the superfluous brace
tokens from the AST in the \verb!ftWordsValue! production.

\begin{figure}[h]
\begin{verbatim}
ftWordsValue : literal | (LBRACESi! expr RBRACSi!);
\end{verbatim}
\caption[AST exclusion operator example]{Using the exclusion operator to exclude
the superfluous brace tokens from the AST, leaving only the expression subtree}
\label{code:ast:exoperator}
\end{figure}

Further the hat operator which is, as previously mentioned, used to promote
tokens to be roots in a subtree was utilized to augment certain expressions into
taking a natural hierarchial tree structure. Figure \ref{code:ast:hatoperator}
illustrates how not only how the hat operator is used to create subtrees from
expressions, but also how precedence is implicitly encoded into the grammar as
mentioned in section \ref{sect:xquery:precedence}.

\begin{figure}[h]
\begin{verbatim}
additiveExpr : multiplicativeExpr ((PLUSSi | MINUSSi)^ multiplicativeExpr)*;
  multiplicativeExpr : unionExpr ((STARSi | DIV | IDIV | MOD)^ unionExpr)*;
    unionExpr : intersectExceptExpr ((UNION | PIPESi)^ intersectExceptExpr)*;
\end{verbatim}
\caption[AST root node promotion operator example]{Using the root node promotion
operator to define structuring of expressions such as arithmetic addition and set
operations}
\label{code:ast:hatoperator}
\end{figure}

This is behaviour is illustrated in figure \ref{tree:ast:arithmetic}, which is
the AST generated for the simple arithmetic expression \verb!1+2*3-2*15!.

\begin{figure}[h]
\Tree [.{-} [.{+} 1 [.{*} 2 3  ]  ] [.{*} 2 15  ] ]
\caption{Arithmetic AST example}
\label{tree:ast:arithmetic}
\end{figure}

\subsection{FLWOR expressions}
fLWORExpr : (fc+=forClause | lc+=letClause)+ whereClause?
            orderByClause? RETURN exprSingle
            -> \^{}(AST\_FLWOR \$fc* \$lc* whereClause? orderByClause? exprSingle);
	