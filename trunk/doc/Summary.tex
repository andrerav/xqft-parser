\chapter{Summary}

\underline{\textbf{\LARGE //TODO:}}
\begin{itemize}
	\item Forbedringer/fremtidig arbeid:
	
	\begin{itemize}
		\item lage lexer for haand, kan bli mindre kompleks
		\item ingen reserved keywords - i hennhold til spesifikasjonen
		\item sjekke noesting av xml-tagger, ikke krevd i spesifikasjonen, men kan vaere praktisk
		\item lage flere lexere, og skifte mellom disse naar vi naa skifter state
		(island grammars)
	\end{itemize}
	
	\item \item stolte for mye paa EBNF fra W3C, sannsynlig ment for mennesker, ikke datamaskiner
	\item arbeidsprossess? / metoder?
\end{itemize}

\textbf{\LARGE flyttet fra implementation - blindveier}

\subsection{Lexer vs. parser syntax}
The Antlr parser generator can generate parsers and lexers from a single grammar
file. The distinction between terminals and non-terminals is simply a matter of
convention, where terminals are assumed to start with uppercase letters, and
non-terminals are assumed to start with lowercase letters.

In the grammar specified by the W3C, all the productions (terminals and
non-terminals) all start with uppercase letters. Initially this caused some
confusion, because this grammar naturally generated a very big lexer and a very
small and non-functional parser.

This was mitigated by converting non-terminal productions to start with
lowercase letters.

\subsection{Rewriting the W3C 'dash' operator}
This operator is not supported in Antlr, so it was necessary to rewrite
these productions using \emph{semantic predicates} where necessary.

\underline{\textbf{\LARGE //ODOT:}}