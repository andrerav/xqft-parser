\chapter{Summary}

\section{Future work and improvements}
\subsection{Improving lexer error handling}
As noted in sections \ref{sect:error_handling:syntax_errors} and
\ref{sec:impl:errorhandling}, the nextToken() method in the lexer base class
would ``hijack'' lexical exceptions and handle them by printing an error message
to stderr and then attempting to recover from the error by simply consuming the
offending character and ignoring it in the following construction of tokens. It is 
not immediately appareant, but the nextToken() method can not be overridden and
forced to throw the exception onwards, due to the method signature itself which
does not allow exceptions to be thrown. 

However, one possible solution to this problem could be to override the
nextToken() method and employ the observer design
pattern to allow a
simple and decoupled way of flagging an exception to the parser.

\subsection{Type checking}
Currently the parser will not perform type checking on the parsed queries. This
is an essential feature and will be necessary to implement for the parser to be
applicable in any realistic setting. A type checking system with proper type
inference and synthesis could be a complex feature to implement in a language
such as XQuery, and might require considerable effort, especially in quality
insurance. 

\subsection{Scoping and symbol tables on AST}
Currently the scoping and symbol tables are being used directly in the grammar -
that is, during parse time. It would be benefitial to move this into being
performed in run time, and combine with type checking functionality.

\label{sect:summary:future_work}
\underline{\textbf{\LARGE //TODO:}}
\begin{itemize}
	\item Forbedringer/fremtidig arbeid:
	\begin{itemize}
		\item lage lexer for haand, kan bli mindre kompleks
		\item ingen reserved keywords - i hennhold til spesifikasjonen
		\item sjekke noesting av xml-tagger, ikke krevd i spesifikasjonen, men kan vaere praktisk
		\item lage flere lexere, og skifte mellom disse naar vi naa skifter state
		(island grammars)
	\end{itemize}
	
	\item stolte for mye paa EBNF fra W3C, sannsynlig ment for mennesker, ikke
	datamaskiner
	\item arbeidsprossess? / metoder?
\end{itemize}


\textbf{\LARGE flyttet fra implementation - blindveier}


\underline{\textbf{\LARGE //ODOT:}}