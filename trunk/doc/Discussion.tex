\chapter{Discussion}
\label{chapter:discussion}

\textbf{\underline{\LARGE TODO:}} innledning

\begin{itemize}
  \item (1,2) + 3 kan oppdages som feil ved at man merker iteratorvariable, atomiske og sekvenser + at man synes
  det er flott for optimalisering av sequencebygging
  \item kan sette context node til noe n\aa r man starter\ldots Tenke en virtuell /collection/doc struktur\ldots
  ellerno
	\item Hva med \$i = (1,2,3) \$i/hatt -> typefeil? kj\o re isInScope(scope) p\aa~noe som ikke har scope kolonne?
\end{itemize}

\section{Loop Lift vs ``insertNameHere''}
\label{sect:discussion:llvsmXr}
\begin{itemize}
  \item vise til ll eksemplet sect \ref{sect:translation:ll:example} og f\aa~frem at ll krever DAGz ganske hardt,
  mye som blir brukt om igjen. Og fra begynnelsen til slutten for den saks skyld.
  \item fordeler vs ulemper..
  \item dra frem at markXremove bruker f\ae rre operatorer
  \item men st\o tter ikke all verden -> fundamental feil? -> ordering iallefall, types ogs\aa~til en viss grad
  \item hva med en switch if(!all verden) -> markXremove else LoopL
  \item pathfinder way kommer ikke til \aa~dra nytte av den mer ekspressive mars-algebraen\ldots Men er ekspressiv
	  bedre? Synes jeg s\aa~ noe i en av pathfinder artiklene hvor de sa at jo mer restriktiv, jo bedre
	  \aa~optimisere\ldots snakke med thorbj\o rnsen om dette..
\end{itemize}

\section{Rewriting}
\label{sect:discussion:rewriting}
\begin{itemize}
  \item fordeler vs ulemper med \aa~skrive om til core
  \item man mister jo informasjon\ldots. Hvis den er p\aa~denne m\aa ten --> gj\o re det akkurat
	  slik, en sp\o rring som skal gi tilsvarende svar er ikke sikkert at man kan
	  skrive p\aa~den samme m\aa ten helt uten videre..  
  \item samme svar = samme utf\o relse = er dette en fordel?
  \item kan man utnytte kunnskap om translation til \aa~optimisere xquery queries?
  \item hva med \aa~bare skrive om det man trenger? Ala det vi gjorde med FLWORz?
\end{itemize}

\section{Manual vs. automated tree parser construction - {DROPPES?}}
ANTLR provides a utility for automated construction of AST parsers. This
utility requires the specification of a separate ``tree grammar''. This tree
grammar is almost identical to the original parser grammar. Practically, the
parser grammar can be copied verbatime, renamed, modified slightly and used as
a tree  grammar. This process is described in detail in \cite{definitiveAntlr},
section 8.1.

This introduces a high level of redundancy. The two grammar specifications are
required to be somewhat identical with regards to their grammar structure; that
is,  redundant tokens can be removed. However, the rewrite rules are required to
be identical.

This creates a problem with maintainability. As the parser grammar and rewrite
rules are not freezed at this point but rather highly subject to change, any
changes made in the parser grammar will need to be transferred to the tree
grammar, and vice versa. 

In \cite{translators_should_use_tree_grammars}, Terence Parr argues that the
traditional visitor
pattern\footnote{http://en.wikipedia.org/wiki/Visitor\_pattern} only provides a
simplistic facility for triggering events on the AST, that no tree structure
validation is implicitly available, and that context information has to be
passed down through the tree during the parse or by setting global variables.

In another point of view strongly polar to that of Terence Parr, Andy Tripp
argues\cite{manual_tree_walking_is_better} that manual tree parsing is
better. He establishes the following points of argument which are of particular
interest to this project:
\begin{itemize}
  \item Duplication of code and effort -- the concept of "what is a valid AST"
  would have to be implemented in both the parser and the AST transformer phase
  [as a rebuttal to validation of AST]
  \item With regards to contextual information, There seems to be nothing wrong
  with depending on the physical structure of the AST 
  \item Defining a traditional parser in grammar is practical because the grammar
  usually resembles the ouput AST. In the case of a tree parser proposed by Parr
  where the grammar actually resembles the input AST, this mapping may break
  down completely if the output is another tree structure
\end{itemize}

In particular, the last point holds a strong indication that a tree grammar
may not be suited for this project, as this tree parser will transform the AST
into a relational algebra tree.

\section{XQuery Features Not Supported - {MADS}}
\label{sect:disc:notSupported}
\textbf{\LARGE TODO: Mads}
\begin{itemize}
\item noen av disse tingene er ikke st\o tta pga typesystem, andre, slik som stemming og thesaurus er fordi ikkeno
slik i mars enn\aa~kan foresl\aa~det med parametere til lookup evt contextting ala \textsf{index()}.
  \item Dette avhenger jo seff av hvor mye vi har l\o st men disse b\o r kunne
  l\o ses:
  	\begin{itemize}
  		\item proximity (kanskje l\o sbar)
  		\item declare variable er nesten st\o tta\ldots men external og greier..
  		\item function declarations, b\o r g\aa~greit, bare ha en function table ala symbol table.
  		\item schema / schema validation -> typeting? Hvordan l\o ser pathfinder dette\ldots synes \aa~ha sett noe om
  		det\ldots
  		\item namespacezz\ldots.
  		\item node comparisons\ldots\ldots tviler p\aa~at vi f\aa r til dette glatt\ldots
  		\item order by med alle ting..
  		\item ordered and unordered -> lurer p\aa om (markXremove + tainting = TD uten $index$ og
  		\textsf{numberate()}) fikser dette ganske bra\ldots
  		\begin{itemize}
			\item MarkXRemove funker bra i unordered m0de tror jeg\ldots. Den er ogs\a~normalisert ref purely relational
			flwors, som sier LL er denormalisert. (bare normalisert innenfor en flwor.. den unormaliserer seg n\aa r den
			g\aa r ut av l\o kka (cross /m const))
			\item Order/unorder kan dra nytte av kontekstsensitive visitors\ldots\ldots
		\end{itemize}
  		\item \textbar, \texttt{union}, \texttt{intersect, except}
  		\item Range expressions $e_1$ \texttt{to} $e_2$ (begge m\aa~v\ae re integer tror jeg -> skrive om det i type?)
  		\item Prologs and modules
  		\item Quantified Expression: (some | every) \$b in $e_1$ satisfies $e_2$
  		\end{itemize}
	\end{itemize}

\subsection{Order By - {MADS}}
\label{sect:disc:orderby}
\textbf{\LARGE TODO: {MADS}}
\begin{itemize}
  \item st\o tte alle de sorteringsspesifiseringene.. st\o tte sortering over flere exprz.
\end{itemize}
	
	
	
	\subsection{XQuery Functions - {MADS}}
\label{sect:disc:functions}
\textbf{\LARGE TODO: {MADS}}
\begin{itemize}
  \item hvordan ordne XQuery funksjoner?
  \item Tror det skal v\ae re lagt til rette for \aa~ha en \textsf{function(FUNCTIONNAME; operator(list?)} operator
  \end{itemize}
  
fra 3.1.5:
\begin{quote}
  Additional functions may be declared in a Prolog, imported from a library module, or provided by the external
  environment as part of the static context.
  \end{quote}


\section{Order}
\label{sect:discussion:order}
\begin{itemize}
  \item hva skjer om noe er sortert.. s\aa~blir det kryssa? M\aa~vi alltid gi ting ordernumber? Herregud s\aa~likt
  pathfinder i s\aa fall =/
  \item kanskje denne section og den om type kan sl\aa s sammen til noe om representering av data?
\end{itemize}

\section{Type System}
\label{sect:discussion:typeSystem}
\begin{itemize}
  \item Hvordan f\aa~til noe typesystem?
  \item lagre false som ``false'' enn s\aa~lenge.. kjipt med /a/b[/a/b/c] hvis c er slik: <c>false</c>
  \item Mars st\o tter ikke forskjellige typer innenfor samme felt
  \item En sekvens er en sekvens i XQuery\ldots ikke en sekvens av booleans
  eller noder etc
  \item Et ekstra felt som sier type?
  \item Hva skjer med /a/b/c/text() vs /a/b/c ?
  \item hva skjer om man lager en <a> hei <b> jeje </b> </a> variabel? Dette
  m\aa~kunne representeres.
  \item Hvis vi hadde hatt statisk og sterk typing s\aa~ hadde mye v\ae rt
  ordna, f.eks \verb!for $i as xs:int in (1,2,3) return /a[$i]! s\aa vet man
  med en gang at \verb![$i]! er en ``numeric predicate''.
\end{itemize}

\section{Optimisations}
\label{sect:discussion:optimisations}
\begin{itemize}
  \item legg merke til bruk av Uk-skrivem\aa te (s vs z)
  \item enkeltverdier i sammenligning b\o r bli putta inn i selecten.. ikke lag
  eget sett og kryss
  \item enkeltverdier etterhverandre b\o r lages i en go, ikke \texttt{union}es
  sammen
  \item step for step pathexpr er ikke effektivt\ldots SCHADE ulempe med rene uttrykk: /a/b/c = trengs ingen
  joins egentlig
  \item et step b\o r kanskje egentlig joines med konteksten sin f\o r predikatet sitt -> /a/b[c] ((a join a/b)
  join a/b/c) ikke (a join (a/b join /a/b/c)
  \item vi har ordna, uten \aa~vite om det at man kan selecte i stedet for \aa~projecte n\aa r man er i logisk
  kontekst.. yeah!
  \item flere relative pathexprs i predikater b\o r egentlig kunne mergeInJoines sammen f\o r de blir merginjoina
	med det som predikatet st\aa r til\ldots dette er kanskje en optimalisering\ldots
  \item 
\end{itemize}