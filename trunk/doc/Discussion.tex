\chapter{Discussion}
\label{chapter:discussion}

\textbf{\underline{\LARGE TODO:}} innledning


Hva med disse punktene?:
\begin{itemize}
	\item MarkXRemove funker bra i unordered m0de tror jeg\ldots. Den er ogs\a~normalisert ref purely relational
	flwors, som sier LL er denormalisert. (bare normalisert innenfor en flwor.. den unormaliserer seg n\aa r den g\aa
	r ut av l\o kka (cross /m const))
	\item Order/unorder kan dra nytte av kontekstsensitive visitors\ldots\ldots
\end{itemize}

\section{REsulTAt diskusjon - {{ANDREAS}}}
\label{sect:disc:res}
\textbf{\LARGE TODO: {{ANDREAS}}}
\begin{itemize}
  \item usikkerhetsgreier, forenklinger / er det forenklinger? LL folger ikke helt slik diskutert i teori tror jeg,
  \item paralellisering? jeje..
\end{itemize}

\section{SingelT0n vs sequence Ye = finn p\aa~overskrift etter innhold - {MADS}}
\label{sect:disc:singelton}
\textbf{\LARGE TODO: {MADS}}
\begin{itemize}
   \item (1,2) + 3 kan oppdages som feil ved at man merker iteratorvariable, atomiske og sekvenser + 
   \item at man synes det er flott for optimalisering av sequencebygging 
   \item order by (1,2) kan ogs\aa~oppdages. 
\end{itemize}

\section{Effective Boolean Value - {MADS}}
\label{sect:disc:effBool}
\textbf{\LARGE TODO: {MADS}}
\begin{itemize}
\item VI M\AA~SJEKKE ATOMIZATION.
\item Effective Boolean Value
\item Kommer til \aa~bli feil n\aa r en sekvens blir brukt som boolean value n\aa~s\aa fremt vi ikke kan groupe\ldots
\item En evt groupingoperator m\aa~slippe igjennom tall\ldots. Evt select index=1, m\aa~se om (1,2,3) er bogus
\item The translator is in the logical context, $\Lambda$, if the AST node it is currently visiting is a successor
of a boolean operator or within the condition part of an \texttt{if..then..else} expression. In all other cases the
translator is in the default context, $\Delta$. If no context is mentioned in the inference rules the default
context is assumed. 
\item test expr i if
\item barna til and og or
\end{itemize}

\section{Context Item - {MADS}}
\label{sect:disc:ctxItem}
\textbf{\LARGE TODO: {MADS}}
\begin{itemize}
  \item m\aa~nesten dytte den til symtab hele tiden, koster ikke mye, s\aa~lenge den ikke blir brukt.
  \item Man skal vel ha tilgang til den hele tiden omtrent.
  \item kan sette context node til noe n\aa r man starter\ldots Tenke en virtuell /collection/doc struktur\ldots
  ellerno
  \item XQuery has a set of functions that provide access to input data. These functions are of particular
  importance because they provide a way in which an expression can reference a document or a collection of
  documents. The input functions are described informally here; they are defined in [XQuery 1.0 and XPath 2.0
  Functions and Operators]. An expression can access input data either by calling one of the input functions or by
  referencing some part of the dynamic context that is initialized by the external environment, such as a variable
  or context item. -> Appendix C.2 Dynamic Context Components
\end{itemize}


\section{Loop Lift vs Tainting Dependencies - {{ANDREAS}}}
\textbf{\LARGE TODO: {ANDREAS}}
\label{sect:disc:llvsTD}
\begin{itemize}
  \item vise til ll eksemplet sect \ref{sect:trans:ll:example} og f\aa~frem at ll krever DAGz ganske hardt,
  mye som blir brukt om igjen. Og fra begynnelsen til slutten for den saks skyld.
  \item fordeler vs ulemper..
  \item dra frem at TD bruker f\ae rre operatorer
  \item pathfinder way kommer ikke til \aa~dra nytte av den mer ekspressive mars-algebraen\ldots Men er ekspressiv
	  bedre? Synes jeg s\aa~ noe i en av pathfinder artiklene hvor de sa at jo mer restriktiv, jo bedre
	  \aa~optimisere\ldots snakke med thorbj\o rnsen om dette..
\end{itemize}

\section{Rewriting - {ANDREAS}}
\label{sect:disc:rewriting}
\begin{itemize}
  \item fordeler vs ulemper med \aa~skrive om til core
  \item man mister jo informasjon\ldots. Hvis den er p\aa~denne m\aa ten --> gj\o re det akkurat
	  slik, en sp\o rring som skal gi tilsvarende svar er ikke sikkert at man kan
	  skrive p\aa~den samme m\aa ten helt uten videre..  
  \item samme svar = samme utf\o relse = er dette en fordel?
  \item kan man utnytte kunnskap om translation til \aa~optimisere xquery queries?
  \item hva med \aa~bare skrive om det man trenger? Hva trenger man \aa~skriveom?
  \item sl\aa~sammen scopes blir frem og tilbake og frem igjen om man tenker core\ldots /a/b/c-> for..for..for ->
  /a/b/c
  \item se om det er mer tungvinne ting Core har\ldots har ikke satt meg helt inn i alle detaljene der.. hva skjer
  med predikater? blir de if.then.else? kan det hende at vi f\aa r un\o dvendige kryssprodukt?
  \item sluppet where-regel hvis man skriver om til if-then-else (dagens where er mer effektiv enn en if() then no
 e else tomt -> den VET at det som er usant blir kasta bort- jfr regel og select-operator)
\end{itemize}

\section{Manual vs. automated tree parser construction - {DROPPES?}}
ANTLR provides a utility for automated construction of AST parsers. This
utility requires the specification of a separate ``tree grammar''. This tree
grammar is almost identical to the original parser grammar. Practically, the
parser grammar can be copied verbatime, renamed, modified slightly and used as
a tree  grammar. This process is described in detail in \cite{definitiveAntlr},
section 8.1.

This introduces a high level of redundancy. The two grammar specifications are
required to be somewhat identical with regards to their grammar structure; that
is,  redundant tokens can be removed. However, the rewrite rules are required to
be identical.

This creates a problem with maintainability. As the parser grammar and rewrite
rules are not freezed at this point but rather highly subject to change, any
changes made in the parser grammar will need to be transferred to the tree
grammar, and vice versa. 

In \cite{translators_should_use_tree_grammars}, Terence Parr argues that the
traditional visitor
pattern\footnote{http://en.wikipedia.org/wiki/Visitor\_pattern} only provides a
simplistic facility for triggering events on the AST, that no tree structure
validation is implicitly available, and that context information has to be
passed down through the tree during the parse or by setting global variables.

In another point of view strongly polar to that of Terence Parr, Andy Tripp
argues\cite{manual_tree_walking_is_better} that manual tree parsing is
better. He establishes the following points of argument which are of particular
interest to this project:
\begin{itemize}
  \item Duplication of code and effort -- the concept of "what is a valid AST"
  would have to be implemented in both the parser and the AST transformer phase
  [as a rebuttal to validation of AST]
  \item With regards to contextual information, There seems to be nothing wrong
  with depending on the physical structure of the AST 
  \item Defining a traditional parser in grammar is practical because the grammar
  usually resembles the ouput AST. In the case of a tree parser proposed by Parr
  where the grammar actually resembles the input AST, this mapping may break
  down completely if the output is another tree structure
\end{itemize}

In particular, the last point holds a strong indication that a tree grammar
may not be suited for this project, as this tree parser will transform the AST
into a relational algebra tree.

\section{XQuery Features Not Supported - {MADS}}
\label{sect:disc:notSupported}
\textbf{\LARGE TODO: Mads}
\begin{itemize}
\item noen av disse tingene er ikke st\o tta pga typesystem, andre, slik som stemming og thesaurus er fordi ikkeno
slik i mars enn\aa~kan foresl\aa~det med parametere til lookup evt contextting ala \textsf{index()}.
  \item Dette avhenger jo seff av hvor mye vi har l\o st men disse b\o r kunne
  l\o ses:
  	\begin{itemize}
  		\item proximity (kanskje l\o sbar)
  		\item declare variable er nesten st\o tta\ldots men external og greier..
  		\item function declarations, b\o r g\aa~greit, bare ha en function table ala symbol table.
  		\item schema / schema validation -> typeting? Hvordan l\o ser pathfinder dette\ldots synes \aa~ha sett noe om
  		det\ldots
  		\item namespacezz\ldots.
  		\item node comparisons\ldots\ldots tviler p\aa~at vi f\aa r til dette glatt\ldots
  		\item order by med alle ting..
  		\item ordered and unordered -> lurer p\aa om (markXremove + tainting = TD uten $index$ og
  		\textsf{numberate()}) fikser dette ganske bra\ldots
  		\begin{itemize}
			\item MarkXRemove funker bra i unordered m0de tror jeg\ldots. Den er ogs\a~normalisert ref purely relational
			flwors, som sier LL er denormalisert. (bare normalisert innenfor en flwor.. den unormaliserer seg n\aa r den
			g\aa r ut av l\o kka (cross /m const))
			\item Order/unorder kan dra nytte av kontekstsensitive visitors\ldots\ldots
		\end{itemize}
  		\item \textbar, \texttt{union}, \texttt{intersect, except}
  		\item Range expressions $e_1$ \texttt{to} $e_2$ (begge m\aa~v\ae re integer tror jeg -> skrive om det i type?)
  		\item Prologs and modules
  		\item Quantified Expression: (some | every) \$b in $e_1$ satisfies $e_2$
  		\end{itemize}
	\end{itemize}

\subsection{Order By - {MADS}}
\label{sect:disc:orderby}
\textbf{\LARGE TODO: {MADS}}
\begin{itemize}
  \item st\o tte alle de sorteringsspesifiseringene.. st\o tte sortering over flere exprz.
\end{itemize}
	
	
	
	\subsection{XQuery Functions - {MADS}}
\label{sect:disc:functions}
\textbf{\LARGE TODO: {MADS}}
\begin{itemize}
  \item hvordan ordne XQuery funksjoner?
  \item Tror det skal v\ae re lagt til rette for \aa~ha en \textsf{function(FUNCTIONNAME; operator(list?)} operator
  \end{itemize}
  
fra 3.1.5:
\begin{quote}
  Additional functions may be declared in a Prolog, imported from a library module, or provided by the external
  environment as part of the static context.
  \end{quote}


\section{Implementasjonen - {ANDREAS}}
\label{sect:disc:contextSens}
\textbf{\LARGE TODO: {ANDREAS}}
\begin{itemize}
  \item innhold
  
\end{itemize}


\section{Order By - {MADS}}
\label{sect:disc:orderby}
\textbf{\LARGE TODO: {MADS}}
\begin{itemize}
  \item st\o tte alle de sorteringsspesifiseringene.. st\o tte sortering over flere exprz.
\end{itemize}

\section{Type System - {ANDREAS}}
\label{sect:disc:typeSystem}
\textbf{\LARGE TODO: {ANDREAS}}
\begin{itemize}
  \item Hvordan f\aa~til noe typesystem?
  \item lagre false som ``false'' enn s\aa~lenge.. kjipt med /a/b[/a/b/c] hvis c er slik: <c>false</c>
  \item Mars st\o tter ikke forskjellige typer innenfor samme felt
  \item En sekvens er en sekvens i XQuery\ldots ikke en sekvens av booleans
  eller noder etc
  \item Et ekstra felt som sier type?
  \item Hva skjer med /a/b/c/text() vs /a/b/c ?
  \item hva skjer om man lager en <a> hei <b> jeje </b> </a> variabel? Dette
  m\aa~kunne representeres.
  \item Hvis vi hadde hatt statisk og sterk typing s\aa~ hadde mye v\ae rt
  ordna, f.eks \verb!for $i as xs:int in (1,2,3) return /a[$i]! s\aa vet man
  med en gang at \verb![$i]! er en ``numeric predicate''.
  \item typeswitch / instance of / cast / castable / treat as
  \item There is however also a need to represent explicitly stated XML-nodes, as well
	as differentiate between the number \texttt{1} and the string \texttt{"1"}.
	This, and other issues about representing XQuery types will be treated in
	section \ref{sect:disc:typeSystem}.  
\item Hva med \$i = (1,2,3) \$i/hatt -> typefeil? kj\o re isInScope(scope) p\aa~noe som ikke har scope kolonne?
	\end{itemize}

\section{XQuery Functions - {MADS}}
\label{sect:disc:functions}
\textbf{\LARGE TODO: {MADS}}
\begin{itemize}
  \item hvordan ordne XQuery funksjoner?
  \item Tror det skal v\ae re lagt til rette for \aa~ha en \textsf{function(FUNCTIONNAME; operator(list?)} operator
  \item 
\end{itemize}

\section{Optimisations - {MADS}}
\label{sect:disc:optimisations}
\textbf{\LARGE TODO: {MADS}}
\begin{itemize}
  \item Med bare \emph{EN} referanse til context item i predikat i path expressions kan hele symtab bajabajaz
  droppes, og bare prune p\aa~scope\ldots
  \item holde styr p\aa~scopes, bruke scope operatoren mer flittig = smaller relasjoner \aa~joine.
  \item kanskje bruke den alt-fremover omskrivinga av pathexprs
\end{itemize}