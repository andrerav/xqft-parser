% Theory / XQUery
% Kort utledning om XQuery
\section{XQuery / XPath With Full Text Extensions}
\label{sect:theory:xquery}
XML is a markup language and a form of data representation capable of storing
data from a vast number of diverse data sources. These data sources include
relational databases and in particular data structures bearing a resemblance to
trees. XQuery is primarily an XML document query language, developed as an
recommendation by the World Wide Web Consortium\cite{w3c00} (abbreviated W3C).
This implies that XQuery is capable of querying any data structure which may be
represented as an XML document.

The full text extensions to XQuery 1.0 and XPath 2.0 employs tokenisation and
phrase search to provide more accurate search results, and generally give the
benefits known from modern search engines. This is also believed to provide
better support for diverse languages, and more accurate search queries \cite{w3c02}.

\subsection{XPath / Path Expressions}
XPath is, as well as a recommendation by the W3C \cite{w3c01}, a subset of
XQuery. However XPath predates XQuery and is also limited only to path
expressions, which includes axis specifiers for addressing nodes in a document
as well as node tests, predicates, and arithmetic operators.

Consider this trivial example\cite{w3s00}:\verb!/bookstore/book[price>35]/title! 
This strongly resembles a reference to a file or folder on a standard Unix file
system, with the exception of the conditional expression enclosed within
angular brackets. In terms of XPath, this particular example would translate
into "fetch all title nodes where the price attribute is higher than 35".

\subsection{FLWOR}
FLWOR is an integral part of XQuery, and can be compared to the
select-from-where in SQL by a loose analogy. The abbreviation FLWOR means:
\begin{itemize}
\item For - imperative for-loop for iterating over tuples
\item Let - assignment of tuples to variables
\item Where - conditional filtering of tuples
\item Order by - ordering of the resulting tuples
\item Return - not to be mistaken with return statements from imperative
languages. Returning in this context is more similar to yielding known from
functional programming. The return statement is executed once for every
iteration   
\end{itemize}
The example in figure \ref{code:theory:flwor} from \cite{styl00} illustrates
the usage of FLWOR, and  compares it to its SQL counterpart.
\begin{figure}
\begin{Verbatim}
(: FLWOR example :)
for $v in doc("videos.xml")//video
where $v/year = 1999
return $v/title

-- Corresponding SQL example 
SELECT v.title FROM video v WHERE v.year = 1999
\end{Verbatim}
\caption[Simple FLWOR example]{Simple FLWOR example with SQL counterpart}
\label{code:theory:flwor}
\end{figure}

\subsection{Full Text Extensions}
The XQuery 1.0  and XPath 2.0 specifications have been extended with full text
capabilities\cite{w3c02}, which enables the use of stemming, thesaurus,
and similar tools in queries. Specifically, the full-text extensions
extend XQuery 1.0 and XPath 2.0 in the following ways:
\begin{itemize}
\item A new and essential expression, FTContainsExpr, as well as new operators
such as FTOr and FTAnd 
\item The addition of score variables to FLWOR expressions, for keeping track
of search result scores, also known as rank, as seen in most modern search
engines 
\item Static context declarations for full-text match options to the query
prologue 
\end{itemize}

Figure \ref{code:theory:flwor_fulltext} shows an example with a FLWOR expression
where full-text matching is used.

\begin{figure}
\begin{Verbatim}
for $result at $rank in
for $book score $score in //book[title ftcontains "usability" 
                                    && "testing" with stemming]
order by $score descending
return $book
where $rank <= 10
return $result
\end{Verbatim}
\caption[Simple FLWOR full-text example]{Simple FLWOR example with full-text}
\label{code:theory:flwor_fulltext}
\end{figure}

\subsection{Precedence}
\label{sect:xquery:precedence}
The operator precedence in XQuery 1.0 is specified implicitly in the grammar,
according to W3C\cite{w3cprecedence}. It is important to note that operators that 
have a lower precedence cannot be contained by operators with a higher 
precedence.
