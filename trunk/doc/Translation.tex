\chapter{Translation process}


\textbf{\underline{\LARGE TODO:}} innledning
\begin{itemize}
  \item presenterer markXremove, loop lifting (ref) og en endret versjon av
  markXremove\ldots
  \item ellers vet jeg ikke
\end{itemize}


\section{Inference Rules Language Explanation}
\label{sect:translation:inferenceExplanation}
During this chapter we will present some inference rules. This is an explanation
of the various typographical representations:

\begin{tabular}{l|l}

  $\Longrightarrow$  & Translates into \\
  $\vartheta$ & A set of iteration variable references \\
  \texttt{typewriter} & MQL expressions \\
  \textit{italic} & XQuery expressions \\
  $\alpha , \beta, \gamma$ & Generic expressions \\
  \textbf{bold} & Operations done during the generation of the algebra \\
  $\Delta$ & The default context \\
  $\Lambda$ & The logical or boolean context \\
  
\end{tabular}





\section{MarkXremove}
\label{sect:translation:markXremove}

One of the greatest challenges in transforming XQuery to relational algebra
lies in preserving the semantics of the iterative FLWOR construct. One
course of action for solving this is a method we have called ``MarkXremove'' --
pronounced ``mark, cross and remove''. The basis of the method is that the
result from the \textit{return} clause will always be crossed (cartesian product
see \ref{sect:theory:crossProduct}) with the sequence bound in the
corresponding \textit{for} clause. Any referal in the \textit{return} clause to a
variable bound in a \textit{for} clause will have to be marked with a number
corresponding to the iteration number the instance of the variable exists in.
After the cross product is calculated, rows not conforming to predicates
calculated in the \textit{return} clause will be removed. One of these is that
the variable instance will have to correspond to the interation number.

In the following sections we will present the MarkXremove method for
translating various aspects of XQuery to relational algebra.

\subsection{Basics}
The algebra will be generated by traversing the abstract syntax tree bottom-up,
left-to-right. After visiting a node the algebra for this node and its children
as well as a set, $\vartheta$, of references made to iterator variables. When
$\vartheta$ is refered to in an MQL, it represents a comma separated list of
the names of the fields holding the iteration number for the iterator
variables. This will be further explained in section
\ref{sect:translation:variables}.

It is assumed that a symbol table exists which handles scoping. The functions
\textbf{put(}\textit{X}, $\alpha$\textbf{)} and \textbf{get(}\textit{X}\textbf{)}
will respectively store and fetch algebra trees based on a key \verb!X!. When
moving out of a scope, the symbol table will remove all entries linked to this scope.

The translator is in the logical context, $\Lambda$, if its current node is a
successor of a boolean operator or within the $\alpha$ part of an
\textit{if($\alpha$) then $\beta$ else $\gamma$} expression. In all other cases
the translator is in the default context, $\Delta$. If no context is mentioned
in the inference rules the default context is assumed.

For simplicity all data fields (e.g \texttt{dokumentId, scope}) will be
referred to as \verb!fields!, if one field is not explicitly needed.

\subsection{Literals and Numbers}
\label{sect:translation:litAndNumbers}
To be able to use litterals and numbers in the algebra processing, they need a
relational representation. A simple way to do this is hold the value in a
created relation with a single field, ``value'' (rule
\ref{eq:translation:literalsAndNumbers} ).

\begin{equation}
\centering
\frac{\alpha \in \left\{ Literals, Numbers \right\} }{\alpha}
\Longrightarrow
\texttt{make(name := "value",} \alpha \texttt{)}
\label{eq:translation:literalsAndNumbers}
\end{equation}

There is however also a need to represent explicitly stated XML-nodes, as well
as differentiate between the number \textit{1} and the string \textit{"1"}.
This, and other issues about representing XQuery types will be treated in
section \ref{sect:discussion:typeSystem}.

Rule \ref{eq:translation:literalsAndNumbers} implies that any sequence of
explicitly stated items will first have to be made as one-tuple relations and
then spliced together. As the \texttt{make} operator supports multiple items, a
better solution would be to collect all items in one MQL operator. Further, if
the item is part of a singleton sequence, there is no need for a representation
in form of an relation, as the item could be made part of the parameters of an
operator (e.g. $\alpha$ \textit{> 1} $\Longrightarrow$
\texttt{select(gt(value, 1) $\alpha$)}). Both these issues will be discussed
further in section \ref{sect:discussion:optimisations}.



\subsection{Sequence Making}
\label{sect:translation:sequencing}
As XQuery sequences can only be one dimensional (section
\ref{sect:theory:xquery}), the creation of a sequence can be translated into a
splice of the relations corresponding to the items constituting the sequence.
This can be seen in rule \ref{eq:translation:sequencing}. 

\begin{equation}
\centering
\texttt{(} \alpha , \beta \texttt{)}
\Longrightarrow
\begin{array}{l}
\mbox{\texttt{union(}} 
\alpha \mbox{\texttt{,}} 
\beta \mbox{\texttt{)}}
\end{array}
\label{eq:translation:sequencing}
\end{equation}

\marginpar{\underline{\Large TODO} \footnotesize B\o r dette flyttes til
method:mars:assumedOperatorz i stedet? Det med NULL verdier i radene m\aa~
ogs\aa~skrives om.. blir mye om det her kanskje ja\ldots}
This method assumes that the \texttt{union} operator accepts relations with
different schemas, and does not remove duplicates (but contrary to disjoint
union, there is no need to differansiate elements by their orgin set in the
resultset). The schema for the result relation can be described as:
\begin{equation*}
schema(\texttt{union(}\alpha, \beta\texttt{)}) = schema(\alpha) \cup
schema(\beta).
\end{equation*}
Where $\cup$ is the ordinary set theory operator. In case this is a
unreasonable task to implement, it can be done manually by the translator with
the help of \texttt{project} operators.

As two fields with different types cannot be spliced together, we presume all
\texttt{value} fields are strings for the time being. This will be further
discussed in section \ref{sect:discussion:typeSystem}.



\subsection{Path Expressions}
\label{sect:translation:smpPathExpr}
XQuery implements XPath 2.0 path expressions as described in section
\ref{sect:theory:xqueryPathExpressions}. The \texttt{scope} operator of
MQL(section \ref{sect:method:marsOperators}) makes it possible to
filter results based on the scope. The operator does however only support paths
equivalent to path expressions with only \textit{child} axis steps. Methods to
translate expressions with other types of axis steps will be discussed in
section \ref{sect:discussion:notAxis}.

Figure \ref{fig:translation:pathAST} shows the abstract syntax tree for the
generic child axis and name test only path expression \textit{/a/\ldots/y/z}. It
should be noted that for the sake of simplicitly the AST node \textbf{AST\_STEPEXPR} is
omitted from the representation. The corresponding translation into MQL of such
expressions can be seen in rule \ref{eq:translation:pathExpr}.


\begin{figure}[h]
\centering
\tikzstyle{astNode}=[circle, draw=blue!70,fill=blue!20,solid,thick, minimum
size=26pt]
\begin{tikzpicture}[grow via three points={one child at (0,-1.5) and two
children at (-1.5,-1.0) and (1.5,-1.0)}]

\node at (0,0) [ellipse,
draw=blue!70,fill=blue!20]{AST\_PATHEXPR\_SGL} 
child{node [astNode] {/}
	child{node [astNode] {/}
		child{node [astNode] {/} 
			child{node [astNode] {a} edge from parent [draw, solid, thin]}
			child{node [astNode] {\ldots} edge from parent [draw, solid, thin]}
			edge from parent [draw,dotted, thick] 
			}
		child{node [astNode] {y}}
		}
	child{node [astNode] {z}}
	};
\end{tikzpicture}
\label{fig:translation:pathAST}
\caption[AST of path expression]{A simplified version of the abstract syntax
tree of a simple path expression}
\end{figure}


\begin{equation}
\centering
\textit{/a/\ldots/y/z}
\Longrightarrow
\begin{array}{l}
	\mbox{\texttt{scope(/a/\ldots/y;}} \\
	\qquad \mbox{\texttt{index(valloc; lookup(\$z))}}
\end{array}
\label{eq:translation:pathExpr}
\end{equation}

Rule

\begin{equation}
\centering
\alpha \texttt{/} \beta
\Longrightarrow
\begin{array}{l}
	\mbox{\texttt{project([fields,)}} \vartheta \mbox{\texttt{];}} \\ \qquad
 	\mbox{\texttt{select(isInScope(l.scope, scopePrefix(\textbf{n}, r.scope));}}
 	\\ \qquad \qquad
 	\mbox{\texttt{join([documentId], [documentId], [fields];}} \alpha ; \beta
 	\mbox{\texttt{)))}};
\end{array}
\label{eq:translation:singleStep}
\end{equation}

\subsection{Variables}
\label{sect:translation:variables}
When a variable is declared, be it in the form of a \textit{for} clause,
\textit{let} clause or a \textit{declare variable} expression, the right side
of such an expression is transformed into algebra. In the case of the
variable being an iterator variable, the $\vartheta$ of this tree will, as well
as possible references made to other iterator variables, include the variable
to which this tree will be bound. The tree is then stored in the symbol
table by means of the previously mentioned \textbf{put(}\textit{X},
$\alpha$\textbf{)}. Rule \ref{eq:translation:variableNotIterator} shows the
translation of the reference of a variable that is not an iterator.

\begin{equation}
\centering
\frac{\displaystyle \textit{X} \notin \vartheta}{\displaystyle \textit{\$X}}
\Longrightarrow
\begin{array}{l}
	\mbox{\textbf{get(}\textit{X}\textbf{)}}
\end{array}
\label{eq:translation:variableNotIterator}
\end{equation}

When entering a predicate of a path expression, the algebra corresponding to
the steps up to the predicate will be stored in the symbol table as the
iterator variable \textit{``.''} (the context item). The context node may
implicitly be used with a relative path expression within the predicate. Rule
\ref{eq:translation:iteratorVariable} shows the translation of using a
iterator variable. The same principles is behind the inference rule for using
the context node within a path expression predicate.

\begin{equation}
\centering
\frac{\displaystyle \textit{X} \in \vartheta}{\displaystyle \textit{\$X}}
\Longrightarrow
\begin{array}{l}
	\mbox{\texttt{project([Xnumb = counter(), fields,}} \vartheta
	\mbox{\texttt{];}}
	\\ \qquad
	\mbox{\textbf{get(}\textit{X}\textbf{)} \texttt{)}}
\end{array}
\label{eq:translation:iteratorVariable}
\end{equation}


Rule \ref{eq:translation:iteratorVariable} also shows the introduction of a
-\verb!numb! field. This field contains information about which iteration the
instance of the variable in that row belongs to.

\subsection{Predicates}
As described in section \ref{sect:theory:xqueryPredicates}, predicates are used
to restrict the items returned from the expression it is assigned to. The
predicate expression returns either true/false or an integer value. Rule
\ref{eq:translation:predicate} shows how predicates, which neither implisitly
nor explicitly makes use of the context item, are translated into MQL.
Translation of predicates which employs the context item are treated as an
interation expression (section \ref{sect:translation:iteration}).

\marginpar{\underline{\Large TODO} \footnotesize litt feil bruk av $\vdash$
her.. kan det tillates?}
\begin{equation}
\centering
%\begin{array}{c}
	\frac{\displaystyle \textbf{put(ctx,}\alpha\textbf{)}\vdash
	\textbf{ctx}\notin\beta . \vartheta}{\alpha \textit{[}\beta\textit{]}}
%	\\
	\Longrightarrow 
%	\\
	\begin{array}{l}
		\mbox{\texttt{project([fields = l.fields,}} \vartheta \mbox{\texttt{];}} 
		\\ \qquad \mbox{\texttt{select(}\textbf{check}\textit{;}} \\ \qquad \qquad
		\mbox{\texttt{cross(}} 
		\alpha \texttt{; }
		\beta \mbox{\texttt{)))}}
	\end{array}
%\end{array}
\label{eq:translation:predicate}
\end{equation}


To be more easily read, in rule \ref{eq:translation:predicate} \textbf{check}
replaces \texttt{ifThenElse(isNumber(r.value), scopeIndex(scope, r.value),
xqBool(r.value))}. These functions are explained in section
\ref{sect:method:marsAddedOperators}. The use of \texttt{scopeIndex(scope,
pred)} may however not be valid in cases where there are multiple predicates,
or when the predicate is assigned to a path expression which utilises some
axes, e.g. \textit{decendant-or-self}. This will be discussed in section
\ref{sect:discussion:predicates}.


\subsection{Iteration Expressions}
\label{sect:translation:iteration}

for \$x in expr1 where expr2 return expr3  -->

scope.put(expr1)

project(~xnumb, ~number
    select(isNull(xnumb), true, eq(number, xnumb))             
    //only if \$x is  used in expr3 
    select(isNull(w.xnumb), true, eq(number, w.xnumb))         
    // only if \$x is used in expr2 
        cross(                     
    // alt fra venstre, varrefs venstre og høyre. (ikke velge value fra h\o yre(
            cross(                                             
          //  n\aa r den bare har value) gir enn\aa~ ordentlig kryss?
                project(number= count(), 
                    expr1)
                expr3
        select(xqueryBoolean(value),                                  
            expr2

\subsection{Comparison Operators}
\label{sect:translation:compOps}

\begin{itemize}
  \item husk at det finnes lambda
  \item project ikke select
  \item fordi noe over vil kanskje ha negativen\ldots f.eks ifthenelse og not()
  \item crazy
\end{itemize}
op $\in$ {< | > | = | <= | >= | != | and }        // and ->
and(xqueryBoolean(l.val), xqueryBoolean(r.val))

\subsection{If then else}
\label{sect:translation:ifThenElse}
if
now

then

else

\subsection{Examples}
\label{sect:translation:markxremoveExamples}
\begin{itemize}
\item steg for ste
\item litt stort eksempel m predikat sammenligning og for
\item kanskje en liten where
\end{itemize}
----------------------------------------------------------------------------------

\begin{itemize}
\item merke bruk av variabler
\item for er alltid = kryss
\item vaske merker mot riktig iterasjon -> ergo er iterasjon virkelig iterasjon
\item blir det noe av det samme for predikater?
\end{itemize}
\section{Loop Lifting}
\label{sect:trans:loop_lifting}
\label{sect:theory:loop_lifting}
Loop lifting is a method of translating XQuery iteration expressions into relational algebra. The method was
developed by Torsten Grust and Jens Teubner and originally presented in \cite{pathfinder_mothertongue}. It is a
part of the Pathfinder project\cite{pathfinderHome} (see section \ref{sect:theory:pathfinder}).

In this section we will present Loop Lifting mainly based on the two articles \cite{pathfinder_purelyRelational}
and \cite{pathfinder_mothertongue}. The articles present the method for a subset of XQuery Core (Pathfinder
rewrites queries to Core, see section \ref{sect:theory:pathfinder}), of which we will only present the elements
relevant in a comparison between Loop Lifting and the Tainting Dependencies method. Thus, the translation of path
expressions and XML-element construction will not be handled, as pathfinder's XML-tree representation(section
\ref{sect:theory:pathfinder}) is incompatible with MARS.

Pathfinder generates relational operator directed asyclic graphs (DAGs) rather than operator trees. The Loop
Lifting method does not require such a structure, but as we will see, it will gain advantage by it, as much
evaluation relies on earlier evaluations. 

Accompanying the translation method is also methods for analysis, simplification and optimisation of the generated
relational algebra, such as the Peep-Hole plan simplification\cite{pathfinder_purelyRelational}.


\subsection{Operators}
\label{sect:trans:ll:Operators}
Loop-lifting utilises a set of relational algebra operators, out of which the ones used in this chapter is
presented in table \ref{tab:trans:ll:Operators}.
\begin{table}[h]
\centering
\begin{tabular}{l|l} 
$\pi_{a_{1}:b_{1},\ldots,a_{n}:b_{n}}$ 	& projection and renaming	\\ 	\hline
$\sigma_{a}$					   		& selection             	\\ 	\hline
$\dot\cup$ 							& disjoint union			\\	\hline
$\times$								& cartesian product			\\	\hline
$\bowtie_{a=b}$							& equi-join					\\ 	\hline
$\varrho_{b:(a_{1},\ldots,a_{n})/p}$	& numbering operator		\\	\hline
$\circledcirc_{b:(a_{1},\ldots,a_{n})}$	& $n$-ary arithmetic/comparison operator $\circ$ \\ \hline
\scriptsize \begin{tabular}{c|c} $a$& $b$\\\hline\end{tabular} & literal table
\end{tabular}
\caption[of the Pathfinder relational algebra]{Operators of the Pathfinder relational algebra. $a$, $b$ and $p$
represents attributes}
\label{tab:trans:ll:Operators}
\end{table}

Most of the operators are quite standard, and can easily be understood by comparing with the operators from
general relational algebra (section \ref{sect:theory:relAlg}) and MQL (section \ref{sect:method:marsOperators}).

Only a very restricted selection
is utilised, written $\sigma_{a}$, which only returns tuples satisfying $a \neq 0$.  Considering the numbering
operator, $p$ denotes the partitioning attribute, $(a_{1},\ldots,a_{n}$ the attributes to be sorted on and $b$ is
an added attribute holding the result of the
numbering (equal to the proposed \textsf{numberate}-operator of MQL, section \ref{sect:method:marsAddedOperators}).
{\scriptsize{\begin{tabular}{c|c}$a$&$b$\\\hline\end{tabular}}} represents the creation of a relation with
attributes $a$ and $b$.

Operator $\circledcirc_{b:(a_{1},\ldots,a_{n})}$ will evaluate the arithmetic/comparison expression $a_{1} \circ
\ldots \circ a_{n}$ and place the result in $b$. Where $\circ \in \left\{ +,- , <, =, \ldots  \right\} $.


\subsection{Basics}
\label{sect:trans:ll:Basics}
XQuery expressions evaluate to finite, ordered sequences of items. As a sequences are one-dimensional, it can be
represented by a single relation where each tuple encodes a sequence item. The order of the sequence is
maintained by an attribute \textit{pos}. The value of the item is held in an attribute \textit{item}. 

During this section concerning Loop Lifing, variables, expressions and scopes is denoted like this (ref. section
\ref{sect:theory:xquery:Flwor}):
\[
s \left\{
\begin{array}{l}
\qquad \qquad \quad \vdots \\
\mbox{\texttt{for \$}}v_{0}\mbox{\texttt{ in }} e_{0} \mbox{\texttt{ return}} \\
\quad s_{0} \left\{ e_{0}' \right. \\
\qquad \qquad \quad \vdots
\end{array}
\right.
\]

Generally, a scope $s_{x \cdot y}$ identifies the $y$th child scope of scope $x$, $x \in \left\{
\mathbb{N}\right\}, y \in \left\{ \mathbb{N} \right\}$. Expression $e_{x\cdot y}$ evaluates to an iterator sequence
and is bound to the variable $v_{x \cdot y}$. $e_{x \cdot y}'$ constitutes the coresponding iterator body, and $I_{x \cdot y}$ the whole iterator expression.

$q_{x}(e)$ is used to denote the relational representation of expression $e$ in scope $s_{x}$.


\subsection{Constant Subexpressions}
\label{sect:trans:ll:ConstExprs}

For a iterator expression $i_{x}$ with $n$ iterations there exists a relation $loop_{x}$, consisting of a
single column, \textit{iter}, with values 1,2,\ldots,$n$. In the outermost scope, $loop$ has a single tuple with
the value 1.

A constant value $c$ in scope $s_{x}$ is \textit{lifted} like this:
\begin{equation}
q_{x}(c) =  loop_{x} \times \mbox{\scriptsize \begin{tabular}{c|c} \textit{pos}&\textit{item} \\
\hline 1 & \textit{c}
\end{tabular}}
\label{eq:ll:constLoopLift}
\end{equation}

A tuple ($iter,pos,item$) in a loop lifted relation for subexpression $e_{x}'$ can be understood as during the
$iter$th iteration, the item in position $pos$ in $e_{x}'$ has the value $item$.

\subsection{Bound Variables}
\label{sect:trans:ll:boundVar}

An iterator sequence expression $e_{x \cdot y}$ is evaulated in scope $s_{x}$. This sequence is then iterated over
and each item is successively bound to the iterator variable $v_{x \cdot y}$. The evaluation of $e_{x \cdot y}'$
is in scope $s_{x \cdot y}$ and may utilise these bindings. 

Considering this, a representation of $v_{x \cdot y}$ in scope $s_{x \cdot y}$ may therefore be calculated by
retaining the values of $q_{x}(e_{x \cdot y})$, introducing a $iter$ attribute with consecutive numbers and
holding the $pos$ attribute to the constant value 1. In terms of algebra, the representation of $v_{x \cdot y}$ is
computed like this:
\begin{equation}
q_{x \cdot y}(\mbox{\texttt{\$}}v_{x \cdot y}) = \mbox{\scriptsize \begin{tabular}{c} $pos$ \\\hline 1
\end{tabular}} \times \pi_{iter:inner,item}(\varrho_{inner:(iter,pos)}(q_{x}(e_{x \cdot y})))
\label{eq:ll:qxy_vxy}
\end{equation}
The introduction of the $inner$ attribute is used to denote evaluation of the loop in scope $s_{x \cdot y}$. The
$iter$ attribute of $q_{x}(e_{x \cdot y})$ can be viewed as an atttribute $outer$, as it denotes the iterations in
the outer loop of scope $s_{x}$.

Loop lifting requires maintenance of a $loop$ relation to ensure independent iterations. The iterator body in
scope $s_{x \cdot y}$ needs to be evaluated once for each binding of the iterator variable $v_{x \cdot y}$.
Thus, the $loop$ relation needs to be redifined based on $q_{x \cdot y}(v_{x \cdot y})$:
\begin{equation}
loop_{x \cdot y} = \pi_{iter}(q_{x \cdot y}(v_{x \cdot y}))
\label{eq:ll:loopxy}
\end{equation}


\subsection{Free Variables}
\label{sect:trans:ll:freeVar}

XQuery expressions may use any iterator variable bound in enclosing scopes. That is, $v_{x}$ bound in
scope $s_{x}$ may also be referred to within any of its child scopes. When looking at one of these child scopes,
$s_{x \cdot y}$, by itself, the variable $v_{x \cdot y}$ appears to be a free variable.

Consider a iterator expresion $I_{x \cdot y}$ within another iterator expression $I_{x}$, both with iterator
sequences of length two. If $v_{x}$ is referred to within scope $s_{x \cdot y}$, from $s_{x \cdot y}$'s point of
view, $v_{x}$ is free. For each binding of $v_{x}$ in the \textit{outer} iteration expression, two
evaluations of the \textit{inner} iteration expresion occur. A relation capturing the relationship between number
of iterations of these two iterator expressions can be defined like this:

\begin{center}
\begin{tabular}{|c|c|}\hline
\textit{outer}	& \textit{inner} 	\\ \hline
1				& 1		\\ \hline
1				& 2		\\ \hline
2				& 3		\\ \hline
2				& 4		\\ \hline
\end{tabular}
\end{center}

Where a tuple $(outer, inner)$ is read as for the $inner$th iteration of the inner iterator expression, the outer
iterator expression is in its $outer$th iteration. This relation is called $map_{x, x\cdot y}$ as it maps
representations between scopes $s_{x}$ and $s_{x \cdot y}$. It can be calculated like this:
\begin{equation}
map_{x, x\cdot y} = \pi_{outer:iter,inner}(\varrho_{inner:(iter,pos)}(q_{x}(e_{x \cdot y})))
\label{eq:ll:mapx_xy}
\end{equation}

With this relationship defined it is now possible to represent the free variable $v_{x}$ in the scope $s_{x \cdot
y}$ with the help of an equi-join:
\begin{equation}
q_{x \cdot y}(\mbox{\texttt{\$}}v_{x}) = \pi_{iter:inner, pos,
item}(q_{x}\left(\mbox{\texttt{\$}}v_{x})\bowtie_{iter=outer} map_{x, x \cdot y}\right)
\label{eq:ll:qxy_vx}
\end{equation}

\subsection{Mapping Back}
\label{sect:trans:ll:mappingBack}

All steps and equations this far have been helpful to represent sequences and variables in a lower scope level. But
the result of a query will have to be in form of its representation in the outermost scope $s$. So a way to
represent an expression $e_{x,y}'$ in its scope's parent scope $s_{x}$ is needed. Once again the $map$ relation
may be of use, combined with an equi-join:
\begin{equation}
q_{x}(e_{x \cdot y}') =
\begin{array}{l}
 \pi_{iter:outer, pos:pos1, item}(\\ \qquad\varrho_{pos1:(iter,pos)/outer}(q_{x \cdot
y}(e_{x \cdot y}')\bowtie_{iter = inner}map_{x, x \cdot y}))
\end{array}
\label{eq:ll:qx_exymark}
\end{equation}


\subsection{Other Expression Types}
\label{sect:trans:ll:OtherExpr}

The sequence construction $e_{1}$\texttt{, }$e_{2}$ is essentially a disjont union of the
relational representations of the expressions, that is, $q_{x}(e_{1})$ and $q_{x}(e_{2})$. By temporarily adding a
attribute $ord$ to these relations before a renumbering of the result with $\varrho$, the proper ordering of the
sequence is aquired. Construction of sequences can therefore be expressed like this:
\begin{equation}
q_{x}(e_{1}\mbox{\texttt{, }}e_{2})=
\begin{array}{l}


\pi_{iter,pos:pos1,item}
\left( \right.\\ \qquad

\varrho_{pos1:(ord,pos)/iter}
	\left( \right. \\ \qquad \qquad
	\left. \left.
		\left(
		\frac{ord}{1} \times q_{x}(e_{1})
		\right)
		\dot\cup
		\left(
		\frac{ord}{1} \times q_{x}(e_{2})
		\right)		
	\right)
\right)
\end{array}
\label{eq:ll:secuence}
\end{equation}

The $\circledcirc$ operator meets the requirement of evaluating comparison and arithmetic operations on atomic
values. Given two XQuery values $e_{1}$ and $e_{2}$ in multiple iterations, with relational representations as before,
the expression $e_{1}$ \texttt{ + } $e_{2}$ can be translated by first joining $q_{x}(e_{1})$ and $q_{x}(e_{2})$
on their iteration number, i.e. $iter$. Then, for each tuple, store the sum of the values of both of the $item$
attributes, before cleaning up the resulting relation with a project. Expressed as an equation, the translation of
sum expressions looks like this:
\begin{equation}
q_{x}(e_{1} \mbox{\texttt{ + }} e_{2}) =
\begin{array}{l}
\pi_{iter,pos,item:res}\left(\right. \\ \qquad
\oplus_{res:(item,item')}
\left( \right. \\ \qquad \qquad

	q_{x}(e_{1})
	\bowtie_{iter = iter'}
	 \\ \qquad \qquad \qquad
	\left.\left(\pi_{iter':iter, item':item}(q_{x}(e_{2}))
	\right)
\right)
\end{array}
\label{eq:ll:sumexpr}
\end{equation}

The \texttt{if(}$e_1$\texttt{) then }$e_2$\texttt{ else }$e_3$, is one of the more complex translations of loop
lifting. First the boolean expression $e_1$ is compilated. The result is split into two new loop relations,
$loop1$ and $loop2$, which uses selection on all $true$ and $false$ values respectively. $loop2$ is used as current
$loop$ relation for the compilation of $e_2$ and $loop3$ as $loop$ relation for the mapping of $e_3$. A equi-join with
their corresponding $loop$ relation on $iter$ will filter out all unnnecesary iterations. The result is the union
of both branches.

\begin{equation}
\begin{array}{l}
q_{x}(\mbox{\texttt{if }}e_1\mbox{\texttt{ then }}e_2\mbox{\texttt{ else }}e_3) = \\ \qquad
\begin{array}{l}
\pi_{iter,pos,item}(q_{x}(e_2)\bowtie_{iter=iter'}(\pi_iter':iter(loop2))\,\dot\cup \\ \qquad
\pi_{iter,pos,item}(q_{x}(e_3)\bowtie_{iter=iter'}(\pi_iter':iter(loop3)) \\
loop2 = \pi_{iter}(\sigma_{item=TRUE}(q_{x}(e_1))) \\
loop3 = \pi_{iter}(\sigma_{item=FALSE}(q_{x}(e_1))) \\
\end{array}
\end{array}
\label{eq:ll:ifthenelse}
\end{equation}

\subsection{Example}
\label{sect:trans:ll:example}

Only looking at equations may be a bit too abstact to fully understand Loop Lifting. To concretise we will present
a simple example of evaluating a query with the method and show intermediate results. The naming of expressions,
scopes and variables will, where possible, be the same as earlier in this section. This query is the basis of this
evaluation:

\begin{figure*}[h!]
\centering
\begin{math}
s\left\{
\begin{array}{l}
\mbox{\texttt{for \$v0 in (10,20) return}} \\ \;
s_{0}\left\{
\begin{array}{l}
\mbox{\texttt{(\$v0, for \$v00 in (7,8,9) return}} \\ \;
s_{0,0}\left\{ \mbox{\texttt{\$v0 + \$v00)}}\right.
\end{array}
\right.
\end{array}
\right.
\end{math}
\end{figure*}

The goal of the evaluation is, after all other calculations, to have a representation of $e_{0}'$ in scope $s$,
that is, $q(e_{0}')$. This is done by nesting inwards until the deepest scope, while calculating needed helping
relations on the way, before evaluating the subexpressions one by one as one nests outwards until the outermost
scope.

Firstly a representation of the outermost loop is needed. With the help of equation \ref{eq:ll:constLoopLift} we
find a representation of \texttt{(10, 20)} in scope $s$, $s(e_{0})$. Then, employing equation \ref{eq:ll:qxy_vxy}
yields \texttt{\$v0} in scope $s_{0}$, the result of which is shown in figure \ref{fig:trans:ll:q0_v0}.
$loop_{0}$ and $map_{\, ,0}$ can now be calculated by using equations \ref{eq:ll:loopxy} and \ref{eq:ll:mapx_xy}
and are shown in figure \ref{fig:trans:ll:loop0} and \ref{fig:trans:ll:map_0} respectevely (remember
$loop$ consists of a single tuple with value 1).

\begin{figure}[!h]
\centering
\subfigure[$q_{0}($\texttt{\$v0}$)$]{
%q0_v0
\begin{tabular}{|c|c|c|} \hline
$pos$	& $iter$	& $item$ \\ \hline
1		& 1			& 10 	\\ \hline
1		& 2			& 20 	\\ \hline
\end{tabular}
\label{fig:trans:ll:q0_v0}
%\caption{$q_{0}($ \texttt{\$v0} $)$}
}
\qquad \quad
\subfigure[$loop_{0}$]{

\begin{tabular}{|c|} \hline
$iter$ \\\hline
1 \\\hline
2 \\\hline
\end{tabular}
\label{fig:trans:ll:loop0}
%\caption{$loop_{0}$}
}
\qquad \quad
\subfigure[$map_{\, ,0}$]{
\begin{tabular}{|c|c|} \hline
$outer$ & $inner$ \\ \hline
1 & 1 \\ \hline
1 & 2 \\ \hline
\end{tabular}
\label{fig:trans:ll:map_0}
%\caption{$map_{\, ,0}$}
}
\label{fig:trans:ll:outerIntermediate}
\caption{Outer loop intermediate results}
\end{figure}

Before we evaluate the sequence expression in scope $s_0$, we need to evaluate the inner \texttt{for} loop. By the
same measure as with the outer loop we first calculate $q_{0\cdot 0}($\texttt{\$v00}$)$, $loop_{0 \cdot 0}$ and
$map_{0, 0 \cdot 0}$. The results are shown in figure \ref{fig:trans:ll:innerIntermediate}.

\begin{figure}[!h]
\centering
\subfigure[$q_{0\cdot 0}($\texttt{\$v00}$)$]{
\begin{tabular}{|c|c|c|} \hline
$pos$	& $iter$	& $item$ \\ \hline
1		& 1			& 7 	\\ \hline
1		& 2			& 8 	\\ \hline
1		& 3			& 9 	\\ \hline
1		& 4			& 7 	\\ \hline
1		& 5			& 8 	\\ \hline
1		& 6			& 9 	\\ \hline
\end{tabular}
\label{fig:trans:ll:q00_v00}
%\caption{$q_{0}($ \texttt{\$v0} $)$}
}
\qquad
\subfigure[$loop_{0 \cdot 0}$]{
\quad
\begin{tabular}{|c|} \hline
$iter$ \\\hline
1 \\\hline
2 \\\hline
3 \\\hline
4 \\\hline
5 \\\hline
6 \\\hline
\end{tabular}
\label{fig:trans:ll:loop00}
%\caption{$loop_{0}$}
\quad
}
\qquad 
\subfigure[$map_{0, 0 \cdot 0}$]{
\begin{tabular}{|c|c|} \hline
$outer$ & $inner$ \\ \hline
1 & 1 \\ \hline
1 & 2 \\ \hline
1 & 3 \\ \hline
2 & 4 \\ \hline
2 & 5 \\ \hline
2 & 6 \\ \hline
\end{tabular}
\label{fig:trans:ll:map0_00}
%\caption{$map_{\, ,0}$}
}
\caption{Inner loop intermediate results \label{fig:trans:ll:innerIntermediate}}
\end{figure}

To be able to calculate the sum-expression, $e_{0 \cdot 0}'$, we first need a representation of the variable
\texttt{\$v0} in scope $s_{0 \cdot 0}$. As this variable is a free variable in this scope, this is done by
applying equation \ref{eq:ll:qxy_vx} on $q_{0}($\texttt{\$v0}$)$ (figure \ref{fig:trans:ll:q0_v0}). This
result is shown in figure \ref{fig:trans:ll:q00_v0}. Now that we have both \texttt{\$v0} and \texttt{\$v00}
expressed in scope $s_{0 \cdot 0}$, we can employ equation \ref{eq:ll:sumexpr} to sum the two
variables together. The resulting relation can be seen in figure \ref{fig:trans:ll:q00_e00m}.

\begin{figure}[!h]
\centering
\subfigure[$q_{0\cdot 0}($\texttt{\$v0}$)$]{
\qquad 
\begin{tabular}{|c|c|c|} \hline
$pos$	& $iter$	& $item$ \\ \hline
1		& 1			& 10 	\\ \hline
1		& 2			& 10	\\ \hline
1		& 3			& 10	\\ \hline
1		& 4			& 20	\\ \hline
1		& 5			& 20	\\ \hline
1		& 6			& 20	\\ \hline
\end{tabular}
\label{fig:trans:ll:q00_v0}
\qquad 
%\caption{$q_{0}($ \texttt{\$v0} $)$}
}
\subfigure[$q_{0\cdot 0}(e_{0\cdot0}')=q_{0\cdot 0}($\texttt{\$v0 + \$v00}$)$]{ 
\qquad 
\begin{tabular}{|c|c|c|} \hline
$pos$	& $iter$	& $item$ \\ \hline
1		& 1			& 17 	\\ \hline
1		& 2			& 18 	\\ \hline
1		& 3			& 19 	\\ \hline
1		& 4			& 27 	\\ \hline
1		& 5			& 28 	\\ \hline
1		& 6			& 29 	\\ \hline
\end{tabular}
\label{fig:trans:ll:q00_e00m}
\qquad
}
\caption{Innermost expression intermediate results \label{fig:trans:ll:innerExpr}}
\end{figure}

The result of the summation, $q_{0\cdot 0}(e_{0\cdot0}')$ is expressed in scope $s_{0 \cdot 0}$ and will have to
be mapped up to scope $s_{0}$. This is done with help from equation \ref{eq:ll:qx_exymark} and $map_{0, 0\cdot 0}$
which we calculated earlier, and the result can be seen in figure \ref{fig:trans:ll:q0_e00m}. With
$q_{0}(e_{0\cdot0}')$ evaluated, and with $q_{0}($\texttt{\$v0}$)$ from earlier, the sequence building can be completed. This operation requires
equation \ref{eq:ll:secuence}, and yields the relation shown in figure \ref{fig:trans:ll:q0_e0m}.

\begin{figure}[!h]
\centering
\subfigure[$q_{0}(e_{0\cdot0}')$]{
%q0_v0
\begin{tabular}{|c|c|c|} \hline
$pos$	& $iter$	& $item$ \\ \hline
1		& 1			& 17 	\\ \hline
2		& 1			& 18 	\\ \hline
3		& 1			& 19 	\\ \hline
1		& 2			& 27 	\\ \hline
2		& 2			& 28 	\\ \hline
3		& 2			& 29 	\\ \hline
\end{tabular}
\label{fig:trans:ll:q0_e00m}
%\caption{$q_{0}($ \texttt{\$v0} $)$}
}
\qquad
\subfigure[$q_{0}(e_{0}')$]{
\begin{tabular}{|c|c|c|} \hline
$pos$	& $iter$	& $item$ \\ \hline
1		& 1			& 10 	\\ \hline
2		& 1			& 17 	\\ \hline
3		& 1			& 18 	\\ \hline
4		& 1			& 19 	\\ \hline
1		& 2			& 20 	\\ \hline
2		& 2			& 27 	\\ \hline
3		& 2			& 28 	\\ \hline
4		& 2			& 29 	\\ \hline
\end{tabular}
\label{fig:trans:ll:q0_e0m}
%\caption{$loop_{0}$}
}
\qquad
\subfigure[$q(e_{0}')$]{
\begin{tabular}{|c|c|c|} \hline
$pos$	& $iter$	& $item$ \\ \hline
1		& 1			& 10 	\\ \hline
2		& 1			& 17 	\\ \hline
3		& 1			& 18 	\\ \hline
4		& 1			& 19 	\\ \hline
5		& 1			& 20 	\\ \hline
6		& 1			& 27 	\\ \hline
7		& 1			& 28 	\\ \hline
8		& 1			& 29 	\\ \hline\end{tabular}
\label{fig:trans:ll:q_e0m}
%\caption{$map_{\, ,0}$}
}
\label{fig:trans:ll:endExample}
\caption{Intermediate and final result}
\end{figure}

Finally the built sequence will have to be expressed in terms of scope $s$. Achieving this only requires the use
of equation \ref{eq:ll:qx_exymark} one last time in combination with $map_{ ,0}$. The complete result of the query
is shown in figure \ref{fig:trans:ll:q_e0m}.


\section{Combining TH00Ze c00l methodz}
\begin{itemize}
  \item Hva skjer om man modifiserer markXremove med \aa~alltid krysse ting med
  noe ala ``loop''?
  \item bare at $loop_{n} = counter(tilordninga_{n}) \times loop_{n-1}$
  (alts\aa~ kolonner for alle iterator variable som eksisterer i scopet)
  \item for-delen blir da ikke et kryss, men en equijoin\ldots
  \item dette minner om loop lifting\ldots men man har ikke mapz.. trenger ikke
  \aa~lage de for hver gang man skal ut av et for-scope
  \item navngi scopez med tall.. g\aa r greit siden s\o skenscopes ikke lever
  samtidig
\end{itemize}
