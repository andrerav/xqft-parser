\section{Tree parsing}
\label{sect:method:tree_parsing}
In section \ref{sect:theory:parser:tree_parsing} some common design patterns
for tree parsing were described. Based on some important traits of XQuery, such
as being an \textit{orthogonal language} (see section
\ref{sect:theory:xquery:basics}), it seems natural to employ the
\textit{context-sensitive visitor pattern}, as described in section
\ref{sect:theory:contextVisitorPattern}. The benefits of this decision is the
possibility to create specialised visitors for certain tasks, such as rewrite
visitors for preprocessing the syntax tree (hinted upon in section
\ref{definition:normalisation}), or predicate visitors, or any situation where
propagation of context/state information is required.

The implementation of the context-sensitive visitor pattern is described in
section \ref{sect:impl:context_sens_visitor}. 
