\section{Licensing}
\label{sect:method:licensing}
One requirement for the parser is a liberal open source software license.
Several such liberal licenses exist, however here we shall concentrate on a
short selection of popular and widely adopted licenses. These have been gathered
from the Open Source Initiative (OSI) website\cite{osi_website}, from the
category
``\emph{Licenses that are popular and widely used or with strong communities}''.
\subsection{GPL -- General Public License}
The GPL license is a 

The GNU General Public License (GNU GPL or simply GPL) is a widely used free
software license, originally written by Richard Stallman for the GNU project.
It is the license used by the Linux kernel. The GPL is the most popular and
well-known example of the type of strong copyleft license that requires derived
works to be available under the same copyleft. Under this philosophy, the GPL
is said to grant the recipients of a computer program the rights of the free
software definition and uses copyleft to ensure the freedoms are preserved,
even when the work is changed or added to. This is in distinction to permissive
free software licences, of which the BSD licenses are the standard examples.   
     
\subsection{Apache License, Version 2.0}
\subsection{BSD}
