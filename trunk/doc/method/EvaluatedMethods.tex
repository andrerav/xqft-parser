\section{Evaluated methods for manual tree parsing}
\label{sect:method:evaluated_methods}

\subsection{Manual tree walker (non-visitor)}
\label{sect:method:evaluated_methods:manual_walker}
Motivation: one of the simplest forms of tree parsing
Problem: does not scale well with regards to maintainability

One of the simplest forms of tree parsing is the ``manual non-visitor''
methodology.

This way of parsing a tree structure is particularly suited for trivial trees
where context is not important. However, for operations that are contextually 
sensitive -- such as context-sensitive code generation, as is the case for this
project -- this methodology can be difficult to maintain as the problem domain 
expands. This particular problem is rooted in that any contextual information
must be extracted from parent nodes, and is thus not available implicitly.

Another problem with this technique is that the data structure (the tree) and
the logic to parse it will often be tied together quite closely, creating
another maintanence problem when/if the tree structure changes to
accommodate new specifications.

\subsection{Visitor pattern}
\label{sect:method:visitorPattern}
Motivation: clean separation of logic and data structure 
Problem: overly complex visitors

The visitor pattern is a common design pattern for parsing tree structures. 

\subsection{Context-sensitive visitor pattern}
\label{sect:method:contextVisitorPattern}

referere til Island Grammars


Motivation: avoid complex state mechanisms
Problem: design decisions
Explain: why return visitor instead of nodes/tree?

skviik 1: Assume that expressions are mutable / have side-effects

skviik 2: Builds tree outside instead of traditional recursive building