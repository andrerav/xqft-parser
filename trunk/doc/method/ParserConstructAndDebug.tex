\section{Parser construction}
Writing a parser from scratch was ruled out early for being too time consuming.
Instead it was decided to use tools for compiler and parser construction to
generate a parser from the XQuery 1.0 and XPath 2.0 grammar specifications
developed by the W3C.

\subsection{Evaluated alternatives}
\subsubsection{JFlex/CUP}
JFlex and CUP is a versatile combination consisting of JFlex which is a lexer
generator, and CUP which is a parser generator. These tools can be interfaced to
generate a complete parser with a separate lexical analyzer.

JFlex and CUP produces only LALR parsers, and since the W3C has specified an
LL(1) grammar for XQuery 1.0 and XPath 2.0, the combination of JFlex and CUP was
rejected from this project.

\subsubsection{JavaCC}
JavaCC could have been a viable alternative as it produces LL(k) parsers,
however compared to Antlr its grammar specification syntax deviated so much
from the W3C EBNF syntax, that grammar would have had to be extensively
rewritten.

\subsubsection{Antlr}
ANTLR is a renowned tool for parser generation, and can generate LL(*) parsers.
Additionally, Antlr accepts a grammar specification syntactically very close to
the EBNF used by the W3C. This is the parser generater chosen for our project,
based on the criteria outlined in this section.

\subsection{Limitations in ANTLR}
\subsubsection{Unicode}
\label{sect:parserconstructanddebug:limitations}
It is important to note, however, that ANTLR has limited support for unicode.
In this project this implies that our parser will not accept unicode characters
in the range from and above 0x10000. This will exclude the supplementary
multilingual plane (SMP) range of unicode characters, as well as the
supplementary ideographic plane (SIP) and the supplementary special-purpose
plane (SSP). These are seldomly used, but this is an important limitation
nonetheless. The Antlr developers have indicated that support for this unicode
range will be added in future versions of Antlr.

As a remedy for this situation it is possible to couple an external lexer with
ANTLR which will accept unicode characters in the above mentioned character
ranges. For the sake of simplicity this has not been further investigated nor
implemented in this project.

\section{Debugging}
There are several approaches to debugging. Antlrworks \cite{antlrwrks00} is a
simple tool for writing, testing and debugging Antlr grammars. The debugging
interface is useful in that it draws a realtime step-by-step parse tree as the
input is being parsed, as well as displaying a list of parser events. ANTLRWorks also helps eliminate grammar nondeterminisms by drawing the syntax diagram associated with a grammar and highlighting nondeterministic paths. A big drawback, though, is as the grammar grows bigger, the application tends to become unstable.

When working properly, ANTLRWorks is a great tool for debuging the parser, but no so much for the lexer. It will show the tokens returned to the parser, but no information or decision path indicating the reason for returning just that token. A more appropriate tool for such a task would be the standard debuging environment in Eclipse (\cite{eclipse}).
