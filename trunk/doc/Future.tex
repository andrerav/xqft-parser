\chapter{Future Work}
\label{chapter:future}

TODO: jeg tror jeg var full da jeg skrev det her

The outcome of this project is a fairly well-defined method of translation,
there is still headroom for improvement and continued research. In particular,
this relates to performance benchmarking, simplifications and optimisations,
and improvement of XQuery feature support.

We propose the following research related to performance benchmarking:
\begin{itemize}
  \item \textbf{Execute the XMark benchmark suite and compare results}. Given that FAST
  develops a working implementation of an MQL processor some time in the future, it could be interesting to extend or rewrite the
  prototype developed here, and execute the
  XMark\footnote{\htmladdnormallink{http://www.xml-benchmark.org/}{http://www.xml-benchmark.org/}} 
  benchmark suite and compare the result to other existing implementations
  \item \textbf{Execute algebra generated by TD on MonetDB}. If a working MQL
  processor can not or will not be developed in the foreseeable future, then
  the prototype and the rules of Tainting Dependencies (TD) may be modified
  and interfaced with MonetDB, and a benchmark could be performed on this
  combined system\footnote{However, the translation of path expressions may
  prove difficult since TD does not support staircase join}
  \item \textbf{Further research on optimisation and simplification of the TD
  methodology}. This thesis suggests some simplifications (section
  \ref{sect:trans:TD:simplifications}), however we suspect there are still
  substantial gains to be made on this account
  \item \textbf{Investigate applicability of parallell execution of subtrees in the
  algebra tree}. MQL supports threading/branching within the language itself,
  and this may be exploited to parallellise the execution of algebra and boost
  performance
\end{itemize}

Furthermore, we propose the following improvements:
\begin{itemize}
  \item Improved support for interfacing with the external environment (as
  described in section \ref{sect:disc:ctxItem})
  \item Improved support of XQuery functionality, including: full-text
  operations, ordered/unordered mode, binary operators not currently supported,
  multiple order specifications for the orderby-clause, and user-defined
  functions as well as built-in XQuery functions (within the \texttt{fn} namespace)
  \item Implementation of the full XQuery type system into TD, which may also
  possibly be exploited for optimisations
  \item Translation of \texttt{if..then..else} expressions can be optimised
  further by assuming execution of DAGs and not trees, as described in section
  \ref{sect:disc:optimisations}
\end{itemize}