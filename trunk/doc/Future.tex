\chapter{Future Work}
\label{chapter:future}
The outcome of this project is a fairly well-defined method of translation,
there is still headroom for improvement and continued research. In particular,
this relates to performance benchmarking, simplifications and optimisations,
and improvement of XQuery feature support.

We propose the following research related to performance benchmarking:
\begin{itemize}
  \item Given that FAST develops a working implementation of an MQL processor
  some time in the future, it could be interesting to extend or rewrite the
  prototype developed here, and execute the
  XMark\footnote{\htmladdnormallink{http://www.xml-benchmark.org/}{http://www.xml-benchmark.org/}} 
  benchmark suite and compare to other existing implementations
  \item If a working MQL processor can not or will not be developed in the
  foreseeable future, then the prototype and the rules of ``Tainting
  Dependencies'' (TD) may be modified and interfaced with MonetDB, and a
  benchmark could be performed on this combined system\footnote{Note that this may prove challenging in the case of
  the MQL \texttt{scope()} operator for which MonetDB does not have a known
  counterpart}
  \item Further research on optimisation and simplification of the TD
  methodology
  \item Investigate applicability of parallell execution of subtrees in the
  algebra tree
\end{itemize}

Furthermore, we propose the following:
\begin{itemize}
  \item Improved support for interfacing with the external environment (as
  described in section \ref{sect:disc:ctxItem})
  \item Improved support of XQuery functionality, including: full-text
  operations, ordered/unordered mode, binary operators not currently supported,
  multiple order specifications for the orderby-clause, user-defined functions
  as well as built-in XQuery functions (within the \texttt{fn} namespace)
  \item Full consideration of the XQuery type system into TD, which may be
  exploited for optimisations
\end{itemize}