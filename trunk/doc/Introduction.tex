\chapter{Introduction}
\label{chapter:introduction}
%\pagenumbering{arabic}
% \begin{itemize}
%   \item motiv for oppgave
%   \item vi har funnet opp ``Tainting Dependencies'' -- en metode for \aa~oversette XQuery til MQL
%   \item vi har sammenlignet mot Loop Lifting (Pathfinder) og det viser seg \aa~v\ae re god stemning
%   \item \ldots
%   \item \ldots 
% 
% \end{itemize}


The search engines of today are capable of finding relevant documents based
on simple search terms as well as weighting and ranking schemes of varying
complexity. However, few are capable of joining several query results,
performing structural queries, and filtering by complex full-text expressions
in a single unified query operation. 

XQuery is an XML query language capable of performing complex nested
queries, extendable with full-text searching, including linguistics such as
stemming and thesaurus. In theory, XQuery queries may be translated into
relational algebra for execution on a suitable algebra processor engine.

iAd \cite{iadcentre} (Information Access Disruptions) is an ongoing research
effort in ``next generation information access solutions'', in which this
project partake. Our specific goal is to develop a method of translating XQuery queries into MARS relational
algebra, and compare this to existing technology. Furthermore, a prototype will be
implemented as a proof of concept.

This report is structured as follows:
In chapter \ref{chapter:theory} we will examine XQuery itself in further detail
and investigate the current state of XQuery translator implementations and
research. In chapter \ref{chapter:method} we will detail the tools and methods
used. In chapter \ref{chapter:translation} we will describe our novel
translation method dubbed ``Tainting Dependencies''. Continuing to chapter
\ref{chapter:implementation} we expound on the implementation of a prototype
which serves as a proof of concept. Eventually the results will be
presented and discussed in chapters \ref{chapter:results} and
\ref{chapter:discussion}. Our work is concluded in chapter
\ref{chapter:conclusion}, and we propose future work and improvements in
chapter \ref{chapter:future}.
