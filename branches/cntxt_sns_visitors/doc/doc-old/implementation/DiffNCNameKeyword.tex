\section{QName, NCName and Keywords}
\label{sect:rewritegrammar:keywordNCName}

All 139 keywords and 73 symbols are explicitly defined as tokens, as opposed to being declared inline in the non-terminal productions. The main reason for this is that predicates can be employed to control when the lexer should and should not try to match these rules. Another benefit from defining the tokens in the lexer is that ANTLR compiler error messages and warnings are much more readable this way. When defined inline ANTLR would just assign any unique name to the tokens, making it hard for humans to understand which token is referred to. 

We defined the names of the tokens for keywords as the keyword in capital letters with dashes replaced by underscore, e.g. \verb!BASE_URI : 'base-uri'!. A descriptive name with the suffix \verb!Si! constitutes the token names for symbols, e.g. \verb!ASSIGNSi : ':='!.

The non-terminal productions refer both to \verb!NCName! and \verb!QName!, where W3C defines \verb!QName! as (simplified):
\begin{Verbatim}
QName       ::= (Prefix ':')? LocalPart
Prefix      ::= NCName
LocalPart   ::= NCName
\end{Verbatim}

Which means that there is a ambiguity between the terminals \verb!QName! and \verb!NCName!, as \verb!QName! can consist of just a \verb!NCName!. This is solved by demanding that the terminal \verb!QName! contains both a prefix and a localpart, and introducing a non-terminal production \verb!qName! which also covers the unprefixed cases:
\begin{Verbatim}
// the non-terminal
qName       : QName
            | NCName;
// the terminal
QName       : NCName COLONSi NCName;
\end{Verbatim}

Another solution would be to just turn \verb!QName! into a non-terminal in a matter similar to the terminal specification by W3C shown earlier. But by employing our approach, we meet the requirement that no whitespace are allowed between the \verb!:!-sign and neither of the \verb!NCName! occurrences. In addition, necessary lookahead is reduced as the parser would know that it sees a \verb!qName! by only looking at a single token.

As we will see in section \ref{sect:impl:parser_controlled_state_driven_lexer}, our lexer depends on a master production \verb!TOKENSWITCH!, and gated semantic predicates to control which rules are available in which states. To have a single on/off switching point, all references to \verb!NCName!, \verb!QName! and the keywords are gathered under the umbrella production \verb!LexLiterals! (figure \ref{fig:lexLitterals}).

\begin{figure}[h!]
\begin{Verbatim}
fragment LexLiterals : (QName)=> QName
                       | n=NCName {
                          if($n.getText().equals("all")) 
                             this.tokenType=ALL;
                          else if(...
                             ...
                          else 
                             this.tokenType=NCName; };
\end{Verbatim}
\caption[Keywords, \texttt{NCName} and \texttt{QName} gathered under \texttt{LexLiterals}]{Keywords, \texttt{NCName} and \texttt{QName} gathered under the umbrella production \texttt{LexLiterals}}
\label{fig:lexLitterals}
\end{figure}

A syntactic predicate is inserted to ensure that the lexer matches a \verb!QName! if it is possible, and not a \verb!NCName! followed by a \verb!:! and another \verb!NCName!.

In section \ref{sect:antlr:lexer} we mentioned that ANTLR will generate a lexer correctly differentiating between explicit and implicit defined character sequences, which in our case are keywords and \verb!NCName! respectively. It actually resolves this by ranking the explicit ones over the implicit ones, and surpressing warnings about ambiguity. This works, as the highest ranked alternative will always be chosen. But because we defined these productions as \verb!fragment! rules, the warnings are not surpressed.

The solution is to implement all these tokens as \verb!NCName!. An \verb!if..else if..! structure checks if the matched token should be one of the keywords, and in such a case change the type appropriately. This manner of fixing the problem is not a particularly good one, and other solutions will be discussed in section \ref{sect:future:improvements}.
