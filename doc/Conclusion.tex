\chapter{Conclusion}
Throughout this project, we have done the following:
\begin{itemize}
  \item Investigated the current state of XQuery implementations (with and
  without full-text extensions) as well as XQuery as language
  \item Researched and evaluated parser generators suitable
  for our project based on requirements, and chosen one for this project
  \item Examined the ANTLR parser generator in detail and documented
  limitations and features
  \item Researched and evaluated lexer strategies, and ultimately chosen one
  for this project
  \item Refactored the W3C grammar specification into a suitable grammar for
  ANTLR
  \item Investigated, documented and resolved several unexpected problems and
  ambiguities in the grammar
  \item \underline{\textbf{\LARGE //TODO:}} Mads: legge til noe spess paa de to
  punktene over?
  \item Added grammar rewrite rules and operators to generate proper abstract
  syntax trees and evaluated their structure
  \item Researched and implemented a prototype for scoping, symbol tables, and
  lookups
  \item Improved the default ANTLR error handling by giving the responsibility
  of error handling to the calling program
  \item Developed, executed and evaluated detailed coverage tests based on the
  official XQuery test suite
  \item Proposed several improvements and suggestions for future work
\end{itemize}


\underline{\textbf{\LARGE //TODO:}}
s\aa~konklusjon

In this project we have developed a front-end parser for XQuery with support for
full-text extensions. The parser produces a usable AST that can be easily
modified, and is suitable for further usage.

hva ble gjort for \aa ~f\aa~ opp dekningskrav hovedgrunner, litt fremtid, litt
av alt egentlig? 

Vi maa si noe om hva vi synes selv ogsaa, om vi er fornoyde osv. Vi maa ha en
definitiv konklusjon, f.eks ``Vi har utviklet en parser av grei kvalitet som vi
er fornoyd med'' 
-AR

\underline{\textbf{\LARGE //ODOT:}}