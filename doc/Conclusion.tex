\chapter{Conclusion}
\label{chapter:conclusion}
Throughout this project, we have explored the nature of XQuery and relational
algebra. We have studied one well-researched method of translation dubbed ``loop
lifting'', and found points of improvement as well as untapped potential in the
fact that the target language MQL is a more expressive form of algebra
which may allow more creative and efficient translations.

We then proceeded to develop a novel method of translation dubbed ``Tainting
Dependencies'' (TD) which seeks to avoid unecessary denormalisation of
intermediate results, and which is also designed specifically for translation
to MQL. Our method supports a substantially large subset of the XQuery language
-- however lacking functionality has been accounted for, and suggestions for
solutions have been proposed.

Further, we developed a prototype as a proof of concept, which is capable of
translating XQuery queries containing basic constructs such as FLWOR
expressions, sequence constructions, and conditional expressions (if-then-else).

Finally, based on a method of measurement for complexity defined by Oystein
Torbj\o rnsen at FAST, we staged a comparison of our prototype TD implementation
and an implementation of loop lifting dubbed Pathfinder developed by Teubner
et. al at the University of Konstanz. Not withstanding the weaknesses of this
method of comparison, we then empirically suggested that our method may
produce less complex and more efficient relational algebra.

Further research may be required, however the outcome of this project is a
fairly complete, novel and well-documented method for translation of XQuery to
MQL -- a method which is designed to perform equally or better than existing
implementations.

% 
% \begin{itemize}
%   \item Vi har beskrevet XQuery, klassisk relasjonell algebra, og loop lifting
%   \item Vi fant visse deficiencies ved loop lifting, samt at MQL er mer
%   expressiv
%   \item Vi har funnet opp metode (TD)
%   \item Vi har implementert metoden (proof of concept)
%   \item vi har sammenlignet metoden (complexity comparison)
%   \item vi har funnet ut at det kan hende vi er inne p\aa~noe
%   \item \ldots
% \end{itemize}