\chapter{Introduction}
The search engines of today are well suited for finding relevant documents based
on simple search terms and various weighting and ranking schemes. However, few 
are capable of joining several query results, performing structural queries, and
filtering by complex full-text expressions in a single unified query operation.

One can imagine the following scenario: A young couple wants move to a new city.
She is a computer engineer. He is a mathematician. And so, they need to find
a city with available jobs for the both of them. Further, they may need a house
with a swimming pool to live in within a vincity of 10 miles of the city. And,
they don't want to live in a city where the density of public elementary schools
is too low, to make sure that their children can go to a school they truly like.

In order to find a city matching these criteria with todays commodity search
engine technology, they would need to perform several search queries across a
multitude of data sources. Further, they would need to compare, join and
rank the results manually. Indeed this would be a lot of work!

Enter XQuery, an XML query language capable of performing complex nested
queries, extendable with full-text extensions, including linguistics such as
stemming and thesaurus. Additionally, XQuery queries can be translated into
querying relational
data\cite{pathfinder_comptech}\cite{pathfinder_staircase}\cite{galatex}\cite{datadirect} 
as well as XML data.

iAd \cite{iadcentre} (Information Access Disruptions) is an ongoing research
effort in ``next generation information access solutions'', in which this
project partake. Our specific goal is thus to develop an experimental XQuery
parser front end, with full-text extensions.

In chapter \ref{chapter:theory} we will examine XQuery itself in further detail
and investigate the current state of XQuery implementations and research. In
chapter \ref{chapter:method} we will detail the tools and methods used. In
chapter \ref{chapter:antlr} we will document the features and 
limitations in the ANTLR parser generator. Continuing to chapter
\ref{chapter:implementation} we expound on the parser
implementation itself. Eventually the results will be presented and discussed in
chapters \ref{chapter:results} and \ref{chapter:discussion}. Our work is 
concluded in chapter \ref{chapter:conclusion}, and we propose future work and
improvements in chapter \ref{chapter:future}.