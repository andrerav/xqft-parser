\section{Theoretical Algebra}
\label{sect:result:theoretical_algebra}
In this section, a collection of XQuery query examples and their translation to
relational algebra are presented. The translation is done manually using the
``Tainting Dependencies'' method described in Chapter \ref{sect:translation},
and includes the simplifications specified in section
\ref{sect:trans:TD:simplifications}. For the sake of brevity, only the rules
used throughout the translation will be noted. Intermediate translations are
not shown here, however they are all shown in their entirety in appendix
\ref{appendix:transl}.

\subsection{Extensive FLWOR}
This example will illustrate the translation of the following FLWOR expression:

\begin{Verbatim}
for $a in (1,2,3) let $b := 2
  where $a gt $b
  order by $a
  return ($a, $b)
\end{Verbatim}

The translation process in its entirety is shown step by step in Appendix \ref{appendix:transl:ext_flwor}, page
\pageref{appendix:transl:ext_flwor}. And the result of the translation is shown in figure
\ref{fig:results:query_ext_flwor_result}. The operator tree can be converted to the DAG seen in Figure
\ref{fig:results:query_ext_flwor_dag}.

\begin{figure}[!htp]
\begin{center}
  \includegraphics[width=1.0\textwidth]{img/graphs/ext_flwor}
  \caption{Complete translation of expression extensive FLWOR expression}
  \label{fig:results:query_ext_flwor_result}
\end{center}
\end{figure}

\begin{figure}[!htp]
\begin{center}
  \includegraphics[width=0.8\textwidth]{img/graphs/ext_flwor_dag}
  \caption{DAG representation of the translated extensive FLWOR expression}
  \label{fig:results:query_ext_flwor_dag}
\end{center}
\end{figure}

\newpage
\subsection{Path expression with predicate}
This example will illustrate the translation of a path expression with a predicate:

\begin{Verbatim}
/a/b[@id eq 2] 
\end{Verbatim}

The translation process in its entirety is shown step by step in Appendix \ref{appendix:transl:pathPred}, page
\pageref{appendix:transl:pathPred}, and the result of the translation is shown in figure
\ref{fig:results:query_pathpred_result}. The operator can be converted to the DAG seen in Figure
\ref{fig:results:query_pathpred_result_dag}.

\begin{figure}[!htp]
\begin{center}
  \includegraphics[width=0.85\textwidth]{img/graphs/TD_patExprPred}
  \caption{Complete translation of the path expression.}
  \label{fig:results:query_pathpred_result}
\end{center}
\end{figure}

\newpage
\begin{figure}[!htp]
\begin{center}
  \includegraphics[width=0.9\textwidth]{img/graphs/TD_patExprPred_dag}
  \caption{DAG representation of the translated path expression.}
  \label{fig:results:query_pathpred_result_dag}
\end{center}
\end{figure}

\newpage

\subsection{If-then-else}
This example will illustrate the translation of a conditional expression:
\begin{Verbatim}
for $a in (1,2,3) return
  if $a gt 2 then $a else 3
\end{Verbatim}
The translation process in its entirety is shown step by step in Appendix \ref{appendix:transl:ifthenelse}, page
\pageref{appendix:transl:ifthenelse}, and the result of the translation is shown in figure
\ref{fig:results:query_ifthenelse_result}. The operator tree can be converted to the DAG seen in Figure
\ref{fig:results:query_ifthenelse_result_dag}.

\begin{figure}[!htp]
\centering
  \includegraphics[width=0.8\textwidth]{img/graphs/ifthenelse}
  \caption{Complete translation of the conditional expression }
  \label{fig:results:query_ifthenelse_result}
\end{figure}


\begin{figure}[!htp]
\centering
  \includegraphics[width=0.6\textwidth]{img/graphs/ifthenelse_dag}
  \caption{DAG representation of the translated conditional expression}
  \label{fig:results:query_ifthenelse_result_dag}
\end{figure}

\newpage

\section{Algebra Generated By Implementation}
\label{sect:result:implementation_algebra}
In this section, a collection of queries are translated to relational
algebra using the implemented proof of concept described in chapter
\ref{chapter:implementation}. Naturally, this implementation also uses the
``Tainting Dependencies'' method, however the results from these translations
will be used in a comparison with Pathfinder in the next section.

\subsection{Trivial FLWOR}
\label{sect:results:algebra:generated:trivial_flwor}
\subsubsection{Query premise}
\begin{figure}[!htp]
\begin{center}
\begin{Verbatim}
for $a in (1,2,3) return $a
\end{Verbatim}
  \caption{Trivial FLWOR query premise}
  \label{fig:results:query_trivial_flwor}
\end{center}
\end{figure}

\subsubsection{Result}
\begin{figure}[!htp]
\begin{center}
  \includegraphics[width=1.0\textwidth]{img/graphs/td_impl_flwor_simple_xq_relalg} \caption{Complete translation of expression in figure
  \ref{fig:results:query_trivial_flwor}}
  \label{fig:results:query_trivial_flwor_result}
\end{center}
\end{figure}

\subsection{Complex FLWOR}
\label{sect:results:algebra:generated:complex_flwor}
\subsubsection{Query premise}
\begin{figure}[!htp]
\begin{center}
\begin{Verbatim}
for $a in (1,2) return (3, for $b in (4,5) return ($a, $b, 6))
\end{Verbatim}
  \caption{Complex FLWOR query premise}
  \label{fig:results:query_complex_flwor}
\end{center}
\end{figure}

\subsubsection{Result}
\begin{figure}[!htp]
\begin{center}
  \includegraphics[angle=90,height=0.7\textheight]{img/graphs/td_impl_flwor_complex_xq_relalg} \caption{Complete
  translation of expression in figure
  \ref{fig:results:query_complex_flwor}}
  \label{fig:results:query_complex_flwor_result}
\end{center}
\end{figure}


The algebra tree in Figure \ref{fig:results:query_complex_flwor_result} can
be converted to the DAG in figure
\ref{fig:results:query_complex_flwor_result_dag}

\begin{figure}[!htp]
\begin{center}
  \includegraphics[width=1.0\textwidth]{img/graphs/td_impl_flwor_complex_xq_relalg_dag}
  \caption{Complete translation of expression in figure
  \ref{fig:results:query_complex_flwor} converted to a DAG}
  \label{fig:results:query_complex_flwor_result_dag}
\end{center}
\end{figure}
\newpage

\subsection{FLWOR with conditional}
\label{sect:results:algebra:generated:conditional_flwor}
\subsubsection{Query premise}
\begin{figure}[!htp]
\begin{center}
\begin{Verbatim}
for $a in (10,20) return if ($a > 15) then $a else 15
\end{Verbatim}
  \caption{Conditional FLWOR query premise}
  \label{fig:results:query_conditional_flwor}
\end{center}
\end{figure}

\subsubsection{Result}
\begin{figure}[!htp]
\begin{center}
  \includegraphics[width=1.0\textwidth]{img/graphs/td_impl_flwor_ifthenelse_xq_relalg}
  \caption{Complete translation of expression in figure
  \ref{fig:results:query_conditional_flwor}}
  \label{fig:results:query_conditional_flwor_result}
\end{center}
\end{figure}

The algebra tree in Figure \ref{fig:results:query_conditional_flwor_result} can
be converted to the DAG in figure
\ref{fig:results:query_conditional_flwor_result_dag}

\newpage
\begin{figure}[!htp]
\begin{center} 
  \includegraphics[width=1.0\textwidth]{img/graphs/td_impl_flwor_ifthenelse_xq_relalg_dag}
  \caption{Complete translation of expression in figure
  \ref{fig:results:query_conditional_flwor} converted to a DAG}
  \label{fig:results:query_conditional_flwor_result_dag}
\end{center}
\end{figure}