\chapter{Implementation}
\label{chapter:implementation}

\begin{itemize}
  \item intro
  \item Vi lager en liten prototype = ``proof of concept'' -> et begrep vi m\aa~bruke =)
  \item VarRefs m\aa~sendes oppover fra child nodes (ref til tainting deps)
  \item Optimalisering: child nodes vet om de er atomiske/singleton eller ikke
\end{itemize}

\begin{itemize}
  \item Bruk av parseren (den som er generert av antlr osv), XQFTTree-klassen,
  evt. UML
  \item Interfacing av parseren mot oversetteren v\aa r (som oversetter til
  relalg) og UML
  \item Bygging av relasjonsalgebra-tre, UML av operatorklassene
  \item Implementering av visitors
  \item Scoping og symtab (nok en gang..)
  \item hvordan metadata som varrefs og singletonindikering blir sendt oppover
  (Metadata)
  \item Implementering av ``tainting dependencies''
  \item Dataflyt ()
  \item Visible external API
  \item Interface til systemet p\aa~ kommandolinjen
  \item Optimaliseringer, om noen? 
\end{itemize}

\section{Prerequisites}
This proof of concept was implemented in Java 5.0, using regular object
oriented techniques, and is licensed under a liberal BSD license. 

\section{Using the XQFT Parser}
The \textit{XQFT Parser}\cite{ourselves} (described in section
\ref{sect:theory:xqftparser}) is a prerequisite for providing the abstract
syntax tree for this XQuery translator. This section will outline how this
parser was used and interfaced with the implementation.

The \textit{XQFT Parser} is a parser generated by the ANTLR parser generator.
Thus, there is a loosely standardised API available for any implementor
utilising a parser generated by ANTLR. In the case of \textit{XQFT Parser}, two
classes are generated: \texttt{XQFTParser} and \texttt{XQFTLexer}. These
classes are used in conjunction on an input string to produce an abstract syntax
tree (see next subsection, and also section
\ref{sect:theory:xqftparser:ast_construction}).

A typical use case to achieve this is shown in figure
\ref{figure:impl:using_xqft}, which is copied almost verbatim from the
implementation.

\begin{figure}[!htp]
\begin{center}
  \begin{Verbatim}
    CharStream cs 
        = new ANTLRStringStream(
            "for $i in (1,2,3) return $i");

    XQFTLexer lexer = new XQFTLexer(cs);

    UnbufferedCommonTokenStream tokens 
        = new UnbufferedCommonTokenStream();
	tokens.setTokenSource(lexer);

    XQFTParser parser = new XQFTParser(tokens);
    parser.setTreeAdaptor(new XQFTTreeAdaptor());
    parser.setLexer(lexer);

    XQFTTree ast = parser.module().getTree();
  \end{Verbatim}
  \caption{Using the XQFTParser and XQFTLexer classes}
  \label{figure:impl:using_xqft}
\end{center}
\end{figure}
Note the use of \texttt{ANTLRStringStream},
\texttt{UnbufferedCommonTokenStream}, and \texttt{XQFTTreeAdaptor}. The latter,
\texttt{XQFTTreeAdaptor}, is a specialised class required to create instances of
the \texttt{XQFTTree} class to represent nodes in the abstract syntax tree.

The actual parsing is performed by calling the method \texttt{module()}, which
is the top-level production rule in the grammar for the XQFT parser (see
appendix \ref{appendix:xquery_ebnf}).

The \texttt{XQFTTree} class represents a node in the produced syntax tree. When
a syntax tree is returned from the parser, the root node is an instance of this
class, as well as all children (see figure \ref{figure:impl:using_xqft})

To make practical use of the XQFT Parser, what remains is nothing more than to
translate the abstract syntax tree acquired from the call to
\texttt{getTree()}, which is the object \texttt{ast} on the last line of code
in figure \ref{figure:impl:using_xqft}.


\section{Constructing the MQL algebra tree}
\subsection{Operators and parameters}
\subsection{Concepts}
\subsection{Usage}

\section{Context-sensitive visitor}
\subsection{Basics}
\subsection{Modifying the XQFTTree class}
\subsection{Rewrite visitor}
\subsection{Translation visitor}

\section{Scoping and symbol tables}
\begin{itemize}
  \item Holder styr p\aa~variabler, og om de er iterasjonsvariabler i n\aa v\ae
  rende scope
  \item Scope pushes rett f\oe r forclause, og poppes etter at returnclausen har
  blitt evaluert
  \item Symbolene er innkapslet i en symtabentry som inneholder metadata
\end{itemize}

\section{Passing metadata between nodes}
\subsection{Variable references}
\subsection{Singleton nodes}
\subsection{The TraverseReturn class}

\section{Tainting dependencies}
Dette blir den lengste seksjonen i dette kapittelet, h\aa per jeg.
\subsection{FLWOR expressions}
\subsection{Sequences}
\begin{itemize}
  \item behandler parantes istf. komma for sekvenser (ref spec) 
\end{itemize}

\section{Data flow}
\begin{itemize}
  \item grammatikk -> antlr -> parser
  \item xquery -> parser -> visitor -> relalg
\end{itemize}

\section{Visible external API}
\begin{itemize}
  \item Vise hvordan oversetteren kan brukes i sin helhet i andre programmer
\end{itemize}

\section{Command line interface}
Dette er ikke s\aa~veldig viktig \aa~skrive om, men greit \aa~ha med.
\begin{itemize}
  \item Argsengine
  \item Aksepterer flere strenger og filer
  \item Outputter til graphviz/dot og pdf, hvis tilgjengelig
  \item Dependencies (jar-filer og drit)
\end{itemize}

\section{Optimalisations?}

\section{Summary}
\label{sect:impl:summary}
\begin{itemize}
  \item sammendrag av dette kap
\end{itemize}