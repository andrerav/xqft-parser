\section{Using the XQFT Parser}
The \textit{XQFT Parser}\cite{ourselves} (described in section
\ref{sect:theory:xqftparser}) is a prerequisite for providing the abstract
syntax tree for this XQuery translator. This section will outline how this
parser was used and interfaced with the implementation.

The \textit{XQFT Parser} is a parser generated by the ANTLR parser generator.
Thus, there is a loosely standardised API available for any implementor
utilising a parser generated by ANTLR. In the case of \textit{XQFT Parser}, two
classes are generated: \texttt{XQFTParser} and \texttt{XQFTLexer}. These
classes are used in conjunction on an input string to produce an abstract syntax
tree (see next subsection, and also section
\ref{sect:theory:xqftparser:ast_construction}).

A typical use case to achieve this is shown in figure
\ref{figure:impl:using_xqft}, which is copied almost verbatim from the
implementation.

\begin{figure}[!htp]
\begin{center}
  \begin{Verbatim}
    CharStream cs 
        = new ANTLRStringStream(
            "for $i in (1,2,3) return $i");

    XQFTLexer lexer = new XQFTLexer(cs);

    UnbufferedCommonTokenStream tokens 
        = new UnbufferedCommonTokenStream();
	tokens.setTokenSource(lexer);

    XQFTParser parser = new XQFTParser(tokens);
    parser.setTreeAdaptor(new XQFTTreeAdaptor());
    parser.setLexer(lexer);

    XQFTTree ast = parser.module().getTree();
  \end{Verbatim}
  \caption{Using the XQFTParser and XQFTLexer classes}
  \label{figure:impl:using_xqft}
\end{center}
\end{figure}
Note the use of \texttt{ANTLRStringStream},
\texttt{UnbufferedCommonTokenStream}, and \texttt{XQFTTreeAdaptor}. The latter,
\texttt{XQFTTreeAdaptor}, is a specialised class required to create instances of
the \texttt{XQFTTree} class to represent nodes in the abstract syntax tree.

The actual parsing is performed by calling the method \texttt{module()}, which
is the top-level production rule in the grammar for the XQFT parser (see
appendix \ref{appendix:xquery_ebnf}).

The \texttt{XQFTTree} class represents a node in the produced syntax tree. When
a syntax tree is returned from the parser, the root node is an instance of this
class, as well as all children (see figure \ref{figure:impl:using_xqft})

To make practical use of the XQFT Parser, what remains is nothing more than to
translate the abstract syntax tree acquired from the call to
\texttt{getTree()}, which is the object \texttt{ast} on the last line of code
in figure \ref{figure:impl:using_xqft}.
