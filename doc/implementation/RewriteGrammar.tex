\section{Enclosed Composite Lexer Productions}
\label{sect:rewriteGrammar:enclosedComposite}
\textbf{\LARGE //TODO:} 

Regler som PI, Pragma, XMLComment and CDATA kan fanges i sin helhet i lexeren. Samme med saertilfellet XQuerykommentarer. 

Fint for aa skille alternativer, men man ikke faar tak i deltokens... saa derfor ----v

\subsection{Emiting More Than One Token Per Production}
\textbf{\LARGE //TODO:} 

Omskriving av lexer for a la produksjoner avgi mer enn ett token

\subsection{PI, Pragma, XMLComment and CDATA Sections}
\textbf{\LARGE //TODO:} 

Sammensatte omhyllede lexerregler (<? noe ?>, <!-- noe --> etc)

\subsection{Nested XQuery Comments}
XQuery allows nested comments, for example:
\begin{verbatim}
(: this is a comment (: this comment is nested :) :)
\end{verbatim}
This is a classic problem in compiler construction, however it can be solved using standard ANTLR syntax, without resorting to custom functions/methods for consuming input and keeping track of nesting. The original EBNF as specified by W3C is as follows:
\begin{verbatim}
Comment ::= "(:" (CommentContents | Comment)* ":)"
\end{verbatim}
At first glance, this seems uncomplicated and straight forward, but this grammar needs to be rewritten to be accepted by an LL parser. A suggestion for a solution to this problem was initially found on the Antlr mailing list\footnote{http://www.antlr.org:8080/pipermail/antlr-interest/2005-July/012967.html}, and we loosly based our implementation on such an approach. This lexer rule will correctly detect and allow nested comments, and hide them from the parser:
\begin{verbatim}   
Comment   : LXQCOMMENTSi 
           ({(input.LA(1)=='(' && input.LA(2)==':')}?Comment 
           | {input.LA(2)!=')'}?=>COLONSi
           | {input.LA(2)!=':'}?=>LPARSi
           | ~(LPARSi | COLONSi | NotChar))*
            RXQCOMMENTSi; {$channel=HIDDEN;}
    fragment LXQCOMMENTSi     : '(:';
    fragment RXQCOMMENTSi     : ':)';
\end{verbatim}
Where the disambiguating semantic predicate on the second line can be understood as "if it looks like a comment, it is a comment". The gated semantic predicates on the third and fourth line guards the production from being greedy, i.e. they hide the posibility of matching a \verb!':'! if it is followed by a \verb!')'!, and \verb!'('! if it is followed by a \verb!':'!. By using \verb!$channel=HIDDEN! ANTLR will put this token in an different virtual channel than the default one, making it invisible for the parser unless explicitly asked for. 

\section{Differentiating NCName and Keywords}

\textbf{\LARGE //TODO:} 

Systemet med NCNames vs keywords. \\
Husk: Keywords er uttrykt som tokens for aa sperre av for parsing og for oversiktlige feilmeldinger.

\section{Extra-grammatical Constraints}

\textbf{\LARGE //TODO:} 

Whitespace, leading dash etc..

\section{Resolving Non-determinisms in the Parser }

\textbf{\LARGE //TODO:} 

pathExpr og itemtype occurenceindicator trengte syntaktiske predikat

\section{Reserved Keywords}

\underline{\textbf{\LARGE //TODO:}} dette maa flyttes eller skrives om.

A particular feature in XQuery is the lack of reserved keywords. This creates a
series of problems when a lexer based on the verbatim grammar specification from
the W3C is trying to recognize tokens. 

vaar parser har enna reserved keywords, flytte dette til future work? \\
\underline{\textbf{\LARGE //ODOT:}} 



