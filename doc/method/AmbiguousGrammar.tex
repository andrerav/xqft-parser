\section{Ambiguous Terminals}
The W3C EBNF specification of XQuery Full Text (\cite{w3c01}) defines among other these terminals (somewhat simplified):
\begin{verbatim}
ElementContentChar         ::= Char - [{}<&]
QuotAttrContentChar        ::= Char - ["{}<&]
AposAttrContentChar        ::= Char - ['{}<&]
IntegerLiteral             ::= Digits
NCName                     ::= NCNameStartChar NCNameChar*
NCNameChar                 ::= Letter | Digit | '.' | '-' | '_'
NCNameStartChar            ::= Letter | '_'
\end{verbatim}
Where \verb!Char! denotes all possible legal characters in XQuery, \verb!Digits! is all possible number characters and \verb!Letter! denotes all possible letter characters. In their specification, the W3C uses a dash operator, which has the following
semantic meaning in a grammar (\cite{w3c03}, section 6):
\begin{quote}
A - B: matches any string that matches A but does not match B.
\end{quote}
In addition, the specification contains explicit defined litterals and symbols such as \verb!'for'! and \verb!'::'!, aswell as \verb!StringLitteral! which is indeed comparable with the non-terminal \verb!dirAttributeList! (simplified):
\begin{verbatim}
dirAttributeValue          ::= ('"' (QuotAttrContentChar)* '"')
                             | ("'" (AposAttrContentChar)* "'")
StringLiteral              ::= ('"' ([^"&])* '"') 
                             | ("'" ([^'&])* "'")
\end{verbatim}
Where \verb!QuotAttrContentChar! and \verb!AposAttrContentChar! is as defined earlier, and the hat operator (\verb!^!) is defined as follows(\cite{w3c03}, section 6):
\begin{quote}
[\^{}abc] matches any Char with a value not among the characters given.
\end{quote}
Meaning that e.g. \verb![^"&]! equals \verb!Char - ["&]!. It is easy to see that these productions would overlap significantly, thus the lexer must be aware of the context of a incomming character to differanciate between the rules. We considered several alternatives, as discussed in the following sections.

\subsection{State Driven Lexer}
With this method the lexer would only sets of words when in an appropriate state. The acceptance of a word in one state may lead to a state transition. 