\section{Unit testing the grammar}
Unit testing can be a powerful tool for asserting functionality and can be a
helpful aid in debugging and prevention of regression errors.  For unit testing the
grammar specification, gUnit \cite{gunit00} was employed. This tool uses a
syntax similar to Antlr itself, however instead of defining productions, one
defines a set of inputs for some rule, as well as the expected result. Consider
this example:

\begin{verbatim}
gunit XQFT;
@header{package no.ntnu.xqft.parse;}

piTarget: // Test piTarget rule

    // Any case permutation of 'XML' must fail
    "Xml" FAIL
    "XMl" FAIL
    "XML" FAIL
    "XmL" FAIL
\end{verbatim}

This is a complete input file for gUnit, and will automatically discover the
classes XQFTLexer and XQFTParser in the package no.ntnu.xqft.parse. gUnit will
then proceed to invoke the lexer with ``Xml'', ``XMl'', ``XML'', and ``XmL'' as
input, and pass the lexer to an instance of XQFTParser and execute the production
piTarget. For all these inputs, it will assert that the parser emits an error
(i.e it must fail to pass the test).

In case of a test where the parser should not fail, the syntax is as follows:
\begin{verbatim}
forClause:
	"for $a in document(\"abc.xml\")/a/b/text()" OK
\end{verbatim}
Here gUnit will assert that the parser will not fail for the given inpu (i.e it
must not fail to pass the test).

gUnit is also capable of parsing abstract syntax trees built by the generated
parser, but this feature has not been used in this project.
