\chapter{Discussion}
\label{chapter:discussion}
In this chapter, some important aspects of this project will be discussed. Some
of them are concrete problems or challenges, others are of a more predictive
nature and may serve as a basis for further research. Also, possible
solutions or ideas are proposed where applicable. 

Most importantly, XQuery features not
currently supported by ``Tainting Dependencies'' are discussed, as well as
possible optimisations. The proof of concept whose implementation was
described in chapter \ref{chapter:implementation} is discussed, and in
particular implementation-specific details. Further, the results presented in
the previous chapter are discussed, and insecurities and sources of errors are
accounted for. This chapter is finalised by discussion of syntax tree rewriting
and normalisation, a discussion of problems related to sequences, and a
discussion of the XQuery type system and related problems.

\textbf{\LARGE TODO:} skal dette inn igjen?
, we have a few assumptions. A scope defined as a parameter to
the \textsf{scope}-operator starting with a slash (\textsf{/}) is an absolute scope. That means if the parameter
is e.g. \textsf{/a/b}, the operator will remove any tuples in the input relation where the $scope$ attribute does
not define the tuple to stem from a \texttt{b} scope within the root-scope \texttt{a}. Without a slash first in
the parameter, the scope is relative. E.g. \texttt{a/b} as a parameter will lead to removal of all tuples where
the $scope$ attribute does not define the tuple to stem from a \texttt{b} scope within an \texttt{a} scope.



\section{Context Item - {MADS}}
\label{sect:disc:ctxItem}
\textbf{\LARGE TODO: {MADS}}
\begin{itemize}
  \item m\aa~nesten dytte den til symtab hele tiden, koster ikke mye, s\aa~lenge den ikke blir brukt.
  \item Man skal vel ha tilgang til den hele tiden omtrent.
  \item kan sette context node til noe n\aa r man starter\ldots Tenke en virtuell /collection/doc struktur\ldots
  ellerno
  \item XQuery has a set of functions that provide access to input data. These functions are of particular
  importance because they provide a way in which an expression can reference a document or a collection of
  documents. The input functions are described informally here; they are defined in [XQuery 1.0 and XPath 2.0
  Functions and Operators]. An expression can access input data either by calling one of the input functions or by
  referencing some part of the dynamic context that is initialized by the external environment, such as a variable
  or context item. -> Appendix C.2 Dynamic Context Components
\item declare variable er nesten st\o tta\ldots men external og greier..
\end{itemize}
\section{XQuery Features Not Supported}
\label{sect:discussion:notSupported}
\begin{itemize}
\item noen av disse tingene er ikke st\o tta pga typesystem, andre, slik som stemming og thesaurus er fordi ikkeno
slik i mars enn\aa~kan foresl\aa~det med parametere til lookup evt contextting ala \textsf{index()}.
  \item Dette avhenger jo seff av hvor mye vi har l\o st men disse b\o r kunne
  l\o ses:
  	\begin{itemize}
  		\item proximity (kanskje l\o sbar)
  		\item declare variable er nesten st\o tta\ldots men external og greier..
  		\item typeswitch? kan bli stress.. har ikke s\aa~mye typer enn\aa \ldots skal v\ae re diskusjon om typer i
  		sect \ref{sect:disc:typeSystem}\ldots
  		\item function declarations, b\o r g\aa~greit, bare ha en function table ala symbol table.
  		\item schema / schema validation -> typeting? Hvordan l\o ser pathfinder dette\ldots synes \aa~ha sett noe om
  		det\ldots
  		\item namespacezz\ldots.
  		\item node comparisons\ldots\ldots tviler p\aa~at vi f\aa r til dette glatt\ldots
  		\item constructors\ldots element, CDATA, attribute etc etc
  		\item order by med alle ting..
  		\item ordered and unordered -> lurer p\aa om (markXremove + tainting = TD uten $index$ og
  		\textsf{numberate()}) fikser dette ganske bra\ldots
  		\item \textbar, \texttt{union}, \texttt{intersect, except}
  		\item Range expressions $e_1$ \texttt{to} $e_2$ (begge m\aa~være integer tror jeg -> skrive om det i type?)
  		\item sequencetype\ldots
  		\item Prologs and modules
  		\item Quantified Expression: (some | every) \$b in $e_1$ satisfies $e_2$
  		\end{itemize}
	\end{itemize}

\section{Optimisations - {MADS}}
\label{sect:disc:optimisations}
\textbf{\LARGE TODO: {MADS}}
\begin{itemize}
  \item Med bare \emph{EN} referanse til context item i predikat i path expressions kan hele symtab bajabajaz
  droppes, og bare prune p\aa~scope\ldots
  \item holde styr p\aa~scopes, bruke scope operatoren mer flittig = smaller relasjoner \aa~joine.
  \item kanskje bruke den alt-fremover omskrivinga av pathexprs
  \item pushe projects sammen ex: pathgreia i resultatkap. (sikkert allerede gjort i Mars) Fjerne attributter som
  aldri blir lest, og ikke er en del av resultatet\ldots
  \item if(e1) then e2 else e3 -> e1 inneholder samme dependencies som e2 og e3, b\o r disse iterasjonene fjernes
  fra e2 og e3 hvis mulig\ldots -> mindre utregninger.
\end{itemize}

fra 2.3.4 snakk om tilstrekkelig kunnskap til \aa~ evaluere expr:
\begin{quote}
These rules apply to all the operands of an expression considered in combination: thus if an expression has two
operands E1 and E2, it may be evaluated using any samples of the respective sequences that satisfy the above rules.

The rules cascade: if A is an operand of B and B is an operand of C, then the processor needs to evaluate only a
sufficient sample of B to determine the value of C, and needs to evaluate only a sufficient sample of A to
determine this sample of B.

The effect of these rules is that the processor is free to stop examining further items in a sequence as soon as
it can establish that further items would not affect the result except possibly by causing an error. For example,
the processor may return true as the result of the expression S1 = S2 as soon as it finds a pair of equal values
from the two sequences.

Another consequence of these rules is that where none of the items in a sequence contributes to the result of an
expression, the processor is not obliged to evaluate any part of the sequence. Again, however, the processor
cannot dispense with a required cardinality check: if an empty sequence is not permitted in the relevant context,
then the processor must ensure that the operand is not an empty sequence.
\end{quote}

\begin{itemize} 
   \item In some cases, a processor can determine the result of an expression without accessing all the data that
   would be implied by the formal expression semantics. For example, the formal description of filter expressions
   suggests that \$s[1] should be evaluated by examining all the items in sequence \$s, and selecting all those
   that satisfy the predicate position()=1.
\end{itemize}

fra 2.3.4 \cite{w3c00}
\begin{quote}
Consider an expression Q that has an operand (sub-expression) E. In general the value of E is a sequence. At an
intermediate stage during evaluation of the sequence, some of its items will be known and others will be unknown.
If, at such an intermediate stage of evaluation, a processor is able to establish that there are only two possible
outcomes of evaluating Q, namely the value V or an error, then the processor may deliver the result V without
evaluating further items in the operand E. For this purpose, two values are considered to represent the same
outcome if their items are pairwise the same, where nodes are the same if they have the same identity, and values
are the same if they are equal and have exactly the same type.

There is an exception to this rule: If a processor evaluates an operand E (wholly or in part), then it is required
to establish that the actual value of the operand E does not violate any constraints on its cardinality. For
example, the expression \$e eq 0 results in a type error if the value of \$e contains two or more items. A processor
is not allowed to decide, after evaluating the first item in the value of \$e and finding it equal to zero, that the
only possible outcomes are the value true or a type error caused by the cardinality violation. It must establish
that the value of \$e contains no more than one item.
\end{quote}

\section{Implementasjonen - {ANDREAS}}
\label{sect:disc:contextSens}
\textbf{\LARGE TODO: {ANDREAS}}
\begin{itemize}
  \item innhold
  
\end{itemize}
\section{Results}
\label{sect:disc:res}
Chapter \ref{chapter:results} presented a serious of algebra trees --
some calculated by hand using Tainted Dependencies, and others generated using
the prototype implementation described in chapter \ref{chapter:implementation}.
Furthermore we compared algebra generated by the prototype with that generated
by Pathfinder. This section will discuss these results in detail.

\subsection{Translation output}
\label{sect:disc:res:translation_output}
In sections \ref{sect:result:theoretical_algebra} and
\ref{sect:result:implementation_algebra}, a series of XQuery queries were
translated using the novel Tainting Dependencies (TD) methodology developed
and described in chapter \ref{sect:translation}. In section
\ref{sect:result:theoretical_algebra}, where hand-computed translations were
presented, a series of simplifications were applied (these simplifications were
described in section \ref{sect:trans:TD:simplifications}). However, the
prototype developed in chapter \ref{chapter:implementation} did not implement
any of these simplifications. This was an important point to keep in mind when
later comparing this algebra to that generated by Pathfinder, and is
discussed more thoroughly in the next section.

One characterisation of the algebra generated by TD is that nodes with more
than one parent node are typically located far towards the bottom of the algebra
tree. 

Another characteristic of the algebra is that is seems to maintain a fairly
compact form. This is partly due to the fact that the tainting process does not
affect constant subexpressions, and thus the potential size of the algebra is
reduced significantly. Compare this to loop lifting used by
Pathfinder, where all expressions within a loop body are loop lifted --
as explained in sections \ref{sect:trans:ll:ConstExprs},
\ref{sect:trans:ll:mappingBack}, and \ref{sect:trans:ll:OtherExpr}.

\subsection{Complexity comparison}
\label{sect:disc:res:comparison}
The complexity calculation method (see section \ref{sect:method:complexity} on
page \pageref{sect:method:complexity}) defined by \O ystein Torbj\o rnsen at
FAST was used to compare complexity in the algebra generated by the prototype
implementation to that of Pathfinder. This comparison was
based on three queries (\emph{Trivial}, \emph{Complex}, and
\emph{Conditional}). For each of these queries, algebra was generated on both
the prototype implementation as well as Pathfinder. Then the described
method of calculating complexity was applied to these trees, and the result was
used to compare the prototype and Pathfinder.

We found that in terms of counting tuple and field creations, TD seems to excel
in large and complex queries. In the case of more trivial queries, loop lifting
and TD seem to perform similarly. Furthermore, in the case of tuple
input/output in join and sort operators, we saw that for the ``complex'' query
example, TD did not produce any joins at all, while loop lifting produced 3
joins for the same expression.

Though an interesting comparison, this is a sparse source of data -- it
is difficult to be conclusive based this data alone. However, with the
exception of the most trivial query, it seems that ``Tanting Dependencies'' (TD)
generates substantially less complex algebra than Pathfinder. As one
may consider TD more specialised method than the general loop lifting
technique, this should emerge as a natural consequence.

However, there are some sources of uncertainty for this comparison. It is not
known exactly to which degree simplifications and optimalisations have been
applied to the algebra generated by Pathfinder. In any case, the
algebra generated by TD is not simplified or optimised, and as such puts these
implementations on equal footing. 

It is also known that Pathfinder does \emph{not}:
\begin{itemize}
  \item generate algebra using pure loop lifting as would be expected from
  \cite{pathfinder_mothertongue} and \cite{pathfinder_purelyRelational} -- this
  is deduced by comparing the output from Pathfinder with the output
  from the rules defined in \cite{pathfinder_mothertongue} and
  \cite{pathfinder_purelyRelational}
  \item apply all simplifications described in
  \cite{pathfinder_purelyRelational} -- again, this is deduced by comparing the
  output from Pathfinder with the supposed output from the simplifications in
  \cite{pathfinder_purelyRelational}
\end{itemize}

Additionally, in section \ref{sect:method:complexity:assumptions} on page
\pageref{sect:method:complexity:assumptions} it was assumed that
the \texttt{Diff} and \texttt{Distinct} operators utilised by Pathfinder both
creates 0 tuples and 0 fields, only relaying the input to output. It is
natural to assume that both of these are in fact costly operators in some
aspects. However, given that \texttt{Diff} computes a difference between
result sets, this does not imply that it creates new tuples and/or fields.
Furthermore, the \texttt{Distinct} operator only removes duplicates, and as
such it should be safe to assume that it does not generate new tuples and/or
fields. This implies that these assumptions may favour Pathfinder, but likely
not in the context of the method for calculation of complexity used here.

Furthermore, it is natural to assume that Pathfinder generates algebra
which is tailored for execution on the MonetDB database system (as indicated
in \cite{pathfinder_purelyRelational}).

With regards to performance measurement, and with the lack of availability of a
proper implementation of a MQL processor (as mentioned in section
\ref{sect:method:mql} on page \pageref{sect:method:mql}), it could have been
of interest to generate algebra using TD modified for MonetDB, and compare
actual performance of loop lifting vs. TD on this database system. This notion
is further detailed in section \ref{chapter:future}.

Finally, it is important to note that the complexity comparison performed did
not in any way account for disk I/O or CPU and memory usage, and the results
must not be interpreted as such. Again, see section \ref{sect:method:complexity} on
page \pageref{sect:method:complexity} for an detailed account of this method.

\subsection{Loop Lifting vs Tainting Dependencies}
\label{sect:disc:llvsTD}
This project has studied two approaches for translating XQuery to relational
algebra; loop lifting as implemented by Pathfinder, and ``Tainting
Dependencies'' (TD), which is a novel method developed in this project which
still shares a few common traits with loop lifting. However, the motivation for
the development of TD was the fact that the more expressive MQL algebra allowed
more flexibility in the translation. Furthermore, loop lifting had the
disadvantage of full denormalisation, as noted in
\cite{pathfinder_purelyRelational}: 

\begin{quote}
[..] loop lifting consequently leads to a fully denormalised representation for
$e$ and thus to -- at least potentially -- significant data redundancy
\end{quote}

And so a major motivation for the development of TD was avoiding this level of
denormalisation.

Another common trait of loop lifting, which is also noted in
\cite{pathfinder_mothertongue} and \cite{pathfinder_purelyRelational}, is that
the algebra trees will quickly grow very large. Consider the example in section 
\ref{sect:trans:ll:example} on page \pageref{sect:trans:ll:example}; here, the
intermediate results grow in size very quickly. In particular, they are
comparatively large seen in the context of the trivial query they are produced
from.

When the algebra trees generated by loop lifting are converted to DAGs, this
trait may not seem so appareant. However, it is easily recognised by the fact
that nodes with more than one parent often are located in the middle and higher
parts of the tree, for example as seen in figure
\ref{fig:result:comparison:conditional_pathfinder_dag} on page
\pageref{fig:result:comparison:conditional_pathfinder_dag}. If this particular
DAG was converted to a tree, it would indeed be substantially larger.

Further, when comparing the rules in loop lifting (described in
\cite{pathfinder_mothertongue}) and TD, it appears that TD will in most cases
produce less operators and less complex trees than loop lifting, as well as
substantially smaller intermediate results. This comparison does not consider
simplifications, however. In some situations, especially for trivial queries,
optimised loop lifting trees \emph{may} perform better than unoptimised TD. 

\subsection{Considerations for executing MQL}
The ``Tainting Dependencies'' method produces relatively simplistic algebra
when compared to the full feature set of MQL. In section \ref{sect:method:mql},
only the operators used throughout this project was described. However, this is
only a subset of the features in MQL. This was done with the intent of creating
algebra which can be optimised using common techniques already available.
Introducing new and exotic operators complicates this process, and so this was
avoided. However, in the light of the expressiveness available in MQL, it may
be benefitial to employ a wider array of operators available when constructing
MQL algebra. In any case, this requires further documentation and concrete
performance measurements.
\section{Rewriting - {ANDREAS}}
\label{sect:disc:rewriting}
\begin{itemize}
  \item fordeler vs ulemper med \aa~skrive om til core
  \item man mister jo informasjon\ldots. Hvis den er p\aa~denne m\aa ten --> gj\o re det akkurat
	  slik, en sp\o rring som skal gi tilsvarende svar er ikke sikkert at man kan
	  skrive p\aa~den samme m\aa ten helt uten videre..  
  \item samme svar = samme utf\o relse = er dette en fordel?
  \item kan man utnytte kunnskap om translation til \aa~optimisere xquery queries?
  \item hva med \aa~bare skrive om det man trenger? Hva trenger man \aa~skriveom?
  \item sl\aa~sammen scopes blir frem og tilbake og frem igjen om man tenker core\ldots /a/b/c-> for..for..for ->
  /a/b/c
  \item se om det er mer tungvinne ting Core har\ldots har ikke satt meg helt inn i alle detaljene der.. hva skjer
  med predikater? blir de if.then.else? kan det hende at vi f\aa r un\o dvendige kryssprodukt?
  \item sluppet where-regel hvis man skriver om til if-then-else (dagens where er mer effektiv enn en if() then no
 e else tomt -> den VET at det som er usant blir kasta bort- jfr regel og select-operator)
 \item TD forutsetter ikke fullversjon XQuery (anti core), da core er subset av fullversjon
\end{itemize}
\section{XQuery sequences}
\label{sect:disc:singelton}

There is no distinction between an item, that is, a node or an atomic value, and a singleton sequence containing
that item in XQuery. An item is equivalent to a singleton sequence containing that item and vice versa. A sequence
may contain nodes, atomic values, or any mixture of nodes and atomic values. But it may be advantageous for a
translator to differentiate singleton sequences from other sequences. 

As we saw in section \ref{sect:trans:TD:simplifications}, by knowing that all subexpressions return singleton
sequences, the translation of the sequence construction expression may be simplified. If the
\texttt{return}-clause expression is a singleton sequence the translation of iterator ordered FLWOR expressions
may also be simplified. If the FLWOR only contains only one iterator and no \texttt{where}-clause the renumbering
can be replaced by a renaming of the $-numb$ field corresponding to the iterator to $index$. Understanding that
this works can be done by considering the rule for translation of iterator ordering (rule
\ref{rule:trans:TD:itOrd}). As the \texttt{return}-clause is a singleton, the $index$ fields will have the
constant value 1. $\beta$ will contain only one $-numb$ attribute, holding information of which iteration the
value in $value$ occurs. The iteration number will then become the $index$ field of the sequence created by the FLWOR.

Some expressions, such as arithmetic expressions, \texttt{order by}-expressions and value comparisons, require
their subexpressions or operands to be singleton sequences. This means that a query such as \texttt{(1, 2) + 3}
will raise a type error. By having the knowledge of the cardinality of the sequence returned to such an
expression, the translator may raise the error, and avoid a faulty query being run on the MQL processor.
Evaluating the cardinality of seqences returned from expressions is in many cases a simple task. Some expressions
will always return singletons, such as logical, comparison and aritmetic expressions, iterator variable references
and literals. The cardinality of sequences constructed of such expressions may also be calculated in the
translator. A problem arises, however, when dynamic content (not from the query
itself) is included. Consider the following query:

\begin{center}
\begin{tabular}{l}
\texttt{for \$a in doc("people.xml")//person} \\
\texttt{order by \$a/surname} \\
\texttt{return \$a}
\end{tabular}
\end{center}

As previously stated, the \texttt{order by}-expression only accepts singletons. If the document contains a
\texttt{person} node containing two \texttt{surename} node the query should fail. The translator does
however not have the ability to evaluate if a type error should be raised or not. The query stated will, without a
check for multiple \texttt{surname} node per \texttt{person}, result in a sequence where the \texttt{person}
nodes containing more than one \texttt{surname} node will occur more than one time. 

The problem lies in the fact that the query is not a erroneous MQL query, but a erroneous XQuery query. One
solution would be to implement a check in MQL, which would inform the MQL processor of any potential error. This
can be done e.g.\ by a MQL function \textsf{raiseError()}, which would abort the evaluation of the query and e.g.\
throw an exception. All items in the relational representation of sequences in
Tainting Dependencies are marked with their position within the sequence with
the $index$ field. One way to check if an expression $e$ does not return a singleton sequence can be the following:
\begin{center}
\begin{tabular}{l}
\textsf{select(ifthenelse(eq(index, 1), true, raiseError());} \\ \quad
\textbf{r(}$e$\textbf{)}\textsf{)}
\end{tabular}
\end{center}

The check may of course be omitted if the translator is sure, by the means discussed earlier, $e$ will return a
singleton.

\subsection{Effective boolean value}
\label{sect:disc:effBool}
Within certain circumstances it is necessary to find the effective boolean value
of a sequence. The effective boolean value is by W3C defined as follows\cite{w3c00}: 
\begin{quote}
The effective boolean value of a value is defined as the result of applying the \texttt{fn:boolean} function to
the value.
\end{quote}

Where the function is declared as \texttt{fn:boolean(\$arg as item()*) as xs:boolean} and the dynamic semantics are
defined as follows\cite{w3cfuncOps}:

\begin{enumerate}
  \item If \texttt{\$arg} is the empty sequence, \texttt{fn:boolean} returns \textit{false}.
  \item If \texttt{\$arg} is a sequence whose first item is a node, \texttt{fn:boolean} returns \textit{true}.
  \item If \texttt{\$arg} is a singleton value of type \texttt{xs:boolean} or a derived from \texttt{xs:boolean},
  \texttt{fn:boolean} returns \$arg.
  \item If \texttt{\$arg} is a singleton value of type \texttt{xs:string} or a type derived from \texttt{xs:string},
  \texttt{xs:anyURI} or a type derived from \texttt{xs:anyURI} or \texttt{xs:untypedAtomic}, \texttt{fn:boolean}
  returns \textit{false} if the operand value has zero length; otherwise it returns \textit{true}.
  \item If \texttt{\$arg} is a singleton value of any numeric type or a type derived from a numeric type,
  \texttt{fn:boolean} returns \textit{false} if the operand value is \textit{NaN} or is numerically equal to zero;
  otherwise it returns \textit{true}.
  \item In all other cases, \texttt{fn:boolean} raises a type error.
\end{enumerate}

In the chapter presenting Tainting Dependencies, chapter \ref{sect:translation}, we solved resolving the effective
boolean value of sequences with the help of an assumed MQL function \textsf{xqBoolean()} and an undefined abstract
function \textbf{B()}. In that chapter we said it was enough to consider it as a grouping function returning one
tuple per unique iteration (as implied by the iterator dependencies). The points 3-5 of the definition of the
effective boolean value are all handled by \textsf{xqBoolean()} without problems, as they all concern singleton
sequences. Points 2 and 6 together, however, creates a interesting situation.
They imply that when finding the effective boolean value of all possible XQuery sequences, all should cause an error, except the sequences where a
node is the first element, regardless of the other items in the sequence.

This means that a simple check as proposed for evaluating the cardinality of operands for e.g.\ arithmetic
expressions earlier in this section will not do. A possible solution for the \textbf{B()} function would be to run
a \textsf{group} operator on the relation, grouping on the unique iterations, and run a \textsf{count} and some
aggregator functions selecting the value and type of the first item of each group. Further, the result will then
have to be checked similar to the check proposed earlier.
\begin{center}
\begin{tabular}{l}
\textsf{select(or(eq(type, node), ifthenelse(eq(count,1), true, raiseError());} \\\quad
\textsf{group((}$e.\vartheta$\textsf{), count(), selProj(eq(index,1), value), selProj(eq(index,1), type);}
\\\quad\quad
\textbf{r(}$e$\textbf{)}\textsf{))}
\end{tabular}
\end{center}
Where \textsf{selProj(}$pred$, $field$\textsf{)} is a kind of selection and projection hybrid, selecting field
$field$ if predicate $pred$ is \textit{true}. Such a aggregator function may however seem strange, and there is
nothing similar implemented in the MQL processor.

An alternative can be to utilise the \textsf{groupby} operator. The semantics of this operator is the same as for
the \textsf{group} operator, except that the input relation is unchanged and returned as the output. Further, the
result of the grouping is returned as a separate, named relation (``result set'' in MQL lingo). The
\textsf{resultset} operator with the name as its parameter will fetch this named result:

\begin{center}
\begin{tabular}{l}
\textsf{select(and(eq(index,1), or(eq(type,node), ifthenelse(eq(count,1), true, raiseError())));} \\ \quad 
\textsf{hhjoin([}$e.\vartheta$\textsf{], [}$e.\vartheta$\textsf{], [l.value, r.count];} \\ \quad\quad
\textsf{groupby(countrelation, (}$e.\vartheta$\textsf{), count();} \\ \quad\quad\quad
\textbf{r(}$e$\textbf{)}\textsf{);} \\ \quad\quad
\textsf{resultset(countrelation)))}
\end{tabular}
\end{center}

Both solutions assume a type system where a field $type$ holds the type of the item represented in the tuple. The
last proposal of a implementation of the \textbf{B()} function only uses operators and functions already
implemented in the MQL processor. A disadvantage with this solution, however, is that it will probably be more
resource-demanding as it consists of both a grouping and a join.

Point \#1 in the definition of evaluation of effective boolean value is also a
bit problematic. This is because Tainting Dependencies, as it is presented in chapter \ref{sect:translation},
does not handle empty sequences explicitly. And in most cases this is sufficient, as non-empty sequences containing
empty items should be normalised. In the following query it is clear that the result must be all the
\texttt{maybe} nodes of the document which is a child of a \texttt{elem} node:

\begin{center}
\begin{tabular}{l}
\texttt{fn:doc("nodes.xml")//elem/maybe}
\end{tabular}
\end{center}

The \texttt{elem} nodes which does \emph{not} contain a \texttt{maybe} node (making \texttt{../elem/maybe}
an empty sequence) are irrelevant to the evaluation of this query. The only time the need for knowledge of a
possible empty sequence arises is when evaluating the effective boolean value. Consider the following XQuery
expression:

\begin{center}
\begin{tabular}{l}
\texttt{if(\$a/maybe) then} \\ \quad
\texttt{"exists"} \\ 
\texttt{else} \\ \quad
\texttt{"empty"}
\end{tabular}
\end{center}

Further, let \texttt{\$a} be an iterator variable consecutively bound to one and one \texttt{elem} node. If
some of the \texttt{elem} nodes does not contain a \texttt{maybe} node, some items of the resulting sequence
should be the string literal \texttt{"empty"}.

A possible solution to this would be to differantiate between the times the empty sequences are needed, and the
times they are not. This can be done by evaluating all the descendant
expressions of an expression which is to be calculated into a effective boolean value, in its own context. To evaluate some expressions in another matter
according to the context is made simple by implementing a context sensitive visitor pattern (section
\ref{sect:theory:contextVisitorPattern}). The difference in translating in the logical context as opposed to the
default context will only be that all joins on common dependencies will have to be made into full outer joins. In
addition joins such as in the axis+name test translation (rule \ref{rule:trans:TD:pathStep}) where only the left
operand may have dependencies, will have to be turned into a left-outer join. Always employing outer joins will
ensure that no $-numb$ field corresponding to one unique iteration will be removed from the relation. Using this
solution \textsf{xqBoolean()} must return $false$ if it is run with \textit{NULL} as input.
\section{Type System Considerations}
\label{sect:disc:typeSystem}
Currently, neither ``Tainted Dependencies'' (TD) nor the prototype
implementation assumes any form of availability of a type system. However
according to the formal semantics specification\cite{xquery_semantics} (and
also noted in \cite{pathfinder_compiling_xquery}, XQuery Core is inherently
fully statically typed. This suggests that full normalisation of queries to
XQuery Core would imply the availability of this. Static typing could help
solve some problems, such as distinguishing numeric predicates. However, XQuery
Core has semantics for solving this exact problem -- the predicate normalisation
mapping applies a typeswitch construct which contains the necessary logic to
differentiate between numeric predicates and other predicates. And regardless,
numeric predicates were solved in TD as per rule \ref{rule:trans:TD:predicate}.

Another related challenge is the lack of explicit typing in the MQL language.
The only concrete specification given by FAST is the fact that if a column is
typed, then if can only contain fields of that type. MQL has typed columns,
however it is not possible to specify type. For example, the \textsf{make()}
operator has no parameters for type specification. This is complicated further
by the fact that a XQuery sequence is simply just a sequence. The individual
items themselves can have vastly different types.

The intricate challenges related to typing are numerous. For example, consider
the sequence \texttt{(1, <a><b>2</b></a>, "3")}. This sequence can not be
represented in a relation without resorting to a BLOB\footnote{Binary
Large Object}-like data type for the value column. However, that implies that
the semantics of the second item as an XML element is lost unless it is somehow
serialised in a common format with which MQL is compatible. That again implies
that metainformation about fields may be required to indicate the type of the
contents.

Furthermore, the \texttt{isInScope()} function proposed in section
\ref{sect:method:marsAddedOperators} on page
\pageref{sect:method:marsAddedOperators} requires a scope column in the tuple
being tested. Consider this non-sensical example:

\begin{Verbatim}
for $i in (1,2,3) return $i/a
\end{Verbatim}

Somehow, this error must be discovered and prevented. Likewise, the following
example must be allowed without errors:

\begin{Verbatim}
for $i in (<a><b>1</b></a>, <a><b>2</b></a>, <a><b>3</b></a>) return $i/b
\end{Verbatim}

On a final note, non-heterogeneous sequences are seldom of practical use, and
can appear irrational. Path expressions always return nodes, string operations
always return strings, and so on. Non-heterogeneous sequences are, as far as
known to the authors, only specifiable by an end-user of XQuery, for example by
attempting to execute a query such as this:

\begin{Verbatim}
for $i in (1, <a><b>2</b></a>, "3") return $i * 2;
\end{Verbatim}

This particular example appears non-sensical, and will likely not execute. For
example, the Saxon parser returns this run-time error message when attempting
to execute the above query:

\begin{Verbatim}
XPTY0004: Unsuitable types for * operation
Query processing failed: Run-time errors were reported
\end{Verbatim}

Naturally, it appears that typing is an important but complex aspect of XQuery,
and several issues such as the ones described here needs to be solved for a
proper implementation to take place.

% \begin{itemize}
%   \item Hvordan f\aa~til noe typesystem?
%   \item Hvordan ordner Pathfinder typer? Kan vi gj\o re det samme? Finnes det noen andre vi kan dra kunnskap fra?
%   \item lagre false som ``false'' enn s\aa~lenge.. kjipt med /a/b[/a/b/c] hvis c er slik: <c>false</c>
%   \item Mars st\o tter ikke forskjellige typer innenfor samme felt
%   \item En sekvens er en sekvens i XQuery\ldots ikke en sekvens av booleans
%   eller noder etc
%   \item Et ekstra felt som sier type?
%   \item Hva skjer med /a/b/c/text() vs /a/b/c ?
%   \item hva skjer om man lager en <a> hei <b> jeje </b> </a> variabel? Dette
%   m\aa~kunne representeres.
%   \item Hvis vi hadde hatt statisk og sterk typing s\aa~ hadde mye v\ae rt
%   ordna, f.eks \verb!for $i as xs:int in (1,2,3) return /a[$i]! s\aa vet man
%   med en gang at \verb![$i]! er en ``numeric predicate''.
%   \item typeswitch / instance of / cast / castable / treat as
%   \item There is however also a need to represent explicitly stated XML-nodes, as well
% 	as differentiate between the number \texttt{1} and the string \texttt{"1"}.
% 	This, and other issues about representing XQuery types will be treated in
% 	section \ref{sect:disc:typeSystem}.
% 	\item sequencetype\ldots  
% \item Hva med \$i = (1,2,3) \$i/hatt -> typefeil? kj\o re isInScope(scope) p\aa~noe som ikke har scope kolonne?
%   		\item constructors\ldots element, CDATA, attribute etc etc
%   		\item typeswitch? kan bli stress.. har ikke s\aa~mye typer enn\aa \ldots skal v\ae re diskusjon om typer i
%   		sect \ref{sect:disc:typeSystem}\ldots
% 	\end{itemize}