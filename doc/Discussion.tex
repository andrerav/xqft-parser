\chapter{Discussion}
\label{chapter:discussion}
In this chapter, some important aspects of this project will be discussed. Some
of them are concrete problems or challenges, others are of a more predictive
nature and may serve as a basis for further research. Also, possible
solutions or ideas are proposed where applicable. 

Most importantly, XQuery features not
currently supported by ``Tainting Dependencies'' are discussed, as well as
possible optimisations. The proof of concept whose implementation was
described in chapter \ref{chapter:implementation} is discussed, and in
particular implementation-specific details. Further, the results presented in
the previous chapter are discussed, and insecurities and sources of errors are
accounted for. This chapter is finalised by discussion of syntax tree rewriting
and normalisation, a discussion of problems related to sequences, and a
discussion of the XQuery type system and related problems.

\textbf{\LARGE TODO:} skal dette inn igjen?
, we have a few assumptions. A scope defined as a parameter to
the \textsf{scope}-operator starting with a slash (\textsf{/}) is an absolute scope. That means if the parameter
is e.g. \textsf{/a/b}, the operator will remove any tuples in the input relation where the $scope$ attribute does
not define the tuple to stem from a \texttt{b} scope within the root-scope \texttt{a}. Without a slash first in
the parameter, the scope is relative. E.g. \texttt{a/b} as a parameter will lead to removal of all tuples where
the $scope$ attribute does not define the tuple to stem from a \texttt{b} scope within an \texttt{a} scope.



\section{Context Item - {MADS}}
\label{sect:disc:ctxItem}
\textbf{\LARGE TODO: {MADS}}
\begin{itemize}
  \item m\aa~nesten dytte den til symtab hele tiden, koster ikke mye, s\aa~lenge den ikke blir brukt.
  \item Man skal vel ha tilgang til den hele tiden omtrent.
  \item kan sette context node til noe n\aa r man starter\ldots Tenke en virtuell /collection/doc struktur\ldots
  ellerno
  \item XQuery has a set of functions that provide access to input data. These functions are of particular
  importance because they provide a way in which an expression can reference a document or a collection of
  documents. The input functions are described informally here; they are defined in [XQuery 1.0 and XPath 2.0
  Functions and Operators]. An expression can access input data either by calling one of the input functions or by
  referencing some part of the dynamic context that is initialized by the external environment, such as a variable
  or context item. -> Appendix C.2 Dynamic Context Components
\end{itemize}
\section{XQuery Features Not Supported - {MADS}}
\label{sect:disc:notSupported}
\textbf{\LARGE TODO: Mads}
\begin{itemize}
\item noen av disse tingene er ikke st\o tta pga typesystem, andre, slik som stemming og thesaurus er fordi ikkeno
slik i mars enn\aa~kan foresl\aa~det med parametere til lookup evt contextting ala \textsf{index()}.
  \item Dette avhenger jo seff av hvor mye vi har l\o st men disse b\o r kunne
  l\o ses:
  	\begin{itemize}
  		\item proximity (kanskje l\o sbar)
  		\item declare variable er nesten st\o tta\ldots men external og greier..
  		\item function declarations, b\o r g\aa~greit, bare ha en function table ala symbol table.
  		\item schema / schema validation -> typeting? Hvordan l\o ser pathfinder dette\ldots synes \aa~ha sett noe om
  		det\ldots
  		\item namespacezz\ldots.
  		\item node comparisons\ldots\ldots tviler p\aa~at vi f\aa r til dette glatt\ldots
  		\item order by med alle ting..
  		\item ordered and unordered -> lurer p\aa om (markXremove + tainting = TD uten $index$ og
  		\textsf{numberate()}) fikser dette ganske bra\ldots
  		\begin{itemize}
			\item MarkXRemove funker bra i unordered m0de tror jeg\ldots. Den er ogs\a~normalisert ref purely relational
			flwors, som sier LL er denormalisert. (bare normalisert innenfor en flwor.. den unormaliserer seg n\aa r den
			g\aa r ut av l\o kka (cross /m const))
			\item Order/unorder kan dra nytte av kontekstsensitive visitors\ldots\ldots
		\end{itemize}
  		\item \textbar, \texttt{union}, \texttt{intersect, except}
  		\item Range expressions $e_1$ \texttt{to} $e_2$ (begge m\aa~v\ae re integer tror jeg -> skrive om det i type?)
  		\item Prologs and modules
  		\item Quantified Expression: (some | every) \$b in $e_1$ satisfies $e_2$
  		\end{itemize}
	\end{itemize}

\subsection{Order By - {MADS}}
\label{sect:disc:orderby}
\textbf{\LARGE TODO: {MADS}}
\begin{itemize}
  \item st\o tte alle de sorteringsspesifiseringene.. st\o tte sortering over flere exprz.
\end{itemize}
	
	
	
	\subsection{XQuery Functions - {MADS}}
\label{sect:disc:functions}
\textbf{\LARGE TODO: {MADS}}
\begin{itemize}
  \item hvordan ordne XQuery funksjoner?
  \item Tror det skal v\ae re lagt til rette for \aa~ha en \textsf{function(FUNCTIONNAME; operator(list?)} operator
  \end{itemize}
  
fra 3.1.5:
\begin{quote}
  Additional functions may be declared in a Prolog, imported from a library module, or provided by the external
  environment as part of the static context.
  \end{quote}
\section{Optimisation}
\label{sect:disc:optimisations}

In section \ref{sect:trans:TD:simplifications} we presented some situations where the translation from XQuery into
MQL can be specialised and simplified. In this section we will outline some ideas for further simplification and
optimisation.

As can be seen of the tree in figure \ref{fig:results:query_ifthenelse_result} on page
\pageref{fig:results:query_ifthenelse_result}, Tainting Dependecies can produce relational algebra trees with
consecutive \textsf{project} operators. This is of cource unnecessary, and such operators can be merged into one
single \textsf{project}. Attributes which is not part of the result or part of evaluation of the result can be
pruned. This may however not be anything the translator would need to consider, as the Mars optimiser allready
implements methods for detecting and simplifying such trees.

Rule \ref{rule:trans:TD:conditional} on page \pageref{rule:trans:TD:conditional} describes the translation of
XQuery \texttt{if..then..else} expressions. It was implemented considering the result would be a operator tree and
not a DAG, and the number of operations should be minimised. In this translation both the \texttt{then}-expression
and the \texttt{else}-expression will be evaluated and their results spliced together. From this relation the
tuples stemming from the right expression according to the result of the test expression is selected. If the MQL
processor labours DAGs it may be a better solution to evaluate the two expressions only for the cases where they
are to be returned. 

This may be done in a matter very similar to the Loop Lift solution (equation
\ref{eq:ll:ifthenelse} on page \pageref{eq:ll:ifthenelse}). In this solution different $loop$ relations are made
depending on the outcome of the test-expression. One of the relations are used to evaluate the
\texttt{then}-expression, another the \texttt{else}-expression. As Tainting Dependencies does not utilise $loop$
relations, some changes would have to be made. Evaluation of the test-expression will reveal which unique
iterations which expression is to be evaluated in. Based on this information iterations can be pruned from the two
expressions before they are evaluated.

\subsection{XQuery Semantics}

As XQuery is a purely functional language an implementation is always free to evaluate the operands of an operator
in any order\cite{w3c00}. This means that a MQL optimiser is able to rearrange operators as it sees fit, with the
possible outcome of a less expensive query execution. One such case would be in a path expression such as this:
\begin{center}
\texttt{\$i/child::elem/descendant::maybe}
\end{center}
If the optimiser has some notion of the frequency of occurence of the \texttt{maybe} and \texttt{elem} elements,
as well as the cardinality of the relational representation of \texttt{\$i}, it may not be evaluated from left to
right. If there exists only a few \texttt{maybe} elements and the \texttt{\$i} relation is relatively large, the
most effective execution plan for this path expression would most likely be from right to left.

In some cases, a XQuery implementation can determine the result of an expression without accessing all the data
that is implied by the formal expression semantics. A consequence of this is that some errors goes undetected. W3C
specify to which extent an implementation may optimise its access to data at the cost of not detecting errors
like this \cite{w3c00}:

\begin{quote}
Consider an expression Q that has an operand (sub-expression) E. In general the value of E is a sequence. At an
intermediate stage during evaluation of the sequence, some of its items will be known and others will be unknown.
If, at such an intermediate stage of evaluation, a processor is able to establish that there are only two possible
outcomes of evaluating Q, namely the value V or an error, then the processor may deliver the result V without
evaluating further items in the operand E. [\ldots]

There is an exception to this rule: If a processor evaluates an operand E (wholly or in part), then it is required
to establish that the actual value of the operand E does not violate any constraints on its cardinality. [\ldots] 
\end{quote}

This feature could be utilised in situations where path expressions is to be evaluated to effective boolean values
(as described in section \ref{sect:disc:effBool}). If the path expression is to return either nodes or an empty
sequence, it is sufficient to find \emph{one} node per unique iteration. This might prove difficult to implement
on an MQL processor however, as it is hard to know which unique iteration a element belongs to before the element
relation and a relation with representations in all unique iterations are joined together.

Another situation where not accessing all data is required is with numeric predicates. The formal description of
filter expressions\cite{xquery_semantics} says that a expression such as \texttt{\$s[1]} should be evaluated by
consecutively examining the items of the sequence \texttt{\$s}, and selecting all items where
\texttt{position()=1}. A better solution would be to only pick the first item of the sequence. But as Tainting
Dependencies stores sequences from many iterations in the same relation, and no ordering can be assumed, this can
not be done easily. The MQL processor has no way of knowing where the tuples holding the items which are the
first item of their respected sequences unless there is some kind of indexing of the relation on the $index$
fields. In such a case the processor could lookup the tuples where $index$ has the value 1, the alternative is a
sequencial scan of the relation, checking each tuple.

\subsection{Path Expressions}
Quite a lot of research has been done on optimising XPath (Path expressions in XQuery stem from XPath) since it
became a W3C Recommendation in 1999. Of the documents produced by this research we would recommend
\cite{optimize_logic}, \cite{optimize_michiels} and \cite{optimize_xsltPath}. As this is outside the scope of this
report the contents of these will not be resited here. However, \cite{optimize_forward} presents a reverse axis
removal algorithm which may be interesting in a MQL and Taiting Dependencies specific setting. The algorith
recognises path expressions containting reverse axis steps, and rewrites them into pure forward axis step
expressions with possible predicates. This may be helpful as the MQL \textsf{scope} operator only accepts steps
equivalent to the \texttt{child} axis.

By letting the translator keep track over valid consecutive \texttt{child} axis steps, the \textsf{scope} operator
may be employed to filter the results from the lookup (ref rule \ref{rule:trans:TD:pathStep}, page
\pageref{rule:trans:TD:pathStep}), most likely reducing the tuples involved in the subsequent join. Consider the
following path expression:

\begin{center}
\texttt{/a/b[g]/c//d[h]/e}
\end{center}

Here, the result of looking up \texttt{g} may be filtered by a \textsf{scope} operator with \textsf{/a/b/g} as its
parameter. Similarly, the lookup of \texttt{c} may be filtered on \textsf{/a/b/c}. Because the name test \texttt{d}
is not part of a \texttt{child} axis step (rather a \texttt{descendant-or-self} axis step), no filtering can be
done. \texttt{e}, however, may be filtered by \textsf{scope} with the parameter \textsf{d/e}. Some consideration
will still have to be done concerning how much cheaper the join will become with the filtering compared to the cost
of filtering one of its operands.

One idea would be to move all predicates out of the path expression, and apply them as post filters. This would
probably reduce the number of needed joins -- at least in some cases. In other cases it may be to costly to build
a filter fitted to the whole path expression. If e.g. \texttt{h} from the last example query was a number instead,
the path expression up until \texttt{d} would have to be evaluated, filtered with the \texttt{g} predicate, and
finally joined with the evaluation of the path expression without the last predicate on their \texttt{d} step in
the $scope$ fields.

The rule for translating predicates, rule \ref{rule:trans:TD:predicate} on page \pageref{rule:trans:TD:predicate},
is a general rule. Here, a reference to the context item from the outside of the predicate is ``copied'' inside the
predicate (containting $sprDotNumb$). This copy is then operated on, and when the predicate expression is
evaluated, the predicated expression and the predicate expression is joined on the reference to the context item.
An example of this process is seen in figure \ref{fig:results:query_ifthenelse_result_dag} on page
\pageref{fig:results:query_ifthenelse_result_dag}. Here, the copying is in form of a upwards split, and the If the
context item is only referred to once within the predicate this translation may however be simplified.

Consider the following XQuery query:
\begin{center}
\texttt{//person[name eq "Robert"]}
\end{center}

This can be solved by joining the \texttt{name} relation with the \texttt{person} relation, and keeping all the
attributes from both relations. The attributes stemming from the \texttt{name} relation would have to be marked in
some way, as these are not a part of the result of the query. This relation will have to be filtered with a
selection removing the tuples where the $scope$ stemming from the \texttt{name} is not a child scope of the
$scope$ stemming from the \texttt{person} relation. Further, only the tuples where the $value$ stemming from the
\texttt{name} relation is \texttt{"Robert"} is retained. Finally the attributes stemming from the \texttt{name}
relation are projected away.

A similar solution may be considered for other types of predicates as well, but if there is more than one
reference to the context item within the predicate things get more cumbersome. In such a case the method would
have to be sure that no reference to the context item is removed from the relation at any time before the
finalisation of the evaluation.
\section{Implementation}
\label{sect:disc:contextSens}
Chapter \ref{chapter:implementation} describes how a prototype was implemented
to demonstrate the ``Tainted Dependencies'' method. This implementation
was dependent on a number of constraints:
\begin{itemize}
  \item The availability of a free\footnote{By ``free'' is meant a liberal
  license and availability of source code} pre-made XQuery parser capable of
  producing abstract syntax trees
  \item The ability to parse and manipulate syntax trees and re-write them into
  new structures
  \item The ability to translate syntax trees into MQL (MARS relational algebra)
\end{itemize}

In this section, the methods chosen to achieve the goals of the implementation
are discussed and elaborated upon.

\subsection{Manual vs. automated tree parser construction}
ANTLR provides a utility for automated construction of AST parsers. This
utility requires the specification of a separate tree grammar. The tree
grammar is almost identical to the original parser grammar. Practically, the
parser grammar can be copied verbatime, renamed, modified slightly and used as
a tree  grammar. This process is described in detail in \cite{definitiveAntlr},
section 8.1.

This introduces some redundancy. If not all nodes in the AST can be matched by
the tree grammar, the parser will throw an error for encountering an unknown
token. This implies that an ANTLR tree grammar will need to recognise all tokens
in a syntax tree, thus the tree grammar in some ways depends on being
synchronised with the parser grammar to be able to function properly.

This creates a potential problem with maintainability. As the parser grammar and
rewrite rules are not freezed at this point but rather highly subject to
change, any changes made in the parser grammar will need to be transferred to
the tree grammar, and vice versa.

In \cite{translators_should_use_tree_grammars}, Terence Parr argues that the
traditional visitor pattern (section \ref{sect:theory:visitorPattern}) only
provides a simplistic facility for triggering events on the AST, that no tree
structure validation is implicitly available, and that context information has to be
passed down through the tree during the parse or by setting global variables.

In another point of view strongly polar to that of Terence Parr, Andy Tripp
argues that manual tree parsing is better\cite{manual_tree_walking_is_better}. He establishes the following points
of argument which are of particular interest to this project:
\begin{itemize}
  \item Duplication of code and effort -- the concept of ``what is a valid AST''
  would have to be implemented in both the parser and the AST transformer phase
  (as a rebuttal to validation of AST).
  \item With regards to contextual information, There seems to be nothing wrong
  with depending on the physical structure of the AST.
  \item Defining a traditional parser in grammar is practical because the grammar
  usually resembles the ouput AST. In the case of a tree parser proposed by Parr
  where the grammar actually resembles the input AST, this mapping may break
  down completely if the output is another tree structure.
\end{itemize}

In particular, the last point holds a strong indication that a tree grammar
may not be suited for this project, as the goal of this tree parser would be to
transform the AST into a relational algebra tree.

\subsection{AST rewriting and the visitor pattern}
In chapter \ref{chapter:method}, methods to achieve the goals
of the implementation were presented. The method chosen for parsing of the
abstract syntax tree (see section \ref{sect:method:tree_parsing}) was the
\emph{context sensitive visitor pattern}. This pattern laid the foundation for
a clean and simplistic implementation. The semantics of the tree parsing process
itself did not interfer unecessarily with the rest of the implementation.

The process of rewriting the abstract syntax tree was implemented as a
stand-alone visitor (the \texttt{RewriteVisitor} class). This implementation
exploited the visitor pattern extensively, resulting in a clear separation of
concerns. In particular, it seems to hold true that the visitor pattern
typically will cleanly separate a data structure from an algorithm which is
operating on that structure.

\subsection{Constructed algebra trees and performance}
As explained in section \ref{sect:impl:construct_mql} on page
\pageref{sect:impl:construct_mql}, the MQL is constructed as an in-memory tree
structure. This was done by instantiating a new \texttt{Operator} subclass
(the exact class depending on context) for every node in the tree. It is
important to note that even though this could become a performance bottleneck
for very large and complex queries, it is still an important trade-off. In
exchange for a theoretical performance bottleneck, the implementation achieves a
higher level of maintainability.
\section{Results}
\label{sect:disc:res}
\textbf{\LARGE TODO: {{ANDREAS}}}

\subsection{Translation output}
\label{sect:disc:res:translation_output}
In sections \ref{sect:result:theoretical_algebra} and
\ref{sect:result:implementation_algebra}, a series of XQuery queries were
translated using the novel ``Tainting Dependencies'' (TD) methodology developed
and described in chapter \ref{sect:translation}. In section
\ref{sect:result:theoretical_algebra}, where hand-computed translations were
presented, a series of simplifications were applied (these simplifications were
described in section \ref{sect:trans:TD:simplifications}). However, the
prototype developed in chapter \ref{chapter:implementation} did not implement
any of these simplifications. This was an important point to keep in mind when
later comparing this algebra to that generated by Pathfinder, and is
discussed more thoroughly in the next section.

One characterisation of the algebra generated by TD is that nodes with more
than one parent node are typically located far towards the bottom of the algebra
tree. 

Another characteristic of the algebra is that is seems to maintain a fairly
compact form. This is partly due to the fact that the tainting process does not
affect constant subexpressions, and thus the potential size of the algebra is
reduced significantly. Compare this to loop lifting used by
Pathfinder, where all expressions within a loop body are loop lifted --
as explained in sections \ref{sect:trans:ll:ConstExprs},
\ref{sect:trans:ll:mappingBack}, and \ref{sect:trans:ll:OtherExpr}.

TODO: parallellisering \^ ?

\subsection{Complexity comparison}
\label{sect:disc:res:comparison}
The complexity calculation method (see section \ref{sect:method:complexity} on
page \pageref{sect:method:complexity}) defined by \O ystein Torbj\o rnsen at
FAST was used to compare complexity in the algebra generated by the prototype
implementation to that of Pathfinder. This comparison was
based on three queries (dubbed \emph{Trivial}, \emph{Complex}, and
\emph{Conditional}). For each of these queries, algebra was generated on both
the prototype implementation as well as Pathfinder. Then the described
method of calculating complexity was applied to these trees, and the result was
used to compare the prototype and Pathfinder.

Though an interesting comparison, this is a sparse source of data -- it
is difficult to be conclusive based this data alone. However, with the
exception of the most trivial query, it seems that ``Tanting Dependencies'' (TD)
generates substantially less complex algebra than Pathfinder. As one
may consider TD an enhancement of the loop lifting technique, this should
emerge as a natural consequence.

However, there are some sources of uncertainty for this comparison. It is not
known exactly to which degree simplifications and optimalisations have been
applied to the algebra generated by Pathfinder. In any case, the
algebra generated by TD is not simplified or optimised, and as such puts these
implementations on equal footing. 

It is also known that Pathfinder does \emph{not}:
\begin{itemize}
  \item generate algebra using pure loop lifting as would be expected from
  \cite{pathfinder_mothertongue} and \cite{pathfinder_purelyRelational}
  \item apply all simplifications described in
  \cite{pathfinder_purelyRelational}
\end{itemize}

Furthermore, it is natural to assume that Pathfinder generates algebra
which is tailored for execution on the MonetDB database system (as indicated
in \cite{pathfinder_purelyRelational}).

With regards to performance measurement, and with the lack of availability of a
proper implementation of a MQL processor (as mentioned in section
\ref{sect:method:mql} on page \pageref{sect:method:mql}), it could have been
benefitial to generate algebra using TD for MonetDB, and compare
actual performance of loop lifting vs. TD on this database system. This notion
is further detailed in section \ref{chapter:future}.

Finally, it is important to note that the complexity comparison performed did
not in anyway account for I/O or CPU usage, and the results must not be
interpreted as such. Again, see section \ref{sect:method:complexity} on
page \pageref{sect:method:complexity} for an detailed account of this method.

\subsection{Loop Lift vs Tainting Dependencies}
\label{sect:disc:llvsTD}
This project has studied two approaches for translating XQuery to relational
algebra; loop lifting as implemented by Pathfinder, and ``Tainting
Dependecies'' (TD), which is a novel method developed in this project which
still shares a few common traits with loop lifting. TODO: skrive om likheter

A common trait of loop lifting, which is also noted in
\cite{pathfinder_mothertongue} and \cite{pathfinder_purelyRelational}, is that
the algebra trees will quickly grow very large. Consider the example in section 
\ref{sect:trans:ll:example} on page \pageref{sect:trans:ll:example}; here, the
intermediate results grow in size very quickly. In particular, they are
comparatively large seen in the context of the trivial query they are produced
from.

When the algebra trees generated by loop lifting are converted to DAGs, this
trait may not seem so appareant. However, it is easily recognized by the fact
that nodes with more than one parent often are located in the middle and higher
parts of the tree, for example as seen in figure
\ref{fig:result:comparison:conditional_pathfinder_dag} on page
\pageref{fig:result:comparison:conditional_pathfinder_dag}. If this particular
DAG was converted to a tree, it would indeed be substantially larger.

TODO: blande inn parallellisering her ogsaa? Mmm parallellisering. Deilig!

Further, when comparing the rules in loop lifting (described in
\cite{pathfinder_mothertongue}) and TD, it appears that TD will in most cases
produce less operators and less complex trees than loop lifting, as well as
substantially smaller intermediate results. This comparison does not consider
simplifications, however. In some situations, especially for trivial queries,
optimised loop lifting \emph{may} perform better than TD. 

\begin{itemize}
  \item vise til ll eksemplet sect \ref{sect:trans:ll:example} og f\aa~frem at
  ll krever DAGz ganske hardt, mye som blir brukt om igjen. Og fra
  begynnelsen til slutten for den saks skyld.
  \item fordeler vs ulemper..
  \item dra frem at TD bruker f\ae rre operatorer
  \item pathfinder way kommer ikke til \aa~dra nytte av den mer ekspressive
  mars-algebraen\ldots Men er ekspressiv  bedre? Synes jeg s\aa~ noe i en av
  pathfinder artiklene hvor de sa at jo mer restriktiv, jo bedre
  \aa~optimisere\ldots snakke med thorbj\o rnsen om dette.. 
\end{itemize}
\section{Normalisation and rewriting}
\label{sect:disc:rewriting}
In section \ref{sect:method:ast_rewrite} on page
\pageref{sect:method:ast_rewrite}, some methods for rewriting (normalising)
certain expressions to XQuery Core were described. In the prototype implementation,
these rewrites are made using the \texttt{RewriteVisitor} class. The advantage
of normalising to XQuery Core is simplification of the syntax tree while
maintaining full semantics. That is -- the final syntax tree may be bigger and
appear more complex. However, expressions such as FLWOR and path expressions
are split into smaller subexpressions that are easier to parse by themselves. 

This project has taken an pragmatic approach to normalisation. The rules
defined in chapter \ref{chapter:translation} do not rely on normalisations.
However, for the sake of simplicity in the prototype, a rewrite visitor was 
applied to simplify FLWOR expressions. This must be seen in the light of the
fact that XQuery Core is a very extensive specification\cite{xquery_semantics},
and so strict adherence to this specification would imply a substantially
larger amount of effort into normalisation.

Consider the normalisation of $RelativePathExpr/StepExpr$ which is normalised
into a FLWOR expression\footnote{See
http://www.w3.org/TR/xquery-semantics/\#id-axis-steps for details}. This may be
counter-productive as the usage of the \textsf{scope()} operator in MQL will
imply that this normalisation will somehow have to be reversed.

Furthermore, the normalisation of FLWOR expressions themselves require that
where-clauses are rewritten to if-then-else expressions. The rules for this
normalisation process is shown in section
\ref{sect:theory:xquery:core:rewriting_flwor} and figure
\ref{figure:xquery:where_mapping_rule} on page 
\ref{figure:xquery:where_mapping_rule}. However, in ``Tainting Dependencies''
(TD) the translation of a where-clause (rule \ref{rule:trans:TD:where} on page
\pageref{rule:trans:TD:where}) is optimised and shown to be more efficient than
the translation of an equivalent if-then-else expression (rule
\ref{rule:trans:TD:conditional} on page \pageref{rule:trans:TD:conditional}).
This is a paradoxical situation, and raises the question of whether other
normalisation rules may also affect the efficiency of the resulting
translation.

On a final note, the TD method in its current state does not rely on
denormalised XQuery -- however it is compatible with XQuery Core since XQuery
Core is a subset of XQuery.

% TODO: si seg ferdig med dette?
% 
% \begin{itemize}
%   \item fordeler vs ulemper med \aa~skrive om til core
%   \item man mister jo informasjon\ldots. Hvis den er p\aa~denne m\aa ten --> gj\o re det akkurat
% 	  slik, en sp\o rring som skal gi tilsvarende svar er ikke sikkert at man kan
% 	  skrive p\aa~den samme m\aa ten helt uten videre..  
%   \item samme svar = samme utf\o relse = er dette en fordel?
%   \item kan man utnytte kunnskap om translation til \aa~optimisere xquery queries?
%   \item hva med \aa~bare skrive om det man trenger? Hva trenger man \aa~skriveom?
%   \item sl\aa~sammen scopes blir frem og tilbake og frem igjen om man tenker core\ldots /a/b/c-> for..for..for ->
%   /a/b/c
%   \item se om det er mer tungvinne ting Core har\ldots har ikke satt meg helt inn i alle detaljene der.. hva skjer
%   med predikater? blir de if.then.else? kan det hende at vi f\aa r un\o dvendige kryssprodukt?
%   \item sluppet where-regel hvis man skriver om til if-then-else (dagens where er mer effektiv enn en if() then no
%  e else tomt -> den VET at det som er usant blir kasta bort- jfr regel og select-operator)
%  \item TD forutsetter ikke fullversjon XQuery (anti core), da core er subset av fullversjon
% \end{itemize}
\section{XQuery Sequences - {MADS}}
\label{sect:disc:singelton}
\textbf{\LARGE TODO: {MADS}}
\begin{itemize}
   \item (1,2) + 3 kan oppdages som feil ved at man merker iteratorvariable, atomiske og sekvenser + 
   \item at man synes det er flott for optimalisering av sequencebygging 
   \item order by (1,2) kan ogs\aa~oppdages. 
   \item In some cases, a processor can determine the result of an expression without accessing all the data that
   would be implied by the formal expression semantics. For example, the formal description of filter expressions
   suggests that \$s[1] should be evaluated by examining all the items in sequence \$s, and selecting all those
   that satisfy the predicate position()=1.
\end{itemize}

fra 2.3.4 \cite{w3c00}
\begin{quote}
Consider an expression Q that has an operand (sub-expression) E. In general the value of E is a sequence. At an
intermediate stage during evaluation of the sequence, some of its items will be known and others will be unknown.
If, at such an intermediate stage of evaluation, a processor is able to establish that there are only two possible
outcomes of evaluating Q, namely the value V or an error, then the processor may deliver the result V without
evaluating further items in the operand E. For this purpose, two values are considered to represent the same
outcome if their items are pairwise the same, where nodes are the same if they have the same identity, and values
are the same if they are equal and have exactly the same type.

There is an exception to this rule: If a processor evaluates an operand E (wholly or in part), then it is required
to establish that the actual value of the operand E does not violate any constraints on its cardinality. For
example, the expression \$e eq 0 results in a type error if the value of \$e contains two or more items. A processor
is not allowed to decide, after evaluating the first item in the value of \$e and finding it equal to zero, that the
only possible outcomes are the value true or a type error caused by the cardinality violation. It must establish
that the value of \$e contains no more than one item.
\end{quote}

\subsection{Effective Boolean Value - {MADS}}
\label{sect:disc:effBool}
\textbf{\LARGE TODO: {MADS}}
\begin{itemize}
\item VI M\AA~SJEKKE ATOMIZATION.
\item Effective Boolean Value
\item Kommer til \aa~bli feil n\aa r en sekvens blir brukt som boolean value n\aa~s\aa fremt vi ikke kan groupe\ldots
\item En evt groupingoperator m\aa~slippe igjennom tall\ldots. Evt select index=1, m\aa~se om (1,2,3) er bogus
\item The translator is in the logical context, $\Lambda$, if the AST node it is currently visiting is a successor
of a boolean operator or within the condition part of an \texttt{if..then..else} expression. In all other cases the
translator is in the default context, $\Delta$. If no context is mentioned in the inference rules the default
context is assumed. 
\item test expr i if
\item barna til and og or
\end{itemize}
\section{Type System - {ANDREAS}}
\label{sect:disc:typeSystem}
\textbf{\LARGE TODO: {ANDREAS}}
\begin{itemize}
  \item Hvordan f\aa~til noe typesystem?
  \item lagre false som ``false'' enn s\aa~lenge.. kjipt med /a/b[/a/b/c] hvis c er slik: <c>false</c>
  \item Mars st\o tter ikke forskjellige typer innenfor samme felt
  \item En sekvens er en sekvens i XQuery\ldots ikke en sekvens av booleans
  eller noder etc
  \item Et ekstra felt som sier type?
  \item Hva skjer med /a/b/c/text() vs /a/b/c ?
  \item hva skjer om man lager en <a> hei <b> jeje </b> </a> variabel? Dette
  m\aa~kunne representeres.
  \item Hvis vi hadde hatt statisk og sterk typing s\aa~ hadde mye v\ae rt
  ordna, f.eks \verb!for $i as xs:int in (1,2,3) return /a[$i]! s\aa vet man
  med en gang at \verb![$i]! er en ``numeric predicate''.
  \item typeswitch / instance of / cast / castable / treat as
  \item There is however also a need to represent explicitly stated XML-nodes, as well
	as differentiate between the number \texttt{1} and the string \texttt{"1"}.
	This, and other issues about representing XQuery types will be treated in
	section \ref{sect:disc:typeSystem}.  
\item Hva med \$i = (1,2,3) \$i/hatt -> typefeil? kj\o re isInScope(scope) p\aa~noe som ikke har scope kolonne?
  		\item constructors\ldots element, CDATA, attribute etc etc
  		\item typeswitch? kan bli stress.. har ikke s\aa~mye typer enn\aa \ldots skal v\ae re diskusjon om typer i
  		sect \ref{sect:disc:typeSystem}\ldots
	\end{itemize}