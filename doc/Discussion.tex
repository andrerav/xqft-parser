\chapter{Discussion}
\underline{\textbf{\LARGE //TODO:}} tekst her... innledning
\subsubsection{Write Parser}
\underline{\textbf{\LARGE //TODO:}} Noe Torbj\o rnsen snakka om.. Hiv p\aa~ om du kommer p\aa~ mer. Dette kan v\ae re innledning?

Hvorfor skrive en parser i det heletatt, hvorfor ikke bare bruke en eksisterende. Fordi det er oppgaven, hva med lisens? Hvorfor valgte vi denne oppgaven, og ikke den andre...? Sette seg inn i kode tar masse tid, lager vi selv har vi mer oversikt...

\underline{\textbf{\LARGE //ODOT:}}

husk generelt: Gjorde vi riktige valg? Ble ting som forventet? Hva hadde skjedd om vi brukte annet alternativ etc..

\section{Design decisions}
\begin{itemize}
\item ting vi har gjort, som vi senere maatte forandre (f.eks. validerende semantiske predikat og CharNotMinus)
\item 
\end{itemize}
\section{Adapting the W3C Grammar}
\label{sect:discussion:adaptW3C}

In the beginning of this project, very inexperienced parser developers as we
were, had the na\"{i}ve idea that since the XQuery grammar was given and
expressed in EBNF, our task was to run this through a parser generator,
possibly with some syntax changes, and the job would be done. After a while of
trying to adapt the given grammar to something ANTLR would accept, we came to
the conclusion that W3C probably intended the specification to be read by
humans, and not by computers. 

An alternative to insert and adapt would be to write a grammar from scratch
while keeping the semantics specified by W3C. We belive that by doing so there
is a risk of over-simplifying, causing latent semantics such as operator
precedence to be distorted, or in worst case lost alltogether. Another negative
aspect with this approach is that it does not properly utilize the work already
done, making it time consuming compared to just adapting.

The terminal productions had to be completely rewritten. By choosing to write
these rules from scratch in the first place, we would in all likelyhood have
discovered the problem with the ambiguous terminals earlier, but at the cost of
not having a parser up and running (albeit a very reduced one) until the
grammar was completly finished. This would have prohibited us from working with
e.g. the scoping system in parallell with the grammar.

Considering the non-terminal productions, their rewrites were not as extensive.
Most of the work done on the parser grammar were in the form of left factoring
and augmenting with syntactic predicates to reduce required lookahead, or in the form of
acommodating for the extra-grammatical constrains. We can not see that any
particular benefit would be gained from writing the non-terminal productions
from scratch, compared to adapting the W3C specification.

\section{ANTLR}
Rett valg?
Vi kjipa med CUP + JFlexxx, hva skjedde med \aa~skrive for h\aa nd?
Har ikke peiling p\aa~ CUP og JFlexxx, AST mye lettere med ANTLR? Mindre
omskriving syntaktisk? Har de predikater der? Kunne vi l\o st statetingen p\aa~
en annen m\aa te med CUP og Flexz?   

In this project we used ANTLR to generate a parser from a grammar
specification.As detailed in section \ref{sect:method:alternatives}, we
evaluated several alternatives before deciding to use ANTLR. One argument for
choosing a parser generator rather than rewriting the parser from scratch was to
save time. In particular, considering the ambiguities in the grammar
specification, it seems obvious that writing a parser from scratch would
have required an order of magnitude more time. Additionally, the quality of the
resulting parser would most likely have been questionable at best. However
we would have had more detailed control over the code, which when seen from a
more distant point of view, could have been benefitial with regards to
maintanence, documentation with javadoc, and quality assurance.

A major design decision was made when we decided to use a LL(k) parser rather
than a LALR parser. This decision was discussed and made in section 
\ref{sect:method:alternatives}.



\section{Covrage Test Results}

\underline{\textbf{\LARGE //TODO: Mads}}
\begin{itemize}
\item Testresultater, bra/d\aa rlig
\item testresultater (hva var de forskjellige feilene)
\item Generelt resultater vi skrev om i forrige kapittel
\item Resultatene maa sees relativt - vi har ikke faatt testet hele settet siden
vi ikke kan kjore run-time tester, og vi returnerer heller ikke noe som kan
sammenlignes med forventet resultat
\end{itemize}

Covrage er ikke h\o yere mest sannsynlig p\aa~grunn av vi har reserved keywords, og noen ganger ser parseren for langt frem... skal se om jeg f\aa~r fiksa noe av dette til helga, sp\o rs p\aa~ hvor lang vi har kommet med rapporten.... Covrage testene er eXtremt bra til \aa~finne bugz.

\underline{\textbf{\LARGE //ODOT:}}


\section{Error Handling}
\underline{\textbf{\LARGE //TODO: Andreas}}

Pass p\aa~ at du ikke skriver om at nextToken() er kjip for mange ganger... =)

Hardkodet catching av exceptions i nextToken() begrenset muligheten for aa
fange opp *alle* feil, men dette er mulig aa fikse (definere egne baseklasser)

Feilhaandteringen

Hvorfor skrudde vi av default error handling? Var dette bra? Hva var kjipt da? Kunne vi gjort det p\aa~ noen annen m\aa te?


\textbf{sakset fra future work: litt omskriving n\oe dvendig?}
\subsubsection{Lexer Error Handling}
\label{sect:future_work:lexer_error_handling}
As noted in sections \ref{sect:error_handling:syntax_errors} and
\ref{sec:impl:errorhandling}, the \verb!nextToken()! method in the lexer base class
would ``hijack'' lexical exceptions and handle them by printing an error message
to stderr and then attempting to recover from the error by simply consuming the
offending character and ignoring it in the following construction of tokens. It is 
not immediately appareant, but the \verb!nextToken()! method can not be overridden and
forced to throw the exception onwards, due to the method signature itself which
does not allow exceptions to be thrown. 

However, one possible solution to this problem could be to override the \verb!nextToken()! method and employ the observer design pattern to allow a simple and decoupled way of flagging an exception to the parser.



\underline{\textbf{\LARGE //ODOT:}}


\section{Scope andz Typecheckz}

\underline{\textbf{\LARGE //TODO: Andreas}} Saksa det her fra future work... kan det bli mer diskusjon i stedet? Evt ogs\aa~dele det opp i to sections...

\subsubsection{Type Checking}
Currently the parser will not perform type checking on the parsed queries. This
is an essential feature and will be necessary to implement for the parser to be
applicable in any realistic setting. A type checking system with proper type
inference and synthesis could be a complex feature to implement in a language
such as XQuery, and might require considerable effort, especially in quality
assurance. 

\subsubsection{Scoping and Symbol Tables on AST}
Currently the scoping and symbol tables are being used directly in the grammar -
that is, during parse time. It would be benefitial to move this into being
performed in run time, and combine with type checking functionality.

\underline{\textbf{\LARGE //ODOT:}}

\section{Dead Ends}
\label{sect:discussion:deadEnds}
\underline{\textbf{\LARGE //TODO: Mads, om ikke du har noen p\aa~lager?}}



Hva har vi gjort som m\aa tte gj\o res om igjen?

\begin{itemize}
\item Skrive om dash: Validerende semantiske predikat, saa CharNotMinus etc
\item Feil splitting Lexer vs Parser, mye feilmeldingjakt
\item Keywords deklarert i @tokens -> vant alltid.
\item NCName med syntaktiske predikat
\end{itemize}

/* Hentet fra implementation: \\
In the grammar specified by the W3C, all the productions (terminals and
non-terminals) all start with uppercase letters. Initially this caused some
confusion, because this grammar naturally generated a very big lexer and a very
small and non-functional parser. \\
*/

\underline{\textbf{\LARGE //ODOT:}}