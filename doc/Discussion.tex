\chapter{Discussion}
\underline{\textbf{\LARGE //TODO:}} tekst her... innledning


husk generelt: Gjorde vi riktige valg? Ble ting som forventet? Hva hadde skjedd om vi brukte annet alternativ etc..

\section{Design Decisions}
\label{sect:discussion:designDecisions}

Of the alternatives in section \ref{sect:ambiguousgrammar:ambigTerm}, we ended up, as previously mentioned, on the parser controlled state driven lexer alternative. This strategy depends upon the lexer not generating any tokens before they are needed by the parser, dealt with by our implementation of \verb!UnbufferedCommonTokenStream!, and that most terminal alternatives are augmented with semantic predicates controlling which alternatives are available in which state.

\subsection{Parser Controlled Lexer}
Making the lexer emit one and one token instead of processing the input as a
whole, most probably leads to a slightly prolonged execution time as the
runtime controll will be shunted between the parser and lexer multiple times
per query parsed. Yet we do not belive that this will cause too much problems
as input queries are approximated to be about one hundred lines long. That this
strategy is viable in a real world situation can be examplified by the fact
that the eXist XML database (section \ref{sect:stateOfTheArt:eXist}) in a
manner similar to our system employs a parser controlled strategy, and that
parser of Pathfinder (section \ref{sect:soa:pathfinder}), like our parser,
generates one and one token.        

The option would be to implement a pure state driven lexer. This would let the
lexer operate completely autonomous, though at the cost of a much more complex
control structure than is the case of the parser controlled solution. All in
all this strategy would probably not lead to any significant performance gain,
but rather the oposite. In addition, as mentioned in section
\ref{sect:amiguousgrammar:stateDriven}, we made a prototype with this strategy
for a subset of XQuery, and decided that this would be a quite complicated task
for the full version of the grammar -- a task that would result in a greater
deal of time being consumed by implementation and bug fixing.

Our grammar, by being dependant on our custom token stream, would render
gUnit(section \ref{sect:method:gUnit}) and ANTLRWorks(section
\ref{sect:method:debugging}) a greatly diminished utility value. This is
because these tools both depend on their own proprietary token stream
implementation. The reduction of the unit testing capabilities were not a big
loss however, as they were mended by the manual coverage tests, which in fact
proved to be a simpler and more thorough way of testing our parser.

Neither was the loss of certain ANTLRWorks functionality; because of its
instability, it never was the grammar editor of choice. Furthermore, the
drawing of syntax diagrams -- a handy way of resolving non-determinisms in the
parser -- is done during grammar compile time, and is therefore still
functional. And finally, by tricking the application with a simple hack, it can
still parse input and draw the corresponding parsing tree, though not in a
step-by-step manner.

\subsection{State Driven Lexer}
\label{sect:discussion:stateDriven}
The states of our lexer are implemented by gated semantic predicates and the
\verb!TOKENSWITCH! construct. This may have made our grammar quite complex and
less easily readable. An alternative to this approach would be to write the
lexer by hand. In this way we could have made a much easier state controll
structure, with a more complete division between each states functionallity.
But future modificatibility would be a problem, as a lot rewriting would be
necessary for even the simplest change in grammar semantics.

Using island grammars (section \ref{sect:amiguousgrammar:islandGrammar}) would render each subgrammar very simple and readable. Simple grammars intuitively also generates simple recognizers, giving an additional benefit in performace time. Though a performace overhead would come in form of switching between the lexers, we belive this is significantly outweighted by the gain by having completly unambiguous grammars. The catch is that using multiple lexers are not naitively supported by ANTLR as of yet, and the tampering needed to make this work is something we are not prepared to do. If however ANTLR is to include this in the near future, we will alter the grammar accordingly.

As there are only a few productions that are available per state, exept for in the \verb!DEFAULT! state, a third option would be a hybrid of island grammars and writing the lexer by hand. In this solution, a lexer rule not a member of this state would be removed from the grammar, and be replaced by corresponding handwritten lexer in the form of a method in the lexer class called by a inline action in the ANTLR grammar. The problem with this option is that much of the lexer specification would not be in form of ANTLR grammar, thus decreasing readability and ease of change. Another issue would be that the manual lexer would have to conform to the ANTLR generated lexer's manner of operation with regards to generating tokens, handling the inputstream and pointers etc., which may not be a trivial task to implement.
\section{Adapting the W3C Grammar}
\label{sect:discussion:adaptW3C}

In the begining of this project, very inexperienced parser developers as we
were, had the na\"{i}ve idea that since the XQuery grammar was given and
expressed in EBNF, our task was to run this through a parser generator,
possibly with some syntax changes, and the job would be done. After a while of
trying to adapt the given grammar to something ANTLR would accept, we came to
the conclusion that W3C probably intended the specification to be read by
humans, not by computers. 

An alternative to insert and adapt would be to write a grammar from scratch while keeping the semantics specified by W3C. We belive that by doing so there is a risk of over-simplifying, causing latent semantics such as operator precedence to be distorted, or in worst case lost alltogether. Another negative aspect with this approach is that it does not properly utilize the work allready done, making it time consuming compared to just adapting.

The terminal productions had to be completly rewritten. By choosing to write these rules from scratch in the first place, we would in all likelyhood have discovered the problem with the ambiguous terminals earlier, but at the cost of not having a parser up and running (albeit a very reduced one) until the grammar was completly finished. This would have prohibited us from working with e.g. the scoping system in parallell with the grammar.

Considering the non-terminal productions, their rewrites were not as extensive. Most of the work done on the parser grammar were in the form of left factoring and augmenting with syntactic predicates to reduce required lookahead, or acommodating for the extra-grammatical constrains. We can not see that any particular benefit would be gained from writing the non-terminal productions from scratch, compared to adapting the W3C specification.
\section{Dead Ends}
\label{sect:discussion:deadEnds}
\underline{\textbf{\LARGE //TODO: Mads, om ikke du har noen p\aa~lager?}}



Hva har vi gjort som m\aa tte gj\o res om igjen?

\begin{itemize}
\item Skrive om dash: Validerende semantiske predikat, saa CharNotMinus etc
\item Feil splitting Lexer vs Parser, mye feilmeldingjakt
\item Keywords deklarert i @tokens -> vant alltid.
\item NCName med syntaktiske predikat
\end{itemize}

/* Hentet fra implementation: \\
In the grammar specified by the W3C, all the productions (terminals and
non-terminals) all start with uppercase letters. Initially this caused some
confusion, because this grammar naturally generated a very big lexer and a very
small and non-functional parser. \\
*/

\underline{\textbf{\LARGE //ODOT:}}


\section{Write Parser}
\underline{\textbf{\LARGE //TODO:}} Noe Torbj\o rnsen snakka om.. Hiv p\aa~ om du kommer p\aa~ mer

Hvorfor skrive en parser i det heletatt, hvorfor ikke bare bruke en eksisterende. Fordi det er oppgaven, hva med lisens? Hvorfor valgte vi denne oppgaven, og ikke den andre...? Sette seg inn i kode tar masse tid, lager vi selv har vi mer oversikt...

\underline{\textbf{\LARGE //ODOT:}}



\section{ANTLR}
\underline{\textbf{\LARGE //TODO: Andreas}}

Rett valg?
Vi kjipa med CUP + JFlexxx, hva skjedde med \aa~skrive for h\aa nd?
Har ikke peiling p\aa~ CUP og JFlexxx, AST mye lettere med ANTLR? Mindre omskriving syntaktisk? Har de predikater der? Kunne vi l\o st statetingen p\aa~ en annen m\aa te med CUP og Flexz? 

\underline{\textbf{\LARGE //ODOT:}}




\section{Covrage Test Results}

\underline{\textbf{\LARGE //TODO: Mads}}
\begin{itemize}
\item Testresultater, bra/d\aa rlig
\item testresultater (hva var de forskjellige feilene)
\item Generelt resultater vi skrev om i forrige kapittel
\end{itemize}

Covrage er ikke h\o yere mest sannsynlig p\aa~grunn av vi har reserved keywords, og noen ganger ser parseren for langt frem... skal se om jeg f\aa~r fiksa noe av dette til helga, sp\o rs p\aa~ hvor lang vi har kommet med rapporten.... Covrage testene er eXtremt bra til \aa~finne bugz.

\underline{\textbf{\LARGE //ODOT:}}

\section{AST}
\underline{\textbf{\LARGE //TODO: Andreas}}

AST'en som blir produsert vs syntakstre

Generelle resultat, bra/d\aa rlig (AST) \#\# typesjekking (bedre \aa~gj\o re p\aa~AST'en) -> putt i egen section i discussion -------v
symtabs (bedre \aa~gj\o re p\aa~ AST'en))

Gjorde vi riktige valg? Ble ting som forventet? Hva hadde skjedd om vi brukte alternativ m\aa te? 

\underline{\textbf{\LARGE //ODOT:}}

\section{Error Handling}
\underline{\textbf{\LARGE //TODO: Andreas}}

Pass p\aa~ at du ikke skriver om at nextToken() er kjip for mange ganger... =)

Hardkodet catching av exceptions i nextToken() begrenset muligheten for aa
fange opp *alle* feil, men dette er mulig aa fikse (definere egne baseklasser)

Feilhaandteringen

Hvorfor skrudde vi av default error handling? Var dette bra? Hva var kjipt da? Kunne vi gjort det p\aa~ noen annen m\aa te?


\textbf{sakset fra future work: litt omskriving n\oe dvendig?}
\subsubsection{Lexer Error Handling}
\label{sect:future_work:lexer_error_handling}
As noted in sections \ref{sect:error_handling:syntax_errors} and
\ref{sec:impl:errorhandling}, the \verb!nextToken()! method in the lexer base class
would ``hijack'' lexical exceptions and handle them by printing an error message
to stderr and then attempting to recover from the error by simply consuming the
offending character and ignoring it in the following construction of tokens. It is 
not immediately appareant, but the \verb!nextToken()! method can not be overridden and
forced to throw the exception onwards, due to the method signature itself which
does not allow exceptions to be thrown. 

However, one possible solution to this problem could be to override the \verb!nextToken()! method and employ the observer design pattern to allow a simple and decoupled way of flagging an exception to the parser.



\underline{\textbf{\LARGE //ODOT:}}


\section{Scope andz Typecheckz}

\underline{\textbf{\LARGE //TODO: Andreas}} Saksa det her fra future work... kan det bli mer diskusjon i stedet? Evt ogs\aa~dele det opp i to sections...

\subsubsection{Type Checking}
Currently the parser will not perform type checking on the parsed queries. This
is an essential feature and will be necessary to implement for the parser to be
applicable in any realistic setting. A type checking system with proper type
inference and synthesis could be a complex feature to implement in a language
such as XQuery, and might require considerable effort, especially in quality
assurance. 

\subsubsection{Scoping and Symbol Tables on AST}
Currently the scoping and symbol tables are being used directly in the grammar -
that is, during parse time. It would be benefitial to move this into being
performed in run time, and combine with type checking functionality.

\underline{\textbf{\LARGE //ODOT:}}