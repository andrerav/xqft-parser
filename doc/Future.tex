\chapter{Future Work and Improvements}
\label{sect:summary:future_work}
\underline{\textbf{\LARGE //ODOT:}} tekst.... innledning..


\section{Tree Parser and Transformation}
\underline{\textbf{\LARGE //TODO: Andreas?}}

Planen er \aa~ transformere til relasjonsalgebra etterhvert, dette blir flott med treeparser? skissere enkelt hvordan dette kan gj\o res

\underline{\textbf{\LARGE //ODOT:}}

\section{Approaching LL(1)}

\underline{\textbf{\LARGE //TODO: Mads}}

LL(1) tilnaerme, kan sikre at states funker for alle tilfeller, samt kanskje(!) la KeywordOrNCName bli virkelighet

\underline{\textbf{\LARGE //ODOT:}}

\section{No Reserved Keywords}

\underline{\textbf{\LARGE //TODO: Mads}}

ingen reserved keywords - i hennhold til spesifikasjonen

\underline{\textbf{\LARGE //ODOT:}}

\section{Other Improvements}

\underline{\textbf{\LARGE //TODO: Mads, om ikke du har noe \aa~legge til?}}

sjekke navn ved noesting av xml-tagger, ikke krevd i spesifikasjonen, men kan vaere praktisk

automatisk gjoere om entityref og breaksymboler i container tokens (ElementContent etc)

bedre skille NCName og keywords i lexeren.

\underline{\textbf{\LARGE //ODOT:}}

\section{Known Bugs}
\label{sect:future:knownBugs}

\underline{\textbf{\LARGE //TODO: Mads, om ikke du har noe \aa~legge til?}}
\begin{itemize}
		\item prefikset funksjonskall i enclosedexpr i attributecontent bruker for mye lookahead -> oedelegger for states. Fikse med ny QName eller evt syntaktiske/semantiske predikat i parseren.
		\item tillater WS foerst i elementtagz
		\item tillater WS i qNamez
	\end{itemize}

\underline{\textbf{\LARGE //ODOT:}}


\section{Improving the Lexer}



\subsection{Lexer Error Handling}
\label{sect:future_work:lexer_error_handling}
As noted in sections \ref{sect:error_handling:syntax_errors} and
\ref{sec:impl:errorhandling}, the \verb!nextToken()! method in the lexer base class
would ``hijack'' lexical exceptions and handle them by printing an error message
to stderr and then attempting to recover from the error by simply consuming the
offending character and ignoring it in the following construction of tokens. It is 
not immediately appareant, but the \verb!nextToken()! method can not be overridden and
forced to throw the exception onwards, due to the method signature itself which
does not allow exceptions to be thrown. 

However, one possible solution to this problem could be to override the \verb!nextToken()! method and employ the observer design pattern to allow a simple and decoupled way of flagging an exception to the parser.
