\chapter{Future Work}
\label{chapter:future}

The outcome of this project is a fairly well-defined method of translation,
there is still headroom for improvement and continued research. In particular,
this relates to performance benchmarking, simplifications and optimisations,
and improvement of XQuery feature support.

We propose the following research related to performance benchmarking:
\begin{itemize}
  \item \textbf{Run the XMark benchmark suite and compare results}. Given that
  Fast Search \& Transfer develops a working implementation of an MQL processor some time in the future, it could be interesting to extend or rewrite the
  prototype developed here, and execute the
  XMark\footnote{\htmladdnormallink{http://www.xml-benchmark.org/}{http://www.xml-benchmark.org/}} 
  benchmark suite and compare the result to other existing implementations
  \item \textbf{Execute algebra generated by Tainting Dependencies on MonetDB}.
  If a working MQL processor can not or will not be developed in the foreseeable future, then
  the prototype and the rules of TD may be modified
  and interfaced with MonetDB, and a comparison with Pathfinder could be performed on this
  combined system. However, the adaptation of TD to MonetDB
  may be non-trivial, as the method depends on some MQL specific features.
  \item \textbf{Further research on optimisation and simplification of the TD
  methodology}. This thesis suggests some simplifications (section
  \ref{sect:trans:TD:simplifications} and \ref{sect:disc:optimisations}), however we suspect there are still
  substantial gains to be made on this account
  \item \textbf{Investigate applicability of parallell execution of subtrees in the
  algebra tree}. MQL supports threading/branching within the language itself,
  and this may be exploited to parallellise the execution of algebra and boost
  performance
\end{itemize}

Furthermore, we propose the following improvements:
\begin{itemize}
  \item \textbf{Improved support for interfacing with the external environment} (as
  described in section \ref{sect:disc:ctxItem})
  \item \textbf{Improved support of XQuery functionality}, including: full-text
  operations, \texttt{ordered} and \texttt{unordered} mode, binary operators not currently supported,
  multiple order specifications for the \texttt{order by}-clause, and user-defined
  functions as well as built-in XQuery functions (within the \texttt{fn} namespace)
  \item \textbf{Implementation of the full XQuery type system} into TD, which may also
  possibly be exploited for optimisations
  \item \textbf{Optimisation of \texttt{if..then..else} expressions} by assuming
  execution of DAGs and not trees, as described in section
  \ref{sect:disc:optimisations}
\end{itemize}