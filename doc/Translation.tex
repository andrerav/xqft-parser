\chapter{Tainting Dependencies}
\label{sect:translation}
\label{chapter:translation}


One of the greatest challenges in translating XQuery to relational algebra, is
to translate the semantics of iterators. The iterative nature of expressions
such as \texttt{for} expressions and the bulk oriented relational processing
may seem contradictory. But because XQuery is a purely functional language, and
thus free of side effects, it is semantically sound to evaluate all iterations
of the iterator body in parallel.

In this chapter we will first present MarkXRemove. This was the first iterator translation proposal of this
project. As it had some flaws, the method was reinvented and evolved into Tainting Dependencies (TD), accomodating
for the deficiencies of MarkXRemove. The rest of the chapter is dedicated to this method. First some concepts will
be discussed, followed by methods for translating various features of XQuery into MQL relational algebra trees,
including the translation of iterators. Finally, the chapter will discuss some possible simplifications of trees
generated with TD.

\section{MarkXRemove}
\label{sect:trans:MarkXRemove}

Our original proposition to a method for translating XQuery ASTs into relational algebra was named MarkXRemove.
The rationale behind this name will be explained later. Eventhough it has many shortcomings and flaws, we
will in this section give a quick run through of the method. This is because when we in section
\ref{sect:trans:taintingDependencies} present the Tainting Dependancies method, which is an evolution and a
refinement of MarkXRemove, the new method may be easier understood when seen in the perspective of its origins.
Another reason is that in case of further development of the the Tainted Dependancies method, it may be of help to
also know what will \textit{not} work, what will work partially and \textit{why} it is flawed.

\subsection{Basics}
\label{sect:trans:mxr:basics}

The foundation of the method is that an iterator expression is always translated by calculating the cartesian
product of the iterator sequence and the iterator body, hence the ``X'' in the name. The ``remove'' stems from the
removing of tuples who ended up in the wrong iteration in the cross product result. The cartesian product and the
selection of tuples afterwards actually constitutes a kind of natural join (see section
\ref{sect:theory:relAlg:naturalJoin}) as we will see later.

As the translator comes across an iterator variable declaration, with the variable name $\chi$, it will augment
the representation of the iterator sequence belonging to this variable with an attribute $\chi$\textsf{numb}.
This new attribute will hold consecutive values from 1 to $n$ for a $n$ item long sequence, which will symbolise the
iteration number of the iterator expression seen isolated from possible other surrounding iterator expressions. A
function \texttt{counter()} returning the row number of a relation and a \texttt{project} operator will handle
the augmentation. The corresponding algebra tree is be stored in the symbol table. The ``mark'' of the name of the
method is because of this augmentation.

\subsection{FLWOR}
\label{sect:trans:mxr:flwor}
If the translator later comes accross a reference to an iterator variable $\chi$, it will get the tree from the
symbol table and return it to the referring node without any further ado. The translator is also required to know
which subtrees have a child that has referred to which iterator variable. This is because the $\chi$\textsf{numb}
column could be lost in a \texttt{project} operation without this knowledge.

When the translator returns to the iterator expression node for the variable $\chi$ after traversing the iterator
body, it will, as mentioned before, make sure that the cartesian product between the iterator sequence and body is
calculated. From the result of this, the tuples where the $\chi$\texttt{numb} stemming from the iterator
sequence does not line up with the $\chi$\texttt{numb} stemming from the body are removed. Any tuple with a
$\chi$\texttt{numb} value \texttt{NULL} will be kept.

A \texttt{NULL} value in attribute $\chi$\texttt{numb} for a tuple means that this tuple is not marked, i.e. it is
not dependant on which iteration the $\chi$ iterator expression is in. Unmarked tuples can e.g. stem
from the creation of a sequence. MarkXRemove translates sequence building such as \texttt{(}$e_{1}$\texttt{,
}$e_{1}$\texttt{)} to a simple disjoint union, $r(e_{1})\;\dot\cup\;r(e_{2})$. Where $r()$ symbolises a function
translating XQuery expressions into relational algebra.

The method creates quite simple algebra, as exemplified by the following query:
\begin{Verbatim}
for $i in (1,2,3) return 
  ($i, 'yes')
\end{Verbatim}

which is translated into this algebra:

\begin{Verbatim}
select(ifThenElse(isNull(value), true, eq(r.inumb,l.inumb))
  cross(
    project(inumb = counter(), value;
      make(name:=["value"], [1,2,3]))
    union(
      project(inumb = counter(), value;
        make(name:=["value"], [1,2,3]))
      make(name:=["value"], ['yes']))))
\end{Verbatim} 

\subsection{Flaws}
\label{sect:trans:mxr:flaws}
The main problem with MarkXRemove is its dependancy on particular ordering of tuples in a relation which is a
result of a cartesian product. Not only can not the \texttt{cross} operator promise the particular ordering of its
result, it can not promise any ordering at all. The ordering the method depends on is that for each tuple in the
left relation the tuple is repeated for all tuples in the right relation. This means that any item may apear
anywhere in the resulting sequence, which is not acceptable for evaluation of XQuery expressions where all
sequences are ordered.

Another problem with this method is that any sequence built which includes a reference to an iterator variable is
not fully calculated until the cartesian product between the corresponding relation and the variables iterator
sequence is evaluated. This makes it hard to evaluate expressions where such a sequence is a subexpression. The
iterator \textit{body} of this query:

\begin{Verbatim}
for $i (5,10) return
  ($i, 8) > (6,12)
\end{Verbatim}
would be translated into something like this relational algebra:
\begin{Verbatim}
project(value = gt(l.value, r.value), inumb
cross(
  union(
    project(inumb = counter(), value;
      make(name:=["value"], [5,10]))
    make(name:=["value"], [8]))
  make(name:=["value"], [3,12])))
\end{Verbatim}
which again would produce this relation:

\begin{figure*}[!h]
\centering
\begin{tabular}{|c|c|} \hline
$inumb$ & $value$ \\\hline
1 & \texttt{false} \\\hline
1 & \texttt{false} \\\hline
2 & \texttt{true} \\\hline
2 & \texttt{false} \\\hline
\texttt{NULL} & \texttt{true} \\\hline
\texttt{NULL} & \texttt{false} \\\hline
\end{tabular}
\end{figure*}

The query should of course be evaluated to $(true, true)$, as the \texttt{>} operator in XQuery yields $true$ if
\textit{one} item in the sequence on the left side of the operator is larger than \textit{one} item on the right.
This means that the relation would have to be pruned by a \texttt{select} or \texttt{group}, which can not be done
generally in the relations current state. A possibility would be to postpone the pruning until after the relation
is crossed with the iterator sequence, but there would still not be any not-complex solution. This problem lead to
the introduction of variable dependancy tainting, as we will see in section
\ref{sect:trans:taintingDependancies}.

\underline{\textbf{\LARGE TODO:}} her st\aa r det egentlig noe om numeric predicates, men det er kommentert bort. 
% Within the \texttt{scope} field of a relation returned from a \texttt{lookup} operator lies information about
% which child number the scope is relative to its sibling scopes. E.g. if a \texttt{lookup} had been executed on a
% index holding the following XML-document:
% \begin{Verbatim}
% <a>
%   <b>one</b>
%   <b>two</b>
% </a>
% \end{Verbatim}
% it would be possible to extract that the element containing ``two'' is the second \verb!<b>!-child of its parent.
% And its exactly this information MarkXRemove relies on when translating numeric predicate(see section
% \ref{sect:theory:xquery:Predicates}). But when a numeric predicate is applied to a step expression with a reverse
% axis, e.g. \texttt{ancestor::}, the numbering is reversed. This means that


\section{Inference Rule Language Explanation}
\label{sect:trans:TD:langExpl}
During this chapter we will present some inference rules. Table \ref{tab:trans:td:langExpl} explains some of the
the various typographical representations.

\begin{table}[h]
\centering
\begin{tabular}{l|l}

  $\longmapsto$  			& Translates into \\
  $\vartheta$ 				& A set of iterator dependencies \\
  \textsf{sans serif} 		& MQL expressions \\
  \texttt{monospaced} 		& XQuery expressions \\
  $e,\ldots,e_{n}$			& Generic expressions \\
  $\chi,\ldots,\chi_{n}$	& Generic variable names \\
  $I_{\chi}$				& The iterator expression which declares $\chi$ \\
  \textbf{bold} 			& Operations done during the generation of the algebra \\
  \textbf{r(}$e$\textbf{)} 	& Returns the relational algebraic representation of $e$   \\
  \textbf{t(}\textbf{r(}$e$\textbf{)}\textbf{,}$\vartheta'$\textbf{)} & Returns \textbf{r(}$e$\textbf{)} tainted
  with the dependencies $\vartheta'$ \\
  \textbf{B(r(}$e$\textbf{))} & Returns the effective boolean value of \textbf{r(}$e$\textbf{)}.
  
\end{tabular}
\caption{Explanation of inference rule symbols}
\label{tab:trans:td:langExpl}
\end{table}

Inference rules are generally in this format:
\begin{equation*}
\frac{condition}{e}\longmapsto \mbox{\textbf{r(}}e\mbox{\textbf{)}}
\end{equation*}

This should be read as: if condition $condition$ is satisfied, the XQuery expression $e$ will be translated into
\textbf{r(}$e$\textbf{)}.

Often MQL operator trees are depicted like this:
\begin{center}
\begin{tabular}{l}
\textsf{operator1(\ldots,l.attr, r.attr\ldots; } \\ \quad
\textsf{operator2(\ldots);} \\ \quad
\textsf{operator3(\ldots))}
\end{tabular}
\end{center}

This is to be interpreted as ``\textsf{operator2} is the left child of \textsf{operator1} and \textsf{operator3} is
the right child''. MARS does not allow attribute names to contain punctuation marks or allow two attributes with the
same name within one relation. An operator combining two relations will therefore have renaming functionality. A
typical projection list of such an operator combining two relations which both contain the attribute $attr$ would
look something like: \textsf{\ldots rattr=right.attr, lattr=left.attr\ldots}. To make the inference rules
easier to read, this step has been dropped. The rules assume that the equal
named fields will automatically be given a prefix \textsf{l.} (left) or
\textsf{r.} (right) depending on which child the attribute stems from.

We assume the \textsf{union}-operator accepts relations with different schemas. The schema for the result
relation can be described as:
\begin{equation*}
schema(\textsf{union(}\alpha, \beta\texttt{)}) = schema(\alpha) \cup schema(\beta).
\end{equation*}

The tuples which have more fields in the result relation than they did in the relation they stem from will
have a \textit{NULL} value for the introduced attributes. It may be not be
desirable to implement the operator in such a way. In that case each
child-relation will have to be augmented with the attributes needed with an \textsf{project} operator.

\subsection{Basics}
\label{sect:trans:TD:basics}
The method assumes left-to-right traversal of the assymetric syntax tree. In most cases the traversal is
postorder, meaning a subtree can be evaluated independently from its ancestors -- the exception being the logical
context set by boolean operators. The relational algebra will thus be generated bottom-up. In addition to the
evaluated subtree, a node must be able to inform its parent node about its variable dependencies ($\vartheta$),
which we will discuss later.

One XQuery sequence is represented as one relation and one XQuery item is represented as one tuple. This is sound,
as all XQuery items are sequences, and all sequences are one-dimensional (section
\ref{sect:theory:xquery:basics}). As we mentioned in section \ref{sect:trans:MarkXRemove}, the MarkXRemove method
did actually not consider the ordering of items in sequences at all. In Tainting Dependencies, however, we have
introduced an attribute $index$ holding the intra sequence number of the item. Consider the XQuery sequence
\texttt{('a','b',}$\ldots$\texttt{,'z')}. With this attribute, the relational representation will be as follwing:

\begin{center}
\begin{tabular}{|c|c|} \hline
$index$ & $value$ \\\hline
1		& \texttt{'a'} \\\hline
2		& \texttt{'b'} \\\hline
$\ldots$& $\ldots$ \\\hline
$n$		& \texttt{'z'} \\\hline
\end{tabular}
\end{center}


As can be seen, the item value is stored in the $value$ attribute. For the course of this chapter we will, for the
sake of simplicity, treat $value$ as a polymorphic type attribute. This simplification has minimal consequences
for the method and the way XQuery expressions are translated. XQuery types and relational representation of such
will is handled in section \ref{sect:disc:typeSystem}.

Also for simplicity, the $index$, $documentId$, $pos$ and $scope$ attributes have sometimes been left out of the
fields specified in \textsf{project} operators. If the \textsf{project} operator is applied to the result of a
join or cartesian product, these fields will follow the $value$ attribute. That is, if $r.value$ is projected,
then so is $r.documentId$, etc\ldots if applicable.

Tainting Dependencies utilises a symbol table for storing of variables declared. The table has two functions:
\begin{itemize}
  \item \textbf{put(}$\chi$\textbf{, }\textbf{r(}$e$\textbf{))} -- will store the
  algebraic version of the expression bound to the variable \texttt{\$}$\chi$ with $\chi$ as the key.  
  \item \textbf{get(}$\chi$\textbf{)} -- will do a lookup in the table based on the name of the variable
  \texttt{\$}$\chi$ and return the algebraic version of the expression linked to it.
\end{itemize}
The symbol table handles scoping according to XQuery semantics (section \ref{sect:theory:xquery:Flwor}), meaning
the translator will always be able to find the right declared variable based on which node in the AST the
translator is visiting.

\subsection{Iterator Dependency Inheritance}
\label{sect:trans:TD:dependency}

The concept of iterator dependency form the basis of the Tainting Dependency method. Such dependency is
defined as follows:

\noindent
\begin{myDefinition}
An XQuery expression $e$ is \textbf{dependant} on an iterator $I_{\chi}$ if $e$ occurs within the iterator body of
$I_{\chi}$ and if the evaluation of $e$ depends on the iteration number of $I_{\chi}$.
\label{def:iterVarDep}
\end{myDefinition}

An variable reference to an iterator variable \texttt{\$}$\chi$ is by this definition dependent on the iterator
$I_{\chi}$. Intuitively, an expression which subexpression is dependent on an iterator $I_{\chi}$ is also
dependent on this iterator -- we say the dependency is inherited. Consider the example subexpression of figure
\ref{fig:trans:td:varDep}, where \texttt{\$x} and \texttt{\$y} both are iterator variables. Here, the expresion
$e_{1}$ is dependent the two iterators $I_{\mbox{\texttt{x}}}$ and $I_{\mbox{\texttt{y}}}$, while expression
$e_{2}$ is only dependent on $I_{\mbox{\texttt{x}}}$.

\begin{figure}[h]
\centering
\tikzstyle{astNode}=[circle, draw=blue!70,fill=blue!20,solid,thick, minimum
size=26pt]
\begin{tikzpicture}[grow via three points={one child at (0,-1.5) and two
children at (-1.5,-1.0) and (1.5,-1.0)}]
\draw[loosely dotted, thick] (0,0) -- (0,-1);
\node at (0,-1) [astNode, label=above left:$e_{1}$ ] {\texttt{and}} 
child{node [astNode, label=above left:$e_{2}$] {\texttt{+}}
	child{node [astNode] {\texttt{\$x}}}
	child{node [astNode] {\texttt{3}}}
 }
child{node [astNode] {\texttt{\$y}}}
 ;
\end{tikzpicture}
\label{fig:trans:td:varDep}
\caption[Iterator variable dependency]{Iterator variable dependency}
\end{figure}

The iterator dependencies of an expression $e$ are part of the set $e.\vartheta$. As mentioned earlier,
an AST node must be able to inform its parent about the node's dependencies as well as the algebra generated. For
an expression $e$ this can be done by letting $e.\vartheta$ piggyback the \textbf{r(}$e$\textbf{)} returned. The
variable dependencies for an expression $e$ with the subexpressions $e_{1},\ldots,e_{n}$ can be described as
following:
\begin{equation}
e.\vartheta = e_{1}.\vartheta\cup\ldots\cup e_{n}.\vartheta
\label{eq:trans:TD:depInheritance}
\end{equation}

The dependancy on the iterator $I_{\chi}$ manifest itself relationally by the attribute $\chi$$numb$. The value of
$\chi$$numb$ is the iteration number of $I_{\chi}$, that is, for a tuple ($\chi$$numb$, $value$) the value $value$
will appear in the $\chi$$numb$th iteration of $I_{\chi}$.

When an iterator variable \texttt{\$}$\chi$ is declared it is assigned a $\chi$$numb$ by renaming the $index$
field of the corresponding iterator sequence $\chi$$numb$. Which leads us to the inference rule for translating the
(optional) \texttt{for} clause of a FLWOR expression:
\begin{equation}
\frac{}{\mbox{\texttt{for \$}}\chi \mbox{\texttt{ in }} e \mbox{\texttt{\ldots}}}\longmapsto
\begin{array}{l}
\mbox{\textbf{put(}}\chi\mbox{\textbf{, }} \\ \quad
\mbox{\textsf{project(}}\chi\mbox{\textsf{numb = index, index=1, value;}} \\ \quad \quad
\mbox{\textbf{r(}}e\mbox{\textbf{)}\textsf{)}\textbf{)}}
\end{array}
\label{rule:trans:TD:forbind}
\end{equation}
Where the dependancies piggybacking the \textsf{project} operator can be expressed as:
$\vartheta = e.\vartheta \cup \chi$.

For a \texttt{for} clause with multiple variable bindings the rule must be applied once per binding as if there
is one FLWOR expression per binding, and the $n$th binding is a FLWOR expression in the $(n-1)$th bindings
return clause. This is in accordance with the XQuery semantics, and is one of the rewrite rules into XQuery core
(see section \ref{sect:theory:xquery:XQcore}).

From definition \ref{def:iterVarDep} it can be seen that $\chi$ is not part of the set of dependancies the iterator
$I_{\chi}$ returns its parent. This is in fact the only case a variable is removed from a dependency set. Because
of this, we must be careful not to incidentally remove a $\chi$$numb$ attribute from a relation by means of a
\textsf{project} operator. 

When we in this chapter write $\vartheta$ enclosed in MQL syntax it is to be interpreted as a comma seperated list
of all the attributes linked to the dependencies in the set $\vartheta$. As an example, the dependency set
$\vartheta = \left\{\mbox{\texttt{x}},\mbox{\texttt{y}},\mbox{\texttt{z}}\right\}$, is read as \textsf{xnumb,
ynumb, znumb} in an MQL environment.

XQuery variable reference expressions, be it iterator, \texttt{let} or \texttt{declare} variables, are translated
to relational algebra quite simply by fetching the tree linked to the variable name in the symbol table:
\begin{equation}
\frac{}{\mbox{\texttt{\$}}\chi}\longmapsto
\mbox{\textbf{get(}}\chi\mbox{\textbf{)}}
\end{equation}


\subsection{Iterator Dependency Tainting}
\label{sect:trans:TD:tainting}

The iterator body of an iterator with a iterator sequence with length $n$ will have to executed $n$ times. This
can be done by e.g. evaluating the cartesian product between the body or the sequence, as with the MarkXRemove
method. To avoid any denormalised intermediate results, an ideal solution would be to always calculate such
products after all other evaluations of the query is done. Consider the following simple example of the query $e$:

\begin{center}
\begin{tabular}{l}
\texttt{for \$a in (1, 2) return} \\ \qquad
\texttt{for \$b in (3, 4) return} \\ \qquad \qquad
\texttt{5 + 6}
\end{tabular}
\end{center}

For this query the result can be calculated like this:
\noindent
\begin{center}
\textbf{r(}$e$\textbf{)}$=$\textbf{r(}\texttt{(1, 2)}\textbf{)}$\times$\textbf{r(}\texttt{(3,
4)}\textbf{)}$\times$\textbf{r(}\texttt{5 + 6}\textbf{)}.
\end{center}
\noindent

But such a simple solution is not adequate if there is a reference to an iterator variable somewhere within the
iterator body. This was managed by MarkXRemove by implementing inheritance of iterator dependencies, similar to
the concept discussed in section \ref{sect:trans:TD:dependency}, and replacing the cartesian product operator with
something like a natural join operator (section \ref{sect:trans:mxr:basics}).

But MarkXRemove has shorcommings when it comes to evaluating expressions where a sequence constructed with at
least one iterator dependent expression is a subexpression. Tainting Dependencies mend for this by requiring that
all items constituting the result of an iterator dependent expression are iterator dependent. To fulfill til
requirement, dependency tainting is introduced.

\noindent
\begin{myDefinition}
Iterator dependency \textbf{Tainting} is to impose a representation of one expression for each iteration of the
iterators another expression is dependent on.
\end{myDefinition}

During sequence construction, expressions explicitly taint all other expressions part of the construction with
their dependencies. Consider this subexpression:
\begin{center}
\begin{tabular}{l}
\quad \;\, $\vdots$  \\
\texttt{(}$e_{1}$\texttt{, }$e_{2}$\texttt{)}\\
\quad \;\, $\vdots$  
\end{tabular}
\end{center}
Where $e_{1}.\vartheta = \left\{\chi_{1}\right\}$ and $e_{2}.\vartheta = \emptyset$. Here $e_{2}$ will be tainted
by $e_{1}$'s dependency on $\chi_{1}$, but as $e_{2}$ have no dependencies, $e_{1}$ will not be tainted. The
tainting process is carried out by calculating the cartesian product of $e_{2}$ and the $\chi_{1}$$numb$ column of
\texttt{\$}$\chi_{1}$ stored in the symbol table.

More generally, for an sequence constructing expression $e$, \texttt{(}$e_{1}$\texttt{, \ldots, }$e_{n}$\texttt{)},
tainting of an subexpression $e_{i}$ can be expressed like this: \marginpar{\underline{\textbf{\Large TODO:}}
\scriptsize noe \aa~utsette p\aa~bruken av $\Pi$ ?}
\begin{center}
\begin{equation}
\begin{array}{l}
e.\vartheta = e_{1}.\vartheta \cup \ldots \cup e_{n}.\vartheta = \left\{\chi_{1},\ldots\chi_{m}\right\} \\
i \in \left\{1,\ldots,n\right\} \\
\mbox{\textbf{t(r(}}e_{i}\mbox{\textbf{),}}e.\vartheta\mbox{\textbf{)}} = 
\mbox{\textbf{r(}}e_{i}\mbox{\textbf{)}} \times {\displaystyle \prod_{\chi_{j} \in (e.\vartheta -
e_{i}.\vartheta)}} \mbox{\textsf{project(}}\chi_{j}numb\mbox{\textsf{;
}\textbf{get(}}\chi_{j}\mbox{\textbf{)}\textsf{)}}
\end{array}
\label{eq:trans:TD:taint}
\end{equation}
\end{center}

\subsection{Implications Of Iterator Dependencies}
\label{sect:trans:TD:implic}
Consider an XQuery expression consisting of nested iterators $I_{\chi_1},\ldots,I_{\chi_n}$, where $I_{\chi_j}$
($1<j<n$) occurs within the iterator body of $I_{\chi_{j-1}}$. As per XQuery semantics, the iterator body of a
iterator $I_{\chi_j}$ is evaluated once for each of the items in the iterator sequence of $I_{\chi_j}$. And because
of the nesting, $I_{\chi_j}$ will have to be evaluated once per item in the iterator sequence of $I_{\chi_{j-1}}$.
The consequence of this is that the body of $I_{\chi_j}$ is actually evaluated in $card(I_{\chi_j}) \times
card(I_{\chi_{j-1}})$ unique iterations. Where $card(I_{\chi})$ is a function returning the cardinality of the
iterator sequence of $I_{\chi}$.

Of these nested iterators, a subexpression $e$ is dependent on the subset
$\left\{I_{\chi_k},I_{\chi_l}\right\}$. Because of dependency tainting and inheritance, the relational
depiction of $e$ will have a representation in all posible iteration combinations of $I_{\chi_k}$ and
$I_{\chi_l}$. A tuple in $e$, ($index, \chi_k{numb},\chi_l{numb}, value$), represents one of these unique
iterations. When $I_{\chi_k}$ is in its $\chi_k{numb}$th iteration and $I_{\chi_l}$ is in its $\chi_l{numb}$th
iteration, the item in position $index$ of the sequence returned from $e$ will be $value$. Let $I_{\chi_m}$ also
be one of the nested iterators, but one which $e$ is not dependent on. $e$ will evaluate to the same result
regardles of which iteration $I_{\chi_m}$ is in given the iteration number of $I_{\chi_k}$ and $I_{\chi_l}$ is the
same.

When an subexpression such as $e$ is used in further evaluation, it is important to seperate these iterations from
each other. This is done by grouping the relation on all unique combinations of its iterator dependency attributes.
Grouping can be done either by the \textsf{group} operator or by specifying the attributes to group by in the
partition list of the \textsf{numberate} operator.

Often the evaluation of a expression use the values of each of its subexpressions. E.g. an addition expression is
evaluated by adding the value of its first subexpression with the value of the second. To be able to calculate
such expressions, the values of the subexpressions will have to be in the same relation. This can be achieved by
evaluating the cartesian product of the subexpressions. Assumed that the subexpressions are independent of
iterators or are not dependent on the same iterators this is sufficient. But if they are depentent on one or more
iterators in common, the result of the cartesian product will have to be synchronised on the common iterators
iterations. This allows evaluation in each unique iteration, and is solved by turning the cartesian product into
an equi-join.

Generally, for such an expression $e$, with the subexpressions $e_1$ and $e_2$ this can be written
like this:
\begin{equation*}
\mbox{\textbf{r(}}e\mbox{\textbf{)}}=
\begin{array}{l}
\mbox{\texttt{\ldots}} \\
\mbox{\textsf{hhjoin([l.}}(e_1.\vartheta\cap e_2.\vartheta)\mbox{\textsf{], [r.}}(e_2.\vartheta\cap e_1.\vartheta)
\mbox{\textsf{]\ldots}} \\ \quad
\mbox{\textbf{r(}}e_1\mbox{\textbf{)}} \\ \quad
\mbox{\textbf{r(}}e_2\mbox{\textbf{)}\textsf{)}}
\end{array}
\end{equation*}

The dependencies $e.\vartheta$ is described by equation \ref{eq:trans:TD:depInheritance}. If $e_{1}.\vartheta \neq
e_{2}.\vartheta$ each subexpression will be implicitly tainted by the other's unique dependencies.


\section{Sequence Construction}
\label{sect:trans:TD:seqBuild}

A sequence in XQuery can be built with the comma operator --\texttt{,}. But this operator is the XQuery operator
with the lowest precedence, therefore, in most cases a sequence construction expression will be enclosed in
paratheses. This is to solve parser ambiguities, which can be seen in the
excerpt of the W3C XQuery Full Text EBNF specification\cite{w3c01} in figure \ref{fig:trans:TD:seqEBNF}. An \texttt{ExprSingle} can solely consist of a
\texttt{ParenthesizedExpr} via a series of productions omitted from the figure. Also note a \texttt{ExprSingle}
can be a \texttt{FLWORExpr}.

\begin{figure}[h]
\begin{Verbatim}
[33] FLWORExpr         ::= (ForClause | LetClause)+ WhereClause? 
                               OrderByClause? "return" ExprSingle
[45] IfExpr            ::= "if" "(" Expr ")" "then" ExprSingle 
                               "else" ExprSingle
[31] Expr              ::= ExprSingle ("," ExprSingle)*
[89] ParenthesizedExpr ::= "(" Expr? ")"
\end{Verbatim}
\caption[Excerpt from W3C XQuery EBNF]{Excerpt from W3C XQuery EBNF showing
sequence construction}
\label{fig:trans:TD:seqEBNF}
\end{figure}

As also can be seen from the figure the for-clause of a FLWOR expression, as
many other expressions, accepts an \texttt{ExprSingle}. If a sequence is to be
constructed in the for-clause, it will have to be parenthesised.

With the concept of tainting and iterator dependencies explained, we are now ready to introduce the translation of
an XQuery sequence construction expression:

\begin{equation}
\frac{}{e_{1}\mbox{\texttt{, \ldots, }}e_{n}}\longmapsto
\begin{array}{l}
\mbox{\textsf{numberate(index, [sprIdx, index], [}}\vartheta\mbox{\textsf{];}} \\ \quad
\mbox{\textsf{union(}} \\ \quad \quad
\mbox{\textsf{project(sprIdx=1, index, value; }} \\ \quad \quad \quad
\mbox{\textbf{t(r(}}e_{1}\mbox{\textbf{), }}\vartheta\mbox{\textbf{)}\textsf{);}} \\ \quad \quad
\qquad\vdots\\ \quad \quad
\mbox{\textsf{project(sprIdx=}\textit{n}\textsf{, index, value; }} \\ \quad \quad \quad
\mbox{\textbf{t(r(}}e_{n}\mbox{\textbf{), }}\vartheta\mbox{\textbf{)}\textsf{)))}}
\end{array}
\label{rule:trans:TD:seqConstr}
\end{equation}
Where $\vartheta=e_{1}.\vartheta \cup \ldots \cup e_{n}.\vartheta$.

The basis of a sequence construction is the \textsf{union} operator -- as with MarkXRemove. But because we in
Tainted Dependencies have introduced the explicit ordering of items with the $index$ attribute, additional
operations have been added. Each item in the sequences returned from the subexpressions is equipped with a
temporary field $sprIdx$ (superindex) holding the relative position of each subexpression. Based on the
positioning defined by $sprIdx$ and $index$ the \textsf{numberate} operator can now renumberate the resulting
sequence. The numbering must partition on the fields corresponding to the dependencies in $\vartheta$, to separate
the different sequences constructed for all dependent iterations.

\begin{myExample}
\label{ex:trans:TD:simpleSeq}
\begin{figure}[h]
\begin{equation*}
\begin{array}{l}
\mbox{\texttt{for \$a in (10,20) return}} \\ \quad 
\underbrace{ \mbox{\texttt{(\$a, "no")}} }_{e_{1}}
\end{array}
\end{equation*}
\caption{Simple XQuery query}
\label{fig:trans:TD:simpQuery}
\end{figure}
Consider the simple XQuery query of figure \ref{fig:trans:TD:simpQuery}. Here \textbf{r(}\texttt{\$a}\textbf{)}
will taint \textbf{r(}\texttt{"no"}\textbf{)} with its dependency on the iterator $I_{a}$, the result of which is shown in figure \ref{fig:trans:TD:simpl:ryes}. Further, we can see that
for each iteration of $I_{a}$ the return clause will return a sequence of two items. Having in mind that $anumb$
($anb$ in figure) holds the iteration number of $I_{a}$, this can be seen in figure \ref{fig:trans:TD:simpl:rall}.

\begin{figure}[!h]
\centering
\subfigure[\textbf{r(}\texttt{\$a}\textbf{)}]{
\begin{tabular}{|c|c|c|} \hline
$anb$ & $idx$ & $val$ \\ \hline
1 & 1 & 10 \\ \hline
2 & 1 & 20 \\ \hline
\end{tabular}
\label{fig:trans:TD:simpl:ra}
}
\qquad
\subfigure[\textbf{t(r(}\texttt{"no"}\textbf{),}\textbf{r(}\texttt{\$a}\textbf{)}$.\vartheta$\textbf{)}]{
\begin{tabular}{|c|c|c|} \hline
$anb$ & $idx$ & $val$ \\ \hline
1 & 1 & \texttt{"no"} \\ \hline
2 & 1 & \texttt{"no"} \\ \hline
\end{tabular}
\label{fig:trans:TD:simpl:ryes}
}
\qquad
\subfigure[\textbf{r(}\texttt{(\$a, "no")}\textbf{)}]{
\begin{tabular}{|c|c|c|} \hline
$anb$ & $idx$ & $val$ \\ \hline
1 & 1 & 10 \\ \hline
2 & 1 & 20 \\ \hline
1 & 2 & \texttt{"no"} \\ \hline
2 & 2 & \texttt{"no"} \\ \hline
\end{tabular}
\label{fig:trans:TD:simpl:rall}
}

\caption[Example: constructing a sequence]{Applying translation rule \ref{rule:trans:TD:seqConstr} on (a) and (b)
yields (c). Attribute names are shortened \label{fig:trans:TD:simpleSeq}}
\end{figure}

The sequence construction rule also holds even if the subexpressions of the
soon-to-be sequence are within the body of an iterator the sequence is not
dependent on. Expanding the query of figure \ref{fig:trans:TD:simpQuery} we get
the query of figure \ref{fig:trans:TD:expandQuery}. \begin{figure}[h]
\begin{equation*}
\begin{array}{l}
\mbox{\texttt{for \$a in (10,20) return}} \\ \quad
\mbox{\texttt{for \$b in (50,75) return}} \\ \quad \quad
\underbrace{ \mbox{\texttt{(\$a, "no")}} }_{e_{1}}
\end{array}
\end{equation*}
\caption{Query of figure \ref{fig:trans:TD:simpQuery} expanded}
\label{fig:trans:TD:expandQuery}
\end{figure}
In this query, notice the innermost return clause expression, $e_{1}$, is identical to $e_{1}$ in the previous
query. Here, the result of the sequence construction will still be the relation shown in figure
\ref{fig:trans:TD:simpl:rall}, because $e_{1}.\vartheta=\left\{a\right\}$ -- also as before. $e_{1}$ is not aware
of the iterator $I_{b}$ -- and does not need to be either, as the result of $e_{1}$ is independent of which
iteration number $I_{b}$ is in.

\end{myExample}

\subsection{FLWOR Expressions}
\label{sect:trans:TD:simpleFLWOR}
The \texttt{let} clause does not cause any dependencies, only variable binding, and can therefore be translated
into storing the algebraic version of the expression to be bound in the symbol table:
\begin{equation}
\frac{}{\mbox{\texttt{let \$}}\chi \mbox{\texttt{ := }}e \mbox{\texttt{ \ldots}}}\longmapsto
\mbox{\textbf{put(}}\chi\mbox{\textbf{, r(}}e\mbox{\textbf{))}}
\label{rule:trans:TD:letbind}
\end{equation}
The iterator dependences $e.\vartheta$ is stored along with \textbf{r(}$e$\textbf{)} and will piggyback this
algebra tree if it later fetched from the symbol table. If there is more than one variable binding in the
\texttt{let} clause the rule must be applied once per binding as if one binding were one \texttt{let} clause.

\textbf{\Large TODO:} \textsc{Dette kan skrives om til enda mer generelt. S\aa~lenge for-clause biten blir kj\o rt
for hver binding, fra den siste og innover, stemmer det. Vi m\aa~definere et sett som holder alle iteratorvariabler
som blir bundet i denne FLWORen. I return-clausen vil iteratorbodyen bli tainta med dette settet (det skjer ikkeno
om iteratorbody'en allerede er avhengig av alle variablene\ldots)}

\textit{As previously mentioned, multiple variable bindings within one \texttt{for} clause is semanticly equal
to nested FLWOR expressions. Here, the innermost FLWOR will contain the last binding as its \texttt{for} clause and
all other subexpressions of the original FLWOR. Each preceding variable binding correspond to a FLWOR with the
binding as a \texttt{for} clause and the FLWOR corresponding to the succeding binding as its \texttt{return}
clause. Multiple \texttt{for} clauses within one FLWOR expression is to be considered in a similar matter.
Explicit rewriting of these kind of expressions is discussed in section \ref{sect:method:ast_rewrite}.}

\textit{
From the XQuery specification exerpt of figure \ref{fig:trans:TD:seqEBNF}, we see that a FLWOR expression may
appear in many different formats. In this section we will for inference rule readability and simplicty treat these
expressions as if they no more than \emph{one} \texttt{for} clause containing exactly \emph{one} variable binding.}
Further, the translation of FLWORs will be described in a clause-by-clause manner, as if executed as a postorder
traversal of the tree in figure \ref{fig:trans:TD:flworExecute}. The left children (variable bindings) will not
return anything to their parent and are handled by the rules \ref{rule:trans:TD:forbind} and
\ref{rule:trans:TD:letbind}, and the algebra returned from the right children will be referred to as
\textbf{r(}$e_{child}$\textbf{)}.

\begin{figure}[h]
\centering
\tikzstyle{optional}= []%[draw=black,dotted,thick, minimum size=26pt]
\tikzstyle{return}= [] %[ellipse,draw=black,solid,thick, minimum size=26pt]

\begin{tikzpicture}[grow via three points={one child at (1.5,-1.0) and two
children at (-1.5,-1.0) and (1.5,-1.0)}]
\node at (0,0) [optional] {[\textbf{let clause}]*}
child{node [optional] {[\textbf{binding}]+} edge from parent [draw,dotted, thick]}
child{node [optional] {[\textbf{for clause}]?}
	child{node [optional] {[\texttt{\$}$\chi$ \texttt{in} $e_{1}$]?} edge from parent [draw,dotted, thick] }
	child{node [optional] {[\textbf{let clause}]*}
		child{node [optional] {[\textbf{binding}]+} edge from parent [draw,dotted, thick] }
		child{node [optional] {[\texttt{order by} $e_{3}$]?}
			child{node [optional] {[\texttt{where} $e_{2}$]?}
				child{node [return] {\texttt{return} $e_{4}$}}
			}
		} 
	}
}
;
\end{tikzpicture}
\label{fig:trans:TD:flworExecute}
\caption[FLWOR translation order]{Illustration of step-by-step translation of FLWOR.}
\end{figure}

Because the \texttt{let} clause does not imply any iteration, it does not have any consequences for the
translation of FLWOR expresions:
\begin{equation}
\frac{}{\mbox{\textbf{let clause}}}\longmapsto \mbox{\textbf{r(}}e_{child}\mbox{\textbf{)}}
\label{rule:trans:TD:letclause}
\end{equation}

If a \texttt{for} clause is part of a FLWOR expresion it an iterator. A \texttt{return} clause which is part of an
iterator is evaluated once for each item in the iterator sequence. The results of these evaluations are
concatenated to form the result of the FLWOR expression. If the \texttt{return} expression is not dependent on the
iterator, its relational form will not vary with the iteration number. In such a case the result of the iterator
body will have to be tainted with the iterator. The \texttt{return} clause of an iterator $I_{\chi}$ can be
translated like this:
\begin{equation}
\frac{I_{\chi}}{\mbox{\texttt{return }}e_{4}}\longmapsto
\mbox{\textbf{t(r(}}e_{4}\mbox{\textbf{), \{}}\chi\mbox{\textbf{\})}}
\label{rule:trans:TD:returnTaint}
\end{equation}
An iterator body dependent on $I_{\chi}$ will allready have a representation for each iteration, and will not be
tainted (ref. equation \ref{eq:trans:TD:taint}).

If the \texttt{return} clause is part of a \texttt{for}-less FLWOR expression no translation is needed.

Because a FLWOR iterator creates a new sequence, renumbering is needed to calculate the result. No expression in a
sibling or parent scope of the iterator $I_{\chi}$ may be dependent on $I_{\chi}$. Thus, $\chi$ is not part of the
dependencies $I_{\chi}.\vartheta$ returned from the iterator, and the corresponding $\chi{numb}$ attribute must be
removed.
\begin{equation}
\frac{I_{\chi}}{\mbox{\textbf{for clause}}}
\longmapsto
\begin{array}{l}
\mbox{\textsf{numberate(index, [}}\chi\mbox{\textsf{numb, index], [}}\vartheta\mbox{\textsf{];}} \\ \quad
\mbox{\textbf{r(}}e_{child}\mbox{\textbf{)}\textsf{)}}
\end{array}
\label{rule:trans:TD:forReturn}
\end{equation}

Where $\vartheta = e_{child}.\vartheta - \chi$.

The \textsf{numberate} operator will have to partition on the remaining dependencies in $\vartheta$ to seperate
the sequences returned from the iterator for all iterators the result is dependent on, as with the sequence
construction expression.

\begin{myExample}

With the rules for translating \texttt{for} and \texttt{return} clauses, the query of figure
\ref{fig:trans:TD:expandQuery} can be fully evaluated. Expression $e_{1}$ is the \texttt{return} clause expression
of , but is not depentant on $I_{\mbox{\texttt{b}}}$. By rule \ref{rule:trans:TD:returnTaint} $e_{1}$ is tainted,
which result is shown in figure \ref{fig:trans:TD:final:taint}. To calculate the result of $I_{\mbox{\texttt{b}}}$,
renumbering is required, as the iterator will return a sequence of four items (to items for each of the two
iterations). The rule \ref{rule:trans:TD:forReturn} is applied. This will result in the relation illustrated in
figure \ref{fig:trans:TD:final:rIb}. $I_{\mbox{\texttt{b}}}$ consitutes the \texttt{return} clause of
$I_{\mbox{\texttt{a}}}$, and is dependent on $I_{\mbox{\texttt{a}}}$ because one of its subexpressions i a
referral to \texttt{\$a}. Rule \ref{rule:trans:TD:returnTaint} will therefore not have any effect. A renumbering
with rule \ref{rule:trans:TD:forReturn} finalises the evalution of the query. The final result is shown in figure
\ref{fig:trans:TD:final:rIb}.


\begin{figure}[h]
\centering
\subfigure[\textbf{t(r(}$e_{1}$\textbf{), \{}\texttt{b}\textbf{\})}]{
\begin{tabular}{|c|c|c|c|} \hline
$bnb$ & $anb$ & $idx$ & $val$ \\ \hline
1 & 1 & 1 & 10 \\ \hline
1 & 2 & 1 & 20 \\ \hline
1 & 1 & 2 & \texttt{"no"} \\ \hline
1 & 2 & 2 & \texttt{"no"} \\ \hline
2 & 1 & 1 & 10 \\ \hline
2 & 2 & 1 & 20 \\ \hline
2 & 1 & 2 & \texttt{"no"} \\ \hline
2 & 2 & 2 & \texttt{"no"} \\ \hline
\end{tabular}
\label{fig:trans:TD:final:taint}
}
\qquad
\subfigure[\textbf{r(}$I_{\mbox{\texttt{b}}}$\textbf{)}]{
\begin{tabular}{|c|c|c|} \hline
$anb$ & $idx$ & $val$ \\ \hline
1 & 1 & 10 \\ \hline
1 & 2 & \texttt{"no"} \\ \hline
1 & 3 & 10 \\ \hline
1 & 4 & \texttt{"no"} \\ \hline
2 & 1 & 20 \\ \hline
2 & 2 & \texttt{"no"} \\ \hline
2 & 3 & 20 \\ \hline
2 & 4 & \texttt{"no"} \\ \hline
\end{tabular}
\label{fig:trans:TD:final:rIb}
}
\qquad
\subfigure[\textbf{r(}$I_{\mbox{\texttt{a}}}$\textbf{)}]{
\begin{tabular}{|c|c|} \hline
$idx$ & $val$ \\ \hline
1 & 10 \\ \hline
2 & \texttt{"no"} \\ \hline
3 & 10 \\ \hline
4 & \texttt{"no"} \\ \hline
5 & 20 \\ \hline
6 & \texttt{"no"} \\ \hline
7 & 20 \\ \hline
8 & \texttt{"no"} \\ \hline
\end{tabular}
\label{fig:trans:TD:final:rIa}
}

\caption[Example: resolving for..return]{Applying rule \ref{rule:trans:TD:forReturn} on (a) yields (b). Applying
rule \ref{rule:trans:TD:forReturn} on (b) yields (c). Attribute names are shortened
\label{fig:trans:TD:finalizeExp}}
\end{figure}

\end{myExample}


Where
\begin{itemize}
  \item select
  \item kan miste -numbz, men det g\aa r greit?
\end{itemize}

orderby
\begin{itemize}
  \item vente litt med denne?
\end{itemize}


\section{Simple Binary Operator Expressions}
\label{sect:trans:TD:binary}

In this section we will present methods for translating simple XQuery binary
operator expressions, namely arithmetic, logic and comparison operator expressions. XQuery binary operators not covered here will be discussed
in section \ref{sect:disc:not:binary} on page \pageref{sect:disc:not:binary}.

\subsection{Arithmetic Expressions}
\label{sect:trans:TD:atrith}
W3C defines the XQuery arithmetic expressions as shown in figure
\ref{fig:trans:TD:arithEBNF}\cite{w3c00}. Notice how the specified grammar handles operator precedence. \texttt{UnaryExpr} is a decendant production of \texttt{UnionExpr}.

\begin{figure}[h]
\begin{Verbatim}
[50] AdditiveExpr       ::= MultiplicativeExpr ( ("+" | "-") MultiplicativeExpr )*
[51] MultiplicativeExpr ::= UnionExpr ( ("*" | "div" | "idiv" | "mod") UnionExpr )*
[58] UnaryExpr          ::= ("-" | "+")* ValueExpr
\end{Verbatim}
\caption{The arithmetic expressions of XQuery}
\label{fig:trans:TD:arithEBNF}
\end{figure}

The translation of such expressions will have to be separated in binary and unary operators. For the binary
operators the two values to be operated on will have to be in the same relation. To ensure the values to be
operated on are from the same unique iteration (if there is any iterations at all), the relations corresponding to
the two expressions will have to be joined on their common iterator dependencies, as described in section
\ref{sect:trans:TD:implic}. Both the unary and the binary XQuery arithmetic operators accept only singleton
sequences. This is discussed in section \ref{sect:disc:singelton}.

\begin{equation}
\frac{}{e_1 \mbox{\texttt{ OP }} e_2}\longmapsto
\begin{array}{l}
\mbox{\textsf{project(index=1,value=FUNC(l.value,r.value),}}\vartheta\mbox{\textsf{;}} \\ \quad
\mbox{\textsf{hhjoin([}}(e_1.\vartheta\cap e_2.\vartheta)\mbox{\textsf{], [}}(e_2.\vartheta\cap e_1.\vartheta)
\mbox{\textsf{],}} 
\mbox{\textsf{[r.value, l.value, }}\vartheta\mbox{\textsf{];}} \\ \quad \quad
\mbox{\textbf{r(}}e_1\mbox{\textbf{)}} \\ \quad \quad
\mbox{\textbf{r(}}e_2\mbox{\textbf{)}\textsf{))}}
\end{array}
\label{rule:trans:TD:arithmetic}
\end{equation}

Where $\vartheta = e_1.\vartheta \cup e_2.\vartheta$, \texttt{OP} will map to a MQL function replacing
\textsf{FUNC} as shown in table \ref{tab:trans:TD:binOpMap}. The projecting functionality of the
\textsf{hhjoin} operator will in this case remove the $index$ attributes and any possible $documentId$, $scope$
and $pos$ attributes.

\begin{table}[h]
\centering
\begin{tabular}{c|c}
\texttt{OP} & \textsf{FUNC} \\ \hline
\texttt{+} 	& \textsf{sum} \\
\texttt{-} 	& \textsf{subtract} \\
\texttt{*} 	& \textsf{prod} \\
\texttt{div} 	& \textsf{div} \\
\texttt{idiv} 	& \textsf{idiv} \\
\texttt{mod} 	& \textsf{mod} \\
\end{tabular}s
\caption{Mapping XQuery arithmetic operators to MQL functions. \label{tab:trans:TD:binOpMap}}
\end{table}

Considering the unary operators, the \texttt{+} operator will never have any effect, and can therefore be dropped.
The unary \texttt{-} operator will change the sign of the value it is assigned to. This is equal to multiplying
the value with $-1$:
\begin{equation}
\frac{}{\mbox{\texttt{-}}e_1}\longmapsto 
\begin{array}{l}
\mbox{\textsf{project(value = prod(-1, value);}} \\ \quad
\mbox{\textbf{r(}}e_1\mbox{\textbf{)}\textsf{)}}
\end{array}
\label{rule:trans:TD:unaryMin}
\end{equation}

\begin{myExample}
Consider the XQuery query of figure \ref{fig:trans:TD:arithQuery}. Here $e_1$ is an arithmetic expression.

\begin{figure}[h]
\centering
\begin{equation*}
\begin{array}{l}
\mbox{\texttt{for \$a in (1,2) return}} \\ \quad
\mbox{\texttt{for \$b in (3,4) return}} \\ \quad \quad
\underbrace{\mbox{\texttt{\$a + \$b}}}_{e_1}
\end{array}
\end{equation*}
\caption{Example query containing a arithmetic expression.}
\label{fig:trans:TD:arithQuery}
\end{figure}

Both \texttt{\$a} and \texttt{\$b} is translated to simple two-tuple relations by rules
\ref{rule:trans:TD:forbind} and \ref{rule:trans:TD:varRef}. As the two expressions do not have any iterator
dependencies the \textsf{hhjoin} operator of rule \ref{rule:trans:TD:arithmetic}
will be treated as a cartesian product (ref. \ref{sect:trans:TD:implic}). The cross product of the two relations are shown in figure
\ref{fig:trans:TD:aritJoin}. Lastly, the \textsf{project} operator is employed
to calculate the sum for each iteration, the result of which is shown in figure \ref{fig:trans:TD:aritEnd}.

\begin{figure}[h]
\centering
\subfigure[]{
\begin{tabular}{|c|c|c|c|}\hline
$anb$ & $bnb$ & $l.val$ & $r.val$ \\ \hline
1 & 1 & 1 & 3 \\ \hline
2 & 1 & 2 & 3 \\ \hline
1 & 2 & 1 & 4 \\ \hline
2 & 2 & 2 & 4 \\ \hline
\end{tabular}
\label{fig:trans:TD:aritJoin}
}
\qquad
\subfigure[\textbf{r(}$e_1$\textbf{)}]{
\begin{tabular}{|c|c|c|c|}\hline
$idx$ & $anb$ & $bnb$ & $val$ \\ \hline
1 & 1 & 1 & 4 \\ \hline
1 & 2 & 1 & 5 \\ \hline
1 & 1 & 2 & 5 \\ \hline
1 & 2 & 2 & 6 \\ \hline
\end{tabular}
\label{fig:trans:TD:aritEnd}
}
\caption[Results of evaluating expression $e_1$ of figure \ref{fig:trans:TD:arithQuery}.]{Results of evaluating
expression $e_1$ of figure \ref{fig:trans:TD:arithQuery}. (a) shows the result of the cross product. (b) is the
fully evaluated $e_1$. Attribute names are shortened. \label{fig:trans:TD:arithRes}}
\end{figure}

\end{myExample}

\subsection{Logical Expressions}
\label{sect:trans:TD:logical}
An XQuery logical expression is either an \texttt{and} expression or an \texttt{or} expression. If a logical
expression does not raise an error(see section \ref{sect:disc:effBool}), its value is always one of the boolean
values $true$ or $false$.

A logical expression is translated in a matter very similar to the arithmetic expressions. The XQuery logical
operators does however operate on the effective boolean value (see section \ref{sect:theory:xquery:basics}) rather
than the direct value. As the operators require boolean values, and the effective boolean function \textbf{B()}
may return a number, the $pred$ fields will have to be run through the \textsf{xqBoolean()} function.

\begin{equation}
\frac{}{e_1 \mbox{\texttt{ OP }} e_2}\longmapsto
\begin{array}{l}
\mbox{\textsf{project(index=1,value=FUNC(xqBoolean(l.pred),xqBoolean(r.pred)),}}\vartheta\mbox{\textsf{;}} \\ \quad
\mbox{\textsf{hhjoin([}}(e_1.\vartheta\cap e_2.\vartheta)\mbox{\textsf{], [}}(e_2.\vartheta\cap e_1.\vartheta)
\mbox{\textsf{],}} 
\\ \quad \quad \quad \quad\mbox{\textsf{[r.pred, l.pred, }}\vartheta\mbox{\textsf{];}} \\ \quad \quad
\mbox{\textbf{B(r(}}e_1\mbox{\textbf{))}} \\ \quad \quad
\mbox{\textbf{B(r(}}e_2\mbox{\textbf{)}\textsf{)))}}
\end{array}
\label{rule:trans:TD:andOr}
\end{equation}

Where $\vartheta = e_1.\vartheta \cup e_2.\vartheta$, \texttt{OP} will map to a MQL function replacing
\textsf{FUNC} as shown in table \ref{tab:trans:TD:logMap}. 

\begin{table}[h]
\centering
\begin{tabular}{c|c}
\texttt{OP} & \textsf{FUNC} \\ \hline
\texttt{or} & \textsf{or} \\
\texttt{and} & \textsf{and} \\
\end{tabular}
\caption{Mapping XQuery boolean operators to MQL functions \label{tab:trans:TD:logMap}}
\end{table}

\subsection{Comparative Expressions}
\label{sect:trans:TD:compArit}
Comparison expressions allow two values to be compared. XQuery provides three kinds of comparison expressions,
called value comparisons, general comparisons, and node comparisons. The comparison operators as specified by W3C
are shown in figure \ref{fig:trans:TD:compEBNF}. The Tainting Dependency method does at this time not support node
comparisons, but possible solutions are discussed in section \ref{sect:disc:not:binary}.

\begin{figure}[h]
\begin{Verbatim}
[61] ValueComp   ::= "eq" | "ne" | "lt" | "le" | "gt" | "ge"
[60] GeneralComp ::= "=" | "!=" | "<" | "<=" | ">" | ">="
[62] NodeComp    ::= "is" | "<<" | ">>"
\end{Verbatim}
\caption{The comparison operators of XQuery \cite{w3c00}}
\label{fig:trans:TD:compEBNF}
\end{figure}

Value comparisons are used for comparing two single values. With the same premises and restrictions, the rule for
translating arithmetic expressions (rule \ref{rule:trans:TD:arithmetic}) can be used to translate such comparison
expressions. The mapping between the XQuery value comparison operators and MQL
functions can be seen in table \ref{tab:trans:TD:valueComp}.

\begin{table}[h]
\centering
\begin{tabular}{c|c}
\texttt{OP} & \textsf{FUNC} \\ \hline
\texttt{eq} & \textsf{eq} \\
\texttt{ne} & \textsf{neq} \\
\texttt{lt} & \textsf{lt} \\
\texttt{le} & \textsf{leq} \\
\texttt{gt} & \textsf{gt} \\
\texttt{ge} & \textsf{geq} \\
\end{tabular}
\caption{Mapping XQuery value comparison operators to MQL functions \label{tab:trans:TD:valueComp}}
\end{table}

General comparisons are existentially quantified comparisons that may be applied to sequences of any length. If
employing a general comparison operator on any pair of items consisting of one from each of the sequences yields
$true$, the comparison expression yields $true$. As an example, all the comparison expressions of figure
\ref{fig:trans:TD:genCompEx} evaluates to $true$.

\begin{figure}[h]
\centering
\begin{tabular}{l}
\texttt{(1, 2) = (2, 3)} \\
\texttt{(1, 2) != (2, 3)} \\
\texttt{(1, 200) < (10, 20)} \\
\texttt{(1, 200) > (10, 20)} \\
\end{tabular}
\caption[Example general comparisons]{\label{fig:trans:TD:genCompEx}Example
general comparisons: all expressions evaluate to $true$}
\end{figure}

The big difference between general and value comparisons is that the first must accomodate for sequences. This is
solved by grouping expressions' iterator dependencies, meaning that each group will contain the sequence of one
unique iteration. By defining $true$ as having a larger value than $false$, the \textsf{max()} aggregator will
identify the groups with \emph{at least} one $true$ value.

As with the arithmetic expressions, the relational representation of the two operands is joined on their common
dependencies to ensure that both values are from the same iteration.

\begin{equation}
\frac{}{e_1\mbox{\texttt{ OP }}e_2}\longmapsto
\begin{array}{l}
\mbox{\textsf{project(index = 1, value=max, }}\vartheta \textsf{;} \\ \quad
\mbox{\textsf{group((}}\vartheta\mbox{\textsf{), max(value);}} \\ \quad \quad
\mbox{\textsf{project(value=FUNC(l.value,r.value),}}\vartheta\mbox{\textsf{;}} \\ \quad \quad \quad
\mbox{\textsf{hhjoin([}}(e_1.\vartheta\cap e_2.\vartheta)\mbox{\textsf{], [}}(e_2.\vartheta\cap e_1.\vartheta)
\mbox{\textsf{],}} 
\mbox{\textsf{[l.value, r.value, }}\vartheta\mbox{\textsf{];}} \\ \quad \quad
\quad \quad \quad \mbox{\textbf{r(}}e_1\mbox{\textbf{)}} \\ \quad \quad \quad \quad \quad
\mbox{\textbf{r(}}e_2\mbox{\textbf{)}\textsf{))))}}
\end{array}
\label{rule:trans:TD:comparative}
\end{equation}

Where $\vartheta = e_1.\vartheta \cup e_2.\vartheta$. \texttt{OP} wil map to a MQL function replacing
\textsf{FUNC} as shown in table \ref{tab:trans:TD:genCompMap}. 

The result of a general comparison is always a singleton sequence, thus it is safe to project in an $index$
attribute with the value $1$. The $index$ and possible $documentId$, $scope$ and $pos$ attributes are left out of
the project list of the \textsf{hhjoin} operator.

\begin{table}[h]
\centering
\begin{tabular}{c|c}
\texttt{OP} & \textsf{FUNC} \\ \hline
\texttt{=} & \textsf{eq} \\
\texttt{!=} & \textsf{neq} \\
\texttt{<} & \textsf{lt} \\
\texttt{<=} & \textsf{leq} \\
\texttt{>} & \textsf{gt} \\
\texttt{>=} & \textsf{geq} \\
\end{tabular}
\caption{Mapping XQuery general comparison operators to MQL functions. \label{tab:trans:TD:genCompMap}}
\end{table}

\begin{myExample}
Figure \ref{fig:trans:TD:genCompQu} shows a simple XQuery query with a general comparison expression, $e_1$.
\begin{figure}[h]
\centering
\begin{equation*}
\begin{array}{l}
\mbox{\texttt{for \$a in (10, 20) return}} \\ \quad
\underbrace{\mbox{\texttt{\$a > (5, 15)}}}_{e_1}
\end{array}
\end{equation*}
\caption{Example query with a general comparison expression \label{fig:trans:TD:genCompQu}}
\end{figure}
 
Because the operands of the \texttt{>} operator have no common iterator dependencies, the cartesian product of the
two relations is calculated, as seen in figure \ref{fig:trans:TD:gen:join}.
After the inner \textsf{project} operator is employed, the result will be as in figure \ref{fig:trans:TD:gen:projGr}. The double line illustrates
the grouping on $anumb$ ($anb$ in the figure). The maximum value of $value$ for each group is calculated and the
attributes are renamed, which gives the relation of figure \ref{fig:trans:TD:gen:end}.

\begin{figure}[h]
\centering
\subfigure[]{
\begin{tabular}{|c|c|c|} \hline
$anb$ & $l.val$ & $r.val$ \\ \hline
1 & 10 & 5 \\ \hline
1 & 10 & 15 \\ \hline
2 & 20 & 5 \\ \hline
2 & 20 & 15 \\ \hline
\end{tabular}
\label{fig:trans:TD:gen:join}
}
\qquad
\subfigure[]{
\begin{tabular}{|c|c|} \hline
$anb$ & $val$ \\ \hline
1 & $true$ \\ \hline
1 & $false$ \\ \hline \hline
2 & $true$ \\ \hline
2 & $true$ \\ \hline
\end{tabular}
\label{fig:trans:TD:gen:projGr}
}
\qquad
\subfigure[\textbf{r(}$e_1$\textbf{)}]{
\begin{tabular}{|c|c|c|} \hline
$idx$ & $anb$ & $val$ \\ \hline
1 & 1 & $true$ \\ \hline
1 & 2 & $true$ \\ \hline
\end{tabular}
\label{fig:trans:TD:gen:end}
}
\caption[Results of evaluating $e_1$ in figure \ref{fig:trans:TD:genCompQu}]{Results of evaluating expression $e_1$
in figure \ref{fig:trans:TD:genCompQu}. (a) The relations of the operands joined. (b) Each combination of the
sequences evaluated. Double line illustrates grouping. (c) The final result. Attribute names are shortened.
\label{fig:trans:TD:genCompRes}}
\end{figure}
 
\end{myExample}

\section{Conditional Expressions}
\label{sect:trans:TD:ifThenElse}
XQuery supports a conditional expression based on the keywords \texttt{if}, \texttt{then}, and \texttt{else}. The
expression is defined by W3C as seen in figure \ref{fig:trans:TD:condEBNF}.

\begin{figure}[h]
\begin{Verbatim}
[45] IfExpr ::= "if" "(" Expr ")" "then" ExprSingle 
                    "else" ExprSingle
\end{Verbatim}
\label{fig:trans:TD:condEBNF}
\caption{W3C EBNF conditional expression specification.}
\end{figure}

The expression following the \texttt{if} keyword is called the test expression, and the expressions following the
\texttt{then} and \texttt{else} keywords are called the \texttt{then}-expression and \texttt{else}-expression,
respectively. If the effective boolean value of the test expression evaluates to $true$, the
\texttt{then}-expression is returned, if it evaluates to $false$ the \texttt{else}-expression is returned.

Conditional expressions are translated by adding an attribute $alt$ with the value $1$ the
\texttt{then}-expression relation and the relational representation of the \texttt{else}-expression with the same
attribute but with value $2$. These two relations are then spliced together with a \textsf{union} operator. If the
relations have disjoint dependencies, they will have to be tainted first.

The result of the \textsf{union} operation is then joined with the relational representation of the test
expression on their common dependencies. Lastly, a \textsf{select} operator is employed on this relation to select
the tuples where $alt$ is $1$ if the $value$ field from the test expression evaluates to $true$, or $alt$ is $2$
if it does not.

\begin{equation}
\begin{array}{c}
\frac{}{\mbox{\texttt{if }}e_1\mbox{\texttt{ then }}e_2\mbox{\texttt{ else }}e_3} \\
\longmapsto \\
\begin{array}{l}
\mbox{\textsf{project(value = r.value, }}\vartheta\mbox{\textsf{;}} \\ \quad
\mbox{\textsf{select(ifthenelse(xqBoolean(r.value), eq(alt,1), eq(alt,2));}} \\ \quad \quad
\mbox{\textsf{hhjoin([l.}}((e_2.\vartheta \cup e_3.\vartheta)\cap e_1.\vartheta)\mbox{\textsf{],
[r.}}(e_1.\vartheta\cap (e_2.\vartheta\cup e_3.\vartheta))
\mbox{\textsf{],}} 
\\  \quad \quad \quad \quad \quad\mbox{\textsf{[l.value, r.value, }}\vartheta\mbox{\textsf{];}}\\\quad\quad\quad
\mbox{\textsf{union(}} \\ \quad\quad\quad\quad
\mbox{\textsf{project(alt=1, value, }}(e_2.\vartheta \cup e_3.\vartheta)\mbox{\textsf{,}}\\\quad\quad\quad\quad\quad
\mbox{\textbf{t(r(}}e_2\mbox{\textbf{), }}e_3.\vartheta\mbox{\textbf{)}\textsf{);}} \\ \quad\quad\quad\quad
\mbox{\textsf{project(alt=2, value, }}(e_3.\vartheta\cup e_2.\vartheta)\mbox{\textsf{,}}\\\quad\quad\quad\quad\quad
\mbox{\textbf{t(r(}}e_3\mbox{\textbf{), }}e_2.\vartheta\mbox{\textbf{)}\textsf{));}}\\\quad\quad\quad
\mbox{\textbf{r(}}e_1\mbox{\textbf{)}\textsf{)))}}
\end{array}
\end{array}
\label{rule:trans:TD:conditional}
\end{equation}

Where $\vartheta = e_1.\vartheta \cup e_2.\vartheta \cup e_3.\vartheta$. The test expression will have to be
evaluated in the logical context $\Lambda$. $index$ and possible $documentId$, $pos$ and $scope$ attributes will
follow $value$ as described in \ref{sect:trans:TD:basics}.

\begin{myExample}
\begin{figure}[h]
\centering
\begin{equation*}
\begin{array}{l}
\mbox{\texttt{for \$a in (10, 20) return}} \\ \quad
\mbox{\texttt{for \$b in (5, 15) return}} \\ 
e_1 \left\{\begin{array}{l}
           \mbox{\texttt{if(\$b > \$a) then}} \\ \quad
           \mbox{\texttt{\$a}} \\ 
           \mbox{\texttt{else}} \\ \quad
           \mbox{\texttt{\$b}}
           \end{array}\right.
\end{array}
\end{equation*}
\caption{Example query containing a conditional expression \label{fig:trans:TD:condQue}}
\end{figure}

The query of figure \ref{fig:trans:TD:condQue} contains a conditional expression $e_1$. First the
\texttt{then}-expression and the \texttt{else}-expression are tainted with eachothers iterator dependencies. The
result of the tainting is shown in figure \ref{fig:trans:TD:inter1:taint}. Then the two relations are augmented
with the $alt$ attribute and spliced together with an \textsf{union} operator to make the relation depicted in
figure \ref{fig:trans:TD:inter1:union}.

\begin{figure}[h]
\centering
\subfigure[]{
\begin{tabular}{|c|c|c|c|} \hline
$bnb$ & $anb$ & $val$ \\ \hline
1 & 1 & 10 \\ \hline
1 & 2 & 20 \\ \hline
2 & 1 & 10 \\ \hline
2 & 2 & 20 \\ \hline
\end{tabular}
\label{fig:trans:TD:inter1:taint}
}
\phantom{A}
\subfigure[]{
\begin{tabular}{|c|c|c|c|} \hline
$alt$ & $bnb$ & $anb$ & $val$ \\ \hline
1 & 1 & 1 & 10 \\ \hline
1 & 1 & 2 & 20 \\ \hline
1 & 2 & 1 & 10 \\ \hline
1 & 2 & 2 & 20 \\ \hline
2 & 1 & 1 & 5 \\ \hline
2 & 2 & 1 & 15 \\ \hline
2 & 1 & 2 & 5 \\ \hline
2 & 2 & 2 & 15 \\ \hline
\end{tabular}
\label{fig:trans:TD:inter1:union}
}
\phantom{A}
\subfigure[]{
\begin{tabular}{|c|c|c|} \hline
$anb$ & $bnb$ & $val$ \\ \hline
1 & 1 & $false$ \\ \hline
2 & 1 & $false$ \\ \hline
1 & 2 & $true$ \\ \hline
2 & 2 & $false$ \\ \hline
\end{tabular}
\label{fig:trans:TD:inter1:test}
}
\caption[Intermediate results of evaluating $e_1$ in figure \ref{fig:trans:TD:condQue}.]{Intermediate results of
evaluating $e_1$ in figure \ref{fig:trans:TD:condQue}. (a) The \texttt{then}-expresion tainted with $I_{\mbox{\texttt{b}}}$ (b) The \texttt{then}-expression and the
\texttt{else}-expression augmented with an $alt$ attribute and spliced together. (c) The test expression.
Attribute names are shortened. $index$ attribute is left out. \label{fig:trans:TD:inter1}}
\end{figure}

The result of the \textsf{union} operation is then joined with the test expression (figure
\ref{fig:trans:TD:inter1:test}) on their common dependencies ($anumb$ and $bnumb$) to form the relation in figure
\ref{fig:trans:TD:condRes:join}. From this relation, only the tuples where $alt$ has the value 1 and $r.val$ is
$true$ or $alt$ has the value 2 and $r.val$ is $false$ are selected. The result of the selection after the final
renaming is shown in figure \ref{fig:trans:TD:condRes:res}.

\begin{figure}[h]
\centering
\subfigure[]{
\begin{tabular}{|c|c|c|c|c|} \hline
$alt$ & $bnb$ & $anb$ & $l.val$ & $r.val$ \\ \hline
1 & 1 & 1 & 10 & $false$ \\ \hline
2 & 1 & 1 & 5 & $false$ \\ \hline
1 & 2 & 1 & 10 & $true$ \\ \hline
2 & 2 & 1 & 15 & $true$ \\ \hline
1 & 1 & 2 & 20 & $false$ \\ \hline
2 & 1 & 2 & 5 & $false$ \\ \hline
1 & 2 & 2 & 20 & $false$ \\ \hline
2 & 2 & 2 & 15 & $false$ \\ \hline
\end{tabular}
\label{fig:trans:TD:condRes:join}
}
\qquad
\subfigure[]{
\begin{tabular}{|c|c|c|} \hline
$bnb$ & $anb$ & $val$ \\ \hline
1 & 1 & 5 \\ \hline
2 & 1 & 10 \\ \hline
1 & 2 & 5 \\ \hline
2 & 2 & 15 \\ \hline
\end{tabular}
\label{fig:trans:TD:condRes:res}
}
\caption[Evaluating $e_1$ in figure \ref{fig:trans:TD:condQue}.]{Evaluating $e_1$ in figure
\ref{fig:trans:TD:condQue}. (a) The test expression joined with the union of the \texttt{then}- and
\texttt{else}-expression. (b) $e_1$ fully evaluated. Attribute names are shortened. $index$ attribute is omitted.
\label{fig:trans:TD:condRes} }
\end{figure}

\end{myExample}



\section{Quantified Expressions}
\label{sect:disc:not:quantified}

Figure \ref{fig:disc:not:quantified} shows the EBNF specification of the XQuery quantified expression. These
expressions support existential and universal quantification, and will always result in a single $true$ or $false$
value.

\begin{figure}[h]
\begin{Verbatim}
[42] QuantifiedExpr ::= ("some" | "every") "\$" VarName "in" ExprSingle 
                        ("," "\$" VarName "in" ExprSingle)* "satisfies" ExprSingle
\end{Verbatim}
\caption{W3C specification of quantified expressions \cite{w3c00} \label{fig:disc:not:quantified}}
\end{figure}

If the quantifier is \texttt{some}, the expression only returns $true$ if at least one evaluation of the
\texttt{satisfies}-expression yields $true$. For the \texttt{every} quantifier, the expression will only return
$true$ if every evaluation of the \texttt{satisfies}-expression yields $true$.

It can be suitable to treat such expressions as iterators with the \texttt{satisfies}-expression as the iterator
body. The result of the iterator is then checked if every or some of its items have the effective boolean value
$true$. This can be done with a \textsf{group} operator similarily to the general comparison operator expressions.
If we assume $true$ has a bigger value than $false$, a \textsf{max} aggregator will reveal if there exists a tuple
with the value $true$ and a \textsf{min} operator will reveal if there exist a tuple with the value $false$. All
variables bound in quantified expressions will be treated by the iterator binding translation of rule
\ref{rule:trans:TD:forbind}. $\beta$ will refer to the set of all variables bound in one such expression

\begin{equation}
\frac{}{\mbox{\texttt{QUANT \$\ldots  satisfies }}e_1} \longmapsto
\begin{array}{l}
\mbox{\textsf{project(index = 1, value, }}\vartheta\mbox{\textsf{;}} \\ \quad
\mbox{\textsf{group((}}\vartheta\mbox{\textsf{), AGG(value);}} \\ \quad\quad
\mbox{\textsf{project(xqBoolean(value), }}\vartheta\mbox{\textsf{;}} \\ \quad\quad\quad
\mbox{\textbf{B(r(}}e_1\mbox{\textbf{))}\textsf{)))}}
\end{array}
\label{rule:trans:TD:quantified}
\end{equation}

Where $\vartheta=e_1.\vartheta-\beta$, and a quantifier specification \texttt{QUANT} maps to an aggregator
function as seen in table \ref{tab:disc:quant}. The \textsf{xqBoolean()} funcion will have to be run on the
$value$ fields of the \texttt{satisfies} expression, as there is no requirement that it is a boolean expression.

\begin{table}[h]
\centering
\begin{tabular}{c|c} 
\texttt{QUANT} & \textsf{AGG} \\ \hline
\texttt{some} & \textsf{max} \\
\texttt{every} & \textsf{min}
\end{tabular}
\caption{Mapping XQuery quantifiers to MQL aggregators \label{tab:disc:quant}}
\end{table}

\begin{myExample}
Consider the XQuery query of figure \ref{fig:trans:TD:quantQu}. In this query the iterator body of the FLWOR is a
quantified expression.
\begin{figure}[h]
\begin{equation*}
\begin{array}{l}
\mbox{\texttt{for \$a in ("a","b") return}} \\ \quad
\mbox{\texttt{every \$e in (\$a, "b") satisfies \$e eq "b"}}
\end{array}
\end{equation*}
\caption{Example query with quantified expression \label{fig:trans:TD:quantQu}}
\end{figure}

The sequence sequentially bound to the quantifier variable is treated as explained in section
\ref{sect:trans:TD:seqBuild}: where \texttt{\$a} taints \texttt{"b"} and the relations are spliced together. As
quantifier variables are to be handled as iterator variables, the $index$ attribute of this relation will be
renamed $enumb$, as illustrated in figure \ref{fig:trans:TD:quant:seq}. This relation (or rather the algebra
evaluating to it) is stored in the symbol table, and is fetched during the evaluation of the
\texttt{satisfies}-expression. The result of this comparison expression is shown in figure
\ref{fig:trans:TD:quant:satEx}. Here, the double line illustrates grouping.

\begin{figure}[h]
\centering
\subfigure[]{
\begin{tabular}{|c|c|c|c|} \hline
$idx$ & $anb$ & $enb$ & $val$ \\ \hline
1 & 1 & 1 & \texttt{"a"} \\ \hline
1 & 1 & 2 & \texttt{"b"} \\ \hline
1 & 2 & 1 & \texttt{"b"} \\ \hline
1 & 2 & 2 & \texttt{"b"} \\ \hline
\end{tabular}
\label{fig:trans:TD:quant:seq}
}
\qquad
\subfigure[]{
\begin{tabular}{|c|c|c|c|} \hline
$idx$ & $anb$ & $enb$ & $val$ \\ \hline
1 & 1 & 1 & $false$ \\ \hline
1 & 1 & 2 & $true$ \\ \hline\hline
1 & 2 & 1 & $true$ \\ \hline
1 & 2 & 2 & $true$ \\ \hline
\end{tabular}
\label{fig:trans:TD:quant:satEx}
}
\qquad
\subfigure[]{
\begin{tabular}{|c|c|c|} \hline
$idx$ & $anb$ & $val$ \\ \hline
1 & 1 & $false$ \\ \hline
1 & 2 & $true$ \\ \hline
\end{tabular}
\label{fig:trans:TD:quant:quant}
}

\caption[Evaluating the quantified expression]{Evaluating the quantified expression of figure
\ref{fig:trans:TD:quantQu}. (a) The quantifier variable fetched from the symbol table. (b) The result of the
\texttt{satisfies}-expression. The double line illustrates the groups. (c) The result of the quantified
expression. Attribute names are shortened. \label{fig:trans:TD:quantInter} }
\end{figure}

Except for the quantifier variable, this expression is dependent on $I_{\mbox{\texttt{a}}}$, which the result of
the \texttt{satisfies}-expression is grouped on. Each group will be run through the \texttt{min} aggregator,
because the quantifier is \texttt{every}. As $false$ has a lower value than $true$, the aggregator will reveal if
the group contains a $false$. The result of the grouping is shown in figure \ref{fig:trans:TD:quant:quant}. This
result will have to be renumbered to finalise the evaluation of the query.
\end{myExample}
\section{Path Expressions and Predicates}
\label{sect:trans:TD:pathNpred}

XQuery implement XPath 2.0 path expressions and predicates as described in section
\ref{sect:theory:xquery:PathExpressions} and \ref{sect:theory:xquery:Predicates}. In this section we will present
a method for translating some of these expressions into MQL relational algebra.

For the translations in this section to be correct, we have a few assumptions. A scope defined as a parameter to
the \textsf{scope}-operator starting with a slash (\textsf{/}) is an absolute scope. That means if the parameter
is e.g. \textsf{/a/b}, the operator will remove any tuples in the input relation where the $scope$ attribute does
not define the tuple to stem from a \texttt{b} scope within the root-scope \texttt{a}. Without a slash first in
the parameter, the scope is relative. E.g. \texttt{a/b} as a parameter will lead to removal of all tuples where
the $scope$ attribute does not define the tuple to stem from a \texttt{b} scope within an \texttt{a} scope.

We also assume that the tuples returned from a scope lookup in the \textsf{valocc}-index will have information of
the scope it self in the $scope$ attribute, and the contents of the scope in an $value$ attribute. E.g. a lookup of
\textsf{\$c} (the \textsf{\$}-sign indicates it is a scope) in this index may return a tuple with
\textsf{a[1].b[2].c[1]} as its $scope$ value.

The tuples returned from a lookup in the value occurence index will have to be ordered according to document
order. This order will have to hold even after the result is run through a \textsf{scope}


\subsection{Path Expressions}
\label{sect:trans:TD:pathExprs}
A path expression consists of a series of one or more steps, separated by ``/'' or ``//'', and optionally beginning
with ``/'' or ``//''.  Each step is either a axis step or a filter expression, and a axis step consist of a axis
and a node test. This can be seen from the exerpt of the W3C XQuery specification in figure \ref{fig:trans:TD:pathEBNF}.
A node test can be either a kind test or a name test, we will focus on the latter. Filter expressions will not
explicitly be handled.

\begin{figure}[h]
\begin{Verbatim}
[68] PathExpr        ::= ("/" RelativePathExpr?)
                       | ("//" RelativePathExpr)
                       | RelativePathExpr
[69] RelativePathExpr::= StepExpr (("/" | "//") StepExpr)*
[70] StepExpr        ::= FilterExpr | AxisStep
[71] AxisStep        ::= (ReverseStep | ForwardStep) PredicateList
[72] ForwardStep     ::= (ForwardAxis NodeTest) | AbbrevForwardStep
[75] ReverseStep     ::= (ReverseAxis NodeTest) | AbbrevReverseStep
\end{Verbatim}
\label{fig:trans:TD:pathEBNF}
\caption{Path expressions as specified by W3C}
\end{figure}
The semantics of such expressions reading from left to right, the result of one step expression is used as input
for the next. Within one step expression, the result from the preceding step will first be used as input for the
axis expression. The result of this will be filtered by a name or kind test, before this again is filtered by
possible predicates.

The effect of a ``/'' at the beginning of a path expression is to begin the path at the root node of the tree that
contains the context node. A ``//'' at the beginning of a path expression will establish an initial node sequence
that contains the root of the tree in which the context node is found, plus all nodes descended from this root.
Together with a name test these expressions can be translated as follows:

\begin{equation}
\frac{}{\mbox{\texttt{/}}QName_{1}}\longmapsto
\begin{array}{l}
\mbox{\textsf{numberate(index,[] ,[] ;}} \\ \quad
\mbox{\textsf{index(valocc;}} \\ \quad \quad
\mbox{\textsf{scope(/}}QName_{1}\mbox{\textsf{;}} \\ \quad\quad\quad
\mbox{\textsf{lookup(\$}}QName_{1}\mbox{\textsf{))))}}
\end{array}
\label{rule:trans:TD:sglSlash}
\end{equation}

\begin{equation}
\frac{}{\mbox{\texttt{//}}QName_{1}}\longmapsto
\begin{array}{l}
\mbox{\textsf{numberate(index, [], [];}} \\ \quad
\mbox{\textsf{index(valocc;}} \\ \quad \quad
\mbox{\textsf{lookup(\$}}QName_1\mbox{\textsf{)))}}
\end{array}
\label{rule:trans:TD:dblSlash}
\end{equation}

Where $QName_1$ is any XML-qualified name, and the dependencies returned are $\vartheta = \emptyset$. To ensure
correct ordering, the order of the tuples returned from the \textsf{lookup} operator must be the same as the order
of the tuples recieved by the \textsf{numberate} operator. 

Step expressions can be abbreviated. If the axis name is omitted from an axis step, the default axis is
\texttt{child}. \texttt{decendant-or-self} can be replaced by using ``\texttt{//}'' istead of ``\texttt{/}''
between the steps. \texttt{@} is an abbreviation of \texttt{attribute::}. Here we will present translation of
unabbreviated syntax.

A general step expression, axis + name test, can be translated like this:
\begin{equation}
\begin{array}{c}
\frac{}{\displaystyle e_{1}\texttt{/AXIS::}QName_1} \\ 
\longmapsto \begin{array}{c}\mbox{\tiny } \\ \mbox{\tiny } \end{array} \\
\begin{array}{l}
\mbox{\textsf{project(docId, index, value, pos, scope, }}\vartheta\mbox{\textsf{;}} \\ \quad
\mbox{\textsf{numberate(index, [sprIdx, index], [}}\vartheta\mbox{\textsf{];}} \\ \quad\quad
\mbox{\textsf{select(isFUNC(scope, lsc);}} \\ \quad \quad\quad
\mbox{\textsf{hhjoin([docId],[docId],[spridx=l.index,lsc=l.scope,r.value,}}\vartheta\mbox{\textsf{];}}\\
\quad\quad\quad\quad \mbox{\textbf{r(}}e_1\mbox{\textbf{)}\textsf{;}} \\ \quad \quad\quad \mbox{\textsf{numberate(index, [], [];}}\\ \quad\quad\quad\quad\quad
\mbox{\textsf{index(valocc;}} \\ \quad\quad\quad\quad\quad\quad
\mbox{\textsf{lookup(\$}}QName_1\mbox{\textsf{)))))))}}
\end{array}
\end{array}
\label{rule:trans:TD:pathStep}
\end{equation}

Where $\vartheta = e.1.\vartheta$. \textsf{r.value} is short for \textsf{value = right.value, index = right.index,
\ldots etc}, and \textsf{docId} is short for \textsf{documentId}. \texttt{AXIS} will map to an MQL funciton
\textsf{isFUNK} as described in table \ref{tab:trans:TD:axisMap}.

\begin{table}[h]
\centering
\begin{tabular}{c|c}
\texttt{AXIS} & \textsf{isFUNC} \\ \hline
\texttt{child} & \textsf{isChild} \\
\texttt{descendant} & \textsf{isDescendant} \\
\texttt{attribute} & \textsf{isChild} $^{*}$ \\
\texttt{self} & \textsf{isSelf} \\
\texttt{descendant-or-self} & \textsf{isDescendantOrSelf} \\
\texttt{following} & \textsf{isFollowing} \\
\texttt{following-sibling} & \textsf{isFollowingSibling} \\
\texttt{parent} & \textsf{isParent} \\
\texttt{ancestor} & \textsf{isAncestor} \\
\texttt{ancestor-or-self} & \textsf{isAncestorOrSelf} \\
\texttt{preceding} & \textsf{isPreceding} \\
\texttt{preceding-sibling} & \textsf{isPrecedingSibling} 
\end{tabular}
\caption{Mapping between XQuery axes and MQL functions. \label{tab:trans:TD:axisMap}}
\end{table}

$^{*}$For the \texttt{attribute} axis the parameter to \textsf{lookup} will have a \textsf{\$@}-prefix
instead of the \textsf{\$}-prefix described in the rule.

\begin{myExample}
Consider the exerpt of a XML-document of figure \ref{fig:trans:TD:pathExXml}. The subscript numbers are used to
differentiate the different elements with same names, and are not a part of the names. Further, let a non-iterator
variable \texttt{\$a} be bound to a sequence of the \texttt{A} elements of the figure, but \emph{not} in document order.
An illustration of the relational representation of \texttt{\$a} is shown in figure
\ref{fig:trans:TD:pathEx:varA}. The $val$-attribute indicates which XML-element is represented and the
$scope$-attribute indicates the scope of this element.

\begin{figure}[h]
\centering
\begin{equation*}
\begin{array}{l}
\qquad \qquad \vdots \\
\mbox{\texttt{<A}}_{1}\mbox{\texttt{>}} \\ \quad
\mbox{\texttt{<B}}_{1}\mbox{\texttt{/>}}\mbox{\texttt{<B}}_{2}\mbox{\texttt{/>}} \\
\mbox{\texttt{</A}}_{1}\mbox{\texttt{>}} \\
\mbox{\texttt{<A}}_{2}\mbox{\texttt{>}} \\ 
\mbox{\texttt{</A}}_{2}\mbox{\texttt{>}} \\
\mbox{\texttt{<A}}_{3}\mbox{\texttt{>}} \\ \quad
\mbox{\texttt{<B}}_{3}\mbox{\texttt{/>}}\mbox{\texttt{<B}}_{4}\mbox{\texttt{/>}}\mbox{\texttt{<B}}_{5}\mbox{\texttt{/>}}
\\ \mbox{\texttt{</A}}_{3}\mbox{\texttt{>}} \\
\qquad \qquad \vdots
\end{array}
\end{equation*}
\caption[Exerpt of example XML-document.]{Exerpt of example XML-document. The subscript numbers indicate the
instance of the elements, and are not part of the QName \label{fig:trans:TD:pathExXml}}
\end{figure}

Figure \ref{fig:trans:TD:pathQu} shows an exerpt of a query referring to the variable \texttt{\$a}.
\begin{figure}[!h]
\centering
\texttt{\$a/child::B}
\caption{Example XQuery path expression \label{fig:trans:TD:pathQu}}
\end{figure} 

First, a lookup of \texttt{\$B} (where \texttt{\$} indicates to find a element, not a word) is done in the value
occurence index. The result of this is numerated by a \textsf{numberate}-operator, as illustrated in figure
\ref{fig:trans:TD:pathEx:luB}. The $index$-attribute ($idx$ in the figure) now holds the document-order of the
\texttt{B} elements. As there may be more \texttt{B} elements in the document, the $index$ attribute may not start
at the value 1 for the tuples relevant to the query.

\begin{figure}[h]
\centering
\subfigure[\textbf{r(}\texttt{\$a}\textbf{)}]{
\begin{tabular}{|c|c|c|}  \hline
$idx$ & $val$ & $scope$ \\ \hline
1 & \texttt{A}$_{2}$ & \textsf{\ldots{A}[2]} \\ \hline
2 & \texttt{A}$_{3}$ & \textsf{\ldots{A}[3]} \\ \hline 
3 & \texttt{A}$_{1}$ & \textsf{\ldots{A}[1]} \\ \hline
\end{tabular}
\label{fig:trans:TD:pathEx:varA}
}
\qquad
\subfigure[]{
\begin{tabular}{|c|c|c|}  \hline
$idx$ & $val$ & $scope$ \\ \hline
$\ldots$ & \texttt{\ldots} & \textsf{\ldots} \\ \hline
5 & \texttt{B}$_{1}$ & \textsf{\ldots{A}[1].B[1]} \\ \hline
6 & \texttt{B}$_{2}$ & \textsf{\ldots{A}[1].B[2]} \\ \hline
7 & \texttt{B}$_{3}$ & \textsf{\ldots{A}[3].B[1]} \\ \hline
8 & \texttt{B}$_{4}$ & \textsf{\ldots{A}[3].B[2]} \\ \hline
9 & \texttt{B}$_{5}$ & \textsf{\ldots{A}[3].B[3]} \\ \hline
$\ldots$ & \texttt{\ldots} & \textsf{\ldots} \\ \hline  
\end{tabular}
\label{fig:trans:TD:pathEx:luB}
}
\caption[Evaluating the expression of figure \ref{fig:trans:TD:pathQu}]{Illustration of results evaluating the
expression in figure \ref{fig:trans:TD:pathQu}. (a) The variable \texttt{\$a}. (b) Experpt of a lookup on the term
\textsf{\$B}, and the following numbering. Some attribute names are shortened. $val$ attribute indicates which
XML-element is represented in the tuple.
\label{fig:trans:TD:pathEx}}
\end{figure}

The result of the lookup will be joined with the \texttt{\$a} relation on their $documentId$-attribute. As we
assume only one XML-document this attribute is omitted from our example. A \textsf{select} operator is applied to
the result of the join to prune the relation. Only the tuples where the $scope$ attribute stemming from the lookup
defines a scope which is the child scope (as defined by the MQL function \textsf{isChild() in section
\ref{sect:method:marsAddedOperators}}) of the scope defined by the $scope$-attribute stemming from the
\texttt{\$a} relation (called $lsc$). After the selection the result will be as illustrated in figure
\ref{fig:trans:TD:pathEx:joinSel}.

\begin{figure}[h]
\subfigure[]{
\begin{tabular}{|c|c|c|c|c|} \hline
$sIdx$ & $idx$ & $val$ & $scope$ & $lsc$ \\ \hline
2 & 9 & \texttt{B}$_{5}$ & \textsf{\ldots{A}[3].B[3]} & \textsf{\ldots{A}[3]} \\ \hline
2 & 8 & \texttt{B}$_{4}$ & \textsf{\ldots{A}[3].B[2]}& \textsf{\ldots{A}[3]} \\ \hline
2 & 7 & \texttt{B}$_{3}$ & \textsf{\ldots{A}[3].B[1]}& \textsf{\ldots{A}[3]} \\ \hline 
3 & 5 & \texttt{B}$_{1}$ & \textsf{\ldots{A}[1].B[1]} & \textsf{\ldots{A}[1]} \\ \hline
3 & 6 & \texttt{B}$_{2}$ & \textsf{\ldots{A}[1].B[2]} & \textsf{\ldots{A}[1]} \\ \hline
\end{tabular}
\label{fig:trans:TD:pathEx:joinSel}
}
\qquad
\subfigure[]{
\begin{tabular}{|c|c|c|} \hline
$idx$ & $val$ & $scope$ \\ \hline
1 & \texttt{B}$_{3}$ & \textsf{\ldots{A}[3].B[1]}\\ \hline 
2 & \texttt{B}$_{4}$ & \textsf{\ldots{A}[3].B[2]}\\ \hline
3 & \texttt{B}$_{5}$ & \textsf{\ldots{A}[3].B[3]}\\ \hline
4 & \texttt{B}$_{1}$ & \textsf{\ldots{A}[1].B[1]}\\ \hline
5 & \texttt{B}$_{2}$ & \textsf{\ldots{A}[1].B[2]}\\ \hline
\end{tabular}
\label{fig:trans:TD:pathEx:done}
}
\caption[Further evaluation of the path expression]{Further evaluation of the path expression in figure
\ref{fig:trans:TD:pathQu}. (a) The result of the selection of the joining of \textbf{r(}\texttt{\$a}\textbf{)} and
the numerated result of the lookup of \textsf{\$B}. (b) Renumbering an projection on the relation of (a). Some
attribute names are shortened.}
\end{figure}

Finally, renumbering is employed, followed by a projection removing the last attributes stemming from the
\texttt{\$a} relation. The result of the expression is illustrated in figure \ref{fig:tans:TD:pathEx:done}.

\end{myExample}

\subsection{Predicates}
\label{sect:trans:TD:predicates}
\begin{itemize}
  \item isNumber() ifThenElse xqBoolean()
  \item contextnode p\aa~boks\ldots trengs dette alltid? Strengt tatt.. veldig vanskelig \aa~vite n\aa r man
  garantert ikke trenger den iallefall\ldots
\end{itemize}

\section{Simplifications}
\label{sect:trans:TD:simplifications}

\textbf{\LARGE TODO:}innledning

In this section we will present some posible simplifications discovered during the develepment of the Tainted
Dependencies method. The $\Rightarrow$ sign is to be read as ``can be written as''. 

\subsection{Literals}
\label{sect:trans:TD:simpl:lit}      

Rule \ref{rule:trans:TD:literal} of section \ref{sect:trans:TD:litteral} shows a very general way to translate
XQuery literals to a relational format. But creating one relation for each litteral is very often unnecessary, and
often quite a bit more resource consuming than alternative solutions. The parent expression of a litteral should
in most cases be informed that its subexpression is a litteral instead of being handed a one-tuple relation.

One such case is if the litteral will be used in a join (predicate-less) or cartesian product. A better solution
will then be to project the litteral into the other relation. Following is an example of how such a expression
should be written:
\begin{equation}
\begin{array}{lcr}
\begin{array}{l}
\mbox{\quad\quad\textsf{\vdots}} \\
\mbox{\textsf{hhjoin([],[],\ldots}} \\ \quad
\mbox{\textsf{\ldots}\textbf{r(}}e\mbox{\textbf{)\ldots}} \\ \quad
\mbox{\textbf{r(}}Literal\mbox{\textbf{)}} \\
\mbox{\quad\quad\textsf{\vdots}}
\end{array}
&
\Rightarrow
&
\begin{array}{l}
\mbox{\quad\quad\textsf{\vdots}} \\
\mbox{\textsf{project(\ldots, rvalue=}}Literal\mbox{\textsf{\ldots}} \\ \quad
\mbox{\textsf{\ldots}\textbf{r(}}e\mbox{\textbf{)\ldots}} \\
\mbox{\quad\quad\textsf{\vdots}}
\end{array}
\end{array}
\label{simpl:trans:TD:joinLit}
\end{equation}

And if the reason why the literal was joined with the relation was because it was part of a arithmetic, comparison
or logical expression, the literal may be moved inside the \textsf{project} operator responsible for executing the
binary operation, shortening the relational algebra even more:

\begin{equation}
\begin{array}{lcr}
\begin{array}{l}
\mbox{\textsf{project(val=OP(l.val, r.val)\ldots}} \\ \quad
\mbox{\textsf{hhjoin([],[],\ldots}} \\ \quad \quad
\mbox{\textsf{\ldots}\textbf{r(}}e\mbox{\textbf{)\ldots}} \\ \quad \quad
\mbox{\textbf{r(}}Literal\mbox{\textbf{)}} \\
\mbox{\quad\quad\textsf{\vdots}}
\end{array}
&
\Rightarrow
&
\begin{array}{l}
\mbox{\textsf{project(val=OP(val, }}Literal\mbox{\textsf{)\ldots}} \\ \quad
\mbox{\textsf{\ldots}\textbf{r(}}e\mbox{\textbf{)\ldots}} \\
\mbox{\quad\quad\textsf{\vdots}}
\end{array}
\end{array}
\label{simpl:trans:TD:binLit}
\end{equation}

If the cases where a literal will have to be translated to a single-tuple relation, in most cases the $index$
attribute will not be needed. But this will probably detected by the optimiser, and may be left out anyway.

\subsection{Sequence Construction}
\label{sect:trans:TD:simpl:seq}

Informing a parent expression if its subexpressions will evaluate to singleton sequences or not can have some
advantages. One is the posibility of detecting certain type errors, as will be discussed in section
\ref{sect:disc:singelton}. Another advandage is gained when it comes to sequence construction.

Rule \ref{rule:trans:TD:seqConstr} in section \ref{sect:trans:TD:seqBuild} describes a general way to build
sequences. But if all the expressions to build a sequence from will evaluate to singleton sequences, there is no
need for the \textsf{numberate} operator. Further, instead of adding $sprIdx$ fields to specify the order, this
can be done directly on the $index$ fields. A version of rule \ref{rule:trans:TD:seqConstr} can therefore be put
like this:
\begin{equation}
\frac{\left\{e_{1},\ldots,e_{n}\right\}\subset{Singletons}}{e_{1}\mbox{\texttt{, \ldots, }}e_{n}}\longmapsto
\begin{array}{l}
\mbox{\textsf{union(}} \\ \quad 
\mbox{\textsf{project(index=1, value; }} \\ \quad \quad 
\mbox{\textbf{t(r(}}e_{1}\mbox{\textbf{), }}\vartheta\mbox{\textbf{)}\textsf{);}} \\ \quad 
\qquad\vdots\\ \quad 
\mbox{\textsf{project(index=}\textit{n}\textsf{, value; }} \\ \quad \quad 
\mbox{\textbf{t(r(}}e_{n}\mbox{\textbf{), }}\vartheta\mbox{\textbf{)}\textsf{))}}
\end{array}
\label{rule:trans:TD:seqIfSingl}
\end{equation}
Where $\vartheta=e_{1}.\vartheta \cup \ldots \cup e_{n}.\vartheta$.

If a sequence construction expression have only literal subexpressions, the translation may be even more
simplified. As the rule stands now, the sequence will be built by splicing together single-tuple relations with an
\textsf{union} operator. The \textsf{make} operator does however support multiple items, so a better solution
would be to collect all items in one MQL operator:

\begin{equation}
\frac{\left\{e_{1},\ldots,e_{n}\right\}\subset{Literals}}{e_{1}\mbox{\texttt{, \ldots, }}e_{n}}\longmapsto
\begin{array}{l}
\mbox{\textsf{make(name:=["index", "value"],}} \\ \quad\quad\quad
\mbox{\textsf{[1, \ldots, n], [}}e_{1}\mbox{\textsf{, \ldots, }}e_{n}\mbox{\textsf{])}}
\end{array}
\label{rule:trans:TD:seqIfLit}
\end{equation}

These rules may even be combined, as literals are also singleton sequences. If all the items in the soon to 
be sequence are singletons, all singletons which are literals as well can be inserted into a relation with
the same \textsf{make} operator. The $index$ value of the items will have to be according to their relative 
position within the sequence construction expression. Following is an example of a sequence construction with
only singleton items where not all of them are literals:
\begin{equation*}
\mbox{\texttt{('a', 'b', \$b, 'c')}}\longmapsto
\begin{array}{l}
\mbox{\texttt{union(}} \\ \quad
\mbox{\texttt{project(index=3, value, bnumb;}} \\ \quad \quad
\mbox{\textbf{r(}}\texttt{\$b}\mbox{\textbf{)}\textsf{)}} \\ \quad
\mbox{\textbf{t(}}
\mbox{\textsf{make(name:=["index", "value"],}} \\ \quad \quad \quad \quad \quad 
\mbox{\textsf{[1, 2, 4],['a', 'b', 'c'])}\textbf{,
}}\{\mbox{\texttt{b}}\}\mbox{\textbf{)}\textsf{)}}
\end{array}
\end{equation*}
      

\subsection{Path Expressions}
\label{sect:trans:TD:simpl:pathExpr}
The \textsf{scope} operator of MQL can be used to filter tuples based on the value of their $scope$-field. The
operator allows only complete scope descriptions, that is, no wildcards are allowed. \textsf{/} separates the
scopes, and can be read as ``encompasses'' or ``is the parent scope of''. E.g. \textsf{a/b} is read as the scope
where \textsf{a} is the parent scope of \textsf{b}. This can be exploited when translating path expressions with
subsequent \texttt{child} axis or \texttt{parent} axis steps. Multiple subsequent \texttt{child} axis + name test
steps can be translated like this:

\begin{equation}
\begin{array}{c}
\frac{}{\displaystyle e_{1}\texttt{/child::}QName_1\texttt{/child::\ldots/child::}QName_n} \\ 
\longmapsto \begin{array}{c}\mbox{\tiny } \\ \mbox{\tiny } \end{array} \\
\begin{array}{l}
\mbox{\textsf{numberate(index, [sprIdx, index], [}}\vartheta\mbox{\textsf{];}} \\ \quad
\mbox{\textsf{select(isFUNC(scope, lsc);}} \\ \quad \quad
\mbox{\textsf{hhjoin([documentId],[documentId],[spridx=l.index,lsc=l.scope,r.value,}}\vartheta\mbox{\textsf{];}}\\ \quad \quad \quad 
\mbox{\textbf{r(}}e_1\mbox{\textbf{)}\textsf{;}} \\ \quad \quad\quad
\mbox{\textsf{numberate(index, [], [];}}\\ \quad \quad\quad\quad
\mbox{\textsf{index(valocc;}} \\ \quad\quad\quad\quad\quad
\mbox{\textsf{scope(}}QName_1\mbox{\textsf{/\ldots/}}QName_n\\ \quad\quad\quad\quad\quad\quad
\mbox{\textsf{lookup(\$}}QName_n\mbox{\textsf{)))))))}}
\end{array}
\end{array}
\label{rule:trans:TD:pathChild}
\end{equation}

Multiple \texttt{parent} axis steps can recieve corresponding treatment, except the path defined to the
\textsf{scope} operator will have to be reversed. Only clean axis + name test steps can be translated
like this, without any interruption by e.g. a predicate or kind test.

In rule \ref{rule:trans:TD:pathExpr}, the last step is renumbered by a \textsf{numberate} operator. If the last
step does not include some kind of filtering, e.g. in form of a predicate, the operator can be exchanged with a
\textsf{project} operator like this: \textsf{project(index = dotNumb, value, }$\vartheta$\textsf{;\ldots}.

\subsection{Arithmetic Expressions}
\label{sect:trans:TD:simpl:arith}

Rule \ref{rule:trans:TD:unaryMin} describes a translation of the unary \texttt{-} operator. If there is multiple
consecutive unary operators, there is no need to apply the translation rule the same amount of times. The
translator can count the number of unary \texttt{-} operators assigned to one expression. If the number of
operators is odd, the rule is applied, if it is even, no translation is needed.

	\begin{itemize}
      \item singleton inn i and/or -> slippe select (bull?)
    \end{itemize}



\subsection{FLWORs}
\label{sect:trans:TD:simpl:flwor}        
  FLWOR:
  	\begin{itemize}
        \item  kan bli -> project , ved singelton return
      \end{itemize}



\section{Summary}
\label{sect:trans:summary}

In this chapter we have presented Tainting Dependencies -- a method for
translating XQuery expressions into MQL relational algebra trees. Some of the
base concepts behind the method is iterator dependencies and interator dependency tainting. An expression dependent
on an iterator will have a relational representation for each iteration of that iterator. This
dependency can taint another expression, if that expression is a subexpression
of an expression whose evaluation requires representation for all iterations.
We have presented methods for translating features of XQuery complying to the
implications of tainting and dependencies. Finally, we presented some possible
simplifications of trees generated with TD. In section
\ref{sect:disc:notSupported} we will discuss translation of XQuery features
not presented in this chapter.
