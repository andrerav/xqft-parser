\section{Loop Lifting}
\label{sect:translation:loop_lifting}
Loop lifting is a method of translating XQuery iteration expressions into relational algebra. The method was
developed by Torsten Grust and Jens Teubner and originally presented in \cite{pathfinder_mothertongue}. It is a
part of the Pathfinder project\cite{pathfinderHome} (see section \ref{sect:theory:pathfinder}).

In this section we will present Loop Lifting mainly based on the two articles \cite{pathfinder_mothertongue} and
\cite{pathfinder_purelyRelational}. The articles present the method for a subset of XQuery Core (Pathfinder
rewrites queries to Core, see section \ref{sect:theory:pathfinder}), of which we will only present the elements
relevant in a comparison between Loop Lifting and \underline{\textbf{\Large TODO:}} ``insertName''. Thus, the
translation of path expressions and XML-element construction will not be handled, as pathfinder's XML-tree
representation(section \ref{sect:theory:pathfinder}) is incompatible with Mars.

\begin{itemize}
  \item forutsetter det \aa~ jobbe p\aa~ DAGz?
\end{itemize}

\subsection{Operators}
\label{sect:translation:ll:Operators}
Loop-lifting utilises a set of relational algebra operators, out of which the ones used in this chapter is
presented in table \ref{tab:translation:llOperators}.
\begin{table}[h]
\centering
\begin{tabular}{l|l} 
$\pi_{a_{1}:b_{1},\ldots,a_{n}:b_{n}}$ 	& projection and renaming	\\ 	\hline
$\sigma_{a}$					   		& selection             	\\ 	\hline
$\dot\cup$ 							& disjoint union			\\	\hline
$\times$								& cartesian product			\\	\hline
$\bowtie_{a=b}$							& equi-join					\\ 	\hline
$\varrho_{b:(a_{1},\ldots,a_{n})/p}$	& numbering operator		\\	\hline
$\circledcirc_{b:(a_{1},\ldots,a_{n})}$	& $n$-ary arithmetic/comparison operator $\circ$ \\ \hline
\scriptsize \begin{tabular}{c|c} $a$& $b$\\\hline\end{tabular} & literal table
\end{tabular}
\caption[of the Pathfinder relational algebra]{Operators of the Pathfinder relational algebra. $a$, $b$ and $p$
represents attributes}
\label{tab:translation:llOperators}
\end{table}

Most of the operators are quite standard, and can easily be understood by comparing with the operators from
section \ref{sect:theory:relAlg} and \ref{sect:method:marsOperators}.

The selection operator ($\sigma_{a}$) selects tuples where attribute $a \ne 0$. Only a very restricted selection
is used, written $\sigma_{a}$, which only returns tuples satisfying $a \neq 0$.  Considering the numbering
operator, $p$ denotes the partitioning attribute, $(a_{1},\ldots,a_{n}$ the attributes to be sorted on and $b$ is
an added attribute holding the result of the
numbering. {\scriptsize{\begin{tabular}{c|c}$a$&$b$\\\hline\end{tabular}}}  represents the creation
of a relation with attributes $a$ and $b$.


\subsection{Basics}
\label{sect:translation:ll:Basics}
XQuery expressions evaluate to finite, ordered sequences of items. As a sequences are one-dimensional, it can be
represented by a single relation where each tuple encodes a sequence item. The order of the sequence is
maintained by an attribute \textit{pos}. The value of the item is held in an attribute \textit{item}. A sequence
such as \texttt{('a', 'b', 'c')} will be represented relationally like this:

\begin{center}
\begin{tabular}{|c|c|}\hline
\textit{pos}	& \textit{item} 	\\ \hline
1				& \texttt{'a'}		\\ \hline
2				& \texttt{'b'}		\\ \hline
3				& \texttt{'c'}		\\ \hline
\end{tabular}
\end{center}

For the rest of this chapter about Pathfinder's loop lifting, variables, expressions and scopes is denoted like
this (ref. section \ref{sect:theory:xquery:Flwor}:
\[
s \left\{
\begin{array}{l}
\qquad \qquad \quad \vdots \\
\mbox{\texttt{for \$}}v_{0}\mbox{\texttt{ in }} e_{0} \mbox{\texttt{ return}} \\
\quad s_{0} \left\{ e_{0}' \right. \\
\qquad \qquad \quad \vdots
\end{array}
\right.
\]

More generally, a scope $s_{x \cdot y}$ identifies the $y$th child scope of scope $x$, $x \in \left\{ 0,1,\ldots
\right\}, y \in \left\{ 0,1,\ldots \right\}$. Expression $e_{x\cdot y}$ evaluates (in scope $s_{x}$) to an
iterator sequence and is successively bound to the variable $v_{x \cdot y}$. $e_{x \cdot y}'$ constitutes the
coresponding iterator body, and $I_{x \cdot y}$ the whole iterator expression.

$q_{x}(e)$ is used to denote the relational representation of expression $e$ in scope $s_{x}$.


\subsection{Constant Subexpressions}
\label{sect:translation:ll:ConstExprs}

For a iterator expression $i_{x}$ with $n$ iterations there exists a relation $loop_{x}$, consisting of a
single column, \textit{iter}, with values 1,2,\ldots,$n$. In the outermost scope, $loop$ has a single tuple with
value 1.

How a constant value $c$ in scope $s_{x}$ is \textit{lifted} is shown in rule \ref{rule:ll:constant}.

\begin{equation}
\frac{}{c \Longrightarrow loop_{x} \times \mbox{\scriptsize \begin{tabular}{c|c} \textit{pos}&\textit{item} \\
\hline 1 & \textit{c}
\end{tabular}}}
\label{rule:ll:constant}
\end{equation}

A tuple ($iter,pos,item$) in a loop lifted relation for subexpression $expr$ can be understood as that during the
$iter$th iteration, the item in position $pos$ in $expr$ has the value $item$.

\subsection{Bound Variables}
\label{sect:translation:ll:boundVar}
\begin{itemize}
  \item her er vel hele sullamitten
\end{itemize}

\subsection{Fre Variables}
\label{sect:translation:ll:freeVar}
\begin{itemize}
  \item her er vel hele sullamitten
\end{itemize}

\subsection{Other Expression Types}
\label{sect:translation:ll:OtherExpr}
\begin{itemize}
  \item sequence construction
  \item variable binding/usage
  \item plus / generic operator plz?
\end{itemize}

\subsection{Optimisations}
\label{sect:translation:ll:Optimisations}
\begin{itemize}
  \item peep hole plan simplification
  \item dette kan fort bli dr\o yt\ldots v\ae re litt forsiktige her tror jeg
\end{itemize}

