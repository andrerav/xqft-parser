\section{MarkXRemove}
\label{sect:translation:MarkXRemove}

Our original proposition to a method for translating XQuery ASTs into relational algebra was named MarkXRemove.
The rationale behind this name will be explained later. Eventhough it has many shortcomings and flaws, we
will in this section give a quick run through of the method. This is because when we in the next section present
``INSERTNAMEHERE'', which is an evolution and a refinement of MarkXRemove, the new method may be easier understood
when seen in the perspective of its origins. Another reason is that in case of further development of the
``INSERTNAMEHERE'', it may be of help to also know what will \textit{not} work, what will work partially and
\textit{why} it is flawed.

\subsection{Basics}
\label{sect:translation:mxr:basics}

The foundation of the method is that an iterator expression is always translated by calculating the cartesian
product of the iterator sequence and the iterator body, hence the ``X'' in the name. The ``remove'' stems from the
removing of tuples who ended up in the wrong iteration in the cross product result. The cartesian product and the
selection of tuples afterwards actually constitutes a kind of natural join (see section
\ref{sect:theory:relAlg:naturalJoin}) as we will see later.

As the translator comes across an iterator variable declaration, with the variable name $\Beta$, it will augment
the representation of the iterator sequence belonging to this variable with an attribute $\Beta$\textsf{numb}.
This new attribute will hold consecutive values from 1 to $n$ for a $n$ item long sequence, which will symbolise the
iteration number of the iterator expression seen isolated from possible other surrounding iterator expressions. A
function \texttt{counter()} returning the row number of a relation and a \texttt{project} operator will handle
the augmentation. The corresponding algebra tree is be stored in the symbol table.

If the translator later comes accross a reference to this variable, it will get the tree from the symbol table and
return it to the referring node without any further ado.
