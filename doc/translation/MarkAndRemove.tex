\section{MarkXRemove}
\label{sect:trans:MarkXRemove}

Our original proposition to a method for translating XQuery ASTs into relational algebra was named MarkXRemove.
Even though it has many
shortcomings and flaws, we will in this section give a quick overview of the
method. This is because the Tainting Dependencies method is an evolution
and a refinement of MarkXRemove, and may be easier understood when seen in the perspective of its origins. Another reason is that in case of further development of TD, it may be of help to also know what will
\textit{not} work, what will work partially and \textit{why} it is flawed.

\subsection{Basics}
\label{sect:trans:mxr:basics}

The foundation of the method is that an iterator expression is always translated by calculating the cartesian
product of the iterator sequence and the iterator body, hence the ``X'' in the name. ``Remove'' stems from the
removing of tuples who ends up in the wrong iteration in the cross product
result. The cartesian product and the selection of tuples afterwards actually constitutes a kind of natural join (section
\ref{sect:theory:relAlg:naturalJoin}) as we will see later.

As the translator comes across an iterator variable declaration with the variable name $\chi$, it will augment
the relational representation of the iterator sequence belonging to this variable with an attribute
$\chi$\textit{numb}. This new attribute will hold consecutive values from 1 to $n$ for a $n$ item long sequence.
These values symbolise the iteration number of the iterator expression seen isolated from possible other
surrounding iterator expressions. A function \textsf{counter()} returning the row number of a relation and a
\textsf{project} operator will handle the augmentation. The corresponding algebra tree is stored in the symbol
table. The ``mark'' of the name of the method is because of this augmentation.

\subsection{FLWOR}
\label{sect:trans:mxr:flwor}
If the translator later comes across a reference to an iterator variable $\chi$,
it will get the tree from the symbol table and return it to the referring AST node without any further ado. The
translator is also required to know which subtrees have a child that has referred to which iterator variable. This
is because the $\chi$\textit{numb} attribute could be lost in a \textsf{project} operation without this knowledge.

When the translator returns to the iterator expression node for the variable $\chi$ after traversing the iterator
body, it will, as mentioned before, make sure that the cartesian product between the iterator sequence and body is
calculated. From the result of this, the tuples where the $\chi$\textit{numb} stemming from the iterator
sequence does not line up with the $\chi$\textit{numb} stemming from the body are removed. Any tuple with a
$\chi$\textit{numb} value $NULL$ will be kept.

A $NULL$ value in the $\chi$\textit{numb} field of a tuple means that this tuple is not marked, i.e.\ it is
not dependent on which iteration the $\chi$ iterator expression is in. Unmarked
tuples can e.g.\ stem from the creation of a sequence. MarkXRemove translates sequence construction expressions,
e.g.\ \texttt{(}$e_{1}$\texttt{, }$e_{1}$\texttt{)}, to a simple disjoint union,
\textbf{r(}$e_{1}$\textbf{)}$\;\dot\cup\;$\textbf{r(}$e_{2}$\textbf{)}. Where \textbf{r()} symbolises a function
translating XQuery expressions into relational algebra.

The method creates quite simple algebra, as exemplified by the following query:

\begin{center}
\begin{tabular}{l}
\texttt{for \$i in (1,2,3) return } \\ \qquad
\texttt{(\$i, 'yes')}
\end{tabular}
\end{center}

which is translated into this algebra:
\begin{center}
\begin{tabular}{l}
\textsf{select(ifThenElse(isNull(value), true, eq(r.inumb,l.inumb))} \\ \quad
\textsf{cross(} \\\quad\quad
\textsf{project(inumb = counter(), value;}\\\quad\quad\quad
\textsf{make(name:=["value"], [1,2,3]))} \\\quad\quad
\textsf{union(} \\\quad\quad\quad
\textsf{project(inumb = counter(), value;} \\\quad\quad\quad\quad
\textsf{make(name:=["value"], [1,2,3]))} \\\quad\quad\quad
\textsf{make(name:=["value"], ['yes']))))}
\end{tabular}
\end{center}

\subsection{Flaws}
\label{sect:trans:mxr:flaws}
The main problem with MarkXRemove is that it requires a particular ordering of
tuples in a relation which is a result of a cartesian product. Not only can
the MQL \textsf{cross} operator not promise the particular ordering of its
result, it can not promise any particular ordering at all. The ordering the
method requires is that for each tuple in the left relation, the tuple is
repeated for all tuples in the right relation. If this requirement is not met
any item may appear anywhere in the resulting sequence, which is not
acceptable for evaluation of XQuery expressions where all sequences are
ordered.

Another problem with this method is that any sequence built which includes a reference to an iterator variable is
not fully calculated until the cartesian product between the corresponding relation and the variables iterator
sequence is evaluated. This makes it hard to evaluate expressions where such a sequence is a subexpression. The
\textit{iterator body} of this query:

\begin{center}
\begin{tabular}{l}
\texttt{for \$i in (5,10) return } \\ \qquad
\texttt{(\$i, 8) > (6,12)}
\end{tabular}
\end{center}

would be translated into this relational algebra:

\begin{center}
\begin{tabular}{l}
\textsf{project(value = gt(l.value, r.value), inumb} \\ \quad
\textsf{cross(} \\ \quad\quad
\textsf{union(} \\ \quad\quad\quad
\textsf{project(inumb = counter(), value;} \\\quad\quad\quad\quad
\textsf{make(name:=["value"], [5,10]))} \\ \quad\quad\quad
\textsf{make(name:=["value"], [8]))} \\ \quad\quad
\textsf{make(name:=["value"], [3,12])))}
\end{tabular}
\end{center}

which again would produce this relation:

\begin{figure*}[!h]
\centering
\begin{tabular}{|c|c|} \hline
$inumb$ & $value$ \\\hline
1 & \textit{false} \\\hline
1 & \textit{false} \\\hline
2 & \textit{true} \\\hline
2 & \textit{false} \\\hline
\textit{NULL} & \textit{true} \\\hline
\textit{NULL} & \textit{false} \\\hline
\end{tabular}
\end{figure*}

The query should of course be evaluated to $(true, true)$, as the \texttt{>} operator in XQuery yields $true$ if
\textit{one} item in the left operand is larger than \textit{one} item in the right.
This means that the relation would have to be pruned by a \textsf{select} or \textsf{group}, which can not be done
generally in the relations current state. A possibility would be to postpone the pruning until after the relation
is crossed with the iterator sequence, but there would still not be any trivial
solution. This problem leads to the introduction of iterator dependency
tainting, as we will see in Section \ref{sect:trans:TD:tainting}.
