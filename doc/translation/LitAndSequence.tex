\section{Sequence Construction}
\label{sect:trans:TD:seqBuild}

A sequence in XQuery can be built with the comma operator --\texttt{,}. But this operator is the XQuery operator
with the lowest precedence, therefore, in most cases a sequence construction expression will be enclosed in
paratheses. This is to solve parser ambiguities, which can be seen in the exerpt of the W3C XQuery Full Text EBNF
specification\cite{w3c01} in figure \ref{fig:trans:TD:seqEBNF}. An \texttt{ExprSingle} can solely consist of a
\texttt{ParenthesizedExpr} via a series of productions omitted from the figure. Also note a \texttt{ExprSingle}
can be a \texttt{FLWORExpr}.

\begin{figure}[h]
\begin{Verbatim}
[33] FLWORExpr         ::= (ForClause | LetClause)+ WhereClause? 
                               OrderByClause? "return" ExprSingle
[45] IfExpr            ::= "if" "(" Expr ")" "then" ExprSingle 
                               "else" ExprSingle
[31] Expr              ::= ExprSingle ("," ExprSingle)*
[89] ParenthesizedExpr ::= "(" Expr? ")"
\end{Verbatim}
\caption[Exerpt from W3C XQuery EBNF]{Exerpt from W3C XQuery EBNF showing sequence construction}
\label{fig:trans:TD:seqEBNF}
\end{figure}

As also can be seen from the figure the for clause of a FLWOR expression, as many other expressions, accepts an
\texttt{ExprSingle}. If a sequence is to be constructed in the for clause, it will have to be parenthesised. 

With the concept of tainting and iterator dependencies explained, we are now ready to introduce the translation of
an XQuery sequence construction expression:

\begin{equation}
\frac{}{e_{1}\mbox{\texttt{, \ldots, }}e_{n}}\longmapsto
\begin{array}{l}
\mbox{\textsf{numberate(index, [sprIdx, index], [}}\vartheta\mbox{\textsf{];}} \\ \quad
\mbox{\textsf{union(}} \\ \quad \quad
\mbox{\textsf{project(sprIdx=1, index, value; }} \\ \quad \quad \quad
\mbox{\textbf{t(r(}}e_{1}\mbox{\textbf{), }}\vartheta\mbox{\textbf{)}\textsf{);}} \\ \quad \quad
\qquad\vdots\\ \quad \quad
\mbox{\textsf{project(sprIdx=}\textit{n}\textsf{, index, value; }} \\ \quad \quad \quad
\mbox{\textbf{t(r(}}e_{n}\mbox{\textbf{), }}\vartheta\mbox{\textbf{)}\textsf{)))}}
\end{array}
\label{rule:trans:TD:seqConstr}
\end{equation}
Where $\vartheta=e_{1}.\vartheta \cup \ldots \cup e_{n}.\vartheta$.

The basis of a sequence construction is the \textsf{union} operator -- as with MarkXRemove. But because we in
Tainted Dependencies have introduced the explicit ordering of items with the $index$ attribute, additional
operations have been added. Each item in the sequences returned from the subexpressions is equipped with a
temporary field $sprIdx$ (superindex) holding the relative position of each subexpression. Based on the
positioning defined by $sprIdx$ and $index$ the \textsf{numberate} operator can now renumberate the resulting
sequence. The numbering must partition on the fields corresponding to the dependencies in $\vartheta$, to separate
the different sequences constructed for all dependent iterations.

\begin{myExample}
\label{ex:trans:TD:simpleSeq}
\begin{figure}[h]
\begin{equation*}
\begin{array}{l}
\mbox{\texttt{for \$a in (10,20) return}} \\ \quad 
\underbrace{ \mbox{\texttt{(\$a, "no")}} }_{e_{1}}
\end{array}
\end{equation*}
\caption{Simple XQuery query}
\label{fig:trans:TD:simpQuery}
\end{figure}
Consider the simple XQuery query of figure \ref{fig:trans:TD:simpQuery}. Here \textbf{r(}\texttt{\$a}\textbf{)}
will taint \textbf{r(}\texttt{"no"}\textbf{)} with its dependency on the iterator $I_{a}$, the result of which is shown in figure \ref{fig:trans:TD:simpl:ryes}. Further, we can see that
for each iteration of $I_{a}$ the return clause will return a sequence of two items. Having in mind that $anumb$
($anb$ in figure) holds the iteration number of $I_{a}$, this can be seen in figure \ref{fig:trans:TD:simpl:rall}.

\begin{figure}[!h]
\centering
\subfigure[\textbf{r(}\texttt{\$a}\textbf{)}]{
\begin{tabular}{|c|c|c|} \hline
$anb$ & $idx$ & $val$ \\ \hline
1 & 1 & 10 \\ \hline
2 & 1 & 20 \\ \hline
\end{tabular}
\label{fig:trans:TD:simpl:ra}
}
\qquad
\subfigure[\textbf{t(r(}\texttt{"no"}\textbf{),}\textbf{r(}\texttt{\$a}\textbf{)}$.\vartheta$\textbf{)}]{
\begin{tabular}{|c|c|c|} \hline
$anb$ & $idx$ & $val$ \\ \hline
1 & 1 & \texttt{"no"} \\ \hline
2 & 1 & \texttt{"no"} \\ \hline
\end{tabular}
\label{fig:trans:TD:simpl:ryes}
}
\qquad
\subfigure[\textbf{r(}\texttt{(\$a, "no")}\textbf{)}]{
\begin{tabular}{|c|c|c|} \hline
$anb$ & $idx$ & $val$ \\ \hline
1 & 1 & 10 \\ \hline
2 & 1 & 20 \\ \hline
1 & 2 & \texttt{"no"} \\ \hline
2 & 2 & \texttt{"no"} \\ \hline
\end{tabular}
\label{fig:trans:TD:simpl:rall}
}

\caption[Example: constructing a sequence]{Applying translation rule \ref{rule:trans:TD:seqConstr} on (a) and (b)
yields (c). Attribute names are shortened \label{fig:trans:TD:simpleSeq}}
\end{figure}

The sequence constructing rule also holds even if the subexpressions of the soon-to-be sequence are within the
body of an iterator the sequence is not dependant on. Expanding the query of figure \ref{fig:trans:TD:simpQuery}
we get the query of figure \ref{fig:trans:TD:expandQuery}. 
\begin{figure}[h]
\begin{equation*}
\begin{array}{l}
\mbox{\texttt{for \$a in (10,20) return}} \\ \quad
\mbox{\texttt{for \$b in (50,75) return}} \\ \quad \quad
\underbrace{ \mbox{\texttt{(\$a, "no")}} }_{e_{1}}
\end{array}
\end{equation*}
\caption{Query of figure \ref{fig:trans:TD:simpQuery} expanded}
\label{fig:trans:TD:expandQuery}
\end{figure}
In this query, notice the innermost return clause expression, $e_{1}$, is identical to $e_{1}$ in the previous
query. Here, the result of the sequence construction will still be the relation shown in figure
\ref{fig:trans:TD:simpl:rall}, because $e_{1}.\vartheta=\left\{a\right\}$ -- also as before. $e_{1}$ is not aware
of the iterator $I_{b}$ -- and does not need to be either, as the result of $e_{1}$ is independent of which
iteration number $I_{b}$ is in.

\end{myExample}
