\section{Inference rule language explanation}
\label{sect:trans:TD:langExpl}
During this chapter we will present some inference rules. Table \ref{tab:trans:td:langExpl} explains some of the
the various typographical representations.

\begin{table}[h]
\centering
\begin{tabular}{l|l}

  $\longmapsto$  			& Translates into \\
  $\vartheta$ 				& A set of iterator dependencies \\
  \textsf{sans serif} 		& MQL expressions \\
  \texttt{monospaced} 		& XQuery expressions \\
  $e,\ldots,e_{n}$			& Generic expressions \\
  $\chi,\ldots,\chi_{n}$	& Generic variable names \\
  $I_{\chi}$				& The iterator expression which declares $\chi$ \\
  \textbf{bold} 			& Operations done during the generation of the algebra \\
  \textbf{r(}$e$\textbf{)} 	& Returns the relational algebraic representation of $e$   \\
  \textbf{t(}\textbf{r(}$e$\textbf{)}\textbf{,}$\vartheta'$\textbf{)} & Returns \textbf{r(}$e$\textbf{)} tainted
  with the dependencies $\vartheta'$ \\
  \textbf{B(r(}$e$\textbf{))} & Returns the effective boolean value of \textbf{r(}$e$\textbf{)}.
  
\end{tabular}
\caption{Explanation of inference rule symbols}
\label{tab:trans:td:langExpl}
\end{table}

Inference rules are generally in this format:
\begin{equation*}
\frac{condition}{e}\longmapsto \mbox{\textbf{r(}}e\mbox{\textbf{)}}
\end{equation*}

This should be read as: if condition $condition$ is satisfied, the XQuery expression $e$ will be translated into
\textbf{r(}$e$\textbf{)}.

Often MQL operator trees are depicted like this:
\begin{center}
\begin{tabular}{l}
\textsf{operator1(\ldots,l.attr, r.attr\ldots; } \\ \quad
\textsf{operator2(\ldots);} \\ \quad
\textsf{operator3(\ldots))}
\end{tabular}
\end{center}

This is to be interpreted as ``\textsf{operator2} is the left child of \textsf{operator1} and \textsf{operator3} is
the right child''. MARS does not allow attribute names to contain punctuation marks or allow two attributes with the
same name within one relation. An operator combining two relations will therefore have renaming functionality. A
typical projection list of such an operator combining two relations which both contain the attribute $attr$ would
look something like: \textsf{\ldots rattr=right.attr, lattr=left.attr\ldots}. To make the inference rules
easier to read, this step has been dropped. The rules assume that the equal
named fields will automatically be given a prefix \textsf{l.} (left) or
\textsf{r.} (right) depending on which child the attribute stems from.

We assume the \textsf{union}-operator accepts relations with different schemas. The schema for the result
relation can be described as:
\begin{equation*}
schema(\textsf{union(}\alpha, \beta\texttt{)}) = schema(\alpha) \cup schema(\beta).
\end{equation*}

The tuples which have more fields in the result relation than they did in the relation they stem from will
have a \textit{NULL} value for the introduced attributes. It may be not be
desirable to implement the operator in such a way. In that case each
child-relation will have to be augmented with the attributes needed with an
\textsf{project} operator. The effective boolean value of \textbf{B}$()$ is
described in Section \ref{sect:trans:TD:implic}. 
