\section{Simplifications}
\label{sect:trans:TD:simplifications}

In this section we will present some possible simplifications discovered during
the development of the Tainting Dependencies method. The $\Rightarrow$ sign is
to be read as ``can be written as''.

\subsection{Literals}
\label{sect:trans:TD:simpl:lit}      

Rule \ref{rule:trans:TD:literal} of section \ref{sect:trans:TD:literal} shows a very general way to translate
XQuery literals to a relational format. But creating one relation for each literal is very often unnecessary, and
often quite a bit more resource consuming than alternative solutions. The parent expression of a litteral should
in most cases be informed that its subexpression is a literal instead of being handed a one-tuple relation.

One such case is if the literal will be used in a join (predicate-less) or cartesian product. A better solution
will then be to project the literal into the other relation. Following is an
example of how such an expression should be written:
\begin{equation}
\begin{array}{lcr}
\begin{array}{l}
\mbox{\quad\quad\textsf{\vdots}} \\
\mbox{\textsf{hhjoin([],[],\ldots}} \\ \quad
\mbox{\textsf{\ldots}\textbf{r(}}e\mbox{\textbf{)\ldots}} \\ \quad
\mbox{\textbf{r(}}Literal\mbox{\textbf{)}} \\
\mbox{\quad\quad\textsf{\vdots}}
\end{array}
&
\Rightarrow
&
\begin{array}{l}
\mbox{\quad\quad\textsf{\vdots}} \\
\mbox{\textsf{project(\ldots, rvalue=}}Literal\mbox{\textsf{\ldots}} \\ \quad
\mbox{\textsf{\ldots}\textbf{r(}}e\mbox{\textbf{)\ldots}} \\
\mbox{\quad\quad\textsf{\vdots}}
\end{array}
\end{array}
\label{simpl:trans:TD:joinLit}
\end{equation}

If the reason for the join with the relation was because the literal was part of
a arithmetic, comparison or logical expression, the literal may be moved inside the \textsf{project} operator responsible for executing the
binary operation, shortening the relational algebra even more:

\begin{equation}
\begin{array}{lcr}
\begin{array}{l}
\mbox{\textsf{project(val=OP(l.val, r.val)\ldots}} \\ \quad
\mbox{\textsf{hhjoin([],[],\ldots}} \\ \quad \quad
\mbox{\textsf{\ldots}\textbf{r(}}e\mbox{\textbf{)\ldots}} \\ \quad \quad
\mbox{\textbf{r(}}Literal\mbox{\textbf{)}} \\
\mbox{\quad\quad\textsf{\vdots}}
\end{array}
&
\Rightarrow
&
\begin{array}{l}
\mbox{\textsf{project(val=OP(val, }}Literal\mbox{\textsf{)\ldots}} \\ \quad
\mbox{\textsf{\ldots}\textbf{r(}}e\mbox{\textbf{)\ldots}} \\
\mbox{\quad\quad\textsf{\vdots}}
\end{array}
\end{array}
\label{simpl:trans:TD:binLit}
\end{equation}

If the cases where a literal will have to be translated to a single-tuple relation, in most cases the $index$
attribute will not be needed. But this will probably be detected by the
optimiser, and may be left out anyway.

\subsection{Sequence Construction}
\label{sect:trans:TD:simpl:seq}

Informing a parent expression about whether or not its subexpressions will
evaluate to singleton sequences can have some advantages. One is the possibility of detecting certain type errors, as will be
discussed in section \ref{sect:disc:singelton}. Another advantage is gained
when it comes to sequence construction:

Rule \ref{rule:trans:TD:seqConstr} in section \ref{sect:trans:TD:seqBuild} describes a general way to build
sequences. But if all the expressions to build a sequence from will evaluate to singleton sequences, there is no
need for the \textsf{numberate} operator. Further, instead of adding $sprIdx$ fields to specify the order, this
can be done directly on the $index$ fields. A version of rule \ref{rule:trans:TD:seqConstr} can therefore be put
like this:
\begin{equation}
\frac{\left\{e_{1},\ldots,e_{n}\right\}\subset{Singletons}}{e_{1}\mbox{\texttt{, \ldots, }}e_{n}}\longmapsto
\begin{array}{l}
\mbox{\textsf{union(}} \\ \quad 
\mbox{\textsf{project(index=1, value; }} \\ \quad \quad 
\mbox{\textbf{t(r(}}e_{1}\mbox{\textbf{), }}\vartheta\mbox{\textbf{)}\textsf{);}} \\ \quad 
\qquad\vdots\\ \quad 
\mbox{\textsf{project(index=}\textit{n}\textsf{, value; }} \\ \quad \quad 
\mbox{\textbf{t(r(}}e_{n}\mbox{\textbf{), }}\vartheta\mbox{\textbf{)}\textsf{))}}
\end{array}
\label{rule:trans:TD:seqIfSingl}
\end{equation}
Where $\vartheta=e_{1}.\vartheta \cup \ldots \cup e_{n}.\vartheta$.

If a sequence construction expression have only literal subexpressions, the translation may be even more
simplified. As the rule stands now, the sequence will be built by splicing together single-tuple relations with an
\textsf{union} operator. The \textsf{make} operator does however support multiple items, so a better solution
would be to collect all items in one MQL operator:

\begin{equation}
\frac{\left\{e_{1},\ldots,e_{n}\right\}\subset{Literals}}{e_{1}\mbox{\texttt{, \ldots, }}e_{n}}\longmapsto
\begin{array}{l}
\mbox{\textsf{make(name:=["index", "value"],}} \\ \quad\quad\quad
\mbox{\textsf{[1, \ldots, n], [}}e_{1}\mbox{\textsf{, \ldots, }}e_{n}\mbox{\textsf{])}}
\end{array}
\label{rule:trans:TD:seqIfLit}
\end{equation}

These rules may even be combined, as literals are also singleton sequences. If all the items in the soon to 
be sequence are singletons, all singletons which are literals as well can be inserted into a relation with
the same \textsf{make} operator. The $index$ value of the items will have to be according to their relative 
position within the sequence construction expression. Following is an example of a sequence construction with
only singleton items where not all of them are literals:
\begin{equation*}
\mbox{\texttt{('a', 'b', \$b, 'c')}}\longmapsto
\begin{array}{l}
\mbox{\texttt{union(}} \\ \quad
\mbox{\texttt{project(index=3, value, bnumb;}} \\ \quad \quad
\mbox{\textbf{r(}}\texttt{\$b}\mbox{\textbf{)}\textsf{);}} \\ \quad
\mbox{\textbf{t(}}
\mbox{\textsf{make(name:=["index", "value"],}} \\ \quad \quad \quad \quad \quad 
\mbox{\textsf{[1, 2, 4],['a', 'b', 'c'])}\textbf{,
}}\{\mbox{\texttt{b}}\}\mbox{\textbf{)}\textsf{)}}
\end{array}
\end{equation*}
      

\subsection{Path Expressions}
\label{sect:trans:TD:simpl:pathExpr}
The \textsf{scope} operator of MQL can be used to filter tuples based on the value of their $scope$ field. The
operator allows only complete scope descriptions, that is, no wildcards are allowed. \textsf{/} separates the
scopes, and can be read as ``encompasses'' or ``is the parent scope of''. E.g. \textsf{a/b} is read as the scope
where \textsf{a} is the parent scope of \textsf{b}. This can be exploited when translating path expressions with
subsequent \texttt{child} axis or \texttt{parent} axis steps. Multiple subsequent \texttt{child} axis + name test
steps can be translated like this:

\begin{equation}
\begin{array}{c}
\frac{}{\displaystyle \mbox{\texttt{child::}}QName_1\mbox{\texttt{/child::\ldots/child::}}QName_n} \\ 
\longmapsto \begin{array}{c}\mbox{\tiny } \\ \mbox{\tiny } \end{array} \\
\begin{array}{l}
\mbox{\textsf{project(docId, index, value, pos, scope, }}\vartheta\mbox{\textsf{;}} \\ \quad
\mbox{\textsf{numberate(dotNumb, [dotNumb, subIdx], [}}(\vartheta-\mbox{\textbf{dot}})\mbox{\textsf{];}} \\
\quad\quad 
\mbox{\textsf{numberate(index, [index], [}}\vartheta\mbox{\textsf{];}} \\ \quad\quad\quad
\mbox{\textsf{select(isChild(scope, lsc);}} \\ \quad \quad\quad\quad
\mbox{\textsf{hhjoin([docId],[docId],}} \\ \quad\quad\quad\quad\quad\quad\quad
\mbox{\textsf{[lsc=l.scope,subIdx=r.index,r.value,}}\vartheta\mbox{\textsf{];}}\\
\quad\quad\quad\quad\quad \mbox{\textbf{get(dot)}\textsf{;}} \\ \quad \quad\quad \quad\quad
\mbox{\textsf{numberate(index, [], [];}}\\ \quad\quad\quad\quad\quad\quad
\mbox{\textsf{index(valocc;}} \\ \quad\quad\quad\quad\quad\quad\quad
\mbox{\textsf{scope(}}QName_1\mbox{\textsf{/\ldots/}}QName_n\mbox{\textsf{;}}\\
\quad\quad\quad\quad\quad\quad\quad\quad
\mbox{\textsf{lookup(\$}}QName_n\mbox{\textsf{)))))))))}}
\end{array}
\end{array}
\label{rule:trans:TD:pathChild}
\end{equation}
Where $\vartheta$ is the $\vartheta$ returned from \textbf{get(dot)}, $QName_x$ is any XML-qualified name.
\textsf{r.value} is short for \textsf{value = right.value, index = right.index, \ldots etc}, and \textsf{docId} is
short for \textsf{documentId}.

Multiple \texttt{parent} axis steps can receive corresponding treatment, except the path defined to the
\textsf{scope} operator will have to be reversed. Only clean axis + name test steps can be translated
like this, without any interruption by e.g. a predicate or kind test.

Rule \ref{rule:trans:TD:pathExpr} shows how whole path expressions are translated. Here, the last step is
renumbered by a \textsf{numberate} operator. If the last step does not include some kind of filtering, e.g. in
form of a predicate, the operator can be exchanged with a \textsf{project} operator like this:
\textsf{project(index = dotNumb, value, }$\vartheta$\textsf{;\ldots}.

The next to last \textsf{numberate} operator in the rules for translating axis + nametest expressions (rules
\ref{rule:trans:TD:pathChild} and \ref{rule:trans:TD:pathStep}) is employed to ensure that numeric predicates
always can be evaluated. If the translator with the help of a type system be make sure that the predicate
expression does not evaluate to a numeric type, or if the step expression does not contain a predicate, this
operator may be dropped.

\subsection{Arithmetic Expressions}
\label{sect:trans:TD:simpl:arith}

Rule \ref{rule:trans:TD:unaryMin} describes a translation of the unary \texttt{-} operator. If there is multiple
consecutive unary operators, there is no need to apply the translation rule the same amount of times. The
translator can count the number of unary \texttt{-} operators assigned to one expression. If the number of
operators is odd, the rule is applied, if it is even, no translation is needed.
