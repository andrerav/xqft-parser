\section{Conditional Expressions}
\label{sect:trans:TD:ifThenElse}
XQuery supports a conditional expression based on the keywords \texttt{if}, \texttt{then}, and \texttt{else}. The
expression is defined by W3C as seen in figure \ref{fig:trans:TD:condEBNF}.

\begin{figure}[h]
\begin{Verbatim}
[45] IfExpr ::= "if" "(" Expr ")" "then" ExprSingle 
                    "else" ExprSingle
\end{Verbatim}
\label{fig:trans:TD:condEBNF}
\caption{W3C EBNF conditional expression specification\cite{w3c00}.}
\end{figure}

The expression following the \texttt{if} keyword is called the test expression, and the expressions following the
\texttt{then} and \texttt{else} keywords are called the \texttt{then}-expression and \texttt{else}-expression,
respectively. If the effective boolean value of the test expression evaluates to $true$, the
\texttt{then}-expression is returned, if it evaluates to $false$ the \texttt{else}-expression is returned.

Conditional expressions are translated by adding an attribute $alt$ with the value $1$ the
\texttt{then}-expression relation and the relational representation of the \texttt{else}-expression with the same
attribute but with value $2$. These two relations are then spliced together with a \textsf{union} operator. If the
relations have disjoint dependencies, they will have to be tainted first.

The result of the \textsf{union} operation is then joined with the relational representation of the test
expression on their common dependencies. Lastly, a \textsf{select} operator is employed on this relation to select
the tuples where $alt$ is $1$ if the $value$ field from the test expression evaluates to $true$, or $alt$ is $2$
if it does not.

\begin{equation}
\begin{array}{c}
\frac{}{\displaystyle \mbox{\texttt{if }}e_1\mbox{\texttt{ then }}e_2\mbox{\texttt{ else }}e_3} \\
\longmapsto \begin{array}{c}\mbox{\tiny } \\ \mbox{\tiny } \end{array} \\
\begin{array}{l}
\mbox{\textsf{project(value = l.value, }}\vartheta\mbox{\textsf{;}} \\ \quad
\mbox{\textsf{select(ifthenelse(xqBoolean(pred), eq(alt,1), eq(alt,2));}} \\ \quad \quad
\mbox{\textsf{hhjoin([}}((e_2.\vartheta \cup e_3.\vartheta)\cap e_1.\vartheta)\mbox{\textsf{],
[}}(e_1.\vartheta\cap (e_2.\vartheta\cup e_3.\vartheta))
\mbox{\textsf{],}} 
\mbox{\textsf{[l.value, pred, }}\vartheta\mbox{\textsf{,
alt];}}\\\quad\quad\quad \mbox{\textsf{union(}} \\ \quad\quad\quad\quad \mbox{\textsf{project(alt=1, value, }}(e_2.\vartheta \cup e_3.\vartheta)\mbox{\textsf{,}}\\\quad\quad\quad\quad\quad
\mbox{\textbf{t(r(}}e_2\mbox{\textbf{), }}e_3.\vartheta\mbox{\textbf{)}\textsf{);}} \\ \quad\quad\quad\quad
\mbox{\textsf{project(alt=2, value, }}(e_3.\vartheta\cup e_2.\vartheta)\mbox{\textsf{,}}\\\quad\quad\quad\quad\quad
\mbox{\textbf{t(r(}}e_3\mbox{\textbf{), }}e_2.\vartheta\mbox{\textbf{)}\textsf{));}}\\\quad\quad\quad
\mbox{\textbf{B(r(}}e_1\mbox{\textbf{)}\textsf{))))}}
\end{array}
\end{array}
\label{rule:trans:TD:conditional}
\end{equation}

Where $\vartheta = e_1.\vartheta \cup e_2.\vartheta \cup e_3.\vartheta$. $index$ and possible $documentId$, $pos$
and $scope$ attributes will follow $value$ as described in \ref{sect:trans:TD:basics}.

\begin{myExample}
\begin{figure}[h]
\centering
\begin{equation*}
\begin{array}{l}
\mbox{\texttt{for \$a in (10, 20) return}} \\ \quad
\mbox{\texttt{for \$b in (5, 15) return}} \\ 
e_1 \left\{\begin{array}{l}
           \mbox{\texttt{if(\$b > \$a) then}} \\ \quad
           \mbox{\texttt{\$a}} \\ 
           \mbox{\texttt{else}} \\ \quad
           \mbox{\texttt{\$b}}
           \end{array}\right.
\end{array}
\end{equation*}
\caption{Example query containing a conditional expression \label{fig:trans:TD:condQue}}
\end{figure}

The query of figure \ref{fig:trans:TD:condQue} contains a conditional expression $e_1$. First the
\texttt{then}-expression and the \texttt{else}-expression are tainted with each
others iterator dependencies. The result of the tainting is shown in figure \ref{fig:trans:TD:inter1:taint}. Then the two relations are augmented
with the $alt$ attribute and spliced together with an \textsf{union} operator to make the relation depicted in
figure \ref{fig:trans:TD:inter1:union}.

\begin{figure}[h]
\centering
\subfigure[]{
\begin{tabular}{|c|c|c|c|} \hline
$bnb$ & $anb$ & $val$ \\ \hline
1 & 1 & 10 \\ \hline
1 & 2 & 20 \\ \hline
2 & 1 & 10 \\ \hline
2 & 2 & 20 \\ \hline
\end{tabular}
\label{fig:trans:TD:inter1:taint}
}
\phantom{A}
\subfigure[]{
\begin{tabular}{|c|c|c|c|} \hline
$alt$ & $bnb$ & $anb$ & $val$ \\ \hline
1 & 1 & 1 & 10 \\ \hline
1 & 1 & 2 & 20 \\ \hline
1 & 2 & 1 & 10 \\ \hline
1 & 2 & 2 & 20 \\ \hline
2 & 1 & 1 & 5 \\ \hline
2 & 2 & 1 & 15 \\ \hline
2 & 1 & 2 & 5 \\ \hline
2 & 2 & 2 & 15 \\ \hline
\end{tabular}
\label{fig:trans:TD:inter1:union}
}
\phantom{A}
\subfigure[]{
\begin{tabular}{|c|c|c|} \hline
$anb$ & $bnb$ & $pred$ \\ \hline
1 & 1 & $false$ \\ \hline
2 & 1 & $false$ \\ \hline
1 & 2 & $true$ \\ \hline
2 & 2 & $false$ \\ \hline
\end{tabular}
\label{fig:trans:TD:inter1:test}
}
\caption[Intermediate results of evaluating $e_1$ in figure \ref{fig:trans:TD:condQue}.]{Intermediate results of
evaluating $e_1$ in figure \ref{fig:trans:TD:condQue}. (a) The
\texttt{then}-expression tainted with $I_{\mbox{\texttt{b}}}$ (b) The \texttt{then}-expression and the \texttt{else}-expression augmented with an $alt$ attribute and spliced together. (c) The test expression.
Attribute names are shortened. $index$ attribute is left out. \label{fig:trans:TD:inter1}}
\end{figure}

The result of the \textsf{union} operation is then joined with the test expression (figure
\ref{fig:trans:TD:inter1:test}) on their common dependencies ($anumb$ and $bnumb$) to form the relation in figure
\ref{fig:trans:TD:condRes:join}. From this relation, only the tuples where $alt$ has the value 1 and $r.val$ is
$true$ or $alt$ has the value 2 and $r.val$ is $false$ are selected. The result of the selection after the final
renaming is shown in figure \ref{fig:trans:TD:condRes:res}.

\begin{figure}[h]
\centering
\subfigure[]{
\begin{tabular}{|c|c|c|c|c|} \hline
$alt$ & $bnb$ & $anb$ & $val$ & $pred$ \\ \hline
1 & 1 & 1 & 10 & $false$ \\ \hline
2 & 1 & 1 & 5 & $false$ \\ \hline
1 & 2 & 1 & 10 & $true$ \\ \hline
2 & 2 & 1 & 15 & $true$ \\ \hline
1 & 1 & 2 & 20 & $false$ \\ \hline
2 & 1 & 2 & 5 & $false$ \\ \hline
1 & 2 & 2 & 20 & $false$ \\ \hline
2 & 2 & 2 & 15 & $false$ \\ \hline
\end{tabular}
\label{fig:trans:TD:condRes:join}
}
\qquad
\subfigure[]{
\begin{tabular}{|c|c|c|} \hline
$bnb$ & $anb$ & $val$ \\ \hline
1 & 1 & 5 \\ \hline
2 & 1 & 10 \\ \hline
1 & 2 & 5 \\ \hline
2 & 2 & 15 \\ \hline
\end{tabular}
\label{fig:trans:TD:condRes:res}
}
\caption[Evaluating $e_1$ in figure \ref{fig:trans:TD:condQue}.]{Evaluating $e_1$ in figure
\ref{fig:trans:TD:condQue}. (a) The test expression joined with the union of the \texttt{then}- and
\texttt{else}-expression. (b) $e_1$ fully evaluated. Attribute names are shortened. $index$ attribute is omitted.
\label{fig:trans:TD:condRes} }
\end{figure}

\end{myExample}


