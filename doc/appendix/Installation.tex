\chapter{Installation and usage}
\label{appendix:installation}
Visit the project website at
\htmladdnormallink{http://code.google.com/p/xqft-parser/}{http://code.google.com/p/xqft-parser/}
for the latest updates and installation instructions.

\section*{Project Directory Contents}
The project directory contains a multitude of files and folders. Here is a list
of the most important ones with a short explanation:
\begin{itemize}
  \item bin - contains the compiled binaries
  \item doc - documentation (this report)
  \item etc - contains the source grammar file, XQFT.g
  \item lib - necessary runtime dependencies
  \item Makefile - Makefile for GNU Make
  \item src - all source code, the generated parser/lexer is moved to this
  directory after generation
  \item test - contains files related to testing and debugging
  \item tmp - temporary files generated during parser generation
\end{itemize}

\section*{Prerequisites}
A computer with a Unix-based operating system is recommended however not
required. Note that the makefiles provided are not immediately suitable for
win32-based operating systems.

\subsection*{Software}
This software should be available for download from the internet if not already
installed on the system.

\begin{itemize}
  \item GNU Make
  \item Subversion
  \item Java JDK 1.5.0 or newer
\end{itemize}

Optional software for generating AST and algebra graphs:
\begin{itemize}
  \item Graphviz
  \item GNU Sed
\end{itemize}

\section*{Getting the Source}
To download the source code using Subversion, execute the following in a
command line interface:

\verb!svn checkout https://xqft-parser.googlecode.com/svn/trunk/ xqft-parser!

To download the source code as a tarball, please visit the project website at
http://code.google.com/p/xqft-parser/.

\section*{Compiling the Source}
Move to the root directory of the source code, and enter the following command:
\verb!make!. This will generate a new parser/lexer pair, compile them together
with supporting classes, and generate a convenient \verb!ntnu-xqft.jar!-file.

lol section fitte

\section*{Command line interface}
Refer to Section \ref{sect:impl:system:cli} on page
\pageref{sect:impl:system:cli} for a description of the command line interface.
