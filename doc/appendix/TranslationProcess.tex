\chapter{Translation process Using TD}
\label{appendix:transl}
This appendix contains step-by-step translations of some XQuery expressions
using Tainting Dependencies.

\section{An extensive FLWOR expression}
\label{appendix:transl:ext_flwor}
This example shows the translation process for the expression in figure
\ref{fig:results:query_ext_flwor} on page \pageref{fig:results:query_ext_flwor}.

The translation process is initiated by entering the FLWOR expression in the
syntax tree and visiting the for- and let-clauses. Recall from section
\ref{sect:trans:TD:basics} on page \pageref{sect:trans:TD:basics} that
\textbf{r}($e$) returns the translation of the XQuery expression $e$ into
relational algebra. This function is then defined through a set of rules
described throughout Section \ref{sect:translation}.

The translation process starts with the \texttt{for}-clause which we translate using
Rule \ref{rule:trans:TD:forbind}. However, to produce \textbf{r}($e$), we must
translate \texttt{(1,2,3)} using rules \ref{rule:trans:TD:seqConstr} and
\ref{rule:trans:TD:literal}. 

By applying Rule \ref{rule:trans:TD:literal} to each of the elements in the
sequence, we obtain the following translations:
\begin{Verbatim}
make(name:=[index, value], [1], [1])
\end{Verbatim}
\begin{Verbatim}
make(name:=[index, value], [1], [2])
\end{Verbatim}
\begin{Verbatim}
make(name:=[index, value], [3], [3])
\end{Verbatim}

By applying Rule \ref{rule:trans:TD:seqConstr} to this result we obtain the
following translation of the sequence (1,2,3):

\begin{Verbatim}
numberate(index,[sprIdx,index],[];
  union(;
    make(name:=[index, value], [1], [1])
    \end{Verbatim}
    \begin{Verbatim}
    make(name:=[index, value], [1], [2])
    \end{Verbatim}
    \begin{Verbatim}
    make(name:=[index, value], [3], [3])))
\end{Verbatim}

We can now continue translating the \texttt{for}-clause, as the above translation
equates to \textbf{r}($e$). By Rule \ref{rule:trans:TD:forbind}, we obtain the
following translation which is to be entered into the symbol table for this
scope (for the symbol $\chi$ = a):

\begin{Verbatim}
project([anumb = index, index = 1, value];
  numberate(index,[sprIdx,index],[];
    union(;
      make(name:=[index, value], [1], [1])
      \end{Verbatim}
      \begin{Verbatim}
      make(name:=[index, value], [1], [2])
      \end{Verbatim}
      \begin{Verbatim}
      make(name:=[index, value], [3], [3]))))
\end{Verbatim}

Any later reference to \texttt{\$a} is now replaced with a lookup in the symbol
table which will return this algebra expression.

The let clause is translated in a similar manner, however the entry in the
symbol table will not be tainted by \texttt{\$b}. To translate the let-clause,
we apply the Rule \ref{rule:trans:TD:literal} on the expression and obtain:

\begin{Verbatim}
make(name:=[index, value], [1],[2])
\end{Verbatim}

Which is entered into the symbol table as per Rule \ref{rule:trans:TD:letbind}.

Before translating the where- and orderby-clauses, it is benefitial to
translate the return-clause (since the two former requires the translation of
the latter). The translation is obtained using rule
\ref{rule:trans:TD:returnTaint} (return-clause) and rule
\ref{rule:trans:TD:seqConstr} (sequence construction), which is trivial. Recall
that references to \texttt{\$a} and \texttt{\$b} are replaced by their
translations in the symbol table:

\begin{Verbatim}
numberate(index,[sprIdx,index],[];
  union(;
    project([anumb = index, index = 1, value];
      numberate(index,[sprIdx,index],[];
        union(;
          project(sprIdx=1,index=0,value;
            make(name:=[index,value], [1], [1])),
          project(sprIdx=2,index=0,value;
            make(name:=[index, value], [2], [2])),
          project(sprIdx=3,index=0,value;
            make(name:=[index, value], [3], [3]))))),
    project([anumb = index, index = 2, value];
      make(name:=[index, value],[1], [2]))))
\end{Verbatim}

This translation is now used to replace \textbf{r}($e_C$) in the translation
of the where-clause, which is translated using rules \ref{rule:trans:TD:where}
and \ref{rule:trans:TD:arithmetic}:

% WHERE
The translation process is finalised by translating the where-clause using rules
\ref{rule:trans:TD:where} and \ref{rule:trans:TD:comparative}. First we
obtain the following translation from \texttt{\$a gt \$b}:

\begin{Verbatim}
    project(index=1, value=gt(r.value, l.value), anumb;
      hhjoin([], [], [anumb, lvalue=l.value, rvalue=r.value, alt];
        project([anumb = index, index = 1, value];
          numberate(index,[sprIdx,index],[];
            union(;
              project(sprIdx=1,index=0,value;
                make(name:=[index, value], [1], [1])),
              project(sprIdx=2,index=0,value;
                make(name:=[index, value], [2], [2])),
              project(sprIdx=3,index=0,value;
                make(name:=[value], [3], [3]))))),
          make(name:=[index, value],[1], [2])))
\end{Verbatim}

Which is entered together with \textbf{r}($e_C$) into the translation of the
where-clause:

\begin{Verbatim}
hhjoin([],[], [l.value, anumb];
  numberate(index,[sprIdx,index],[];
    union(;
      project([anumb = index, index = 1, value];
        numberate(index,[sprIdx,index],[];
          union(;
            project(sprIdx=1,index=0,value;
              make(name:=[index, value], [1], [1])),
            project(sprIdx=2,index=0,value;
              make(name:=[index, value], [2], [2])),
            project(sprIdx=3,index=0,value;
              make(name:=[value], [3], [3]))))),
      project([anumb = index, index = 2, value];
        make(name:=[index, value],[1], [2]))),
  select(xqBoolean(value);
    project(index=1, value=gt(r.value, l.value), anumb;
      hhjoin([], [], [anumb, lvalue=l.value, rvalue=r.value, alt];
        project([anumb = index, index = 1, value];
          numberate(index,[sprIdx,index],[];
            union(;
              project(sprIdx=1,index=0,value;
                make(name:=[index, value], [1], [1])),
              project(sprIdx=2,index=0,value;
                make(name:=[index, value], [2], [2])),
              project(sprIdx=3,index=0,value;
                make(name:=[value], [3], [3]))))),
          make(name:=[index, value],[1], [2]))))))
\end{Verbatim}

This result is used to replace \textbf{r}($e_C$) in the translation of the
orderby-clause, and we obtain the complete translation:

\begin{Verbatim}
project(value=l.value, index;
  numberate(index, [r.value, index], [];
    hhjoin([anumb], [anumb], [l.value, r.value];
      project([anumb = index, index = 1, value];
        numberate(index,[sprIdx,index],[];
          union(;
            project(sprIdx=1,index=0,value;
              make(name:=[index, value], [1], [1])),
            project(sprIdx=2,index=0,value;
              make(name:=[index, value], [2], [2])),
            project(sprIdx=3,index=0,value;
              make(name:=[value], [3], [3]))))),
      hhjoin([],[], [l.value, anumb];
          project(index=1, value=gt(l.value, r.value), anumb;
            hhjoin([], [], [anumb, lvalue=l.value, rvalue=r.value, alt];
              numberate(index,[sprIdx,index],[];
                  union(;
                    project([anumb = index, index = 1, value];
                      numberate(index,[sprIdx,index],[];
                        union(;
                          project(sprIdx=1,index=0,value;
                            make(name:=[index, value], [1], [1])),
                          project(sprIdx=2,index=0,value;
                            make(name:=[index, value], [2], [2])),
                          project(sprIdx=3,index=0,value;
                            make(name:=[value], [3], [3]))))),
                    project([anumb = index, index = 2, value];
                      make(name:=[index, value],[1], [2])))),
              select(xqBoolean(value);
                project(index=1, value=gt(l.value, r.value), anumb;
                  hhjoin([], [], [anumb, lvalue=l.value, rvalue=r.value, alt];
                    project([anumb = index, index = 1, value];
                      numberate(index,[sprIdx,index],[];
                        union(;
                          project(sprIdx=1,index=0,value;
                            make(name:=[index, value], [1], [1])),
                          project(sprIdx=2,index=0,value;
                            make(name:=[index, value], [2], [2])),
                          project(sprIdx=3,index=0,value;
                            make(name:=[value], [3], [3]))))),
                  project([anumb = index, index = 2, value];
                    make(name:=[index, value],[1], [2])))))))))))
\end{Verbatim}
 
This can be simplified by replacing the sequence construction nodes with a
single operation \texttt{make(name:=[index, value], [1,2,3], [1,2,3]))}:

\begin{Verbatim}
project(value=l.value, index;
  numberate(index, [r.value, index], [];
    hhjoin([anumb], [anumb], [l.value, r.value];
      hhjoin([],[], [l.value, anumb];
        numberate(index,[sprIdx,index],[];
          union(;
            project([anumb = index, index = 1, value];
              make(name:=[index, value], [1,2,3], [1,2,3])),
            project([anumb = index, index = 2, value];
              make(name:=[index, value],[1],[2])))),
        select(xqBoolean(value);
          project(index=1, value=gt(l.value, r.value), anumb;
            hhjoin([l.anumb], [r.anumb], [l.value, r.value, anumb];
              project([anumb = index, index = 1, value];
                make(name:=[index, value], [1,2,3], [1,2,3])),
              make(name:=[index, value],[1],[2]))))),
      project([anumb = index, index = 1, value];
        make(name:=[index, value], [1,2,3], [1,2,3])))
\end{Verbatim}

And further by magic:

\begin{Verbatim}
project(value=l.value, index;
  numberate(index, [r.value, index], [];
    hhjoin([anumb], [anumb], [l.value, r.value];
      hhjoin([anumb],[anumb], [l.value, anumb];
          union(;
            project([anumb = index, index = 1, value];
              make(name:=[index, value], [1,2,3], [1,2,3])),
            project([anumb, index = 2, value = 2];
              make(name:=[index, value], [1,2,3], [1,2,3]))),
        select(xqBoolean(value);
          project(index=1, value=gt(value, 2), anumb;
            project([anumb = index, index = 1, value];
              make(name:=[index, value], [1,2,3], [1,2,3]))))),
      project([anumb = index, index = 1, value];
        make(name:=[index, value], [1,2,3], [1,2,3])))))
\end{Verbatim}


\section{If-then-else}
\label{appendix:transl:ifthenelse}
This translation is initiated precisely as the previous ``Extensive FLWOR''
example, so we will not reiterate the translation of the for-clause here.
However, note that the following expression is obtained and entered into the
symbol table to represent \texttt{\$a}:

\begin{Verbatim}
project([anumb = index, index = 1, value];
  numberate(index,[sprIdx,index],[];
    union(;
      project(sprIdx=1,index=0,value;
        make(name:=[index, value], [1], [1])),
      project(sprIdx=2,index=0,value;
        make(name:=[index, value], [2], [2])),
      project(sprIdx=3,index=0,value;
        make(name:=[value], [3], [3]))))
\end{Verbatim}

% for $a in (1,2,3) return
%   if $a gt 2 then $a else 3

The translation of the if-then-else expression follows rule
\ref{rule:trans:TD:conditional}  on page \pageref{rule:trans:TD:conditional}.
First, however, the boolean expression \texttt{\$a > 2} (which corresponds to
$e_1$ in Rule \ref{rule:trans:TD:conditional}) must be translated. This can be
done using Rule \ref{rule:trans:TD:comparative}, as for the where-clause in the
previous FLWOR example:


\begin{Verbatim}
    project(index=1, value=gt(r.value, l.value), anumb;
      hhjoin([], [], [anumb, lvalue=l.value, rvalue=r.value, alt];
        project([anumb = index, index = 1, value];
          numberate(index,[sprIdx,index],[];
            union(;
              project(sprIdx=1,index=0,value;
                make(name:=[index, value], [1], [1])),
              project(sprIdx=2,index=0,value;
                make(name:=[index, value], [2], [2])),
              project(sprIdx=3,index=0,value;
                make(name:=[value], [3], [3]))))),
          make(name:=[index, value],[1], [2])))
\end{Verbatim}

Next we translate $e_2$ (which is simply a lookup in the symbol table for \texttt{\$a}):

\begin{Verbatim}
project([anumb = index, index = 1, value];
  numberate(index,[sprIdx,index],[];
    union(;
      project(sprIdx=1,index=0,value;
        make(name:=[index, value], [1], [1])),
      project(sprIdx=2,index=0,value;
        make(name:=[index, value], [2], [2])),
      project(sprIdx=3,index=0,value;
        make(name:=[value], [3], [3])))))
\end{Verbatim}

The last expression, $e_3$ (which is tainted by \texttt{\$a}), can be translated
using a simplification where instead of constructing the tuples, an additional
field with the value 3 is projected onto the result from a lookup of
\texttt{\$a} in the symbol table:

\begin{Verbatim}
project(anumb, index = 1, value = 3;
  project([anumb = index, index = 1, value];
    numberate(index,[sprIdx,index],[];
      union(;
        project(sprIdx=1,index=0,value;
          make(name:=[index, value], [1], [1])),
        project(sprIdx=2,index=0,value;
          make(name:=[index, value], [2], [2])),
        project(sprIdx=3,index=0,value;
          make(name:=[value], [3], [3]))))))
\end{Verbatim}

Now the translated representations of $e_1$, $e_2$ and $e_3$ (where the latter
is also tainted) can be entered into Rule \ref{rule:trans:TD:conditional}:

\begin{Verbatim}
project(value=lvalue, anumb;
  select(ifthenelse(xqBoolean(rvalue), eq(alt,1), eq(alt,2));
    hhjoin([l.anumb],[r.anumb], [anumb, lvalue=l.value, rvalue=r.value, alt];
      union(
        project(anumb, alt = 1, value;
          project([anumb = index, index = 1, value];
            numberate(index,[sprIdx,index],[];
              union(;
                project(sprIdx=1,index=0,value;
                  make(name:=[index, value], [1], [1])),
                project(sprIdx=2,index=0,value;
                  make(name:=[index, value], [2], [2])),
                project(sprIdx=3,index=0,value;
                  make(name:=[value], [3], [3])))))),
        project(anumb, alt = 2, value;
          project(anumb, index = 1, value = 3;
            project([anumb = index, index = 1, value];
              numberate(index,[sprIdx,index],[];
                union(;
                  project(sprIdx=1,index=0,value;
                    make(name:=[index, value], [1], [1])),
                  project(sprIdx=2,index=0,value;
                    make(name:=[index, value], [2], [2])),
                  project(sprIdx=3,index=0,value;
                    make(name:=[value], [3], [3])))))))),
      project(index=1, value=gt(l.value, r.value), anumb;
        hhjoin([], [], [anumb, lvalue=l.value, rvalue=r.value, alt];
          project([anumb = index, index = 1, value];
            numberate(index,[sprIdx,index],[];
              union(;
                project(sprIdx=1,index=0,value;
                  make(name:=[index, value], [1], [1])),
                project(sprIdx=2,index=0,value;
                  make(name:=[index, value], [2], [2])),
                project(sprIdx=3,index=0,value;
                  make(name:=[value], [3], [3]))))),
            make(name:=[index, value],[1], [2]))))))
\end{Verbatim}

This can be simplified further by replacing \texttt{make()} as described in Rule \ref{rule:trans:TD:seqIfLit}: 

\begin{Verbatim}
project(value=lvalue, anumb;
  select(ifthenelse(xqBoolean(rvalue), eq(alt,1), eq(alt,2));
    hhjoin([l.anumb],[r.anumb], [anumb, lvalue=l.value, rvalue=r.value, alt];
      union(
        project(anumb, alt = 1, value;
          project([anumb = index, index = 1, value];
            make(name:=[index, value], [1,2,3], [1,2,3]))),
        project(anumb, alt = 2, value;
          project(anumb, index = 1, value = 3;
            project([anumb = index, index = 1, value];
             make(name:=[index, value], [1,2,3], [1,2,3]))))),
      project(index=1, value=gt(value, 2), anumb;
        project([anumb = index, index = 1, value];
          make(name:=[index, value], [1,2,3], [1,2,3]))))))
\end{Verbatim}

The translation is finalised by applying Rule \ref{rule:trans:TD:itOrd}:

\begin{Verbatim}
numberate(index, [anumb, index], [];
  project(value=lvalue, anumb;
    select(ifthenelse(xqBoolean(rvalue), eq(alt,1), eq(alt,2));
      hhjoin([l.anumb],[r.anumb], [anumb, lvalue=l.value, rvalue=r.value, alt];
        union(
          project(anumb, alt = 1, value;
            project([anumb = index, index = 1, value];
              make(name:=[index, value], [1,2,3], [1,2,3]))),
          project(anumb, alt = 2, value;
            project(anumb, index = 1, value = 3;
              project([anumb = index, index = 1, value];
               make(name:=[index, value], [1,2,3], [1,2,3]))))),
        project(index=1, value=gt(value, 2), anumb;
          project([anumb = index, index = 1, value];
            make(name:=[index, value], [1,2,3], [1,2,3])))))))
\end{Verbatim}

\section{Path expression with a predicate}
\label{appendix:transl:pathPred}

It should be noted that the query is not a valid query unless the context item is set before execution. We assume
the context item is stored in the symbol table, and have no dependencies. First the axis steps consisting of the
name tests \texttt{a} and \texttt{b} is translated. As these are consecutive child-axis steps (no axis specifier
is read as \texttt{child::}), Rule \ref{rule:trans:TD:pathChild} may be used once for both steps, instead of using
the more general rule (Rule \ref{rule:trans:TD:pathStep}) twice:
\begin{Verbatim}
project(dotNumb, docId, index, value, pos, scope;
  numberate(dotNumb, [dotNumb, subIdx], [];
    select(isChild(scope, lsc);
      hhjoin([docId], [docId], [dotNumb,lsc=l.scope,subIdx=r.index,
            value=r.value,pos=r.pos,scope=r.scope];
        symtab.get(dot);
        numberate(index, [], [];
          index(valocc;
            scope(a/b;
              lookup($b))))))))
\end{Verbatim}
Also note that the next to last \textsf{numberate} operator from the rule is dropped, as the predicate expression
must evaluate to a boolean type. This is described in Section \ref{sect:trans:TD:simpl:pathExpr}.

Then the predicate will have to be taken into account. The result of \texttt{/a/b} is stored in the symbol table
as \textbf{sprDot}, where $dotNumb$ is copied to $sprDotNumb$ and the $index$ fields will be set to the
value 1. When this is done, the comparison expression in the predicate will be evaluated. First \textsf{@id} is
evaluated. This contains a implicit reference to the context node, and in its verbose form is formed like this:
\texttt{./attribute::id}. This expression is evaluated by Rule \ref{rule:trans:TD:pathStep}. As this path
expression (within the predicate) does not contain any predicate, the next to last \textsf{numberate} operator
will be dropped. As it is the complete path expression it will be finalised by Rule \ref{rule:trans:TD:pathExpr}.
The last step of the path expression (still the one within the predicate) does not contain some kind of filtering,
thus the final \textsf{numberate} operator will be exchanged with a \textsf{project} operator:

\begin{Verbatim}
project(index=dotNumb, docId, value, pos, scope, sprDotNumb
  project(docId, index, value, pos, scope, dotNumb, sprDotNumb;
    numberate(dotNumb, [dotNumb, subIdx], [sprDotNumb];
      select(isChild(scope,lsc);
        hhjoin([docId],[docId],[dotNumb,lsc=l.scope,subIdx=r.index,
              value=r.value,pos=r.pos,scope=r.scope,sprDotNumb]
          project(sprDotNumb=dotNumb, dotNumb, index=1, value, pos,
                docId, scope;
            project(dotNumb, docId, index, value, pos, scope;
              numberate(dotNumb, [dotNumb, subIdx], [];
                select(isChild(scope, lsc);
                  hhjoin([docId], [docId], [dotNumb,lsc=l.scope,
                        subIdx=r.index,value=r.value,pos=r.pos,
                        scope=r.scope];
                    symtab.get(dot);
                    numberate(index, [], [];
                      index(valocc;
                        scope(a/b;
                          lookup($b)))))))))
          numberate(index, [], []
            index(valocc;
              lookup($@id))))))))
\end{Verbatim}

The comparison expression will then be evaluated. By the simplifications described in section
\ref{sect:trans:TD:simpl:lit}, the whole expression will be evaluated by adding a single \textsf{project} operator:

\begin{Verbatim}
project(index=1, value=eq(value, 2), sprDotNumb;
  project(index=dotNumb, docId, value, pos, scope, sprDotNumb
    project(docId, index, value, pos, scope, dotNumb, sprDotNumb;
      numberate(dotNumb, [dotNumb, subIdx], [sprDotNumb];
        select(isChild(scope,lsc);
          hhjoin([docId],[docId],[dotNumb,lsc=l.scope,
                subIdx=r.index,value=r.value,pos=r.pos,
                scope=r.scope,sprDotNumb]
            project(sprDotNumb=dotNumb, dotNumb, index=1, value,
                  pos, docId, scope;
              project(dotNumb, docId, index, value, pos, scope;
                numberate(dotNumb, [dotNumb, subIdx], [];
                  select(isChild(scope, lsc);
                    hhjoin([docId], [docId], [dotNumb,lsc=l.scope,
                          subIdx=r.index,value=r.value,pos=r.pos,
                          scope=r.scope];
                      symtab.get(dot);
                      numberate(index, [], [];
                        index(valocc;
                          scope(a/b;
                            lookup($b))))))))
            numberate(index, [], []
              index(valocc;
                lookup($@id)))))))))
\end{Verbatim}

Then the rule for translating predicates (Rule \ref{rule:trans:TD:predicate}) are employed. The predicate
expression and the predicated expression will have to be joined on their $sprDotNumb$ and $dotNumb$ attributes
respectively, as they have no other common dependencies. Finally the whole path expression is finalised by a
renumbering, as described by Rule \ref{rule:trans:TD:pathExpr}.

\begin{Verbatim}
numberate(index, [dotNumb, index], [];
  project(index,docId,scope,pos,value,dotNumb;
    select(ifthenelse(isNumber(pred),
          eq(index,pred),xqBoolean(pred));
      hhjoin([dotNumb],[sprDotNumb], [value=l.value,scope=l.scope,
            pos=l.pos,docId=l.docId,pred=r.value];
        project(dotNumb, docId, index, value, pos, scope;
          numberate(dotNumb, [dotNumb, subIdx], [];
            select(isChild(scope, lsc);
              hhjoin([docId], [docId], [dotNumb,lsc=l.scope,
                    subIdx=r.index,value=r.value,pos=r.pos,
                    scope=r.scope];
                symtab.get(dot);
                numberate(index, [], [];
                  index(valocc;
                    scope(a/b;
                      lookup($b))))))))
        project(index=1, value=eq(value, 2), sprDotNumb;
          project(index=dotNumb,docId,value,pos,scope,sprDotNumb;
            project(docId,index,value,pos,scope,dotNumb,sprDotNumb;
              numberate(dotNumb, [dotNumb, subIdx], [sprDotNumb];
                select(isChild(scope,lsc);
                  hhjoin([docId],[docId],[dotNumb,lsc=l.scope,
                        subIdx=r.index,value=r.value,pos=r.pos,
                        scope=r.scope,sprDotNumb]
                    project(sprDotNumb=dotNumb,dotNumb,index=1,
                          value,pos,docId,scope;
                      project(dotNumb,docId,index,value,pos,scope;
                        numberate(dotNumb, [dotNumb, subIdx], [];
                          select(isChild(scope, lsc);
                            hhjoin([docId], [docId], [dotNumb,
                                  lsc=l.scope,subIdx=r.index,
                                  value=r.value,pos=r.pos,
                                  scope=r.scope];
                              symtab.get(dot);
                              numberate(index, [], [];
                                index(valocc;
                                  scope(a/b;
                                    lookup($b))))))))
                    numberate(index, [], []
                      index(valocc;
                        lookup($@id)))))))))))))
\end{Verbatim}
