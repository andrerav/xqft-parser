\chapter{Translation Process Using TD}
\label{appendix:transl}
\section{An Extensive FLWOR Expression}
\label{appendix:transl:ext_flwor}
This example shows the translation process for the expression in figure
\ref{fig:results:query_ext_flwor} on page \pageref{fig:results:query_ext_flwor}.

The translation process is initiated by entering the FLWOR expression in the
syntax tree and visiting the for- and let-clauses. Recall from section
\ref{sect:trans:TD:basics} on page \pageref{sect:trans:TD:basics} that
\textbf{r}($e$) returns the translation of the XQuery expression $e$ into
relational algebra. This function is then defined through a set of rules
described throughout section \ref{sect:trans:taintingDependencies}.

The translation process starts with the for-clause which we translate using
rule \ref{rule:trans:TD:forbind}. However, to produce \textbf{r}($e$), we must
translate \texttt{(1,2,3)} using rules \ref{rule:trans:TD:seqConstr} and
\ref{rule:trans:TD:literal}. 

By applying rule \ref{rule:trans:TD:literal} to each of the elements in the
sequence, we obtain the following translations:
\begin{Verbatim}
make(name:=[value],1)
\end{Verbatim}
\begin{Verbatim}
make(name:=[value],2)
\end{Verbatim}
\begin{Verbatim}
make(name:=[value],3)
\end{Verbatim}

By applying rule \ref{rule:trans:TD:seqConstr} to this result we obtain the
following translation of the sequence (1,2,3):

\begin{Verbatim}
numberate(index,[sprIdx,index],[];
  union(;
    project(sprIdx=1,index=0,value;
      make(name:=[value],1)),
    project(sprIdx=2,index=0,value;
      make(name:=[value],2)),
    project(sprIdx=3,index=0,value;
      make(name:=[value],3))))
\end{Verbatim}

We can now continue translating the for-clause, as the above translation
equates to \textbf{r}($e$). By rule \ref{rule:trans:TD:forbind}, we obtain the
following translation which is to be entered into the symbol table for this
scope (for the symbol $\chi$ = a):

\begin{Verbatim}
project([anumb = index, index = 1, value];
  numberate(index,[sprIdx,index],[];
    union(;
      project(sprIdx=1,index=0,value;
        make(name:=[value],1)),
      project(sprIdx=2,index=0,value;
        make(name:=[value],2)),
      project(sprIdx=3,index=0,value;
        make(name:=[value],3))))
\end{Verbatim}

Any later reference to \texttt{\$a} is now replaced with a lookup in the symbol
table which will return this algebra expression.

The let clause is translated in a similar manner, however the entry in the
symbol table will not be tainted by \texttt{\$b}. To translate the let-clause,
we apply the rule \ref{rule:trans:TD:literal} on the expression and obtain:

\begin{Verbatim}
make(name:=[value],2)
\end{Verbatim}

Which is entered into the symbol table as per rule \ref{rule:trans:TD:letbind}.

Before translating the where- and orderby-clauses, it is benefitial to
translate the return-clause (since the two former requires the translation of
the latter). The translation is obtained using rule
\ref{rule:trans:TD:returnTaint} (return-clause) and rule
\ref{rule:trans:TD:seqConstr} (sequence construction), which is trivial. Recall
that references to \texttt{\$a} and \texttt{\$b} are replaced by their
translations in the symbol table:

\begin{Verbatim}
numberate(index,[sprIdx,index],[];
  union(;
    project([anumb = index, index = 1, value];
      numberate(index,[sprIdx,index],[];
        union(;
          project(sprIdx=1,index=0,value;
            make(name:=[value],1)),
          project(sprIdx=2,index=0,value;
            make(name:=[value],2)),
          project(sprIdx=3,index=0,value;
            make(name:=[value],3))))),
    make(name:=[value],2)))
\end{Verbatim}

This translation is now used to replace \textbf{r}($e_C$) in the translation
of the where-clause, which is translated using rules \ref{rule:trans:TD:where}
and \ref{rule:trans:TD:arithmetic}:

% WHERE
The translation process is finalized by translating the where-clause using rules
\ref{rule:trans:TD:where} and \ref{rule:trans:TD:comparative}. First we
obtain the following translation from \texttt{\$a gt \$b}:

\begin{Verbatim}
    project(index=1, value=gt(r.value, l.value), anumb;
      hhjoin([], [], [l.value, r.value, anumb];
        project([anumb = index, index = 1, value];
          numberate(index,[sprIdx,index],[];
            union(;
              project(sprIdx=1,index=0,value;
                make(name:=[value],1)),
              project(sprIdx=2,index=0,value;
                make(name:=[value],2)),
              project(sprIdx=3,index=0,value;
                make(name:=[value],3))))),
        make(name:=[value],2)))))
\end{Verbatim}

Which is entered together with \textbf{r}($e_C$) into the translation of the
where-clause:

\begin{Verbatim}
hhjoin([],[], [l.value, anumb];
  numberate(index,[sprIdx,index],[];
    union(;
      project([anumb = index, index = 1, value];
        numberate(index,[sprIdx,index],[];
          union(;
            project(sprIdx=1,index=0,value;
              make(name:=[value],1)),
            project(sprIdx=2,index=0,value;
              make(name:=[value],2)),
            project(sprIdx=3,index=0,value;
              make(name:=[value],3))))),
      make(name:=[value],2))),
  select(xqBoolean(value);
    project(index=1, value=gt(r.value, l.value), anumb;
      hhjoin([], [], [l.value, r.value, anumb];
        project([anumb = index, index = 1, value];
          numberate(index,[sprIdx,index],[];
            union(;
              project(sprIdx=1,index=0,value;
                make(name:=[value],1)),
              project(sprIdx=2,index=0,value;
                make(name:=[value],2)),
              project(sprIdx=3,index=0,value;
                make(name:=[value],3))))),
        make(name:=[value],2)))))
\end{Verbatim}

This result is used to replace \textbf{r}($e_C$) in the translation of the
orderby-clause, and we obtain the complete translation:

\begin{Verbatim}
project(value=l.value, anumb;
  numberate(index, [r.value, index], [anumb];
    hhjoin([l.anumb], [r.anumb], [l.value, r.value, anumb];
      hhjoin([],[], [l.value, anumb];
          project(index=1, value=gt(l.value, r.value), anumb;
            hhjoin([], [], [l.value, r.value, anumb];
              numberate(index,[sprIdx,index],[];
                  union(;
                    project([anumb = index, index = 1, value];
                      numberate(index,[sprIdx,index],[];
                        union(;
                          project(sprIdx=1,index=0,value;
                            make(name:=[value],1)),
                          project(sprIdx=2,index=0,value;
                            make(name:=[value],2)),
                          project(sprIdx=3,index=0,value;
                            make(name:=[value],3))))),
                    make(name:=[value],2))),
        select(xqBoolean(value);
          project(index=1, value=gt(l.value, r.value), anumb;
            hhjoin([], [], [l.value, r.value, anumb];
              project([anumb = index, index = 1, value];
                numberate(index,[sprIdx,index],[];
                  union(;
                    project(sprIdx=1,index=0,value;
                      make(name:=[value],1)),
                    project(sprIdx=2,index=0,value;
                      make(name:=[value],2)),
                    project(sprIdx=3,index=0,value;
                      make(name:=[value],3))))),
              make(name:=[value],2))))),
      project([anumb = index, index = 1, value];
        numberate(index,[sprIdx,index],[];
          union(;
            project(sprIdx=1,index=0,value;
              make(name:=[value],1)),
            project(sprIdx=2,index=0,value;
              make(name:=[value],2)),
            project(sprIdx=3,index=0,value;
              make(name:=[value],3))))
\end{Verbatim}
 
This can be simplified by replacing the sequence construction nodes with a
single operation \texttt{make(name:=[index, value], [1,2,3], [1,2,3]))}:

\begin{Verbatim}
project(value=l.value, anumb;
  numberate(index, [r.value, index], [anumb];
    hhjoin([l.anumb], [r.anumb], [l.value, r.value, anumb];
      hhjoin([],[], [l.value, anumb];
        numberate(index,[sprIdx,index],[];
          union(;
            project([anumb = index, index = 1, value];
              make(name:=[index, value], [1,2,3], [1,2,3])),
            make(name:=[value],2))),
        select(xqBoolean(value);
          project(index=1, value=gt(r.value, l.value), anumb;
            hhjoin([l.anumb], [r.anumb], [l.value, r.value, anumb];
              project([anumb = index, index = 1, value];
                make(name:=[index, value], [1,2,3], [1,2,3])),
              make(name:=[value],2)))),
      project([anumb = index, index = 1, value];
        make(name:=[index, value], [1,2,3], [1,2,3])))))
\end{Verbatim}
