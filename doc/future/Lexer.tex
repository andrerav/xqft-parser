\section{Improving the Lexer}
\underline{\textbf{\LARGE //TODO:}}

lage lexer for haand, kan bli mindre kompleks evt -----v

lage flere lexere, og skifte mellom disse naar vi naa skifter state (island grammars)

\subsection{Lexer Error Handling}
As noted in sections \ref{sect:error_handling:syntax_errors} and
\ref{sec:impl:errorhandling}, the nextToken() method in the lexer base class
would ``hijack'' lexical exceptions and handle them by printing an error message
to stderr and then attempting to recover from the error by simply consuming the
offending character and ignoring it in the following construction of tokens. It is 
not immediately appareant, but the nextToken() method can not be overridden and
forced to throw the exception onwards, due to the method signature itself which
does not allow exceptions to be thrown. 

However, one possible solution to this problem could be to override the
nextToken() method and employ the observer design
pattern to allow a
simple and decoupled way of flagging an exception to the parser.
