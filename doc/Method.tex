%Relasjonsalgebra
\chapter{Architectual Decisions}
\label{chapter:method}
Because we decided to create the parser with a generator, different alternatives of such tools had to be evaluated. In this chapter we will document this process. Further, we will present other decisions we made before implementating the recognizer. This constitutes the choice of licence and means testing the system. Finally we will present the problem with the ambiguous terminals and evaluate different lexer strategies that would solve this.

\section{Parser construction}
Writing a parser from scratch was ruled out early for being too time consuming.
Instead it was decided to use tools for compiler and parser construction to
generate a parser from the XQuery 1.0 and XPath 2.0 grammar specifications
developed by the W3C.

\subsection{Evaluated alternatives}
\label{sect:method:alternatives}
\subsubsection{JFlex/CUP}
JFlex and CUP is a versatile combination consisting of JFlex which is a lexer
generator, and CUP which is a parser generator. These tools can be interfaced to
generate a complete parser with a separate lexical analyzer.

JFlex and CUP produces only LALR parsers, and since the W3C has specified an
LL(1) grammar for XQuery 1.0 and XPath 2.0, the combination of JFlex and CUP was
rejected from this project.

\subsubsection{JavaCC}
JavaCC could have been a viable alternative as it produces LL(k) parsers,
however compared to Antlr its grammar specification syntax deviated so much
from the W3C EBNF syntax, that grammar would have had to be extensively
rewritten.

\subsubsection{Antlr}
ANTLR is a renowned tool for parser generation, and can generate LL(*) parsers.
Additionally, Antlr accepts a grammar specification syntactically very close to
the EBNF used by the W3C. This is the parser generater chosen for our project,
based on the criteria outlined in this section.

\section{Debugging}
\label{sect:method:debugging}
There are several approaches to debugging. Antlrworks \cite{antlrwrks00} is a
simple tool for writing, testing and debugging Antlr grammars. The debugging
interface is useful in that it draws a realtime step-by-step parse tree as the
input is being parsed, as well as displaying a list of parser events. ANTLRWorks also helps eliminate grammar nondeterminisms by drawing the syntax diagram associated with a grammar and highlighting nondeterministic paths. A big drawback, though, is as the grammar grows bigger, the application tends to become unstable.

When working properly, ANTLRWorks is a great tool for debuging the parser, but no so much for the lexer. It will show the tokens returned to the parser, but no information or decision path indicating the reason for returning just that token. A more appropriate tool for such a task would be the standard debuging environment in Eclipse (\cite{eclipse}).

% Method / Testing
\section{Testing}
\label{sect:method:testing}
To improve quality insurance in this project, it was decided to systematically
write tests for new features as well as perform coverage testing for comparison
with existing implementations. This section will outline the various test
methods used in this project. Test results are presented in section \ref{sect:results:tests}.

\subsection{gUnit}
\label{sect:method:gUnit}
Unit testing can be a powerful tool for asserting functionality and can be a
helpful aid in debugging and prevention of regression errors.  For unit testing the
grammar specification, gUnit \cite{gunit00} was employed. This tool uses a
syntax similar to ANTLR itself, however instead of defining productions, one
defines a set of inputs for some rule, as well as the expected result. Consider
this example:

\begin{Verbatim}
gunit XQFT;
@header{package no.ntnu.xqft.parse;}

piTarget: // Test piTarget rule

    // Any case permutation of 'XML' must fail
    "Xml" FAIL
    "XMl" FAIL
    "XML" FAIL
    "XmL" FAIL
\end{Verbatim}

This is a complete input file for gUnit, and will automatically discover the
classes \verb!XQFTLexer! and \verb!XQFTParser! in the package
\verb!no.ntnu.xqft.parse!. gUnit will  then proceed to invoke the lexer with
``Xml'', ``XMl'', ``XML'', and ``XmL'' as input, pass the lexer to an instance
of \verb!XQFTParser! and execute the production \verb!piTarget!. For all these
inputs, it will assert that the parser emits an error (i.e it must fail to pass
the test).

In case of a test where the parser should not fail, the syntax is as follows:
\begin{Verbatim}
forClause:
	"for $a in document(\"abc.xml\")/a/b/text()" OK
\end{Verbatim}
Here gUnit will assert that the parser will not fail for the given input (i.e it
must not fail to pass the test).

gUnit is also capable of parsing abstract syntax trees built by the generated
parser, but this feature has not been used in this project. gUnit is available
at the gUnit website\cite{gunit00}.

\subsection{jUnit}
In addition to the gUnit tests for the grammar, jUnit was used to write unit tests for
the additional classes needed for this parser. In particular, the classes
related to scoping and symbol tables. Here is one example, asserting that the
setParent() method in the scope class works as expected:
\begin{Verbatim}
    @Test
    public void testSetParent() {
        this.scope = new Scope();
        Scope tmp = new Scope();
        this.scope.setParent(tmp);
        assertEquals(this.scope.getParent(), tmp);
    }
\end{Verbatim}

jUnit is available at the jUnit website\cite{junit_website}.

\subsection{Manual Compliance Tests}
Finally, we employed manual tests for running a subset of the XQuery Test
Suite\cite{w3c05} as a benchmark for XQuery compliance. Specifically, we
simplified this approach into counting any test query accepted by the parser as
passed. Since the parser does not generate any results from queries, it was
impossible to run a comparison to the given expected results.

The original test suite consists of a total of 12584 different queries with a
corresponding set of expected output. These tests span a large amount of XQuery
functionality. Out of these tests, a total of 12478 tests are applicable for
this project. These tests are categorized as scenarios: \emph{standard}, 
\emph{parse-error}, and \emph{trivial}. The remaining tests are only possible to
perform against a XQuery processor, and are thus not applicable in the context of this project.
\section{Licensing}
\label{sect:method:licensing}
One requirement for the parser is a liberal open source software license.
Several such liberal licenses exist, however here we shall concentrate on a
short selection of popular and widely adopted licenses. These have been gathered
from the Open Source Initiative (OSI) website\cite{osi_website}, from the
category
``\emph{Licenses that are popular and widely used or with strong communities}''.
\subsection{GPL -- General Public License}
The GPL license is a 

The GNU General Public License (GNU GPL or simply GPL) is a widely used free
software license, originally written by Richard Stallman for the GNU project.
It is the license used by the Linux kernel. The GPL is the most popular and
well-known example of the type of strong copyleft license that requires derived
works to be available under the same copyleft. Under this philosophy, the GPL
is said to grant the recipients of a computer program the rights of the free
software definition and uses copyleft to ensure the freedoms are preserved,
even when the work is changed or added to. This is in distinction to permissive
free software licences, of which the BSD licenses are the standard examples.   
     
\subsection{Apache License, Version 2.0}
\subsection{BSD}

\section{Ambiguous Terminals}
The W3C EBNF specification of XQuery Full Text (\cite{w3c01}) defines among other these terminals (somewhat simplified):
\begin{verbatim}
ElementContentChar         ::= Char - [{}<&]
QuotAttrContentChar        ::= Char - ["{}<&]
AposAttrContentChar        ::= Char - ['{}<&]
IntegerLiteral             ::= Digits
NCName                     ::= NCNameStartChar NCNameChar*
NCNameChar                 ::= Letter | Digit | '.' | '-' | '_'
NCNameStartChar            ::= Letter | '_'
\end{verbatim}
Where \verb!Char! denotes all possible legal characters in XQuery, \verb!Digits! is all possible number characters and \verb!Letter! denotes all possible letter characters. In their specification, the W3C uses a dash operator, which has the following
semantic meaning in a grammar (\cite{w3c03}, section 6):
\begin{quote}
A - B: matches any string that matches A but does not match B.
\end{quote}
In addition, the specification contains explicit defined litterals and symbols such as \verb!'for'! and \verb!'::'!, aswell as \verb!StringLitteral! which is indeed comparable with the non-terminal \verb!dirAttributeList! (simplified):
\begin{verbatim}
dirAttributeValue          ::= ('"' (QuotAttrContentChar)* '"')
                             | ("'" (AposAttrContentChar)* "'")
StringLiteral              ::= ('"' ([^"&])* '"') 
                             | ("'" ([^'&])* "'")
\end{verbatim}
Where \verb!QuotAttrContentChar! and \verb!AposAttrContentChar! is as defined earlier, and the hat operator (\verb!^!) is defined as follows(\cite{w3c03}, section 6):
\begin{quote}
[\^{}abc] matches any Char with a value not among the characters given.
\end{quote}
Meaning that e.g. \verb![^"&]! equals \verb!Char - ["&]!. It is easy to see that these productions would overlap significantly, thus the lexer must be aware of the context of a incomming character to differanciate between the rules. We considered several alternatives, as discussed in the following sections.

\subsection{Fuzzy Token Lexer}
A fuzzy token lexer is typicaly a simple lexer which leaves the meaning of the tokens somewhat ambiguous. Such a lexer may e.g. have a production \verb!Letters! which bundles all consecutive letter characters into tokens. The tokens would then be refined to their appropriate type during the context sensitive parsing process. Problems would occur, though, if e.g. the parser encounters a \verb!Letters! token but is expecting a token consisting of both letters, numbers and other symbols. The parser would then need to do the comlicated task of splitting and merging such tokens.

\subsection{Island Grammars}
Island grammars is a strategy where ambiguous parts of a grammar is taken out of the main section and turned into its own subgrammar. This will lead to a case where the system will consist of more than one lexer. The parser would then, in the power of being aware of the context, switch between the lexers when necessary. Communication between the lexers is a must, as they need to know where to start scanning in the input stream. ANTLR does not support island grammars by default, allthough there is thoughs about implementing it in future versions \footnote{http://www.antlr.org/wiki/display/ANTLR3/island+grammar+formalization}.

\subsection{State Driven Lexer}
With this method the lexer would only sets of words when in an appropriate state. The acceptance of a word in one state may lead to a state transition. I.e. the lexical recognition would be implemented as transition tables for a push-down automata. A suggestion for such transition tables for XQuery (not Full Text) are published as W3C working draft (\cite{createTokenizer}) with the following disclaimer:
\begin{quote}
The following tables were hand-constructed and have not, at the time of this writing, been exhaustively verified against all possible paths that may be legal in the XQuery and XPath EBNFs, which is to say, it is possible they contain bugs.
\end{quote}
We made a prototype of a state driven lexer for a tiny subset of XQuery, leading us to the conclusion that it would be a cumbersome task to implement for a the full XQuery Full Text grammar.


\section{Summary}
We have now outlined the requirements for our parser generator, evaluated
several alternatives, and chosen one of these parser generators (ANTLR) based on
these requirements. We have also briefly presented debugging
strategies using ANTLRWorks and/or integrated development environments.

We have presented methods for performing general unit/regression testing
using jUnit as well as ANTLR-specific grammar unit testing using gUnit. We also
outlined how to use the XQuery test suite to perform coverage/conformity
benchmarking and compare our results to existing implementations.

Further, a small collection of popular open source licenses were reviewed, and
one (the new BSD license) was selected.

Finally we discussed the problem of ambiguous terminals in the grammar. A number of lexer strategies to solve this were presented and evaluated. We chose the parser controlled state driven lexer due to its intuitivity and ease of implementation.

The next chapter will present the parser generator chosen for this
project, ANTLR, in more detail.