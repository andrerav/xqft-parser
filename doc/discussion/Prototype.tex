\section{Implementasjonen}
\label{sect:disc:contextSens}
Chapter \ref{chapter:implementation} describes how a prototype was implemented
to demonstrate the ``Tainted Dependencies'' (TD) method. This implementation
was dependent on a number of constraints:
\begin{itemize}
  \item The availability of a free\footnote{By ``free'' is meant a liberal
  licensing and availability of source code} pre-made XQuery parser capable of
  producing abstract syntax trees
  \item The ability to manipulate syntax trees and re-write them into new
  structures (in particular to support normalisation to XQuery Core)
  \item The ability to translate syntax trees into MQL (relational algebra)
\end{itemize}

In section chapter \ref{chapter:method}, methods to achieve the goals
of the implementation were presented. The method chosen for parsing of the
abstract syntax tree (see section \ref{sect:method:tree_parsing}) was the
\emph{context sensitive visitor pattern}. This pattern laid the foundation for
a clean and simplistic implementation. The semantics of the tree parsing process
itself did not interfer unecessarily with the rest of the implementation.

The process of rewriting the abstract syntax tree was implemented as a
stand-alone visitor (the \texttt{RewriteVisitor} class). This implementation
exploited the visitor pattern extensively, resulting in a clear separation of
concerns. In particular, it seems to hold true that the visitor pattern
typically will cleanly separate a data structure from an algorithm.

As explain in section \ref{sect:impl:construct_mql} on page
\pageref{sect:impl:construct_mql}, the MQL is constructed as an in-memory tree
structure. This was done by instantiating a new \texttt{Operator} subclass
(the exact class depending on context) for every node in the tree. It is
important to note that even though this could become a performance bottleneck
for very large and complex queries, it is still an important trade-off. In
exchange for a theoretical performance problem, the implementation achieves a
higher level of maintainability.