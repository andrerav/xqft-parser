\section{Implementation}
\label{sect:disc:contextSens}
Chapter \ref{chapter:implementation} describes how a prototype was implemented
to demonstrate the ``Tainted Dependencies'' (TD) method. This implementation
was dependent on a number of constraints:
\begin{itemize}
  \item The availability of a free\footnote{By ``free'' is meant a liberal
  license and availability of source code} pre-made XQuery parser capable of
  producing abstract syntax trees
  \item The ability to parse and manipulate syntax trees and re-write them into
  new structures (in particular to support normalisation to XQuery Core)
  \item The ability to translate syntax trees into MQL (relational algebra)
\end{itemize}

In this section, the methods chosen to achieve the goals of the implementation
are discussed and elaborated upon.

\subsection{Manual vs. automated tree parser construction}
ANTLR provides a utility for automated construction of AST parsers. This
utility requires the specification of a separate ``tree grammar''. This tree
grammar is almost identical to the original parser grammar. Practically, the
parser grammar can be copied verbatime, renamed, modified slightly and used as
a tree  grammar. This process is described in detail in \cite{definitiveAntlr},
section 8.1.

This introduces a high level of redundancy. The two grammar specifications are
required to be somewhat identical with regards to their grammar structure; that
is,  redundant tokens can be removed. However, the rewrite rules are required to
be identical.

This creates a problem with maintainability. As the parser grammar and rewrite
rules are not freezed at this point but rather highly subject to change, any
changes made in the parser grammar will need to be transferred to the tree
grammar, and vice versa. 

In \cite{translators_should_use_tree_grammars}, Terence Parr argues that the
traditional visitor
pattern\footnote{http://en.wikipedia.org/wiki/Visitor\_pattern} only provides a
simplistic facility for triggering events on the AST, that no tree structure
validation is implicitly available, and that context information has to be
passed down through the tree during the parse or by setting global variables.

In another point of view strongly polar to that of Terence Parr, Andy Tripp
argues\cite{manual_tree_walking_is_better} that manual tree parsing is
better. He establishes the following points of argument which are of particular
interest to this project:
\begin{itemize}
  \item Duplication of code and effort -- the concept of "what is a valid AST"
  would have to be implemented in both the parser and the AST transformer phase
  [as a rebuttal to validation of AST]
  \item With regards to contextual information, There seems to be nothing wrong
  with depending on the physical structure of the AST 
  \item Defining a traditional parser in grammar is practical because the grammar
  usually resembles the ouput AST. In the case of a tree parser proposed by Parr
  where the grammar actually resembles the input AST, this mapping may break
  down completely if the output is another tree structure
\end{itemize}

In particular, the last point holds a strong indication that a tree grammar
may not be suited for this project, as the goal of this tree parser would be to
transform the AST into a relational algebra tree.

\subsection{AST rewriting and the visitor pattern}
In section chapter \ref{chapter:method}, methods to achieve the goals
of the implementation were presented. The method chosen for parsing of the
abstract syntax tree (see section \ref{sect:method:tree_parsing}) was the
\emph{context sensitive visitor pattern}. This pattern laid the foundation for
a clean and simplistic implementation. The semantics of the tree parsing process
itself did not interfer unecessarily with the rest of the implementation.

The process of rewriting the abstract syntax tree was implemented as a
stand-alone visitor (the \texttt{RewriteVisitor} class). This implementation
exploited the visitor pattern extensively, resulting in a clear separation of
concerns. In particular, it seems to hold true that the visitor pattern
typically will cleanly separate a data structure from an algorithm which is
operating on that structure.

\subsection{Constructed algebra trees and performance}
As explain in section \ref{sect:impl:construct_mql} on page
\pageref{sect:impl:construct_mql}, the MQL is constructed as an in-memory tree
structure. This was done by instantiating a new \texttt{Operator} subclass
(the exact class depending on context) for every node in the tree. It is
important to note that even though this could become a performance bottleneck
for very large and complex queries, it is still an important trade-off. In
exchange for a theoretical performance problem, the implementation achieves a
higher level of maintainability.