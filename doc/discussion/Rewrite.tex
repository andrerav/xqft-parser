\section{Normalisation and rewriting}
\label{sect:disc:rewriting}
In section \ref{sect:method:ast_rewrite} on page
\pageref{sect:method:ast_rewrite}, some methods for rewriting (normalising)
certain expressions to XQuery Core were described. In the prototype implementation,
these rewrites are made using the \texttt{RewriteVisitor} class. The advantage
of normalising to XQuery Core is simplification of the syntax tree while
maintaining full semantics. That is -- the final syntax tree may be bigger and
appear more complex. However, expressions such as FLWOR and path expressions
are split into smaller subexpressions that are easier to parse by themselves. 

This project has taken an pragmatic approach to normalisation. The rules
defined in chapter \ref{chapter:translation} do not rely on normalisations.
However, for the sake of simplicity in the prototype, a rewrite visitor was 
applied to simplify FLWOR expressions. This must be seen in the light of the
fact that XQuery Core is a very extensive specification\cite{xquery_semantics},
and so strict adherence to this specification would imply a substantially
larger amount of effort into normalisation.

Consider the normalisation of \texttt{RelativePathExpr/StepExpr} which is normalised
into a FLWOR expression\footnote{See
http://www.w3.org/TR/xquery-semantics/\#id-axis-steps for details}. This may be
counter-productive as the usage of the \textsf{scope} operator in MQL will
imply that this normalisation will somehow have to be reversed.

Furthermore, the normalisation of FLWOR expressions themselves require that
\texttt{where}-clauses are rewritten to \texttt{if..then..else} expressions. The rules for this
normalisation process is shown in section
\ref{sect:theory:xquery:core:rewriting_flwor} and figure
\ref{figure:xquery:where_mapping_rule} on page 
\ref{figure:xquery:where_mapping_rule}. However, in ``Tainting Dependencies''
(TD) the translation of a \texttt{where}-clause (rule \ref{rule:trans:TD:where} on page
\pageref{rule:trans:TD:where}) is optimised and shown to be more efficient than
the translation of an equivalent \texttt{if..then..else} expression (rule
\ref{rule:trans:TD:conditional} on page \pageref{rule:trans:TD:conditional}).
This is a paradoxical situation, and raises the question of whether other
normalisation rules may also affect the efficiency of the resulting
translation.

On a final note, the TD method in its current state does not rely on
denormalised XQuery -- however it is compatible with XQuery Core since XQuery
Core is a subset of XQuery.

% 
% \begin{itemize}
%   \item fordeler vs ulemper med \aa~skrive om til core
%   \item man mister jo informasjon\ldots. Hvis den er p\aa~denne m\aa ten --> gj\o re det akkurat
% 	  slik, en sp\o rring som skal gi tilsvarende svar er ikke sikkert at man kan
% 	  skrive p\aa~den samme m\aa ten helt uten videre..  
%   \item samme svar = samme utf\o relse = er dette en fordel?
%   \item kan man utnytte kunnskap om translation til \aa~optimisere xquery queries?
%   \item hva med \aa~bare skrive om det man trenger? Hva trenger man \aa~skriveom?
%   \item sl\aa~sammen scopes blir frem og tilbake og frem igjen om man tenker core\ldots /a/b/c-> for..for..for ->
%   /a/b/c
%   \item se om det er mer tungvinne ting Core har\ldots har ikke satt meg helt inn i alle detaljene der.. hva skjer
%   med predikater? blir de if.then.else? kan det hende at vi f\aa r un\o dvendige kryssprodukt?
%   \item sluppet where-regel hvis man skriver om til if-then-else (dagens where er mer effektiv enn en if() then no
%  e else tomt -> den VET at det som er usant blir kasta bort- jfr regel og select-operator)
%  \item TD forutsetter ikke fullversjon XQuery (anti core), da core er subset av fullversjon
% \end{itemize}