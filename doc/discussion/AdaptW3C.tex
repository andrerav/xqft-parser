\section{Adapting the W3C Grammar}
\label{sect:discussion:adaptW3C}

In the begining of this project, very inexperienced parser developers as we were, had the na\"{i}ve idea that since the XQuery grammar was given and expressed in EBNF, our task was to run this through a parser generator, possibly with some syntax changes, and the job would be done. After a while of trying to adapt the given grammar to something ANTLR would accept, we came to the conclusion that W3C probably intended the specification to be read by humans, not by computers.

An alternative to insert and adapt would be to write a grammar from scratch while keeping the semantics specified by W3C. We belive, however, that by doing so there is a risk of over simplifying, causing latent semantics such as operator precedence to be distorted, or in worst case lost alltogether. Another negative aspect with this approach is that it does not properly utilize the work allready done, making it more time consuming.

Actually, when it comes to the terminal productions

\underline{\textbf{\LARGE //TODO:}}

Rett valg? Putte inn og gj\o re om vs. bruke som spesifikasjon og lage egen..

stolte for mye paa EBNF fra W3C -> ment for mennesker, ikke datamaskiner

\underline{\textbf{\LARGE //ODOT:}}