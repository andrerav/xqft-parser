\section{Adapting the W3C Grammar}
\label{sect:discussion:adaptW3C}

In the begining of this project, very inexperienced parser developers as we were, had the na\"{i}ve idea that since the XQuery grammar was given and expressed in EBNF, our task was to run this through a parser generator, possibly with some syntax changes, and the job would be done. After a while of trying to adapt the given grammar to something ANTLR would accept, we came to the conclusion that W3C probably intended the specification to be read by humans, not by computers.

An alternative to insert and adapt would be to write a grammar from scratch while keeping the semantics specified by W3C. We belive that by doing so there is a risk of over simplifying, causing latent semantics such as operator precedence to be distorted, or in worst case lost alltogether. Another negative aspect with this approach is that it does not properly utilize the work allready done, making it time consuming compared to just adapting.

Considering the non-terminal productions, there were not many productions that would have to be adapted to fit ANTLR semantics. The adjustments came in form of introducing predicates to accommodate for the extra-gramatical constraints. In addition we rewrote some productions to reduce required parser look-ahead. \underline{\textbf{\LARGE //TODO:}} Ble det mer?

However, the terminal productions had to be completly rewritten. By choosing to
