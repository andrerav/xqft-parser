\section{Type System Considerations}
\label{sect:disc:typeSystem}
Currently, neither ``Tainted Dependencies'' (TD) nor the prototype
implementation assumes any form of availability of a type system. However
according to the formal semantics specification\cite{xquery_semantics} (and
also noted in \cite{pathfinder_compiling_xquery}), XQuery Core is inherently
fully statically typed. This suggests that full normalisation of queries to
XQuery Core would imply the availability of this. Static typing could help
solve some problems, such as distinguishing numeric predicates. However, XQuery
Core has semantics for solving this exact problem -- the predicate normalisation
mapping applies a typeswitch construct which contains the necessary logic to
differentiate between numeric predicates and other predicates. And regardless,
numeric predicates were solved in TD as per rule \ref{rule:trans:TD:predicate}.

Another related challenge is the lack of explicit typing in the MQL language.
The only concrete specification given by FAST is the fact that if a column is
typed, then if can only contain fields of that type. MQL has typed columns,
however it is not possible to specify type. For example, the \textsf{make()}
operator has no parameters for type specification. This is complicated further
by the fact that a XQuery sequence is simply just a sequence. The individual
items themselves can have vastly different types.

The intricate challenges related to typing are numerous. For example, consider
the sequence \texttt{(1, <a><b>2</b></a>, "3")}. This sequence can not be
represented in a relation without resorting to a BLOB\footnote{Binary
Large Object}-like data type for the value column. However, that implies that
the semantics of the second item as an XML element is lost unless it is somehow
serialised in a common format with which MQL is compatible. That again implies
that metainformation about fields may be required to indicate the type of the
contents.

Furthermore, the \texttt{isInScope()} function proposed in section
\ref{sect:method:marsAddedOperators} on page
\pageref{sect:method:marsAddedOperators} requires a scope column in the tuple
being tested. Consider this non-sensical example:

\begin{Verbatim}
for $i in (1,2,3) return $i/a
\end{Verbatim}

Somehow, this error must be discovered and prevented. Likewise, the following
example must be allowed without errors:

\begin{Verbatim}
for $i in (<a><b>1</b></a>, <a><b>2</b></a>, <a><b>3</b></a>) return $i/b
\end{Verbatim}

On a final note, non-heterogeneous sequences are seldom of practical use, and
can appear irrational. Path expressions always return nodes, string operations
always return strings, and so on. Non-heterogeneous sequences are, as far as
known to the authors, only specifiable by an end-user of XQuery, for example by
attempting to execute a query such as this:

\begin{Verbatim}
for $i in (1, <a><b>2</b></a>, "3") return $i * 2;
\end{Verbatim}

This particular example appears non-sensical, and will likely not execute. For
example, the Saxon parser returns this run-time error message when attempting
to execute the above query:

\begin{Verbatim}
XPTY0004: Unsuitable types for * operation
Query processing failed: Run-time errors were reported
\end{Verbatim}

Naturally, it appears that typing is an important but complex aspect of XQuery,
and several issues such as the ones described here needs to be solved for a
full implementation to take place.

% \begin{itemize}
%   \item Hvordan f\aa~til noe typesystem?
%   \item Hvordan ordner Pathfinder typer? Kan vi gj\o re det samme? Finnes det noen andre vi kan dra kunnskap fra?
%   \item lagre false som ``false'' enn s\aa~lenge.. kjipt med /a/b[/a/b/c] hvis c er slik: <c>false</c>
%   \item Mars st\o tter ikke forskjellige typer innenfor samme felt
%   \item En sekvens er en sekvens i XQuery\ldots ikke en sekvens av booleans
%   eller noder etc
%   \item Et ekstra felt som sier type?
%   \item Hva skjer med /a/b/c/text() vs /a/b/c ?
%   \item hva skjer om man lager en <a> hei <b> jeje </b> </a> variabel? Dette
%   m\aa~kunne representeres.
%   \item Hvis vi hadde hatt statisk og sterk typing s\aa~ hadde mye v\ae rt
%   ordna, f.eks \verb!for $i as xs:int in (1,2,3) return /a[$i]! s\aa vet man
%   med en gang at \verb![$i]! er en ``numeric predicate''.
%   \item typeswitch / instance of / cast / castable / treat as
%   \item There is however also a need to represent explicitly stated XML-nodes, as well
% 	as differentiate between the number \texttt{1} and the string \texttt{"1"}.
% 	This, and other issues about representing XQuery types will be treated in
% 	section \ref{sect:disc:typeSystem}.
% 	\item sequencetype\ldots  
% \item Hva med \$i = (1,2,3) \$i/hatt -> typefeil? kj\o re isInScope(scope) p\aa~noe som ikke har scope kolonne?
%   		\item constructors\ldots element, CDATA, attribute etc etc
%   		\item typeswitch? kan bli stress.. har ikke s\aa~mye typer enn\aa \ldots skal v\ae re diskusjon om typer i
%   		sect \ref{sect:disc:typeSystem}\ldots
% 	\end{itemize}