\section{ANTLR}
Rett valg?
Vi kjipa med CUP + JFlexxx, hva skjedde med \aa~skrive for h\aa nd?
Har ikke peiling p\aa~ CUP og JFlexxx, AST mye lettere med ANTLR? Mindre
omskriving syntaktisk? Har de predikater der? Kunne vi l\o st statetingen p\aa~
en annen m\aa te med CUP og Flexz?   

In this project we used ANTLR to generate a parser from a grammar
specification.As detailed in section \ref{sect:method:alternatives}, we
evaluated several alternatives before deciding to use ANTLR. One argument for
choosing a parser generator rather than rewriting the parser from scratch was to
save time. In particular, considering the ambiguities in the grammar
specification, it seems obvious that writing a parser from scratch would
have required an order of magnitude more time. Additionally, the quality of the
resulting parser would most likely have been questionable at best. However
we would have had more detailed control over the code, which when seen from a
more distant point of view, could have been benefitial with regards to
maintanence, documentation with javadoc, and quality assurance.

A major design decision was made when we decided to use a LL(k) parser rather
than a LALR parser. This decision was extensively discussed and made in
section \ref{sect:method:alternatives}. In retrospect, this decision was
crucial to this project and could have made a very big difference in the
outcome. ANTLR seems to have been a good choice. However it could have been
interesting, notwithstanding that it would be of mostly academic interest, to  
implement the same parser using alternative parser generators and compare their
performance and scalability.

In the case of the ambiguous terminals, which were solved using a ``parser
controlled state driver lexer'' (see section 
\ref{sect:amiguousgrammar:parserControlled}), this would possibly have been
somewhat easier to implement in JFlex/CUP and JavaCC since they don't prebuffer 
the tokens in their entirety from input before they are sent to the parser. This
problem and its solution was described in detail in section
\ref{sect:impl:parser_controlled_state_driven_lexer}.