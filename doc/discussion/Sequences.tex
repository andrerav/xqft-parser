\section{XQuery Sequences - {MADS}}
\label{sect:disc:singelton}
\textbf{\LARGE TODO: {MADS}}
\begin{itemize}
   \item (1,2) + 3 kan oppdages som feil ved at man merker iteratorvariable, atomiske og sekvenser + 
   \item at man synes det er flott for optimalisering av sequencebygging 
   \item order by (1,2) kan ogs\aa~oppdages. 
   \item In some cases, a processor can determine the result of an expression without accessing all the data that
   would be implied by the formal expression semantics. For example, the formal description of filter expressions
   suggests that \$s[1] should be evaluated by examining all the items in sequence \$s, and selecting all those
   that satisfy the predicate position()=1.
\end{itemize}

fra 2.3.4 \cite{w3c00}
\begin{quote}
Consider an expression Q that has an operand (sub-expression) E. In general the value of E is a sequence. At an
intermediate stage during evaluation of the sequence, some of its items will be known and others will be unknown.
If, at such an intermediate stage of evaluation, a processor is able to establish that there are only two possible
outcomes of evaluating Q, namely the value V or an error, then the processor may deliver the result V without
evaluating further items in the operand E. For this purpose, two values are considered to represent the same
outcome if their items are pairwise the same, where nodes are the same if they have the same identity, and values
are the same if they are equal and have exactly the same type.

There is an exception to this rule: If a processor evaluates an operand E (wholly or in part), then it is required
to establish that the actual value of the operand E does not violate any constraints on its cardinality. For
example, the expression \$e eq 0 results in a type error if the value of \$e contains two or more items. A processor
is not allowed to decide, after evaluating the first item in the value of \$e and finding it equal to zero, that the
only possible outcomes are the value true or a type error caused by the cardinality violation. It must establish
that the value of \$e contains no more than one item.
\end{quote}

\subsection{Effective Boolean Value - {MADS}}
\label{sect:disc:effBool}
\textbf{\LARGE TODO: {MADS}}
\begin{itemize}
\item VI M\AA~SJEKKE ATOMIZATION.
\item Effective Boolean Value
\item Kommer til \aa~bli feil n\aa r en sekvens blir brukt som boolean value n\aa~s\aa fremt vi ikke kan groupe\ldots
\item En evt groupingoperator m\aa~slippe igjennom tall\ldots. Evt select index=1, m\aa~se om (1,2,3) er bogus
\item The translator is in the logical context, $\Lambda$, if the AST node it is currently visiting is a successor
of a boolean operator or within the condition part of an \texttt{if..then..else} expression. In all other cases the
translator is in the default context, $\Delta$. If no context is mentioned in the inference rules the default
context is assumed. 
\item test expr i if
\item barna til and og or
\end{itemize}