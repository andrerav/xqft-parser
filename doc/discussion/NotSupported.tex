\section{XQuery features not supported}
\label{sect:disc:notSupported}

In this section we will present some of the features of XQuery at the time not supported by Tainting Dependencies,
and some ideas around how a possible solution may be realised. Some of the
features are not supported because they involve types, which is discussed in section \ref{sect:disc:typeSystem}. %As the MQL processor is still in development,
%there may be some uncertainty of what will and what will not be a function of the finished implementation. 
 

\subsection{Full-text extensions}
XQuery Full Text\cite{w3c01} is a superset of XQuery. A quick overview
of the extensions made to XQuery can be found in section \ref{sect:theory:xquery:fulltext_ext}. Tainting Dependencies does however at this time
support any of these features. One of the most important expressions of the
extension is the \texttt{ftcontains} expression. An excerpt of the EBNF specification of this expression and some of
its subexpressions can be seen in figure \ref{fig:disc:xqftEBNF}. The \texttt{FTPrimaryWithOptions} production is a
straight descendant of \texttt{FTMildNot}, and \texttt{FTWordsValue} is a straight descendant of
\texttt{FTPrimary}. The complete EBNF specification can be found in appendix \ref{appendix:xquery_ebnf}.
\begin{figure}[h]
\begin{Verbatim}
 [51] FTContainsExpr ::= RangeExpr ( "ftcontains" FTSelection FTIgnoreOption? )?
[144] FTSelection    ::= FTOr FTPosFilter* ("weight" RangeExpr)?
[145] FTOr           ::= FTAnd ( "ftor" FTAnd )*
[146] FTAnd          ::= FTMildNot ( "ftand" FTMildNot )*
...
[149] FTPrimaryWithOptions ::= FTPrimary FTMatchOptions?
[166] FTMatchOption  ::= FTLanguageOption
                       | FTWildCardOption
                       | FTThesaurusOption
                       ...
[152] FTWordsValue   ::= Literal | ("{" Expr "}")
\end{Verbatim}

\caption{Excerpt of W3C EBNF full text specification\cite{w3c01}
\label{fig:disc:xqftEBNF}}
\end{figure}

A simple \texttt{ftcontains} expression checking if a node contains a literal may be quite simple to translate.
This can be done by looking up the literal and joining the result with the node on their $scope$ attributes.
Something like this:
\begin{equation*}
\frac{}{e \mbox{\texttt{ ftcontains }}literal}\leadsto
\begin{array}{l}
\mbox{\textsf{\ldots}} \\ \quad
\mbox{\textsf{hhjoin([l.scope],[r.scope],\ldots;}} \\ \quad\quad
\mbox{\textbf{r(}}e\mbox{\textbf{)}\textsf{;}} \\ \quad\quad
\mbox{\textsf{\ldots}} \\ \quad\quad\quad
\mbox{\textsf{lookup(}}literal\textsf{))}
\end{array}
\end{equation*}

\texttt{ftand} and \texttt{ftor} expressions may extend upon this solution. These operators makes it possible to
check for more than one term per node. If the translator keeps track of which scope is the current scope according
to a path expression, as discussed in section \ref{sect:disc:optim:path}, this can be translated quite nicely into
MQL. MQL supports two operators \textsf{and} and \textsf{or} which when surrounded by a \textsf{scope} operator
will require the results from the two operands to stem from the same scope. A simple \texttt{ftand} expression may
therefore be translated something like this:

\begin{equation*}
\frac{}{e \mbox{\texttt{ ftcontains }}literal_1 \mbox{\texttt{ ftand} }literal_2}\leadsto
\begin{array}{l}
\mbox{\textsf{\ldots}} \\ \quad
\mbox{\textsf{hhjoin([l.scope],[r.scope],\ldots;}} \\ \quad\quad
\mbox{\textbf{r(}}e\mbox{\textbf{)}\textsf{;}} \\ \quad\quad
\mbox{\textsf{\ldots}} \\ \quad\quad\quad
\mbox{\textsf{scope(}}e.scope\mbox{\textsf{;}} \\ \quad\quad\quad\quad
\mbox{\textsf{and(}} \\ \quad\quad\quad\quad\quad
\mbox{\textsf{lookup(}}literal_1\textsf{);}\\ \quad\quad\quad\quad\quad
\mbox{\textsf{lookup(}}literal_2\textsf{)}
\end{array}
\end{equation*}

A problem do however arise when the operands of \texttt{ftcontains} are not a node and a literal. As can be seen
from the specification in figure \ref{fig:disc:xqftEBNF}, a general expression may also be an operand. This is no
simple task to accomodate for. One possible solution would be to let the \textsf{lookup} operator take a relation
as input. There will also be problems if the first operand is not a node. If this is the case, there is no $scope$
attribute to join on. An example of such a query might be: 
\begin{center}
\texttt{"a man and his dog" ftcontains "dog"}
\end{center}
Here, the MQL processor would have to split up and search through the first operand for any matches with the
second.

\texttt{FTMatchOption} contains a great deal of options which modify in the way two terms or phrases are matched.
The options are specified like e.g.\ \texttt{with stemming} and \texttt{with thesaurus}. One possible
solution for accomodating for such options would be a context operator comparable to the \textsf{index} operator
(section \ref{sect:method:marsOperators:index}). The operator would take the match options as parameters, and set
the context for possible \textsf{lookup} operators within its subtree. Another possibility would be to use the
options as parameters directly to the \textsf{lookup} operator.

\subsection{Ordering mode}
\label{sect:disc:not:ord}
XQuery contains \texttt{ordered} and \texttt{unordered} expressions. The purpose of these expressions is to set
the ordering mode in to \texttt{ordered} or \texttt{unordered} for a certain region in a query. The expressions
set up an environment enclosed by curly braces in which the specific ordering mode applies. The default ordering
mode is \texttt{ordered}. A performance advantage may be realised by setting the ordering mode to
\texttt{unordered} for expressions where the ordering of the result is not significant. The system will then be
granted the flexibility to return the result in the order it finds most efficient.

The $index$ attribute and the \textsf{numberate} operator are responsible for ensuring correct order in Tainting
Dependencies. One of the problems with MarkXRemove was that it did not consider the ordering of items, while one
of its advantages was its simplicity. By introducing the concept of tainting to MarkXRemove, this would probably
be a good start for finding a method of translating in \texttt{unordered} mode.
But as TD is an evolution of MarkXRemove and a more complete method, a better
solution might be to simplify this method by removing \textsf{numberate} operators and all operators whose only intent is to manipulate $index$ fields. Utilising context sensitive visitor patterns
(section \ref{sect:theory:contextVisitorPattern}), differentiating translation of \texttt{ordered} and
\texttt{unordered} mode expressions will be made easy.


\subsection{Binary expressions}
\label{sect:disc:not:binary}
In this section we will present some of the XQuery binary operator expressions not handled in section
\ref{sect:trans:TD:binary}, and some ideas for possible translations of them.

\subsubsection{Node comparison operators}

The node comparison operators of XQuery are \texttt{is}, \texttt{<<} and
\texttt{>>}, and are currently not handled by Tainting Dependencies. A comparison with the \texttt{is} operator yields $true$ if the two operand nodes have the
same node identity. Where node identity is defined by W3C like this \cite{w3c04}:
\begin{quote}
Each node has a unique identity. Every node in an instance of the data model is unique: identical to itself, and
not identical to any other node.
\end{quote}

One solution would be to translate such expressions into checking if the two
nodes have the same value in their respective $scope$ and $documentId$ fields:
\begin{equation*}
\frac{}{e_1 \mbox{\texttt{ is }} e_2} \leadsto
\begin{array}{l}
\mbox{\textsf{\ldots}} \\ \quad
\mbox{\textsf{project(value = and(eq(l.docId,r.docId), eq(l.scope, r.scope))\ldots;}} \\ \quad\quad
\mbox{\textsf{hhjoin([}}(e_1.\vartheta\cap e_2.\vartheta)\mbox{\textsf{], [}}(e_2.\vartheta\cap e_1.\vartheta)
\mbox{\textsf{],\ldots;}} \\ \quad\quad \quad
\mbox{\textbf{r(}}e_1\mbox{\textbf{)}} \\ \quad\quad \quad
\mbox{\textbf{r(}}e_2\mbox{\textbf{)}}
\end{array}
\end{equation*}
This holds true as $documentId$ is unique per document, and no two nodes may be in the exactly same position within
one document.

A comparison with the \texttt{<<} and \texttt{>>} operators returns $true$ if, in document order, the left operand
node precedes the right operand node and if the left operand node follows the right operand node, respectively.
Otherwise it returns $false$. These operators have semantics comparable with the \texttt{preceding} and
\texttt{following} axes in path expressions, and may therefore be translated by utilising the \textsf{isPreceding}
and \textsf{isFollowing} functions (section \ref{sect:method:marsAddedOperators}, page
\pageref{sect:method:marsAddedOperators}):

\begin{equation*}
\frac{}{e_1 \mbox{\texttt{ COMP }} e_2} \leadsto
\begin{array}{l}
\mbox{\textsf{\ldots}} \\ \quad
\mbox{\textsf{project(value = isFUNK(l.scope, r.scope),\ldots;}} \\ \quad\quad
\mbox{\textsf{hhjoin([}}(e_1.\vartheta\cap e_2.\vartheta)\mbox{\textsf{], [}}(e_2.\vartheta\cap e_1.\vartheta)
\mbox{\textsf{],\ldots;}} \\ \quad\quad \quad
\mbox{\textbf{r(}}e_1\mbox{\textbf{)}} \\ \quad\quad \quad
\mbox{\textbf{r(}}e_2\mbox{\textbf{)}}
\end{array}
\end{equation*}

When the comparison operator \texttt{COMP} is \texttt{<<} the MQL function
\textsf{isFUNK} must be \textsf{isPreceding}, and when the operator is \texttt{>>} the function must be \textsf{isFollowing}.

\subsubsection{Combining node sequences}
XQuery provides the following operators for combining sequences of nodes:
\begin{itemize}
  \item \texttt{union} and \texttt{\textbar}(read as ``or'') which are equivalent. They take
  two node sequences as operands and return a sequence containing all the nodes that occur in either of the operands.
  \item \texttt{intersect} takes two node sequences as operands and returns a sequence containing all the
  nodes that occur in both operands.
  \item \texttt{except} takes two node sequences as operands and returns a sequence containing all the nodes
  that occur in the first operand but not in the second operand.
\end{itemize}

As all these operators eliminate duplicate nodes from their result sequences (based on node identity), the
operators are the XQuery equivalent to the relational algebra operators union, intersect and difference discussed
in section \ref{sect:theory:relAlg} on page \pageref{sect:theory:relAlg}. The result is also required to be in
document order.

MQL does currently not implement a pure \textsf{distinct} operator, but if it
did it would need partition functionality similar to that of the \textsf{numberate} operator, and a translation of
\texttt{union} and \texttt{\textbar} may have looked something like this:
\begin{equation*}
\frac{}{e_1 \mbox{\texttt{ union }} e_2} \leadsto
\begin{array}{l}
\mbox{\textsf{numberate(index, [documentId, scope], [}}e_1.\vartheta\cup e_2.\vartheta\mbox{\textsf{];}} \\ \quad
\mbox{\textsf{distinct([documentId, scope], [}}e_1.\vartheta\cup e_2.\vartheta\mbox{\textsf{];}} \\ \quad\quad
\mbox{\textsf{union(}} \\ \quad\quad\quad
\mbox{\textbf{r(}}e_1\mbox{\textbf{)};} \\ \quad\quad \quad
\mbox{\textbf{r(}}e_2\mbox{\textbf{)}\textsf{)))}}
\end{array}
\end{equation*}

Where the first parameter list of \textsf{distinct} is the field combinations which need to be distinct, and the
second list is the fields to partition on. This solution also requires the possibility sort on $scope$ fields.

The \texttt{intersect} operator may be implemented as a join on their node identity, that is, $documentId$ and
$scope$. The result will have to be run through a \textsf{distinct} operator like in the case of the
\texttt{union} expression, in case any of the operand sequences contains duplicates. A solution might look
something like:

\begin{equation*}
\frac{}{e_1 \mbox{\texttt{ intersect }} e_2} \leadsto
\begin{array}{l}
\mbox{\textsf{numberate(index, [documentId, scope], [}}e_1.\vartheta\cup e_2.\vartheta\mbox{\textsf{];}} \\ \quad
\mbox{\textsf{distinct([documentId, scope], [}}e_1.\vartheta\cup e_2.\vartheta\mbox{\textsf{];}} \\ \quad\quad
\mbox{\textsf{hhjoin([docId,scope,}}(e_1.\vartheta\cap e_2.\vartheta)\mbox{\textsf{],
[docId,scope,}}(e_2.\vartheta\cap e_1.\vartheta) \mbox{\textsf{],\ldots;}} \\ \quad\quad \quad 
\mbox{\textbf{r(}}e_1\mbox{\textbf{)};} \\ \quad\quad \quad
\mbox{\textbf{r(}}e_2\mbox{\textbf{)}\textsf{)))}}
\end{array}
\end{equation*}


The translation of the \texttt{except} operator may have been solved with the help of a anti-join (section
\ref{sect:theory:relAlg:antiJoin} on page
\pageref{sect:theory:relAlg:antiJoin}). But as MQL does not implement a
anti-join operator, the same effect may be achieved by using a left-outer-join
followed by a selection. A \textsf{distinct} operator is needed here aswell, as the left operand may contain duplicates:

\begin{equation*}
\frac{}{e_1 \mbox{\texttt{ intersect }} e_2} \leadsto
\begin{array}{l}
\mbox{\textsf{numberate(index, [documentId, scope], [}}e_1.\vartheta\cup e_2.\vartheta\mbox{\textsf{];}} \\ \quad
\mbox{\textsf{\ldots}} \\ \quad
\mbox{\textsf{distinct([documentId, scope], [}}e_1.\vartheta\cup e_2.\vartheta\mbox{\textsf{];}} \\ \quad\quad
\mbox{\textsf{select(eq(r.value, NULL);}} \\\quad\quad\quad
\mbox{\textsf{hhjoin([docId,scope,}}(e_1.\vartheta\cap e_2.\vartheta)\mbox{\textsf{],
[docId,scope,}}(e_2.\vartheta\cap e_1.\vartheta)\mbox{\textsf{],}}\\\quad\quad\quad\quad\quad\quad
\mbox{\textsf{\ldots, left;}} \\ \quad\quad \quad\quad 
\mbox{\textbf{r(}}e_1\mbox{\textbf{)};} \\ \quad\quad \quad\quad
\mbox{\textbf{r(}}e_2\mbox{\textbf{)}\textsf{))))}}
\end{array}
\end{equation*}

As can be seen, all these proposals involve multiple resource-expensive
operators such as \textsf{numberate} and \textsf{distinct}. So a better solution would be preferred, or at least to
minimise the use of the sequence combination operators when forming XQuery queries.

\subsubsection{Range expressions}
Range expressions are in the format $e_1$ \texttt{to} $e_2$ and can be used to construct a sequence of consecutive
integers. Both operands must be integers or castable to \texttt{xs:integer}. If a operand is an empty
sequence, or the integer value of first operand is greater than the integer value of the second, the
result is an empty sequence. Otherwise, the result is a sequence containing the two integers and every
integer between the two operands, in increasing order.

If the operands are literals, the translator can turn the expression into a sequence construction expression
containing the needed literals. However, if the value of the operands is not known before runtime, such
expressions will have to be handled by the MQL processor. The most natural
solution to this would be to implement a MQL operator taking one relation as input, two attribute names to create a sequence from and to, and a
list of attributes names for fields that must be a part of the result. The
operator would then create tuples based on the value of the fields specified as the from and to attributes, and augment each produced tuple with a copy of
the value of the fields specified to be part of the result. This would look something like:

\begin{equation*}
\frac{}{e_1 \mbox{\texttt{ to }} e_2} \leadsto
\begin{array}{l}
\mbox{\textsf{\ldots}}\\
\mbox{\textsf{rangeExpr([l.value, r.value], [}}e_1.\vartheta\cup e_2.\vartheta\mbox{\textsf{];}} \\ \quad\quad
\mbox{\textsf{hhjoin([}}(e_1.\vartheta\cap e_2.\vartheta)\mbox{\textsf{],
[}}(e_2.\vartheta\cap e_1.\vartheta) \mbox{\textsf{],\ldots;}} \\ \quad\quad \quad 
\mbox{\textbf{r(}}e_1\mbox{\textbf{)};} \\ \quad\quad \quad
\mbox{\textbf{r(}}e_2\mbox{\textbf{)}\textsf{)))}}
\end{array}
\end{equation*}

The \textsf{rangeExpr} operator would be a very specialised operator, and probably quite complex to implement.
If no better translation option is found, a better solution would be to require range expressions to have literals
as operands.

\subsection{Order by}
\label{sect:disc:orderby}

As can be seen from the \texttt{order by}-clause specification of figure \ref{fig:trans:TD:ordEBNF} on page
\pageref{fig:trans:TD:ordEBNF}, the ordering of tuples returned from a FLWOR expression is very flexible as it may
be set by one or more ordering specifications. Options may also be set for each order specification, called order
modifiers. Currently, Tainting Dependencies only accomodates for a single order specification and no order
modifiers. Expanding the translation rule for \texttt{order by} ordering (rule \ref{rule:trans:TD:orderbyOrd}) to
allow multiple ordering specification may be done by sequentially joining the specifiers on their common
dependencies before joining the result of this with the relation stemming from the \texttt{where} or
\texttt{return}-clause, and finally renumbering while sorting on the values from the orderspecifier in the correct
order:

\begin{equation*}
% dette maa staa her ellers kompilerer det ikke
\frac{}{\mbox{\texttt{order by }}e_{3 \cdot 1}\mbox{\texttt{,\ldots,}}e_{3 \cdot n}} \leadsto
\begin{array}{l}
\mbox{\textsf{project(value = l.value, }}\vartheta\mbox{\textsf{;}}\\ \quad
\mbox{\textsf{numberate(index, [value1,\ldots,value}}n\mbox{\textsf{], [}}\vartheta\mbox{\textsf{];}} \\ \quad
\quad \mbox{\textsf{hhjoin(\ldots [l.value,r.value1,\ldots,r.value}}n\mbox{\textsf{,\ldots];}} \\ \quad
\quad \quad \mbox{\textbf{t(r(}}e_C\mbox{\textbf{), }}\beta\mbox{\textbf{)}\textsf{;}} \\ \quad \quad \quad
\quad \mbox{\textsf{hhjoin(\ldots [value1= l.value,\ldots,r.value}}n\mbox{\textsf{,\ldots];}}
\\\quad\quad\quad\quad\quad
\mbox{\textbf{r(}}e_{3 \cdot 1}\mbox{\textbf{)}\textsf{)))}}\\\quad\quad\quad\quad\quad
\mbox{\textsf{hhjoin(\ldots}}\\\quad\quad\quad\quad\quad
\mbox{\textsf{\ldots}}\\\quad\quad\quad\quad\quad\quad
\mbox{\textsf{hhjoin(\ldots,value}}n\mbox{\textsf{=r.value\ldots;}}\\\quad\quad\quad\quad\quad\quad\quad
\mbox{\textsf{\ldots}}\\\quad\quad\quad\quad\quad\quad\quad
\mbox{\textbf{r(}}e_{3 \cdot n}\mbox{\textbf{)}\textsf{)))}}
\end{array}
\end{equation*}
Where $\vartheta=(e_C.\vartheta \cup e_{3 \cdot 1}.\vartheta \cup \ldots \cup e_{3 \cdot n}.\vartheta) - \beta$.
If the \texttt{order by} clause is defined as \texttt{stable}, $l.index$ will have to be added as a last attribute
to sort on in the \textsf{numberate} operator.

The best way to implement the order modifiers would probably be to allow some corresponding parameters to be
specified in the \textsf{numberate} operator. It may be cumbersome to implement such an operator allowing
modifiers to be specified for each of the attributes to sort on. In such a case, each of the order specifier
relations may have their $value$ fields sorted and updated by their own \textsf{numberate} operator.


	
		
\subsection{XQuery functions}
\label{sect:disc:functions}

XQuery supports numerous built-in functions, such as \texttt{fn:not} and \texttt{fn:count}, all
specified in \cite{w3cfuncOps}. These functions are identified by being a member of the \texttt{fn}-namespace. In
addition, XQuery allows users to declare functions of their own. A function declaration specifies the name of the
function, the names and datatypes of the parameters, and the datatype of the result, as can be seen from the
specification in figure \ref{fig:disc:not:funcEBNF}.

\begin{figure}[h]
\begin{Verbatim}
[26] FunctionDecl ::= "declare" "function" QName "(" ParamList? ")" 
                       ("as" SequenceType)? (EnclosedExpr | "external")
[27] ParamList    ::= Param ("," Param)*
[28] Param        ::= "$" QName TypeDeclaration?
\end{Verbatim}
\caption{W3C XQuery function declaration specification\cite{w3c00}\label{fig:disc:not:funcEBNF}}
\end{figure}

The \texttt{external} keyword means that the function is implemented outside the query environment. This is
discussed in section \ref{sect:disc:ctxItem}. To support built-in and query declared functions with TD, a
function table may be introduced. By querying this table with the name of the function, a operator tree is
returned with pointers to possible references of the parameters within the tree. When the translator comes upon a
function call, it fetches this tree and follows the pointers to the parameter references and inserts the
corresponding algebra tree. The built-in functions will have to be hardcoded and reside within the function table
at startup.
