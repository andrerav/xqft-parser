% Discussion / ScopingAndSymbol
\section{Scoping, Symbol Tables, and Typechecking}
In this section we will discuss scoping, symbol tables, and type
checking/inference.

\underline{\textbf{\LARGE //TODO:}}Andreas: skrive generelt om hvordan scoping
og symtabs funker naa.

\subsection{Scoping and Symbol Tables on AST}
\label{sect:discussion:scoping_and_symtab}
Currently the scoping and symbol tables are being used directly in the grammar -
that is, during parse time, as described in section
\ref{sect:impl:scoping_and_symtab}. It would be benefitial to move this
functionality into being performed in run time, and combine with type checking
functionality. This possibility was previously mentioned in section
\ref{sect:discussion:ast:extend}.

This would decouple some amount of code from the grammar, improving readability
slightly and creating more self-contained parser with no forced semantic checks
since this would be implemented in the AST parser instead.

The current implementation is adequate, but maintainability and extendability
suffers from the constraints of being locked into a monolithic grammar file.

\subsection{Type Checking}
Currently the parser will not perform type checking on the parsed queries. This
is an essential feature and will be necessary to implement for the parser to be
applicable in any realistic setting. A type checking system with proper type
inference and synthesis could be a complex feature to implement in a language
such as XQuery, and might require considerable effort, especially in quality
assurance. 

However, the resulting AST output from the parser (presented in section
\ref{sect:results:parser_output_ast}) should be a solid base to improve upon by
adding proper type inference and type checking. This would require a framework
for dataflow analysis. This is a substantial amount of work, but it would
benefit the project in several ways, for example the possibility of adding
optimizations.