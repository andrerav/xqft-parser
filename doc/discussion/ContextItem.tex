\section{External Environment}
\label{sect:disc:ctxItem}

XQuery provides some different ways to communicate with the external environment, that is, the world outside the
query. One of these features is that variables and functions may be declared with the \texttt{external} keyword. A
query containting such expressions will have to be translated by the translator, leaving space where the
expressions are refered to. These spaces will have to be filled by another part of the system. If the contents of
the variables and functions are known before runtime, a better solution would be to insert them into the symbol
and function tables of the translator before translation commences.

The functions \texttt{fn:doc} and \texttt{fn:collection} also provide access to external data. The \texttt{fn:doc}
function takes a string containing a URI. If that URI is associated with a document among the available documents,
the function returns a document node representing the given document. Conformant to Tainting Dependencies, this
function would need to map the document name to a document id, and return a relation containing one tuple with
this document id in the $documentId$ field and a empty $scope$ field.

A collection can be any sequence of nodes. The \texttt{fn:collection} function also takes takes a string
containing a URI. The URI will have to be associated with a collection, which is returned from the function. To
accomodate for this function with Tainting Dependencies, the collection returned must be in form of a relation
with $index$, $documentId$, $pos$, $scope$ and $value$ fields. The collection may of course be a collection of
document nodes.

The final way for XQuery to access the external environment is through initial values (appendix C.2 of
\cite{w3c00}). Most important of these is the context item and the default collection. The default collection is
referred to with a function call with \texttt{fn:collection} without an URI as parameter. The initial context item
is referred to explicitly with the \verb!.! (dot) operator, or implicitly at the start of a path expression. These
features may be suported in Tainting Dependencies by inserting the values in the function and symbol table,
respectively.